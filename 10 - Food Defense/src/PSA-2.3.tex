\thispagestyle{formato-PI}
\renewcommand{\MayorVer}{2}
\renewcommand{\MenorVer}{0}
\renewcommand{\Codigo}{PSA-3-PRO}
\renewcommand{\FechaPub}{2023--01}
%\renewcommand{\Edit}{2.1}
\renewcommand{\Titulo}{Inspección de producto durante su almacenamiento}

\section{\Titulo} \label{PRO:InspeccionDeAlimentosAlmacenados}\index{Inspección!productos durante su almacenamiento, de}

\subsection{Objetivo}
Establecer un programa de inspección que asegure que las condiciones de almacenamiento se encuentren en condiciones óptimas como resultado de un adecudo mantenimiento de instalaciones, limpieza y temperatura apropiada para la conservación de productos alimenticios.

\subsection{Alcance}
\begin{itemize}
	\item Este documento se destina al área operativa, personal de mantenimiento y personal de aseguramiento de calidad.
\end{itemize}

\subsection{Términos y definiciones}
\begin{description}
	\defglo{alimento}
	\defglo{producto}
\end{description}

\subsection{Documentos y/o normas relacionados}
\begin{itemize}
	\item Programa de \glsfirst{BPD}.
\end{itemize}

\subsection{Procedimiento} \label{sec:pro:inspeccionDeProducto:almacenamiento}
\subsubsection{Materiales}
\begin{itemize}
	\item Linterna \textit{(Opcional).}
\end{itemize}

\subsubsection{Precauciones de Seguridad}
\begin{itemize}
	\item Usar uniforme completo.
\end{itemize}

\subsection{Instrucciones}
\subsubsection{Temperatura de almacenamiento de alimentos refrigerados y congelados}
\begin{enumerate}
	\item Para asegurar que todo \gls{alimento} refrigerado o congelado se mantenga a la temperatura adecuada durante el almacenamiento, todos los productos se deben colocar directamente después de ser recibido en la cámara de refrigeración o congelación según la asignación, una vez que se descarguen.
	\item Las cámaras de refrigeración y de congelación están equipadas con un \emph{termómetro calibrado} ubicado en el \emph{rack,} en el cual se puede verificar la temperatura de cada cámara.
	\item Las lecturas de temperaturas deben ser documentadas \VecesTempManual veces durante el turno los días laborados.
	\item El Supervisor de mantenimiento es responsable de la verificación de temperaturas.
	\item Las puertas del andén de embarques deben permanecer cerradas en todo momento a menos que la unidad este siendo utilizada para carga y descarga de productos.
	\item Las puertas del andén deben de ser abiertas cuando la unidad de transporte ya se encuentre colocada en la rampa, esto con el fin de evitar fuga de temperatura y/o entrada de polvo o fauna nociva.
	\item Cualquier daño a las cámaras de refrigeración o congelación (techo, paredes, puertas, cortinas, etc.) debe ser reparadas en el menor tiempo posible.\footnote{El Supervisor de mantenimiento es responsable de garantizar que todo el equipo esté en buen estado y funcionando adecuadamente.}
\end{enumerate}

\subsubsection{Almacenaje y manejo de Productos Secos / Ambiente, Refrigerados y Congelados}
\begin{enumerate}
	\item Para asegurarse de que los productos almacenables a temperatura ambiente, refrigerados, y congelados son almacenados y protegidos de daños físicos, químicos, biológicos (contaminación), todos los productos deben ser almacenados \qty{\EspacioMinimoParedProducto}{\centi\meter} desde todas las paredes interiores\footnote{Si el diseño del almacén lo permite}, sobre atados limpios, \qty{\EspacioMinimoTechoProducto}{\centi\meter} de las superficies del techo, así como lejos de todas las luces superiores y nunca almacenados directamente en contacto con los pisos\footnote{todo \gls{alimento} almacenado en \gls{RDF} debe de estar sobre una tarima---no puede estar en contacto directo con el piso.}
	\item Todos los productos deben guardarse en su empaque secundario y etiquetados con las fechas de recepción necesarias.
	\item Todos los productos deben ser rotados de acuerdo en el principio de \textit{“First-in-First-Out”} \textbf{(FIFO)}\footnote{En español, \gls{PEPS}} o primeras entradas primero a vencer (O según instrucciones de cada cliente).
	\item En caso de ser requerida, una \emph{área de retención} debe asignarse en el área de almacenamiento para segregar cualquier artículo recuperado, muestras de productos, producto dañado. \index{Área de retención}
	\item Todos los productos deben mantenerse libres de polvo y suciedad en todo momento.
	\item Cualquier o todos los productos químicos almacenados en la instalación deben estar completamente separados de todos los envases y productos alimenticios.
	\item Todos los productos químicos ubicados en la planta deben de tener un MSDS.
\end{enumerate}

\subsubsection{Control y mantenimiento de temperatura}

\begin{enumerate}
	\item Es imperativo que las cámaras en refrigerado y congelado en nuestras instalaciones se mantengan a la temperatura adecuada en todo momento. Esto incluye horas regulares, después de horas regulares, fines de semana y días festivos.
	\item Se tiene que hacer un seguimiento continuo de las temperaturas de las cámaras de almacenamiento, durante el horario normal las temperaturas son supervisadas y documentadas al día\footnote{Inicio de turno, medio turno y fin de turno} de lunes a sábado.
	\item El Supervisor de Mantenimiento es inmediatamente informado de cualquier irregularidad detectada en la temperatura,\footnote{Incluyendo horas después del trabajo, fines de semana y días festivos} de modo que las acciones correctivas se puedan tomar.
\end{enumerate}

\begin{note}[Mediciones de temperaturas] \label{nota:MedicionDeTemperaturas}
	Con motivos de controlar las temperaturas de las cámaras de almacenamiento, se cuentan con múltiples sistemas de medición para mantener una recursividad adecuada:
	\begin{itemize}
		\item El \emph{controlador de las unidades de refrigeración} es la principal fuente de información sobre las temperaturas de las cámaras. La información que el controlador recaba se basa en los termopares individuales de cada cámara, estos están debidamente considerados en el \emph{plan de metrología} y son calibrados anualmente;
		\item Semanalmente se colocan \emph{termograficadores} en las cámaras de almacenamiento y se verifican las temperaturas por el departamento de aseguramiento de calidad.
		\item Diariamente se comprueban las temperaturas de las cámaras de forma manual con termómetros infrarrojos por parte de:
		\begin{itemize}
			\item aseguramiento de calidad;
			\item mantenimiento.
		\end{itemize}
	\end{itemize}
\end{note}

\subsubsection{Plan de contingencia en caso de avería de equipos de refrigeración}\index{Plan de contingencia!en caso de avería de equipos de refrigeración}
Con fin de garantizar que todos los productos refrigerados y congelados se encuentren protegidos contra cualquier peligro microbiológico asociado con el incremento de la temperatura, las cámaras con refrigeración o congelación que presenten una falla y que no puedan ser reparadas en menos en el caso de refrigeración de \qty{\TiempoAveriaRefri}{\hour} y en congelación de \qty{\TiempoAveriaConge}{\hour},\footnote{Siempre y cuando las puertas de cámara se encuentren cerradas y el producto mantenga la temperatura establecida} el producto se \emph{debe} trasladar oportunamente a otra cámara previamente acondicionada a las especificaciones del producto.

\subsection{Responsables de la actividad}

\begin{itemize}
	\item \textbf{Ejecutado} por personal de operaciones y mantenimiento;
	\item \textbf{Monitoreado} por personal de calidad;
	\item \textbf{Verificado} por personal de gerencia.
\end{itemize}

\subsection{Acciones preventivas}

\begin{itemize}
	\item Se llevará a cabo una inspección. Registrando el indicador y medida correspondiente
	\item Si la desviación se repite frecuentemente se dará curso de capacitación al personal de almacén y limpieza para que realicen eficientemente su trabajo.
	\item Si después de haber capacitado al personal de almacén se siguen presentando desviaciones por causas injustificadas, se tomaran acciones más enérgicas con el personal por incumplimiento con sus deberes.
\end{itemize}

\subsection{Acciones correctivas}

\begin{itemize}
	\item En caso contrario la tarea se deberá volver a realizar como se indica en el procedimiento.
	\item En caso de no conformidad reportar en formato de acciones correctivas (\RAC.
\end{itemize}

\subsection{Frecuencia}

\begin{itemize}
	\item \textbf{Verificación manual de temperaturas:}
		\begin{enumerate}
			\item \emph{Dpto.\ de mantenimiento:} \VecesTempManual veces al día;
			\item \emph{Dpto.\ de aseguramiento de calidad:} \VecesTempManual veces al día;
		\end{enumerate}
\end{itemize}

\begin{changelog}[simple, sectioncmd=\subsection*,label=changelog-\thesection-\MayorVer.\MenorVer]

	\begin{version}[v=2.1, date=2023--01, author=Pablo E. Alanis]
		\item Cambio de formato;
		\item Cambios en la serialización de versiones;
		\item Correcciones ortográficas y de estilo;
		\item Se agregó \cref{nota:MedicionDeTemperaturas}.
	\end{version}

	\begin{version}[v=1.8, date=2022--05, author=Alonso M.]
		\item cambio de fecha;
	\end{version}

	\shortversion{v=1.7, date=2021--05, changes=No hubo cambios}
\end{changelog}