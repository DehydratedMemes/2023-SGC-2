\thispagestyle{formato-PI}
\renewcommand{\MayorVer}{2}
\renewcommand{\MenorVer}{0}
\renewcommand{\Codigo}{FD-1-MAN}
\renewcommand{\FechaPub}{2023--01}
%\renewcommand{\Edit}{2.1}
\renewcommand{\Titulo}{Manual de Defensa Alimentaria}

\section{\Titulo} \label{MAN:FoodDefense}\index{Programa!Defensa Alimentaria, de!Manual}
\subsection{Objetivo}
Establecer reglas y procedimientos, que asegure que se tornan las medidas necesarias para evitar cualquier actividad o situación voluntaria o involuntaria que pueda poner en riesgo directa o indirectamente la seguridad de los productos alimenticios que en esta empresa se manejan.

\subsection{Alcance}
\begin{itemize}
	\item Este documento detalla los lineamientos considerados por el programa de \gls{FD} de \gls{RDF}.
\end{itemize}

\subsection{Terminología y definiciones}
\begin{description}
	\item[\glsname{FD}] \glsdesc{FD} 
\end{description}

\subsection{Introducción}
El sitio debe preparar, implementar y mantener un protocolo de defensa alimentaria que se basa en una evaluación de la amenaza de la defensa alimentaria y describe los métodos, las responsabilidades y los criterios para prevenir la adulteración de los alimentos causada por actos deliberados de sabotaje. Este plan debe ser probado y revisado, como mínimo,
anualmente. El proveedor debe designar a un miembro de la alta dirección que sea responsable de la defensa alimentaria. Esta persona responsable debe asegurarse de que existen procedimientos para registrar y controlar el acceso a las áreas del sitio por parte de los empleados, contratistas y visitantes.

El plan de defensa alimentaria deberá ser probado o cuestionado para asegurar que las estrategias de defensa alimentaria son efectivas, apropiadas y documentadas. Las pruebas del plan de defensa alimentaria pueden basarse en la descripción de un escenario ficticio.

El protocolo debe identificar cómo el proveedor limita el acceso a las áreas designadas de la operación sólo a los empleados debidamente autorizados. El proveedor debe aplicar medidas para proteger los puntos de procesamiento sensibles de la contaminación intencionada. El protocolo debe explicar cómo la empresa garantiza el almacenamiento y el transporte seguros
de las materias primas, los envases, los equipos, los productos químicos peligrosos y el producto
acabado.

El protocolo debe definir cómo se van a abordar las zonas restringidas, controladas o de difícil acceso. El proveedor es libre de desarrollar medidas adecuadas para abordar áreas específicas para garantizar el control a través de una amplia variedad de soluciones.

\subsubsection{Registros}
Los siguientes son ejemplos de registros y/o documentos para ayudar a la aplicación y revisión de este tema: 

\begin{itemize}
	\item El plan de defensa alimentaria, que incluye el equipo de defensa alimentaria, la evaluación de la amenaza, las estrategias preventivas, los contactos y el plan de acción.
	\item Resultados de la revisión anual del plan de defensa alimentaria.
	\item Resultados de la prueba anual, o simulacro, al plan de defensa alimentaria.
	\item Hojas de registro, etiquetas de identificación u otros medios para controlar a los visitantes y empleados.
	\item Medidas correctivas de cualquier incumplimiento del plan
\end{itemize}

\subsection{Documentos y/o normas relacionados}
\begin{itemize}
	\item Procedimiento de Orden de entradas y Orden de salidas;
	\item Control de llaves de equipos;
	\item Procedimientos de entrada y salida de visitantes;
	\item Registro de entrada de clientes.
\end{itemize}

\subsection{Procedimiento}\index{Programa!Defensa Alimentaria, de!procedimientos}
Existe control de acceso para ingresar a oficinas y áreas administrativas de \gls{RDF} y acceso a las áreas Operativas de \gls{RDF}.
Reglamento de ingresos a \gls{RDF}
Requisitos de ingreso de entradas y salidas para visitantes y clientes.

\begin{itemize}
	\item El acceso a las áreas operativas y otros puntos cuentan con un sistema de circuito cerrado (CCTV) que trabajan las \qty{24}{\hour}.
	\item Se cuenta con vigilancia durante todo el turno laboral.
	\item El área de vigilancia controla el ingreso de proveedores y visitantes a las instalaciones previamente autorizados por el departamento correspondiente, los controles de acceso son llevados mediante registros.
	\item Cualquier persona autorizada para ingresar a las instalaciones deberá ser acompañada por personal de \gls{RDF} S.A de C.V.
	\item Registrarse en la bitácora de entrada y salida de visitantes el cual consta de los siguientes puntos:
	\begin{enumerate}
		\item Fecha
		\item Nombre
		\item Compañía
		\item Persona a quien visita
		\item Área que visita
		\item Asunto
		\item Hora de entrada y salida
		\item Firma
	\end{enumerate}
	\item Registrarse en la bitácora de entrada y salida de clientes el cual consta de los siguientes puntos:
	\begin{enumerate}
		\item Fecha
		\item Nombre
		\item Proveedor
		\item Cliente
		\item Identificación
		\item Hora de entrada y salida
		\item Folio de entrada y salida
	\end{enumerate}
	\item Si solo viene a las oficinas deberá esperar en recepción hasta ser atendido
	\item Si requiere de ingresar al área operativa deberá cumplir con el reglamento de ingreso y ser acompañado por la persona designada (Ver reglamento o política de entrada de áreas operativas).
	\item La persona ajena a la empresa será acompañado durante el total de su estancia dentro de la empresa
	\item Firmar la hoja de registro con la hora de salida
\end{itemize}

\subsubsection{Requisitos de ingreso para el personal}\index{Programa!Defensa Alimentaria, de!ingreso de personal}
\begin{itemize}
	\item Todo el personal deberá dejar sus pertenencias en el área de lockers (Esta área se encuentra fuera del área de operaciones) y ponerse su uniforme completo, este debe estar limpio.
	\item Antes de ingresar a las áreas operativas el personal debe colocar su lonche en el refrigerados para esta función (Esta área se encuentra fuera de las instalaciones de operación).
	\item El personal debe transitar por la aduana sanitaria y lavarse las manos (Ver reglamento de entrada de personal).
	\item En ningún momento se encuentra permitido el acceso dentro del almacén con mochilas o bolsas de mano y no se permite la entrada con ningún artículo personal dentro de las instalaciones.
	\item Al momento de la salida de planta el personal no puede llevar ningún artículo o producto al menos de que este sea autorizado por el jefe de área.
	\item El personal de vigilancia realiza inspección visual del personal y controla la salida del mismo
\end{itemize}

\subsubsection{Supervisión de empleados subcontratados}\index{Programa!Defensa Alimentaria, de!contratistas}
Toda persona que ingrese a la planta deberá cumplir ciertas reglas establecidas para su ingreso y permanencia en la misma.

\begin{itemize}
	\item Se registrara en la caseta de entrada y/o en oficinas administrativas.
	\item Esperará en recepción a la persona que lo va a atender
	\item Si requiere de ingresar al área operativa deberá cumplir con el reglamento de ingreso a planta.
	\item En todo momento deberá estar acompañado por la persona a la que visita.
	\item Para ingresar a cualquier otro departamento deberá tener la autorización y estar acompañado igual por personal de la planta
	\item Al final de visita la persona debe regresar a firmar la hoja de registro con la hora de salida en la caseta de vigilancia y/o recepción.
En el caso de proveedores que trabajan constantemente dentro de las áreas operativas se extiende un documento en el que la empresa confirma quien es el personal que presta sus servicios y que está autorizada para ingresar a la empresa.
\end{itemize}

\subsubsection{Contratación de personal}\index{Programa!Defensa Alimentaria, de!gestión de recursos humanos!contratación de personal}
\begin{itemize}
	\item Se cuenta con procedimiento para contratación de personal
	\item Se cuenta con el recopilado de información personal de la persona contratada.
	\item El personal se encuentra identificado por sistema, a través de número de nómina; como medio de reconocimiento de sus entradas y salidas a la planta.
	\item Al ser contratado el personal es ingresado a la base de datos de la empresa, se le designa un número de registro.
	\item Se informa al personal sobre el reglamento de trabajo de la empresa y BPD, así como los programas de calidad implementados.
\end{itemize}

\subsubsection{Notificación de baja del personal}\index{Programa!Defensa Alimentaria, de!gestión de recursos humanos!baja de personal}
\begin{itemize}
	\item De acuerdo a lo estableció en el plan de seguridad alimentaria lo importante es detectar alguna contaminación internacional en alguna parte del proceso; por lo que en el momento que se da de baja ya sea renuncia voluntaria o despido de algún trabajador, se da aviso de manera verbal a casetas de vigilancia donde se especifica el personal que ya no labora, con esto evitamos que personal que ya no labora en la empresa entre a las instalaciones sin la autorización requerida.
\end{itemize}

\subsubsection{Capacitación y entrenamiento del personal}\index{Programa!Defensa Alimentaria, de!gestión de recursos humanos!capacitación}
\begin{itemize}
	\item El personal de nuevo ingreso recibe capacitación de introducción, en el cual contiene temas de BPD, \gls{FD}, estos temas con la finalidad que el personal se encuentre entrenado en aspectos básicos del cómo actuar dentro de las áreas operativas y evitar acciones involuntarias de riesgo al producto.
	\item Se cuenta con programa de capacitación anual sobre temas que permitan el mejor desempeño y superación del personal que se verán reflejadas en los resultados, eficiencia y calidad en las actividades globales de las áreas operativas y por consecuencia en el producto que llegara al consumidor. Como resultados de este entrenamiento se busca el minimizar las acciones de peligro involuntarias que puedan poner el riego al producto.
	\item El personal recibe entrenamiento sobre \gls{FD} de manera anual como mínimo
\end{itemize}

\subsubsection{Sistema de monitoreo alarmas}\index{Programa!Defensa Alimentaria, de!sistemas de monitoreo}
En los ingresos se cuenta con sistema de alarma en puertas de acceso y sensores de movimiento en áreas de oficinas, las cuales son monitoreadas las \qty{24}{\hour} por el proveedor contratado.

\subsubsection{Vigilancia de actividad del personal}\index{Programa!Defensa Alimentaria, de!vigilancia}
\begin{itemize}
	\item Dentro de las áreas operativas se cuenta un sistema cerrado de cámaras (CIRCUITO CERRADO con grabación) que trabaja las 24 horas, estas se encuentra en diferentes puntos de las áreas operativas y ayudan a monitorear y controlar los movimientos dentro de las áreas operativas.
	\item La distribución de las cámaras es la siguiente:
	\item En el patio que abarca el área de andares y entrada principal a las áreas operativas
	\item En áreas de andares donde abarcan en total de los accesos principales de las áreas de refrigeración y congelación
	\item Personal autorizado se encarga del monitoreo de todas las cámaras para detectar alguna anormalidad.
	\item El área operativa cuenta con un jefe (Supervisor de almacén) que al mismo tiempo de coordinar las actividades de personal da aviso de cualquier conducta sospechosa.
	\item Anexado se encuentra un mapa de las áreas operativas planta donde se incluyen los puntos de localización de las cámaras.
	\item Las cámaras son verificadas de manera constante por el personal autorizado, las grabaciones quedan guardadas por 15 días. Las cuales pueden ser consultadas en el momento requerido.
\end{itemize}

\subsubsection{Control de llaves de ingreso al almacén}\index{Programa!Defensa Alimentaria, de!control de llaves}
\begin{itemize}
	\item Todos los accesos a las áreas operativas se encuentran cerrados con llave como medida de seguridad al final de las actividades operativas.
	\item Se tiene un llavero maestro al que solo tiene acceso el área de vigilancia y personal asignado para poder hacer uso de estas llaves.
	\item Las llaves están numeradas de acuerdo al flujo de entrada.
	\item En el llavero se encuentra una lista con el número y la descripción de la puerta a la que corresponde.
	\item Todas las llaves están restringidas para uso de personal no autorizado.
	\item Las llaves son entregadas únicamente al personal designado para cada área.
	\item Se cuenta con un registro de control de llaves. Cuando se realiza el préstamo de una llave para cuestiones operativas se debe de registrar.
	\item Las áreas restringidas se identifican en el procedimiento de áreas restringidas.
\end{itemize}

\subsubsection{Seguridad en proceso}\index{Programa!Defensa Alimentaria, de!seguridad en el proceso}
\begin{itemize}
	\item Se tiene implementado un sistema de Trazabilidad para rastrear producto recibido, almacenado, embarcado y distribuido.
	\item Se verifica la integridad de los productos y su empaque antes de su ingreso
	\item Se verifica la integridad de los productos y su empaque antes de su embarque.
	\item El inventario de producto terminado se lleva debidamente controlado y existen registros.
	\item Todos los productos se encuentran registrados en el sistema PROTHEUS.
\end{itemize}

\subsubsection{Áreas restringidas}\index{Programa!Defensa Alimentaria, de!áreas restringidas}
El personal responsable de áreas restringidas debe mantener estos accesos cerrados como son las áreas de:

\paragraph{Planta baja}\index{Programa!Defensa Alimentaria, de!áreas restringidas!planta baja}
\begin{enumerate}
	\item Acceso principal a patios de carga (Acceso restringido con barrera física portón y candado).
	\item Acceso Pasillo periferia de almacén (Acceso restringido con barrera física de puerta y llave).
	\item Puerta principal de acceso al almacén y Oficinas (Acceso restringido con barrera física puerta de seguridad y llave).
	\item Puerta principal de acceso a Oficina Supervisor (Acceso restringido con barrera física puerta de seguridad y llave).
	\item Medidor de Agua (Acceso restringido con barrera física)
	\item Registro de energía (Acceso restringido con barrera física de puerta y candado).
	\item Puertas de carga y descarga en el andén y rampas (Acceso restringido con barrera física puerta y barra de seguridad).
	\item Área artículos de limpieza (Acceso restringido con barrera física puerta y llave)
	\item Puerta de salida emergencia (Acceso restringido con barrera física puerta)
\end{enumerate}

\paragraph{Planta alta}\index{Programa!Defensa Alimentaria, de!áreas restringidas!planta alta}

\begin{enumerate}
	\item Área almacén de material de oficina y consumibles. (Acceso restringido con barrera física puerta y llave).
	\item Cuarto de controles de herramientas (Acceso restringido con barrera física puerta y llave).
	\item Cuarto de controles eléctricos (Acceso restringido con barrera física puerta y llave).
	\item Puerta principal de acceso a oficinas (Acceso restringido con barrera física puerta y llave).
\end{enumerate}

\subsubsection{Control de acceso para sistema de operación PROTHEUS}\index{Programa!Defensa Alimentaria, de!\gls{SGA}}

\begin{itemize}
	\item El acceso para el sistema de operación PROTHEUS para el control de producto recibido, almacenado, embarcado y distribuido se encuentran restringidos. Existe una lista de usuarios autorizados con una clave de acceso y son de los siguientes departamentos:
	\item Supervisor de almacén
	\item Mesa de Control
	\item Administrador General
	\item Facturación
	\item Gerencia de almacén
	\item El personal administrativo y de sistemas de cómputo son los responsables del control de autorizaciones al sistema.
\end{itemize}

\subsubsection{Entrada de Producto}\index{Programa!Defensa Alimentaria, de!ingreso de alimentos}

\begin{itemize}
	\item Se cuenta con inspección de transportes y productos (Ver formato de verificación de transporte), el cual consta de los siguientes puntos de verificación:
	\item Fecha
	\item Folio de verificación de registro
	\item Nombre del cliente
	\item Proveedor del cliente
	\item Temperatura de recibo
	\item Cantidad
	\item Número de lote
	\item Transporte
	\item Condiciones del transporte
	\item Libre de plaga
	\item Tarimas recibidas
	\item Observaciones
	\item Firmas
\end{itemize}

\subsubsection{Aéreas de productos de limpieza}\index{Programa!Defensa Alimentaria, de!áreas restringidas!almacen de productos de limpieza}

\begin{itemize}
	\item Los productos de limpieza se entregan por área específica y el pedido es controlado por el personal encargado del departamento de Mantenimiento de Edificio.
	\item Los productos de limpieza se mantienen como área de acceso restringido y bajo llave, solo el personal capacitado y autorizado para el manejo de productos de limpieza tiene acceso a esta área.
	\item El personal autorizado es el siguiente:
	\begin{itemize}
		\item Mantenimiento de Edificio.
		\item Mantenimiento de Equipos.
	\end{itemize}
\end{itemize}

\subsubsection{Salida de producto}\index{Programa!Defensa Alimentaria, de!salida de alimentos}

\begin{itemize}
	\item Se cuenta con un control de producto por lote durante su almacenamiento y de esta manera se controla el producto a embarcar, para fines de trazabilidad.
	\item Se cuenta con inspección de transportes y productos antes del embarque (Ver formato de verificación de transporte), el cual consta de los siguientes puntos de verificación:
	\item Fecha
	\item Folio de verificación de riesgo
	\item Nombre del cliente
	\item Proveedor del cliente
	\item Ruta
	\item Temperatura de embarque
	\item Pedido
	\item Maniobras de salida
	\item Picking
	\item Número de lote
	\item Transporte
	\item Condiciones del transporte
	\item Libre de plaga
	\item Observaciones
	\item Firmas
	\item Todas las entradas y salidas son registradas en el sistema interno de la empresa.
	\item El personal administrativo y de sistemas de cómputo son los responsables del control de autorización al sistema PROTHEUS.
	\item \gls{PPRO}
\end{itemize}

\subsection{Acciones correctivas}\index{Programa!Defensa Alimentaria, de!acciones correctivas}

\begin{itemize}
	\item Cuando se presenta una no conformidad en la realización del procedimiento, detectada por el supervisor o encargada del área se deberá reportar a su jefe directo
\end{itemize}

\subsection{Frecuencia}\index{Programa!Defensa Alimentaria, de!frecuencia}\index{Frecuencia!revisión del programa de \textit{food defense}}\index{Frecuencia!verificación del programa de \textit{food defense}}

\begin{itemize}
	\item[Revisión] Semestral;
	\item[Verificación] Semestral. 
\end{itemize}

\begin{changelog}[simple, sectioncmd=\subsection*,label=changelog-10.1]

	\begin{version}[v=\MayorVer.\MenorVer, date=2023--01, author=Pablo E. Alanis]
		\item Cambio de formato;
		\item Cambios en la serialización de versiones;
		\item Correcciones ortográficas y de estilo;
		\item Se agregó \cref{nota:MedicionDeTemperaturas}.
	\end{version}

	\begin{version}[v=1.7, date=2022--02, author=Alonso M.]
		\item cambio de fecha;
	\end{version}

	\shortversion{v=1.6, date=2021--02, changes=No hubo cambios}
\end{changelog}