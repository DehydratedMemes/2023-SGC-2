\thispagestyle{formato-PI}
\renewcommand{\MayorVer}{2}
\renewcommand{\MenorVer}{0}
\renewcommand{\Codigo}{FD-1-MAN}
\renewcommand{\FechaPub}{2023--01}
%\renewcommand{\Edit}{2.1}
\renewcommand{\Titulo}{Equipo multidisciplinario de \textit{\glsname{FD}}}

\section{\Titulo} \label{ESP:EquipoFoodDefense}\index{Equipo multidisciplinario!\textit{\glsname{FD}}, de}

\subsection{Objetivo}
\begin{itemize}
	\item Crear el equipo multidisciplinario comprometido para asegurar el control de todos los puntos dentro de la cadena de recepción, almacenamiento y distribución de nuestros productos.
	\item Garantizar la seguridad (Inocuidad) de alimentos para consumo humano.
	\item Reducir, eliminar y evitar que ocurran riesgos significativos intencionales o no intencionales durante la operación.
\end{itemize}

\subsection{Alcance}
\begin{itemize}
	\item A todas las personas responsables para que se lleven a cabo las operaciones que se requieran para el buen cumplimiento de este procedimiento en \gls{RDF}.
	\item Las pautas a seguir en este programa son aplicables a los productos, personal, instalaciones, visitantes, transporte y servicios externos contratados.
\end{itemize}

\subsection{Terminología y definiciones}

\begin{description}
	\defglo{inocuidad-de-los-alimentos}
	\defglo{FD}
\end{description}
% SEGURIDAD ALIMENTARIA: los alimentos por su naturaleza pueden ser blanco de un ataque. Por estar al alcance de mucha gente involucrada en su elaboración y distribución, de ahí que las Buenas Prácticas de Distribución y los sistemas de seguridad contribuyan a la prevención contra acciones deliberadas y mal intencionadas con los alimentos, dando lugar a los Programas de Bioseguridad Alimentaria.

\subsection{Documentos y/o normas relacionadas}

\begin{itemize}
	\item Programa de reuniones semestral.
	\item Procedimientos del Programa de Seguridad Alimentaria.
\end{itemize}

\subsection{Procedimiento}

La creación de un comité de \gls{FD} nos ayudara a proteger la seguridad del producto, apoyados en el personal de las diferentes áreas involucradas en la producción, calidad y almacenamiento y distribución.

\begin{itemize}
	\item El equipo debe ser multidisciplinario para diseñar un sistema global y comprometido preventivo no reactivo para dar la seguridad requerida durante la cadena de proceso y comercialización de producción y asegurar la inocuidad y seguridad de los alimentos.
	\item Este plan de Seguridad Alimentaria será desarrollado por un equipo multidisciplinario que puede constar de miembros de los siguientes departamentos.
	\begin{itemize}
		\item Director General
		\item Gerencia de Operaciones.
		\item Control de Inventarios y Coordinación del cliente.
		\item Mantenimiento a equipos.
		\item Mantenimiento a edificios.
		\item Supervisor de almacén.
		\item Recursos humanos.
		\item Sistemas.
	\end{itemize}
\end{itemize}

\subsubsection{Reuniones del equipo}

El equipo multidisciplinario se reunirá dos veces al año en el lugar previamente designado por el equipo para verificar como base los temas a continuación descritos.

\begin{itemize}
	\item Evaluar artículos de gravedad o riesgo visualizados durante la cadena de recepción, almacenamiento, embarque y distribución del producto.
	\item Comentar las tendencias en la industria de alimentos incluyendo los impactos sobre la seguridad alimentaria y bienestar público, estos temas pueden ser cambios o sugerencias aportados por cursos de capacitación sobre el tema.
	\item Revisar incidencias o acontecimientos significativos que se cataloguen como riesgo dentro de la cadena de recepción, almacenamiento, embarque y distribución del producto
	\item Revisar iniciativas reglamentarias o actualizaciones del sistema.
	\item Autorizar nuevos procedimientos de Seguridad Alimentaria que se adopten.
	\item Revisar los puntos de oportunidad del sistema de Seguridad Alimentaria para su correcta implementación.
	\item Actualización y elaboración de manuales del sistema si así se requiera.
	\item Puntos de oportunidad detectadas en auditorias de implementación del sistema de Seguridad Alimentaria.
	\item Análisis de observaciones y comentarios detectados por historial en el transcurso del año por parte de los integrantes del equipo.
	\item Programar y definir la capacitación del personal en materia de Seguridad Alimentaria.
\end{itemize}

\subsubsection{Registros}

\begin{itemize}
	\item Registrar en bitácora todos los puntos tratados en la reunión (Minutas).
	\item En la minuta se registran los puntos observados o áreas de oportunidad así mismo los responsables del cumplimiento de las acciones correctivas.
\end{itemize}

\subsection{Frecuencia}

Semestral

\begin{changelog}[simple, sectioncmd=\subsection*,label=changelog-10.2]

	\begin{version}[v=\MayorVer.\MenorVer, date=2023--01, author=Pablo E. Alanis]
		\item Cambio de formato;
		\item Cambios en la serialización de versiones;
		\item Correcciones ortográficas y de estilo;
		\item Se agregó \cref{nota:MedicionDeTemperaturas}.
	\end{version}

	\begin{version}[v=1.5, date=2022--02, author=Alonso M.]
		\item cambio de fecha;
	\end{version}

	\shortversion{v=1.4, date=2021--02, changes=No hubo cambios}
\end{changelog}