\thispagestyle{formato-PI}
\renewcommand{\MayorVer}{2}
\renewcommand{\MenorVer}{0}
\renewcommand{\Titulo}{Frecuencia de capacitaciones para el personal del almacén}
\renewcommand{\TipoID}{ESP}
\renewcommand{\FechaPub}{2023--01}

\section{\Titulo}\index{Programa!capacitación, de}\index{Especificación!capacitaciones para el personal del almacén}
\renewcommand{\Codigo}{\Prog--\thesection--\TipoID}

\subsection{Objetivo}
Establecer un programa de capacitación para que el personal de almacén cuente con los conocimientos necesarios para realizar las tareas que le serán encomendadas, dichos conocimientos se debe llevar a la práctica en el almacén con el fin de asegurar la inocuidad de los productos.

\subsection{Alcance}
Personal operativo y personal de aseguramiento de calidad.


\subsection{Documentos y/o normas relacionadas}
\begin{itemize}
	\item Reglamento de ingreso a las instalaciones
	\item Reglamento de almacén
	\item Programa de capacitaciones
\end{itemize}

\subsection{Procedimiento}
\subsubsection{Capacitación de personal de nuevo ingreso}

Todo personal de nuevo ingreso a \gls{RDF} debe ser capacitado antes de entrar a las áreas operativas con las capacitaciones de inducción; \gls{BPD} y \gls{FD}, se le dará a conocer el reglamento de ingreso al almacén y estas capacitaciones estarán registrados dentro de su expediente.

\subsubsection{Capacitación de personal}
\begin{itemize}
	\item A todo el personal del almacén se le dará una capacitación anual de acuerdo con el programa de capacitaciones en el tema que corresponda para reforzar sus conocimientos.
	\item Se les realizará una evaluación con preguntas (al finalizar se revisarán respuestas) esta evaluación puede ser por escrito o de manera oral, también se podrá evaluar con presentación frente a grupo sobre el tema.
	\item A la persona que no alcance la calificación suficiente de \qty{60}{\percent} o que se le detecte por medio del comportamiento que no tiene conocimientos suficientes sobre el tema, se le impartirá nuevamente la capacitación.
\end{itemize}

\subsubsection{Evaluación de curso}
\begin{itemize}
	\item El personal que capacita al personal será avaluado por medio del formato \emph{“Evaluación de Curso”,} esto servirá para encontrar áreas de oportunidad de mejora en la impartición de los temas.
\end{itemize}

\subsubsection{Capacitaciones}
\begin{itemize}
	\item Alérgenos
	\item BPD
	\item Food defense
	\item HACCP
	\item Manejo integral de plagas
	\item Manejo de Químicos
	\item Inocuidad y cadena de frío
	\item Carga, descarga y manejo de producto.
	\item Programa de Limpieza.
	\item 5s
	\item Sanitización
	\item Entre Otras.
\item[\textbf{NOTA 1:}] El personal que debe de tomar cada una de las capacitaciones mencionadas se designara en el programa anual de capacitaciones.
\item[\textbf{NOTA 2:}] Se pudiera presentar cambios en el orden como se encuentran agendadas las
Capacitaciones, siempre y cuando se cumplan los temas en el programa anual
Establecido.
\item[\textbf{NOTA 3:}] En el caso de cambiar algunos de los temas que se encuentran en el programa se deberá registrar en el programa de capacitaciones el motivo de esta decisión.
\end{itemize}

\subsection{Responsables de la actividad}
\begin{itemize}
	\item \textbf{Ejecutado} por personal de aseguramiento de calidad.
\end{itemize}

\subsection{Frecuencia}
Anual

\begin{changelog}[simple, sectioncmd=\subsection*,label=changelog-\thesection-\MayorVer.\MenorVer]
	\begin{version}[v=\MayorVer.\MenorVer, date=2023--01, author=Pablo E. Alanis]
		\item Cambio de formato;
		\item Cambios en la serialización de versiones;
		\item Correcciones ortográficas y de estilo.
	\end{version}

	\begin{version}[v=1.6, date=2022--05, author=Alonso M.]
		\item cambio de fecha;
	\end{version}

	\shortversion{v=1.5, date=2021--05, changes=No hubo cambios}
\end{changelog}