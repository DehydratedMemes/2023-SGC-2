\thispagestyle{formato-PI}
\renewcommand{\MenorVer}{0}
\renewcommand{\MayorVer}{2}
%\renewcommand{\Edit}{\MayorVer.\MenorVer}
% \renewcommand{\Codigo}{HYS-13-IT}
\renewcommand{\FechaPub}{2023--01}
\renewcommand{\Titulo}{Limpieza de utensilios de limpieza de alérgenos}

\section{\Titulo}\index{Limpieza de utensilios!limpieza de alérgenos, de}
% \section{Limpieza de accesorios de alérgenos}
\renewcommand{\Codigo}{\Prog--\thesection--\TipoID}

\subsection{Objetivo}
Establecer un procedimiento que aseguren un mantenimiento y limpieza apropiada en caso de presentarse un derrame de producto Alérgeno.

\subsection{Alcance}
Toda superficie afectada por un derrame de alimento con componente(s) alérgenico(s).

\subsection{Terminología y definiciones}
\begin{itemize}
	\item \textbf{Limpieza:} Proceso de remover los residuos de Orgánicos y suciedad que pueden ser una fuente de contaminación. Los métodos de limpieza adecuados y los materiales dependerán de la naturaleza del alimento.
	\item \textbf{Sanitización:} Proceso realizado después de efectuar la limpieza profunda de equipos, maquinaria e instalaciones la cual nos permite el reducir el número de microorganismos
\end{itemize}
\subsection{Procedimiento}
\subsubsection{Materiales}
\begin{itemize}
	\item Preparación de \emph{Liquid K} a la concentración indicada por el proveedor, \qtyrange{0.13}{0.26}{\percent} o bien \qtyrange{1.3}{2.6}{\milli\liter\per\liter};
	\item Cubeta de \qty{20}{\litre};
	\item Cepillo para piso color Blanco;
	\item Cepillos manuales color Blanco;
	\item Recogedor de plástico;
	\item Jalador color Blanco;
	\item Toallas desechables.
\end{itemize}

\subsubsection{Instrucciones}
\paragraph{Derrame de alérgenos en el piso de almacén, cámaras frías y andenes}

\begin{enumerate}
	\item Verificar la no afectación de ningún otro producto
	\item Recolectar el producto alérgeno de las áreas afectadas con ayuda de toallas desechables (elimine toallas después de su uso)
	\item Prepare detergente a la concentración indicada por el proveedor.
	\item Tallar el piso con agua y jabón y el cepillo asignado para estos casos
	\item Enjuagar el piso
	\item Eliminar la humedad con el jalador asignado
	\item Dar aviso al supervisor de Almacén y Calidad para realizar revisión visual
	\item Registro de lavado por derrame de material alérgeno.
	\item Si se afectó algún otro producto o empaque se procede a eliminar el producto contaminante por medio de limpieza en seco (toallas desechables).
	\item Separar el producto y etiquetar como producto detenido y/o retenido
	\item Valorar el grado de afectación del producto en coordinación con el Coordinador de calidad del cliente en cuestión.
	\item Se precederá según indicaciones sobre la disposición del producto.
	\item Registro de lavado por derrame de material alérgeno.
\end{enumerate}

\subsection{Responsables de la actividad}
\begin{itemize}
	\item \textbf{Ejecutado} por personal de limpieza;
	\item \textbf{Monitoreado} por personal de mantenimiento;
	\item \textbf{Verificado} por personal de Calidad.
\end{itemize}

\subsection{Acciones preventivas}
\begin{itemize}
	\item Se llevará a cabo una inspección diaria. Registrando el indicador y medida correspondiente
	\item Si la desviación se repite frecuentemente se dará curso de capacitación al personal de almacén y limpieza para que realicen eficientemente su trabajo.
	\item Si después de haber capacitado al personal de almacén y limpieza se siguen presentando desviaciones por causas injustificadas, se tomaran acciones más enérgicas con el personal por incumplimiento con sus deberes.
\end{itemize}

\subsection{Acciones correctivas}
\begin{enumerate}
	\item La tarea es liberada una vez que el área se encuentran libre de polvo, suciedad, residuos de materiales extraños, residuos de jabón, y seca. En caso contrario la tarea se deberá volver a realizar como se indica en el procedimiento.
	\item En caso de no conformidad reportar en \RAC.
\end{enumerate}

\subsection{Frecuencia}
Cuando se presente la incidencia.

\begin{changelog}[simple, sectioncmd=\subsection*,label=changelog-\thesection-\MayorVer.\MenorVer3]
	\begin{version}[v=\MayorVer.\MenorVer, date=2023--01, author=Pablo E. Alanis]
		\item Cambio de formato;
		\item Cambios en la serialización de versiones;
		\item Correcciones ortográficas y de estilo.
	\end{version}

	\begin{version}[v=1.7, date=2022--05, author=Alonso M.]
		\item cambio de fecha;
	\end{version}	

	\shortversion{v=1.6, date=2021--05, changes=No hubo cambios}
\end{changelog}