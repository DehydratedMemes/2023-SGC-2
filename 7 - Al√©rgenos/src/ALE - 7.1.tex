\thispagestyle{formato-PI}
\renewcommand{\MayorVer}{2}
\renewcommand{\MenorVer}{0}
\renewcommand{\Titulo}{Programa de control de alérgenos}
\renewcommand{\TipoID}{ESP}
\renewcommand{\FechaPub}{2023--01}

\section{\Titulo}\index{Programa!control de alérgenos, de}\index{Especificación!capacitaciones para el personal del almacén}
\renewcommand{\Codigo}{\Prog--\thesection--\TipoID}

\subsection{Objetivo}
Establecer un procedimiento el correcto manejo de los productos y materiales alérgenos y evitar la contaminación cruzada con otros productos almacenados.

\subsection{Alcance}
Alimentos que contengan componente(s) alérgeno(s).

\subsection{Terminología y definiciones}
\begin{description}
	\defglo{peligro-relacionado-con-la-inocuidad-de-los-alimentos};
	\defglo{contacto-cruzado-alergenos};
	\defglo{alergeno}.
\end{description}

\subsection{Documentos y/o normas relacionadas}
\begin{itemize}
	\item Programa de Higiene y Sanidad.
	\item 27 CFR 7.82 --- Voluntary disclosure of major food alergens.
	\item 21 CFR 117.3 --- Definitions.
	\item Código SQF 2.8.1.
\end{itemize}

\subsection{Control de alérgenos en planta}
\begin{itemize}
	\item Alguno de los productos almacenados de \gls{RDF} son productos preparados formulados y contienen uno a varios productos alérgenos, cabe mencionar que estos productos se encuentran envasados (Empaque original) y emplayados.
	\item Estos productos que contienen ingredientes alérgenos son manejados de la siguiente manera:
	\item Áreas específicas y señaladas con la leyenda \emph{ALERGENO} para almacenamiento de productos que contienen alérgenos.
	\item Cada producto es almacenado por separado y en ningún momento son mezclados.
	\item Independientemente del programa interno de alérgenos algunos de estos productos se encuentran identificados de origen con la leyenda \emph{ALERGENO}
	\item Cada producto DEBE de estibarse con productos de su mismo tipo, tarimas del mismo producto.
	\item Registrar todos los movimientos de estos productos (bitácora de entrada y salida).
	\item Identificar en cada uno de los puntos del proceso con etiqueta de alérgenos, recepción, alimentación y embarque.
\end{itemize}

\subsubsection{Contacto cruzado de alérgenos}
La clave para manejar alérgenos en el procedimiento y manejo, es evitar el \gls{contacto-cruzado-alergenos} Si el mismo ingrediente alergénico fuese utilizado en todas las fórmulas de producto, entonces no habría riesgo de contacto-cruzado usualmente. Se deben utilizar políticas y procedimientos que deben incluir la documentación apropiada para apoyar estas actividades. Elementos clave que deben ser considerados al desarrollar e implementar estas políticas y procedimientos que incluyen.

\begin{itemize}
	\item Limpieza durante cambio de alérgenos
	\item Inspecciones de almacén.
	\item Colocación de productos alérgenos sobre productos alérgenos del mismo tipo.
	\item Barreras físicas laterales como empaque o emplaye.
	\item Colocar o ubicar en el rack productos no alérgenos sobre productos alérgenos.
	\item Códigos de color u otra designación y segregación de tarimas.
	\item Áreas exclusivas.
	\item Todos los productos de este tipo deben ser inspeccionadas por daños (incluyendo fugas, cajas dañadas, etc.). Cualquier producto que se encuentre con fugas DEBEN ser devueltos inmediatamente y documentar en él la bitácora de “Registro de Producto Dañado”. Los esfuerzos de limpieza deben garantizar que no existe contaminación cruzada con otros productos.
	\item La limpieza con agua se recomienda para eliminar cualquier residuo pastoso o pegajoso que contenga material alérgeno. Cuando se utilice un sistema de limpieza en las áreas de operación debe ser examinado para evidenciar áreas que no puedan ser adecuadamente limpiadas y que se puedan atrapar residuos alergénicos.
	\item Tanto la limpieza con agua como la limpieza en seco deben ser validadas periódicamente a través de una revisión visual para verificar que las superficies estén limpias.
	\item Después de una validación inicial del procedimiento de limpieza, se puede usar la examinación visual y diaria de superficies en contacto con el producto para verificar que la limpieza de alérgenos se ha llevado a cabo.
	\item Esta examinación visual debe documentarse en la bitácora destinada para instalaciones en las cuales se utilizó el material alérgeno.
\end{itemize}

\subsection{Responsables de la actividad}
\begin{itemize}
	\item \textbf{Ejecutado} por personal de calidad.
\end{itemize}

\subsection{Acciones correctivas}
\begin{itemize}
	\item Cuando se presente una no conformidad en la realización de los procedimientos, detectada por el supervisor o encargado del área se deberá de repetir este mismo
	\item En caso de que la desviación sea mayor, se deberá de registrar la acción correctiva en el formato “Registro de reporte y acciones correctivas”
	\item Dar aviso al supervisor inmediato.
\end{itemize}

\subsection{Frecuencia}
\begin{itemize}
	\item Cada producto que contiene alérgeno.
\end{itemize}

\begin{changelog}[simple, sectioncmd=\subsection*,label=changelog-\thesection-\MayorVer.\MenorVer]
	\begin{version}[v=\MayorVer.\MenorVer, date=2023--01, author=Pablo E. Alanis]
		\item Cambio de formato;
		\item Cambios en la serialización de versiones;
		\item Correcciones ortográficas y de estilo.
	\end{version}

	\begin{version}[v=1.7, date=2022--05, author=Alonso M.]
		\item cambio de fecha;
	\end{version}

	\shortversion{v=1.6, date=2021--05, changes=No hubo cambios}
\end{changelog}
