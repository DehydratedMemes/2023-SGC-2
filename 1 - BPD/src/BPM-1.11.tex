\thispagestyle{formato-PI}
\renewcommand{\MayorVer}{1}
\renewcommand{\MenorVer}{0}
\renewcommand{\Codigo}{BPD-10-IT}
\renewcommand{\FechaPub}{2023--01}
\renewcommand{\Titulo}{Llenado de formularios de orden de salida}
\section{\Titulo}\index{Procedimiento!llenado de formularios!orden de salida}
%\section{Llenado de formularios de orden de salida}

\subsection{Objetivo}

\begin{itemize}
	\item \textbf{Establecer} las definiciones de los campos solicitados en los formularios de orden de salida (PSA-OP-FORM-1);
	\item \textbf{Estandarizar} el llenado de los formularios de orden de salida.
\end{itemize}

\subsection{Alcance}

\begin{itemize}
	\item Éste documento esta dirigido, más no está limitado, al \emph{departamento de operaciones;}
	\item en este documento no se contempla el llenado de formularios de \emph{orden de entrada (PSA-OP-IT-36).}
\end{itemize}

\subsection{Términos y definiciones}

% \begin{itemize}
% 	\item \textbf{cantidad de tarimas - blancas}
% 	\begin{itemize}
% 		\item Número de tarimas convencionales (no CHEP) requeridas para la carga.
% 	\end{itemize}
% 	\item \textbf{cantidad de tarimas - CHEP}
% 	\begin{itemize}
% 		\item Numero de tarimas CHEP requeridas para la carga.
% 		\item \textbf{Nota:} las tarimas CHEP son fácilmente identificables, ya que son de color azul y tienen impresa la leyenda "Property of CHEP".
% 	\end{itemize}
% 	\item \textbf{cantidad de tarimas - Total}
% 	\begin{itemize}
% 		\item La suma de la cantidad de \emph{tarimas blancas} y \emph{tarimas CHEP} empleadas en la carga.
% 	\end{itemize}
% 	\item \textbf{caja de la unidad de transporte (CUT)}
% 	\begin{itemize}
% 		\item unidad, generalmente con temperatura controlada, unida a un vehículo de transporte.
% 		\item \textbf{Nota:} en la sección \emph{C - Servicios} se usa el término de \emph{caja} en referencia a cantidad unitaria de producto.
% 	\end{itemize}
% 	\item \textbf{carga}
% 	\begin{itemize}
% 		\item El total de \emph{tarimas con producto} solicitadas en una orden de salida por el \emph{cliente} o algún representante de él que puedan ser transportadas en una sola caja.
% 		\item \textbf{Nota:} si en la \emph{orden de salida} se requiere de más de una caja para surtir la demanda, no se considera como \emph{carga} al total solicitado en ésta, si no al total de producto transportado por cada caja.
% 	\end{itemize}
% 	\item \textbf{cliente}
% 	\begin{itemize}
% 		\item Empresa o persona a la que RDF le brinda el servicio de almacenamiento de producto alimenticio.
% 	\end{itemize}
% 	\item \textbf{condición de la caja}
% 	\begin{itemize}
% 		\item Estado visible en general de las paredes, techo, piso y unidad de enfriamiento de la caja;
% 		\item si las paredes y techos tienen rasgaduras, se debe de marcar el campo como \emph{Mala.}
% 		\item \textbf{NOTA:} la condición (visible) de la caja \emph{no asegura} el correcto funcionamiento de la unidad de enfriamiento de la misma. Por lo que si la caja no llega a su destino a la temperatura a la que salió de RDF por algún fallo del sistema de enfriamiento del la misma, RDF no se responsabiliza de la carga;
% 		\item \textbf{NOTA:} si es requerido por el \emph{cliente} o el \emph{sub-cliente,} se pueden incorporar a las cajas termoregistradores proporcionados o pagados por dichas partes.
% 	\end{itemize}
% 	\item \textbf{congelación}
% 	\begin{itemize}
% 		\item En el apartado \emph{B - Condiciones del producto} se considera como congelación una especificación de almacenamiento de producto a óptimamente \SI{-18}{\celsius};
% 		\item \textbf{NOTA:} en caso de que las especificaciones de almacenamiento del producto sean diferentes a \SI{-18}{\celsius}, pero sean menores a \SI{0}{\celsius}, se marcará este campo.
% 	\end{itemize}
% 	\item \textbf{Emplaye (servicio)}
% 	\begin{itemize}
% 		\item proceso en el que se envuelve la tarima cargada con producto con emplaye.
% 	\end{itemize}
% 	\item \textbf{limpieza de la caja}
% 	\begin{itemize}
% 		\item se considerará como \emph{buena} la limpieza de una caja cuando no se encuentren las paredes, techo y piso manchados con residuos como (pero no limitados a):
% 		\begin{itemize}
% 			\item residuos de madera o plástico de cargas anteriores;
% 			\item manchas o salpicaduras de aceite;
% 			\item manchas o salpicaduras de fluidos excretados por cargas anteriores;
% 			\item otras.
% 		\end{itemize}
% 	\end{itemize}
% 	\item \textbf{maniobras de salida}
% 	\begin{itemize}
% 		\item proceso de carga de tarimas a la caja, considerando la extracción de la tarima cargada de la cámara de enfriamiento y su disposición de forma temporal en el área designada para efectuar la carga.
% 	\end{itemize}
% 	\item \textbf{picking}
% 	\begin{itemize}
% 		\item proceso manual en el que \emph{cargas unitarias} de producto se disponen en una tarima o palé vacío según especificaciones acordadas con el \emph{cliente} o \emph{sub-cliente} para su posterior carga en la caja.
% 	\end{itemize}
% 	\item \textbf{presencia de plagas}
% 	\begin{itemize}
% 		\item se debe de marcar positivamente la presencia de plagas si es que se encuentran:
% 		\begin{itemize}
% 			\item excretas;
% 			\item orina
% 			\item restos de plagas;
% 			\item insectos vivos;
% 			\item indicios de roídas;
% 			\item plumas;
% 			\item entre otros.
% 		\end{itemize}
% 	\end{itemize}
% 	\item \textbf{presencia de tarimas NO cargadas en RDF}
% 	\begin{itemize}
% 		\item si la caja viene con una carga parcial de productos que no fueron cargados en RDF, se debe de marcar el campo de forma positiva.
% 	\end{itemize}
% 	\item \textbf{sub-cliente}
% 	\begin{itemize}
% 		\item \emph{o prestador de servicio logístico} es la empresa o persona encargada del transporte de la carga.
% 	\end{itemize}
% 	\item \textbf{refrigeración}
% 	\item En el apartado \emph{B - Condiciones del producto} se considera como refrigeración una especificación de almacenamiento de producto a óptimamente \SI{4}{\celsius};
% 	\begin{itemize}
% 		\item \textbf{NOTA:} en caso de que las especificaciones de almacenamiento del producto sean diferentes a \SI{4}{\celsius} pero sean mayores a \SI{0}{\celsius}, se marcará este campo.
% 	\end{itemize}
% 	\item \textbf{requerimiento de temperatura del producto}
% 	\begin{itemize}
% 		\item especificación cualitativa indicada para el almacenamiento del producto según el \emph{cliente.}
% 	\end{itemize}
% 	\item \textbf{tarima}
% 	\begin{itemize}
% 		\item también conocida como \emph{palé,} es una estructura plana de transporte que soporta productos de manera estable para ser cargados por un montacargas;
% 	\end{itemize}
% 	\item \textbf{temperatura de salida}
% 	\begin{itemize}
% 		\item temperatura expresada en grados Celsius \si{\celsius} que indica la temperatura a la que se encuentra la caja una vez que fue cargada.
% 	\end{itemize}
% 	\item \textbf{temperatura de la CUT adecuada para cargar}
% 	\begin{itemize}
% 		\item se refiere a la temperatura en la que se encuentra la CUT cuando llega al andén y aún no se carga
% 	\end{itemize}
% \end{itemize}

\subsection{Documentos y/o normas relacionadas}

\begin{itemize}
	\item PSA-FORM-OP-2 --- Orden de salida
	\item PSA-FORM-OP-1 --- Orden de entrada
\end{itemize}

\subsection{Instrucciones}

\subsubsection{Materiales}

\begin{itemize}
	\item Formulario \textbf{PSA-FORM-OP-2;}
	\item pluma;
	\item termómetro IR;\@
	\item termoregistrador (opcional).
\end{itemize}

\subsubsection{Instrucciones}

\begin{itemize}
	\item Se debe de llenar de forma clara y con pluma indeleble el formulario;
	\item en caso de cometer algún error al llenar el formulario, tachar el error con pluma y llenar correctamente el campo
	\begin{itemize}
		\item si no es legible el campo después de haber sido corregido y vuelto a llenar, deberá llenarse otro formulario vacío.
	\end{itemize}
\end{itemize}

\paragraph{§A --- Datos de identificación}

\begin{itemize}
	\item Llenar campos de acuerdo a los datos proporcionados;
	\item la hora debe escribirse en formato de \SI{24}{\hour} (HH:MM).
\end{itemize}

\paragraph{§B --- Condiciones del producto}

\begin{itemize}
	\item Las casillas se deben de marcar claramente o con una palomilla, llenándola por completo o con una cruz;
	\begin{itemize}
		\item si se comete un error marcando una casilla, llenar un formulario vacío nuevamente.
	\end{itemize}
	\item en el campo de \emph{Temperatura de salida} se debe de escribir la temperatura según la sección de Términos y definiciones adicionando la unidad de medición \emph{i.e.} \SI{-15}{\celsius}
\end{itemize}

\paragraph{§C --- Servicios}

\begin{itemize}
	\item Se deben de indicar con números legibles los servicios prestados en la carga de la caja.
\end{itemize}

\paragraph{§D --- Observaciones}

\begin{itemize}
	\item Se deben de indicar en observaciones las desviaciones detectadas en la caja de una manera descriptiva;
	\item en el caso de rechazar una unidad, se deben de escribir las razones;
	\item pueden ir observaciones adicionales no consideradas en el formulario.
\end{itemize}

\subsection{Responsables de la actividad}

\begin{itemize}
	\item \textbf{Ejecutado} por personal de operaciones;
	\item \textbf{Monitoreado} por personal de calidad;
	\item \textbf{Verificado} por personal de calidad.
\end{itemize}

\subsection{Acciones preventivas}

\begin{itemize}
	\item Aleatoriamente el \emph{personal de calidad} verificará que se llenen los formularios de manera adecuada;
	\item se dispondrá de este documento en el área de operaciones para su consulta en caso de ser requerido.
\end{itemize}

\subsection{Acciones correctivas}

\begin{itemize}
	\item En caso de que una \emph{orden de salida} presente desviaciones a los requisitos establecidos en este documento, se procederá a capacitar \emph{in situ} al personal;
	\item en caso de que se repita con más empleados, se capacitarán a los involucrados de forma detallada en el llenado de este formulario y aquellos otros que no se llenen correctamente.
\end{itemize}

\subsection{Frecuencia}

\begin{itemize}
	\item \textbf{Verificación y monitoreo:} de forma aleatoria.
	\item \textbf{Ejecución:} cada que se genere una orden de salida.
\end{itemize}

\begin{changelog}[title=Registro de cambios,simple, sectioncmd=\subsection*,label=changelog-\thesection-\MayorVer.\MenorVer1]
	\begin{version}[v=1.0, date=2023--01, author=Pablo E. Alanis]
		%\fixed
			\item Primera versión.
	\end{version}
\end{changelog}
