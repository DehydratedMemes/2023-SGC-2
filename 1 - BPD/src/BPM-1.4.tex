\renewcommand{\MayorVer}{2}
\renewcommand{\MenorVer}{1}
\renewcommand{\Codigo}{BPD-4-CI/CE}
\renewcommand{\FechaPub}{2023--01}
\renewcommand{\Titulo}{Reglamento de ingreso a las instalaciones}

\section{\Titulo}
\index{Reglamento!ingreso a las instalaciones, de}
%\section{Reglamento de ingreso a las instalaciones}

\subsection{Objetivos}
\begin{itemize}
	\item Garantizar la calidad e inocuidad de los productos alimenticios por medio del control del ingreso de personal y visitantes para evitar contaminación cruzada.
	\item Garantizar la eficiencia durante la recepción, almacenamiento y distribución de nuestros productos alimenticios.
\end{itemize}

\noindent
\subsection{Alcance}
\index{Reglamento!ingreso a las instalaciones, de!alcance}
\begin{itemize}	
	\item Este documento se extiende a todo \gls{personal-interno} y \gls{cliente} de \gls{RDF}
\end{itemize}

\subsection{Terminología y definiciones}

\begin{description}
	\defglo{personal-interno}
	\item[\gls{uniforme-completo}] \glsdesc{uniforme-completo}
	\item[\gls{cliente}] \glsdesc{cliente}
	\item[\gls{visitante}] \glsdesc{visitante}  
\end{description}

\subsection{Documentos y/o normas relacionados}
\begin{itemize}
	\item Programa de \textit{Food Defense;}
	\item Programa de \gls{HACCP};
	\item Programa de BPM;
	\item Programa de seguridad alimentaria;
	\item Ley Federal de Sanidad Animal (LFSA);
	\item Reglamento de Ley Federal de Sanidad Animal (RLFSA).
	\item Carta de conformidad con el reglamento interno
\end{itemize}

\subsection{Reglamento de ingreso al almacén}

\subsubsection{Disposiciones generales}
\index{Reglamento!ingreso a las instalaciones, de!disposiciones generales}
\begin{itemize}
	\item Se deben de lavar las manos al entrar al área operativa con jabón y gel antibacteriano;\footnote{En caso de portar guantes, estos deberán estar limpios.}
	\item Traer su uniforme completo;
	\item Prohibido portar joyería;\footnote{Anillos, cadenas, aretes, pulseras, piercings en lugares visibles.}
	\item Prohibido fumar;
	\item Prohibido masticar chicle;
	\item Prohibido comer ó beber en las áreas operativas;
	\item Prohibido escupir;
	\item Usar ropa y/o uniforme limpio;
	\item Prohibido usar pantalones deshilachados o con pedrería;
	\item Prohibido entrar en shorts y camisas sin mangas;
	\item Usar cofias en áreas de dentro de Almacén;
	\item Usar cubrebocas en el Almacén;
	\item No portar artículos en bolsillos superiores a la cintura;
	\item Prohibido introducir artículos de vidrio;
	\item No golpear ni dañar los productos;
	\item Mantener limpias y en orden las áreas y las herramientas de trabajo;
	\item Prohibido el uso de zapatos abiertos;
	\item No pisar los productos ni las tarimas;
	\item No colocar en el piso materiales de contacto con producto \emph{i.e.} emplayes, cajas plástico, etc;
	\item No utilizar celulares personales dentro del almacén;\footnote{Si se tiene que utilizar el celular para una cuestión que involucre la operación, el personal que lo haya usado tendrá que lavarse y desinfectarse las manos nuevamente antes de tocar cualquier producto alimenticio.}
	\item Traer las uñas de las manos cortas;
	\item Prohibido usar barba larga;
	\item Si se tiene el cabello largo, éste debe de estar completamente cubierto por la cofia;
	\item Prohibido el uso de cámara fotográfica y de video de cualquier tipo dentro del almacén sin autorización previa.
\end{itemize}

\subsubsection{Ingreso de personal interno}
\index{Reglamento!ingreso a las instalaciones, de!personal interno}
\begin{itemize}
	\item El personal contratado directamente de RDF debe de apegarse al reglamento interno;
	\item en caso de que se registre un incumplimiento, se deberá de registrar la insatisfacción y capacitar \emph{in situ} a la persona que incumplió con el reglamento.
\end{itemize}

\subsubsection{Ingreso de visitantes o personal contratado por clientes de RDF (PCPC)}
\index{Reglamento!ingreso a las instalaciones, de!PCPC}
\label{PCPC-reglamento}

\begin{itemize}
	\item Los visitantes o personal contratado por clientes de RDF (PCPC) solo pueden ingresar a la planta de almacenamiento con previa autorización y acompañadas por personal autorizado;
	\item los visitantes y PCPC solo deberán tener acceso a las zonas que les son de incumbencia para su operación, por lo que no deben de permanecer sin estar acompañados, a la vista de personal de operaciones constantemente;
	\item los visitantes y PCPC se deben de apegar al reglamento descrito en el documento BPD-CI-1 y en el enumerado en la \cref{PCPC-reglamento};
	\item si los visitantes no portan el uniforme adecuado, no podrán ingresar a las áreas operativas y se les tratará de proporcionar uniforme adecuado.
	\begin{itemize}
		\item En caso de no contar con uniformes adicionales, se le notificará al cliente para determinar que acciones se tomarán.
	\end{itemize}
\end{itemize}

\subsection{Responsable(s) de la actividad}
\index{Reglamento!ingreso a las instalaciones, de!responsables}
\begin{itemize}
	\item \emph{Ejecutado} por personal de operaciones y vigilancia;
	\item \emph{Monitoreado} por personal de operaciones;
	\item \emph{Verificado} por personal de gerencia y calidad.
\end{itemize}

\subsection{Acciones preventivas}

\begin{itemize}
	\item Se llevará a cabo una inspección por el personal de operaciones y en caso de ser observada una insatisfacción, se le pedirá al personal o visitante que incumplió con el reglamento cumplir con el requisito;
	\item El personal de calidad hará una inspección diaria de forma aleatoria y marcará en el formulario correspondiente si se cumple con el reglamento de ingreso de almacén.
	\item Si la desviación se repite frecuentemente se dará curso de capacitación al personal por parte de personal de calidad;
	\item Si después de haber capacitado al personal de almacén se siguen presentando desviaciones por causas injustificadas, será acreedor de una amonestación el personal interno que incumplió con el reglamento interno.
\end{itemize}

\subsection{Acciones correctivas}

\begin{itemize}
	\item En caso contrario la tarea se deberá volver a realizar como se indica en el procedimiento;
	\item Se prohíbe el acceso del personal re-incidente;
	\item En caso de no conformidad reportar en formato de acciones correctivas.
\end{itemize}

\begin{changelog}[simple, sectioncmd=\subsection*,label=changelog-1.4]
	\begin{version}[v=2.1, date=2023--01, author=Pablo E. Alanis]
			\item Cambio de formato;
			\item cambios en la serialización de versiones;
			\item cambio de Codificación;
			\item se contempló el acceso a PCPC.
	\end{version}

	\begin{version}[v=1.7, date=2022-05, author=Alonso M.]
		\item No hubo cambios.
	\end{version}

	\shortversion{v=1.6, date=2021-05, changes=No hubo cambios}
\end{changelog}

\subsection*{Anexos}
\begin{itemize}
	\item Ver \cref{carta.compromiso}
\end{itemize}