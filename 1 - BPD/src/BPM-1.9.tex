\thispagestyle{formato-PI}
\renewcommand{\MayorVer}{2}
\renewcommand{\MenorVer}{0}
\renewcommand{\Codigo}{BPD-9-POL}
\renewcommand{\FechaPub}{2023--01}
\renewcommand{\Titulo}{Política de igualdad laboral}
\section{\Titulo}\index{Politica!igualdad laboral, de}

%\section{Política de igualdad laboral}

\subsection{Objetivo}
\begin{itemize}
	\item \textbf{Establecer} que en \gls{RDF}, NO se hace diferencia de género, raza, religión. edad, orientación política u origen étnico y sus resultados se medirán por el desempeño demostrado dentro del su área laboral.
\end{itemize}

\subsection{Alcance}

A todas las personas que laboren en \gls{RDF}.

\subsection{Documentos y/o normas relacionadas}

\begin{itemize}
	\item Manual de calidad y buenas prácticas de distribución
\end{itemize}

\subsection{Política de igualdad laboral}

\begin{itemize}
	\item En \gls{RDF}, \textbf{NO} se hace diferencia de género, raza, religión, edad, afiliación política, orientación sexual o etnicidad, y sus resultados se medirán por el desempeño demostrado dentro del su área laboral.
	\item RDF puede contratar a menores de edad como responsabilidad social, siempre y cuando esté autorizado por escrito por los padres o tutor, las labores asignadas al personal menor de edad serán en las áreas administrativas, esto con el fin de no generar riesgo en su integridad física.
\end{itemize}

\begin{changelog}[title=Registro de cambios,simple, sectioncmd=\subsection*,label=changelog-\thesection-\MayorVer.\MenorVer]
	\begin{version}[v=2.1, date=2023--01, author=Pablo E. Alanis]
		%\fixed
			\item Cambio de formato;
			\item se adoptaron versiones mayores y menores para revisiones;
			\item cambio en el código del documento.
	\end{version}

	\begin{version}[v=1.8, date=2022--05, author=Alonso M.]
		\item Primera versión.
	\end{version}
\end{changelog}
