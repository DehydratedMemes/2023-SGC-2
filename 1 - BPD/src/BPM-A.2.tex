\renewcommand{\MayorVer}{1}
\renewcommand{\MenorVer}{0}
\renewcommand{\Codigo}{BPD-AP-B}
\renewcommand{\FechaPub}{2023--01}
\renewcommand{\Titulo}{Carta compromiso de cumplimiento del reglamento interno}

\section{\Titulo}
\label{carta.compromiso}
%\section{Reglamento de ingreso a las instalaciones}

\noindent \textit{A quien corresponda,}\bigskip

Se le exhorta a cualquier \emph{personal contratado por clientes de RDF (PCPC)} y \emph{visitante} que se apegue al reglamento de ingreso al almacén descrito abajo \textit{(vide infra).} Con el propósito de resguardar la inocuidad de los productos alimenticios aquí almacenados, así como para prevenir incidentes dentro del área operativa.
\vspace{2\baselineskip}
\begin{center}
    % \fbox{%
        \begin{minipage}{0.9\linewidth}
            \small
            \begin{itemize}
                \item Se deben de lavar las manos al entrar al área operativa con jabón y gel antibacteriano;\footnote{En caso de portar guantes, estos deberán estar limpios y de todas maneras, el procedimiento de Limpieza y desinfección de manos adecuado deberá ser realizado.}
                \item Traer su uniforme completo;
                \item Prohibido portar joyería;\footnote{\textit{i.e.} Anillos, cadenas, aretes, pulseras, piercings en lugares visibles.}
                \item Prohibido fumar;
                \item Prohibido masticar chicle;
                \item Prohibido comer o beber en las áreas operativas;
                \item Prohibido escupir;
                \item Usar ropa y/o uniforme limpio;
                \item Prohibido usar pantalones deshilachados o con pedrería;
                \item Prohibido entrar en shorts y camisas sin mangas;
                \item Usar cofias en áreas de dentro de Almacén;
                \item Usar cubrebocas en el Almacén;
                \item No portar artículos en bolsillos superiores a la cintura;
                \item Prohibido introducir artículos de vidrio;
                \item No golpear ni dañar los productos;
                \item Mantener limpias y en orden las áreas y las herramientas de trabajo;
                \item Prohibido el uso de calzado abierto;
                \item No pisar los productos ni las tarimas;
                \item No colocar en el piso material de contacto con producto \emph{i.e.} emplayes, cajas de plástico, etc.;
                \item No utilizar celulares personales dentro del almacén;\footnote{Si se tiene que utilizar el celular para una cuestión que involucre la operación, el personal que lo haya usado tendrá que lavarse y desinfectarse las manos nuevamente antes de tocar cualquier producto alimenticio.}
                \item Traer las uñas de las manos cortas;
                \item Prohibido usar barba larga;
                \item Si se tiene el cabello largo, este debe de estar completamente cubierto por la cofia;
                \item Prohibido el uso de cámara fotográfica y de video de cualquier tipo dentro del almacén sin autorización previa.
                \item[\textbf{Importante:}] En caso de tener heridas expuestas o tener algúna enfermedad contagiosa, se debe de notificar previo a entrar al almacén.
            \end{itemize}
        \end{minipage}
    % }%
\end{center}

\vfill
\begin{center}
\noindent\begin{tabular}{@{}>{\centering}p{2.5in}>{\centering}p{2.5in}@{}}
    \small
    \hrulefill                         & \hrulefill \tabularnewline
    \textbf{\textit{Nombre del PCPC o visitante\\y empresa en que labora}}      & \textbf{\textit{Firma}}\\  
    \end{tabular}
\end{center}
\vfill
