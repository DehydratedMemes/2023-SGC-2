\renewcommand{\MayorVer}{2}
\renewcommand{\MenorVer}{1}
\renewcommand{\Codigo}{BPD-3-CI/ESP}
\renewcommand{\FechaPub}{2023--01}
\renewcommand{\Titulo}{Manual de calidad y Buenas Prácticas de Distribución}

\section{\Titulo}
%\section{Manual de calidad y Buenas Prácticas de Distribución}
\index{Manual!calidad, de}
\subsection{Objetivos}

\begin{itemize}
	\item Garantizar la calidad e inocuidad de productos por medio del cumplimiento de las Buenas Prácticas de Distribución.
	\item Garantizar la eficiencia en los procesos de recepción, almacenamiento y distribución de productos.
	\item Garantizar que las condiciones de instalaciones, operación y del desempeño del personal son adecuadas para asegurar que los productos dentro del almacén son seguros y de excelente calidad para consumo humano.
\end{itemize}

\subsection{Alcance}

\begin{itemize}
	\item A todas las personas responsables para que se lleven a cabo las operaciones que se requieran para el buen cumplimiento de este procedimiento en las áreas de operación.
	\item Personal que labora en \gls{RDF} S.A. de C.V.
	\item Proveedores, Contratistas y Visitantes.
\end{itemize}

\subsection{Propósito del manual de buenas prácticas de distribución}

\begin{enumerate}
	\item Establecer guías para la administración e implementación de reglas que se refieran a la apariencia, higiene, sanidad y prácticas de manejo de los alimentos por parte de los empleados.
	\item Asegurarse que las personas que trabajan en \gls{RDF}, están conscientes de la importancia de la limpieza personal y las prácticas higiénicas.
	\item Estar seguro de que las reglas que conciernen a la higiene y sanidad han sido entendidas por el personal y que estos las sigan.
	\item Estar seguros de que los artículos recibidos, almacenados y distribuidos son de la más alta calidad y están libres de contaminación.
\end{enumerate}

\subsection{Introducción}

\gls{RDF} se siente complacida de poder darte a conocer sus normas de sanidad y Buenas Prácticas de Distribución aplicables dentro de cualquier operación de la empresa para ayudar a asegurar la distribución de productos alimenticios para el consumo humano.
Para todos nosotros el trabajar en \gls{RDF} representa una gran responsabilidad por ser una empresa que distribuye productos alimenticios, por esta razón todos somos responsables de dar lo mejor de nosotros mismos, para mantener y elevar el nivel sanitario de nuestra empresa, con el fin de prevenir cualquier tipo de contaminación, daño o infestación de nuestros productos.
El propósito principal de nuestra compañía es el de ofrecer a nuestros clientes el mejor servicio y al mejor costo, por esta razón es importante que nuestros hábitos vayan encaminados a obtener la satisfacción de los clientes y consumidores.
Esto lo lograremos solo si cada uno de nosotros realizamos nuestro trabajo bien y a la primera vez, cumpliendo además con las normas de sanidad y las Buenas Prácticas de Distribución que te daremos a conocer en este manual.

\subsection{Protección al producto}

Dentro de este programa se incluyen todos aquellos problemas que contribuyen directamente a la contaminación o daño de los productos, o materiales de empaque, para que el daño o la contaminación no ocurran, debes cumplir con lo siguiente:

\begin{enumerate}
	\item Está estrictamente prohibido introducir a cualquier área de la planta recipientes de vidrio o madera que nos sean del uso de almacenaje.
	\item Dentro de la planta no debe de haber presencia de objetos metálicos como lo son clips metálicos, o cualquier otro tipo de sujeto papel.
	\item Los productos deben de inspeccionarse y clasificarse antes de llevarse a las áreas de almacenaje.
	\item Las áreas de almacenaje deben de estar limpias y libres de materiales extraños antes de estibar los productos para evitar cualquier tipo de contaminación.
	\item Todo producto químico de limpieza y de fumigación debe almacenarse fuera de las áreas de operación.
	\item Está prohibido utilizar cubetas o cualquier otro tipo de recipiente, que no sea el indicado por las normas de seguridad y de higiene.
	\item Asegure al recibir cualquier producto que no haya acumulado polvo viejo o húmedo sobre la mercancía, ya que de ser así, no se podrá recibir dicha mercancía, ya que afecta nuestros procesos de higiene contaminando a los demás productos almacenados, de ser así, comunicarlo al jefe de almacén inmediatamente.
	\item Asegúrate que en los equipos móviles exista una total limpieza para no contaminar el producto.
	\item Asegúrate que el material de empaque contengan el producto marcado en la etiqueta.
	\item Todos los equipos deben de estar libres de óxido, tales como equipo móvil, patines, \textit{rack,} etc.
	\item Dentro de la planta no debe de haber ningún utensilio de limpieza con mango de madera, ni mucho menos cerca de los productos.
	\item Durante el proceso de limpieza realizada no generes polvos ni salpicaduras de agua que puedan contaminar los productos.
	\item Los productos que evidentemente no sean aptos, deben de reportarse, separarse y eliminarse de la planta a fin de evitar mal uso, contaminación e infestaciones.
	\item Todos los productos deben de estar bien identificados y clasificados para evitar una contaminación cruzada.
	\item Se debe asegurar que los equipos que tienen partes lubricadas no contaminen al producto al momento de manióbralos para su acomodo.
	\item En el área de manipulación de productos no debe de almacenarse ninguna sustancia que pudiera contaminarlos, como son los aceites, solventes, tintas, etc.
	\item No almacenar directamente sobre el piso producto, o materiales de empaque, se deben de almacenar sobre tarimas u otros aditamentos.
	\item Todos los vehículos que transporte, se deben revisar antes de cargarlos con el fin de asegurar que se encuentren en buenas condiciones sanitarias.
	\item Todos los rollos de película de empaque que esté sin usar deben de estar cubiertos para evitar que se contaminen con polvo u otros materiales extraños.
	\item Siempre que se detecte un trozo de pintura dentro del almacén, deben de avisar al encargado de área y realizar la corrección.
	\item Todos los productos deben de portar la fecha de caducidad y número de lote con el fin de llevar un mejor control de \textit{PEPS} (primeras entradas, primeras salidas)
	\item No aceptar ningún insumo que no se encuentre identificado.
	\item Antes de efectuar la limpieza de tu equipo y área de trabajo, asegúrate de que no existan productos o materiales de emplaye, tarimas, alrededor que pudieran ser afectados.
	\item Durante la inspección, no manipule el producto con las manos. En este caso y en todo momento se deberá utilizar el equipo apropiado para este propósito, los cuales deberán estar siempre limpios y almacenarse fuera de contacto con el piso o superficies sucias.
\end{enumerate}

\subsection{Control de plagas}
\index{Plagas!control de}
Se tiene por objetivo prevenir o impedir la infestación de las instalaciones, con plagas que además de causar daños a los productos y materiales de empaque, pueden causar enfermedades en los consumidores y deteriorar la imagen de la compañía, para evitar esto, debe de poner en práctica las siguientes normas:

\begin{enumerate}
	\item No mover o dañar las trampas que se encuentran instaladas en el interior y exterior de la planta.
	\item No mover, dañar o substraer el veneno de las trampas que se encuentran en el perímetro externo de la planta.
	\item No desconectar o bloquear las trampas de la luz para insectos voladores que se encuentren en el interior de la planta.
	\item Cada vez que se detecte presencia de alguna plaga o de su evidencia, comunicarlo al responsable del área.
	\item Cuando observes alguna de las trampas dañadas o sin funcionar da aviso al responsable del área, con tu ayuda podremos mantener un buen control de plagas.
	\item No debe de haber nido de pájaros vivos o muertos dentro de la planta.
	\item No debe de existir ningún tipo de plaga dentro de la planta. Si detectas algunas de ellas repórtalas al responsable del área.
\end{enumerate}

\subsection{Práctica de empleados}

La finalidad principal de este programa es la de observar todas las acciones que realizan los empleados al efectuar sus actividades diarias y que pueden afectar directamente a otros programas, mientras los empleados conozcan y entiendan las normas de sanidad, los problemas se minimizaran.

\begin{enumerate}
	\item Todo el personal que opere en las áreas de almacenaje debe entrenarse en las buenas prácticas de higiene y sanidad.
	\item Todos los visitantes internos y externos deben de cubrir su cabello, barba y bigote, además de ropa y calzado adecuados antes de entrar a las áreas de operación.
	\item Cuando uses los sanitarios, asegurarte de lavarte perfectamente y desinfectarte las manos antes de retirarte a tu área de trabajo o consumir tus alimentos.
	\item Deposita el papel higiénico en los botes de basura destinado para ese uso, nunca en los pisos.
	\item Queda estrictamente prohibido defecar u orinar en las áreas externas de la planta.
	\item Es obligatorio el baño diario antes de iniciar tu jornada de trabajo.
	\item No esconder utensilios de limpieza o refacciones dentro del equipo.
	\item Está estrictamente prohibido sustraer o hurtar cualquier producto de nuestra planta.
	\item Se prohíbe rayar, dibujar o escribir en puertas y paredes.
	\item Está prohibido escupir en cualquier parte de la planta.
	\item En caso de utilizar mandiles o guantes, lavarlos periódicamente o en su defecto cambiarlos.
	\item No jugar ni hacer bromas que puedan ocasionar un accidente a ti mismo o a tus compañeros.
	\item No hacer reparaciones improvisadas con cartón, cinta, mecate, alambre, etc.
	\item No tirar la basura en el piso ni en las coladeras o registros de aguas pluviales o residuales.
	\item Es obligatorio el uso del uniforme, el cual siempre tendrá que estar limpio antes de entrar a tu jornada de trabajo.
	\item Prescindir de plumas, lapiceros, termómetros, sujetadores u otros objetos desprendibles en la vestimenta de la cintura hacia arriba.
	\item Las cortadas y heridas deben de cubrirse apropiadamente con material impermeable evitando entrar al área de producción cuando estas se encuentren en partes del cuerpo que estén al contacto directo con el producto y que puedan propiciar la contaminación del mismo. Este personal puede ser colocado en áreas donde realice actividades que no tengan contacto con producto, materia prima o material de empaque (Fuera de áreas de operación)
	\item Evitar que personas con enfermedades contagiosas laboren en contacto directo con los productos. Este personal puede ser colocado en áreas donde realice actividades que no tengan contacto con producto, materia prima o material de empaque (Fuera de áreas de operación). Si la enfermedad es grave el personal no debe de presentarse a su área de trabajo para evitar cualquier riesgo de contaminación del producto.
	\item Pulseras, anillos, collares, relojes u otras joyas no están permitidas en el área de producción.
	\item Las uñas deben de estar limpias, bien cortadas y sin barniz.
	\item El suéter utilizado solo debe de ser el proporcionado por la empresa.
	\item Solamente está permitido el uso de calzado de seguridad, el mismo que deberá mantenerse siempre limpio y en buenas condiciones.
	\item No se permite cabello largo en los hombres, se debe de mantener a nivel del cuello de la camisa como máximo.
	\item Se deberá usar cofia que cubra totalmente el pelo cuando se tenga acceso a la planta.
	\item Las cofias deshilachadas rotas o sucias no están permitidas por lo que se deberá de cambiar inmediatamente.
	\item Se permite bigote nítido, no más ancho que la boca, a lo largo que el inicio del labio superior.
	\item La patilla no deberá extenderse más abajo del lóbulo de la oreja.
	\item Los hombres siempre tendrán que estar bien afeitados.
	\item No está permitido la barba crecida o se tendrá que utilizar cubre barba.
	\item Queda estrictamente prohibido introducir alimentos al área de los almacenes.
	\item Todo alimento y bebida deben de ser consumidos solo en el comedor.
	\item Al terminar de comer, todos los alimentos sobrantes deben de recogerlos, dejando el lugar limpio y ordenado.
	\item No está permitido sacar alimentos, bebidas o similares del área del comedor hacia las áreas de almacenamiento.
	\item Está prohibida la posesión y/o consumo de bebidas alcohólicas en la planta.
	\item No se permite mascar chicle o golosinas dentro de la planta en general.
	\item Está prohibido fumar dentro de la planta en general.
	\item Los lavabos se usarán específicamente para el aseo de las manos y no para el lavado de cabello o cuerpo.
	\item Durante el peinado asegúrate que el cabello haya sido eliminado de los hombros.
	\item Medicamentos de cualquier tipo no deberán introducirse a las áreas de los almacenes.
	\item Al desarmar los equipos para su limpieza, todas sus partes deben de colocarse sobre estantes o carros especiales diseñados para este propósito y nunca directamente sobre el piso.
	\item Evitar rebasar la capacidad de las tarimas con rejas, cubetas y/o producto.
	\item No dejar las puertas o ventanas abiertas, ya que esto deja entrar polvo y plagas.
	\item No mezcles la basura con desperdicios o chatarra.
	\item Está estrictamente prohibido consumir alimentos o fumar en los baños y vestidores.
	\item No caminar sobre las tarimas, producto o carros transportadores.
	\item No usar limpieza húmeda antes de haber hecho una perfecta limpieza en seco.
	\item Cuando una máquina o equipo este en mantenimiento, no dejar piezas, tornillos tuercas, rondanas, etc.\ sobre el piso, tarimas y/o producto.
	\item Evitar obstruir el paso con paquetes u otros objetos en los pasillos.
	\item En ninguna de las áreas de la planta deben de existir tarimas pegadas a la pared.
	\item No bloquear los sistemas contra incendios.
	\item Los recipientes para basura no deben de rebasar su nivel llenado, estos deben de estar en un máximo de tres cuartos de su capacidad.
	\item No verter producto en coladeras o registros.
\end{enumerate}

\subsection{Limpieza de equipos}
\index{Limpieza!equipos, de}
Este programa se enfoca principalmente a observar la eficiente limpieza del equipo para asegurar que no habrá afectación en la calidad sanitaria de los productos provocados por la contaminación cruzada debido a una mala limpieza del equipo

\begin{enumerate}
	\item Debes conocer los procedimientos de limpieza del equipo que se te asigne para limpiar.
	\item Debe de impedirse la acumulación de suciedad, ya que esto facilita la formación de microorganismos.
	\item Los equipos y utensilios se deben mantener limpios en todas sus partes, y desinfectados si así se requiere.
	\item Las superficies de los equipos deben de estar lisas y exentas de orificios y grietas.
	\item Después de un mantenimiento del equipo, este se debe de lavar y desinfectar si lo requiere.
	\item Los equipos deben de estar funcionando y en buen estado.
	\item En el interior del equipo no debe de haber producto incrustado o suciedad en él.
	\item Mientras los equipos se encuentren parados, estos deben de mantenerse limpios.
\end{enumerate}

\subsection{Mantenimiento de equipos y edificio}
\index{Mantenimiento!infraestructura, de}
Este programa está enfocado a detectar cualquier situación presente en el equipo y edificio que pueda afectar cualquiera de los programas de sanidad, es decir, que este programa nos ayuda a asegurar un excelente estado de funcionamiento del equipo y edificio.

\begin{enumerate}
	\item Todos los agujeros de las paredes externas e internas, deben de ser reparadas y las aberturas alrededor de los tubos y cables deben de ser sellados para evitar la entrada de plagas.
	\item Se debe de contar con una excelente iluminación en todas las áreas.
	\item Las juntas de los patios y paredes deben de estar bien sellados.
	\item Eliminar la presencia de muros con orificios y puertas dañadas.
	\item Los pisos deben de tener una superficie lisa, sin grietas, perforaciones, agujeros o deformaciones.
	\item Todos los mecanismos en movimiento deben de contar con su guarda correspondiente.
	\item Todas las áreas deben de contar con un buen pintado.
\end{enumerate}

\subsection{Higiene y sanidad}
\index{Higiene y sandidad!áreas contempladas}
Este programa está enfocado a detectar cualquier situación que de una mala imagen de orden y falta de limpieza en las diferentes áreas de la planta. Para eso debemos de cumplir con lo siguiente 1. No debe de haber basura ni desperdicio en los pisos.
2. Los pisos, plataformas, pasillos, paredes, columnas, trabes, tuberías, deben de estar libres de polvo.

\begin{itemize}
	\item Todas las coladeras y registros, deben de estar limpios y sin bloquear.
	\item No deben de existir montones de desperdicio o basura en los pisos y rincones.
	\item Todas las escaleras de acceso, plataformas y pisos deben de estar siempre limpias.
	\item No bloquear con materiales de empaque, producto, materiales de equipo a puertas, escaleras de acceso o tránsito de personal.
	\item Todas las azoteas deben de estar limpias.
	\item No debe de haber producto tirado.
	\item No deben de existir cubetas o tambos con agua acumulada y con mal olor.
	\item Las áreas para utensilios de limpieza deben de estar limpias y ordenadas.
	\item Los pisos del baño deben de estar limpios, secos y desinfectados.
	\item No debe de haber papel higiénico tirado fuera de las tazas del baño.
	\item En los talleres de mantenimiento no debe de existir desorden.
	\item Todas las plataformas, pisos deben de estar libres de agua estancada.
	\item Todos los pisos y mesas de trabajo deben de estar limpios y ordenadas 16. No debe de haber escurrimientos de aceite en los pisos.
	\item Los patios y pasillos deben de estar limpios y libres de objetos que obstruyan el paso.
	\item Mantener \emph{lockers} limpios y ordenados
	\item No deben de existir pisos manchados de tinta o pintura.
	\item Las cortinas Hawaianas siempre deben de estar limpias.
	\item El objetivo de estas normas, es el de prevenir la contaminación y perdida de nuestros alimentos, en el presente y en el futuro, y lograr la satisfacción de nuestros clientes y consumidores.
	\item Las normas aquí establecidas son políticas de Productos aquí almacenados y su aplicación y cumplimiento son de carácter obligatorio.
	      Estamos seguros de que con tu participación, podremos lograr mantener la satisfacción total de nuestros clientes.
\end{itemize}

\subsection{Diseño, construcción y mantenimiento}

\subsection{Exteriores}
\index{Diseño!exteriores, de}
\begin{itemize}
	\item El \emph{exterior} de la planta debe de estar diseñado, construido y mantenido, de modo que evite infecciones y contaminaciones.
	\item Debe de evitarse que los patios del establecimiento existan condiciones que puedan ocasionar contaminación de producto y proliferación de plagas tales como: equipo mal almacenado, basura, desperdicio y chatarra, formación de maleza o hiervas, drenaje insuficiente o inadecuado (Los drenajes deben de tener cubierta apropiada para evitar la entrada de plagas provenientes del alcantarillado o arias externas) e iluminación inadecuada.
	\item El arreglo de exteriores incrementa la apariencia y la imagen de calidad e integridad.
	\item El drenaje debe de ser apropiado para evitar que se estanque el agua, lo que puede llevar a condiciones insalubres.
	\item Los empleados deben de mantener los alrededores de la planta de forma ordenada y limpia todo el tiempo para prevenir insectos, roedores, suciedades, plagas y otros elementos infecciosas que puedan convertirse en una fuente de contaminación de los alimentos.
	\item Las áreas destinadas a la colección de desechos deberán mantenerse limpias y libres de olores desagradables.
\end{itemize}

\subsubsection{Interiores}
\index{Diseño!interiores, de}
\begin{itemize}
	\item Los edificios y los equipos en los que el producto es almacenado y controlado, deben de ser del tamaño adecuado y facilitar el mantenimiento y las operaciones sanitarias necesarias.
	\item Debe haber suficiente espacio entre los productos almacenados para facilitar los controles sanitarios, permitiendo un tránsito que no represente un posible riesgo para los transeúntes y operarios.
	\item Paredes, pisos y techos deben de estar permanentemente en reparación, y deben de estar construidos de modo que permitan un acceso fácil y eficiente para su limpieza.
	\item Todas las entradas a la planta deben de permanecer cerradas, para impedir la entrada de insectos, aves o roedores.
	\item Las áreas de almacenamiento cuyos techos permitan la entrada de animales, deben ser selladas con plásticos flexibles, todas las entradas a la planta deben de estar diseñadas contra roedores.
	\item Los pisos paredes y techos, deben ser lavados por materiales limpiadores y desinfectantes de la suficiente fuerza para prevenir la proliferación de microorganismos y mantener las condiciones sanitarias.
	\item Los equipos empleados para el acomodo de los productos deben ser limpiados con el producto adecuado, para asegurar una limpieza a fondo.
	\item Se deben de usar agentes químicos para impedir que los equipos se vuelvan campos de crecimiento para bacterias u otros contaminantes.
	\item Debe de evitarse condiciones inseguras por lo que los pisos deberán mantenerse secos.
	\item Los derrames de productos así como las salpicaduras y goteos, deben de ser limpiados lo antes posible por el mismo personal operativo. Todo el personal estará consciente de que la limpieza es una labor de todos. Cada miembro del personal tiene la responsabilidad de mantener su área de trabajo en buenas condiciones sanitarias.
\end{itemize}

\subsubsection{Iluminación}
\index{Diseño!iluminación, de}
\begin{itemize}
	\item Debe de haber una adecuada iluminación en todas las áreas de la planta, de modo que las actividades de almacenaje e inspección puedan ser llevadas a cabo efectivamente.
	\item Las lámparas y focos que se encuentren suspendidos sobre los productos almacenados deben de estar protegidos para evitar contaminación en caso de rupturas.
\end{itemize}

\subsubsection{Ventilación}
\index{Diseño!ventilación, de}
\begin{itemize}
	\item Debe de haber una adecuada ventilación en todas las áreas de la planta de manera que se prevenga una inaceptable acumulación de polvo y deberá extraerse el aire contaminado.
	\item Las entradas de aire deben de estar protegidas con malla para evitar la entrada de contaminantes o insectos.
	\item La dirección de la corriente de aire no debe de ir nunca de un área sucia a un área limpia.
\end{itemize}

\subsubsection{Basura}
\index{Diseño!basura, disposición de}
\begin{itemize}
	\item Toda la basura y desechos de la planta deberán mantenerse en recipientes destinados para tal fin, los cuales se mantendrán limpios, sin escurrimientos y tapados.
	\item Debe de evitarse la acumulación de basura, merma y alimentos descompuestos dentro y fuera de la planta.
	\item Se realizará una clasificación de los desechos en orgánicos e inorgánicos con la finalidad de fomentar y establecer el reciclaje de los materiales.
	\item Los desperdicios deberán ser retirados periódicamente de nuestras instalaciones.
	\item El drenaje debe de estar colocado de modo que los sólidos y/o líquidos no se acumulen a lo largo de él, en ningún lugar, ya que ocasionaría una gran contaminación por microorganismos.
	\item Los drenajes deben estar diseñados de tal manera que eviten la entrada de roedores a la planta, ya que estos pueden ser vectores de muchas enfermedades y hay que poner especial atención para evitarlos dentro de las instalaciones.
	\item Las aguas negras no deben de pasar a través de las zonas de almacenamiento.
\end{itemize}

\subsection{Instalaciones sanitarias}

\subsubsection{Instalaciones para el personal}
\index{Diseño!instalaciones para el personal}
\begin{itemize}
	\item Los retretes deben de estar limpios para evitar enfermedades y que estas constituyan un foco de infección y contagio.
	\item Siempre se debe de contar con abastecimiento continuo de agua.
	\item Debe de contar con suministro de papel sanitario.
	\item Los sanitarios se deben de lavar y desinfectar diariamente para evitar que las personas que los usen contraigan alguna enfermedad e introduzcan microbios a la planta.
	\item Los lavabos deben de estar limpios y con suministro continuo de agua.
	\item Debe de haber jabón de tipo líquido (preferentemente con desinfectante) y el accionamiento de agua debe de ser con una llave en los pies o infrarroja para evitar contacto de las manos limpias con la llave sucia del agua.
	\item Se recomienda colocar en este punto letreros que le recuerden al personal las medidas de higiene que deben de seguir.
	\item El personal contará con área destinada exclusivamente para guardar los artículos personales.
	\item Se prohibirá mantener fuera de las gavetas algún artículo personal.
	\item Cada una de las gavetas estará identificada y deberá contar con candado como medida de seguridad.
	\item Se contará con área de comedor para personal, los cuales deberán mantenerse siempre en orden y en buenas condiciones higiénicas.
	\item Los alimentos y bebidas serán depositados en el área destinada para tal efecto.
	\item Deberá contarse con bote de basura con tapa para evitar la generación de malos olores y proliferación de insectos, así mismo los desechos se eliminarán a la brevedad posible de la planta.
\end{itemize}

\subsection{Lavado de equipo e instalaciones sanitarias:}
\index{Limpieza!equipo, de}
\begin{itemize}
	\item Los pisos, paredes y techos deben ser lavados con materiales limpiadores y desinfectantes de la suficiente fuerza para prevenir la proliferación de microorganismos y mantener las condiciones sanitarias.
	\item Los equipos empleados en el proceso de acomodo de producto, deben ser lavados con el material adecuado, para asegurar una limpieza a fondo.
	\item Se deben de usar agentes químicos para impedir que los equipos se vuelvan campos de crecimiento para bacterias u otros contaminantes.
	\item Debe de evitar condiciones inseguras por lo que los pisos deberá mantenerse secos.
	\item Los derrames de productos, así como salpicaduras y goteos debe de ser limpiados lo antes posible por el mismo personal operativo, todo el personal deberá estar consciente de que la limpieza es una labor de todos. Todo el personal tiene la responsabilidad de mantener su área de trabajo en buenas condiciones sanitarias.
\end{itemize}

\subsection{Operaciones sanitarias}

\begin{itemize}
	\item Todas las instalaciones de la planta estarán siempre en buenas condiciones y de ir de acuerdo con la \textbf{norma oficial mexicana de seguridad}
\end{itemize}

\definecolor{Gallery}{rgb}{0.929,0.929,0.929}
\begin{table}
	\centering
	\caption{Código de colores para las tuberías.}
	\label{tab:cod.colores.tuberias}
	\begin{tblr}{%
		%width = 0.5\linewidth,
		colspec = {cc},
		cells = {c},
		row{even} = {Gallery},
		}
		\toprule
		\textbf{Color} & \textbf{Uso}           \\
		\midrule
		Azul           & Agua                   \\
		Rojo           & Conductores eléctricos \\
		Gris           & Drenaje                \\
		Verde          & Refrigerante           \\
		\bottomrule
	\end{tblr}
\end{table}

\subsection{Calidad y abastecimiento de agua}
\index{Agua!calidad del}
\index{Agua!abastecimiento de}
\begin{itemize}
	\item Debe de disponerse de suficiente abastecimiento de agua la cual debe de provenir de una fuente confiable y suficiente así como de instalaciones apropiadas para su almacenamiento y distribución.
	\item Toda el agua que entra en contacto con cualquier superficie dentro de la planta, o se vaya a beber, debe de tener una pureza y limpieza asegurada.
	\item En todas las áreas debe de haber agua corriente a la temperatura y presión necesaria para estar de acuerdo a los requerimientos de la demanda para proceso de limpieza.
\end{itemize}

\subsubsection{Controles}
\index{Agua!controles microbiologicos}
\begin{itemize}
	\item Se deben de llevar controles por escrito que demuestren la seguridad microbiológica y química, del agua de entrada.
	\item Se debe de realizar la determinación de contenido de cloro en el área de abastecimiento, llevando un registro de este control. Y se recomienda realizar análisis microbiológicos de coliformes totales y fecales.
\end{itemize}

\subsection{Transporte}
\index{Transporte!verificación de}
\begin{itemize}
	\item Se debe de asegurar que los vehículos que cargan o descargan están libres de condiciones que puedan contaminar los productos. No debe de haber evidencia de roedores, pájaros, derrames, olores desagradables o materiales extraños.
	\item El equipo de transporte debe de estar limpio y en buenas condiciones y no tener huecos, escondites, o peligros que puedan servir de albergue a insectos u otras alimañas.
	\item Todos los procedimientos de manipulación durante el transporte deben de ser de tal naturaleza que impidan la contaminación del producto.
\end{itemize}

\subsection{Almacenamiento}
\index{Programas de prerequisitos!almacenamiento, aplicables a}
\begin{itemize}
	\item Todos los productos que se reciban se deben de colocar en forma tal que faciliten la limpieza e implementación del control de insectos, roedores y otras medidas sanitarias.
	\item Prácticas de buena limpieza y mantenimiento así como un programa general de sanidad deben de ser llevados a cabo en forma continua en todos los almacenes y centros de distribución para prevenir la creación de condiciones insalubres.
	\item Debe de impedirse la entrada de animales domésticos en las áreas de almacenes de producto.
\end{itemize}

\subsection{Recepción y almacenamiento de productos}
\index{Almacenamiento!disposiciones generales}
\begin{itemize}
	\item Los productos recibidos deben de ser almacenados preferentemente de \SI{45}{\centi\meter} o mínimo \SI{32}{\centi\meter} de las paredes para facilitar acceso a la inspección, limpieza, una barra para mantener los productos alejados de la pared, espacio para las operaciones de control de plagas y roedores, así como la realización de los inventarios.
	\item Los pasillos adyacentes a las estibas deben de mantenerse limpios y libres de obstáculos.
\end{itemize}

\subsection{Recepción y almacenamiento de productos no-comestibles}
\index{Almacenamiento!productos no comestibles}
\begin{itemize}
	\item Los plaguicidas, detergentes, desinfectantes y otras substancias toxicas, deben de etiquetarse adecuadamente con un rotulo que se informe sobre su toxicidad y empleo. Estos productos deben de almacenarse en áreas o armarios especialmente destinados al efecto, y deben de ser distribuidos o manipulados solo por personal competente. Se debe de poner el mayor cuidado para evitar la contaminación de los productos.
	\item Los materiales sanitizados deben de ser señalados y almacenados en áreas separadas del material de empaque o producto alimenticio.
	\item Todos los materiales empleados para la limpieza deben de almacenarse limpios y secos, para evitar la proliferación de olores desagradables y posible crecimiento microbiano.
\end{itemize}

\subsection{Equipo}

\subsubsection{Diseño e instalación}
\index{Limpieza!equipos y utensilios}
\begin{itemize}
	\item Todos los equipos y utensilios de la planta deben ser diseñados en material de fácil limpieza y mantenimiento.
	\item El diseño y construcción de los equipos de almacenaje, debe de evitar la adulteración del alimento con lubricantes, aceites, fragmentos de metal, agua contaminada, etc.
	\item Las superficies en contacto con los alimentos deben ser resistentes a la corrosión y hechos de un material no toxico.
\end{itemize}

\subsubsection{Superficies en contacto con aliementos}
\index{Limpieza!superficies en contacto con producto, de}
\begin{itemize}
	\item Las superficies que no estén en contacto con el producto deben de limpiarse lo suficiente para prevenir la acumulación de polvo, suciedad, partículas contaminantes u otros materiales. Se debe de tener cuidado especial con los conductos eléctricos.
\end{itemize}

\subsubsection{Calibración y mantenimiento de equipos}
\index{Calibración!equipos, de}
\begin{itemize}
	\item Se debe tener por escrito un programa efectivo de mantenimiento.
	\item El mantenimiento y calibración de los equipos se debe de llevar a cabo por personal entrenado para ello \emph{(SE REALIZA POR COMPAÑÍA EXTERNA).}
	\item Todos los instrumentos de control, deben de estar calibrados, y en condiciones de uso para evitar desviaciones de los patrones de operación.
	\item Al lubricar el equipo se deben de tomar precauciones para evitar contaminación de los productos que se almacenan. Se deben de emplear lubricantes inocuos.
	\item Los equipos deben de estar instalados en forma tal que el espacio entre la pared, el techo y piso permitan su limpieza.
	\item Las bombas, compresores, ventiladores y equipo en general de impulso para el manejo de refrigeración, deben de ser colocadas sobre una base que no dificulta la limpieza y el mantenimiento.
	\item Las partes externas de los equipos que no estén en contacto con los alimentos, deben de estar limpios.
	\item Los equipos y utensilios deben de estar en buenas condiciones de funcionamiento, dándoles el mantenimiento necesario.
\end{itemize}

\subsection{Controles escritos de mantenimiento}
\index{Mantenimiento!registros}
\begin{itemize}
	\item Se deben contar con controles por escrito del mantenimiento que se realiza a los equipos (Principalmente los críticos) especificando la identificación del equipo, la actividad de mantenimiento, la fecha, la razón por la que se efectuó el mantenimiento y la persona encargada de realizarlo.
\end{itemize}

\subsection{Control de calibración}
\index{Calibración!registros}
\begin{itemize}
	\item Se debe de recopilar la información acerca de las calibraciones efectuadas, especificando el nombre del equipo, la fecha de la calibración, los resultados obtenidos y el nombre de la persona o empresa que realizo la calibración \emph{(SE REALIZA POR COMPAÑÍA EXTERNA).}
\end{itemize}

\subsection{Personal}
\index{Personal!contratación}
\begin{itemize}
	\item Es importante, como condición para ser contratado, que todos los candidatos entrevistados y contratados entiendan y cumplan las reglas de buenas prácticas de distribución, especialmente aquellos que esté relacionada con la apariencia, higiene y sanidad del personal.
\end{itemize}

\subsubsection{Capacitaciones}
\index{Capacitaciones}
\paragraph{Entrenamiento en higiene alimentaria}
\index{Capacitaciones!higiene alimentaria, en}
\begin{itemize}
	\item Es importante que el personal de la gerencia esté al tanto de lo que se puede y no se puede en las buenas prácticas de distribución. Ello debe ser un ejemplo en el trabajo día a día, esto hará subir la confianza del personal y ganar su interés en seguir las buenas prácticas de manufactura.
	\item Los supervisores y personal deben de recibir entrenamiento apropiado en las técnicas de manejo de alimentos, principios de protección de los alimentos y deben de estar al tanto de los peligros de contaminación causados por falla en limpieza personal, prácticas insalubres. Personal bien entrenado, con buenos hábitos y buena actitud son esenciales para asegurar continuamente la calidad del producto.
\end{itemize}

\paragraph{Entrenamiento técnico}
\index{Capacitaciones!entrenamiento técnico, en}
\begin{itemize}
	\item El personal responsable de la seguridad, la calidad y la sanidad debe de estar calificado a través de educación o entrenamiento, o una combinación para darle un nivel de competencia.
	\item El personal de supervisión en el área de almacenaje, debe de estar en un número adecuado y tener el entrenamiento y la experiencia para asegurarse de que el producto cumple con los estándares de la compañía y con las regulaciones que existan.
\end{itemize}

\subsubsection{Requisitos de seguridad e higiene}
\index{Requisitos de seguridad!vestimenta}
\begin{itemize}
	\item Se debe de proveer al empleado de vestimenta adecuada que sea capaz de prevenir la contaminación del producto.
	\item Los uniformes o ropa exterior, deben de estar limpios al comienzo de las actividades y mantenerse razonablemente limpios hasta el final de las operaciones.
	\item Se deben de establecer las políticas de la compañía o la planta acerca de la limpieza del personal, así como el manejo y prevención de la contaminación de los alimentos.
	\item La gerencia debe de establecer procedimientos para implementar y monitorear el seguimiento por parte del personal de las reglas establecidas.
	\item Se debe de asegurar que el personal siga buenas prácticas higiénicas y que la planta cuente con un programa de sanitización efectivo.
	\item Es preferible que se usen cierres o broches en lugar de botones en los uniformes.
	\item No se permitirán bolsas en la ropa que estén situados más arriba de la cintura. Si existen estos bolsillos en la ropa, se deben de coser o remover para prevenir que los artículos que pudieran colocarse caigan accidentalmente en el producto.
	\item Los zapatos se deberán de mantener limpios, nítidos y en buenas condiciones.
	\item Solamente serán permitidos botas de hule anti-derrapantes, o zapatos de seguridad para el personal de mantenimiento.
	\item El personal deberá de mantener un alto grado de limpieza personal para prevenir la contaminación de los alimentos.
	\item Los hombres deberán de mantenerse bien afeitados para promover un ambiente de trabajo sanitario.
	\item La barba y bigote deberán estar cubiertos por un cubre pelo adecuado.
	\item Deberá de evitarse en la medida posible el uso de bigote, barba y patillas, de lo contrario deberán de cumplir con los siguientes requisitos:
	\item Que el bigote no se extienda más allá del borde externo de la boca.
	\item Que el bigote bajo el labio no sea de más de \SI{1.2}{\centi\meter} y que no se extienda más allá de la barbilla.
	\item Que las patillas sean recortadas y que no se extiendan más allá de la parte inferior de la oreja.
	\item El pelo debe de mantenerse limpio y cubierto en su totalidad por redes, gorros, cascos o cualquier otra cubierta similar.
	\item Las manos deben mantenerse siempre limpias y se debe de evitar saludar de mano a cualquier persona durante sus labores.
	\item No se permitirá colgar o mantener ropa u otras pertenencias personales en áreas de almacenamiento de producto.
	\item No se permiten alimentos o bebidas para uso personal en la planta, excepto en las áreas autorizadas para este propósito.
	\item Los almuerzos deben de estar guardados fuera de las áreas de almacenaje y de preferencia en lugares cerrados.
	\item No se permite mascar chicles o cualquier tipo de golosina.
	\item No se permite tener en la boca ningún objeto mientras se esté en las áreas de almacenamiento, tampoco se permite tener objetos detrás de las orejas.
	\item Se prohíbe ensuciar, desordenar o crear condiciones insalubres en la planta.
\end{itemize}

\subsubsection{Enfermedades contagiosas y heridas}
\index{Capacitaciones!enfermedades y heridas}
\begin{itemize}
	\item Cuando la compañía o las agencias gubernamentales, requieren que el empleado pase un examen médico, el aplicante solo debe ser contratado después de ser examinado por un medio y de que este haya extendido un certificado de que la condición del aplicante es adecuada para trabajar con alimentos.
	\item Un archivo con todos los exámenes que se hayan practicado al empleado deben de guardarse y debe de ser revisado periódicamente.
	\item Cualquier empleado que regrese a trabajar después de una enfermedad transmisible o después de una enfermedad seria de cualquier tipo, debe de ser certificado por un médico para poder regresar al trabajo en la planta de alimentos.
	\item El médico de la compañía si se cuenta con él, debe de dar la última opinión en todos los casos dudosos.
	\item Los empleados que se enfermen durante horas de trabajo deberán reportarlo al supervisor.
	\item Si se requiere toser o estornudar, la persona deberá alejarse del producto que este manipulando, cubrirse la boca y después lavarse las manos para prevenir la contaminación microbiana.
\end{itemize}

\subsection{Visitantes}
\index{Requisitos de seguridad!visitantes}
\begin{itemize}
	\item Toda persona que visite el CEDIS, deberá de dar aviso de su llegada al personal responsable del CEDIS.
	\item El personal responsable de CEDIS debe de autorizar la entrada de la persona al CEDIS.
	\item Si el visitante se dirige a un área del almacén, el personal encargado deberá de proporcionar una cofia y cubre bocas.
	\item Todo visitante deberá de cumplir con las normas de higiene y seguridad del almacén.
\end{itemize}

\subsection{Prevención de la contaminación cruzada}
\index{Progrmas de prerequisitos!Buenas prácticas de distribución, programa de}
\begin{itemize}
	\item todo el personal que tenga que acceder al \gls{area-operativa} tendrá que lavarse y desfinfectarse las manos en la \gls{aduana};
	\item en caso de usar guantes, el personal tendrá que lavarse y desinfectarse las manos previo a ponerse los guantes;
	\item Se deben tomar medidas para evitar la contaminación del producto por contacto directo o indirecto con materiales que se encuentren almacenados.
\end{itemize}

\subsection{Programa de limpieza}
\index{Progrmas de prerequisitos!limpieza, programa de}
\begin{itemize}
	\item Se debe de establecer por escrito un programa de limpieza y sanitización para todas y cada una de las áreas de la planta.
	\item Todos los equipos de limpieza deben de mantenerse en una solución desinfectante o sobre una superficie limpia cuando no se usen. Los cepillos deben de mantenerse fuera de contacto con el suelo, ya que su superficie contiene microorganismos que pueden adherirse fácilmente a las cerdas.
	\item Los cepillos mojados deben ser colgados hasta que se sequen. El guardar o mantener los cepillos mojados en una superficie plana no es una medida sanitaria y resulta en la deformación de las cerdas, esto causa un daño prematuro al cepillo y un costo adicional de utensilios de limpieza.
	\item Los productos químicos que se ocupan para la limpieza deben de usarse de acuerdo a las indicaciones del fabricante y deben de ser aceptados para su empleo en plantas procesadoras de alimentos.
\end{itemize}

\subsection{Controles sanitarios}
\index{Requisitos de seguridad!controles sanitarios}
\begin{itemize}
	\item Los controles sanitarios deben de incluir la fecha, la persona responsable, lo encontrado, la acción correctiva tomada y de ser posible los resultados obtenidos de pruebas microbiológicas.
\end{itemize}

\subsection{Programa de control de plagas}
\index{Progrmas de prerequisitos!control de plagas, programa de}
\begin{itemize}
	\item Se dará un énfasis primordial a mantener buenas prácticas de distribución, limpieza y mantenimiento, sin embargo estas se reforzaran mediante la aplicación de controles químicos para la prevención de control de plagas.
	\item El trabajo de control de pestes debe ser delegado a una persona responsable que esta apropiadamente entrenada en este trabajo y emplee el equipo adecuado. Una opción puede ser contratar una compañía externa que haga este trabajo.
	\item No debe de haber animales o aves que no sean productos almacenados, por lo cual no deben de ser permitidos en ninguna área donde se almacenen alimentos.
	\item Solo se deben de usar insecticidas y raticidas aprobados para su uso en plantas alimenticias.
	\item Todos los insecticidas y raticidas son venenosos y deben de ser considerados como adulterantes de los alimentos.
	\item Cada pesticida debe de tener un etiquetado claro y legible y debe ser usado de acuerdo con las indicaciones de la etiqueta o con las direcciones de la compañía.
	\item Todo el equipo que se usa para dispersar los pesticidas y otros compuestos químicos deben de ser limpiados apropiadamente después de cada uso y mantenido en buenas condiciones de operación.
	\item Todos los insecticidas y raticidas deben de ser usados de modo que no pueda haber contacto entre ellos y los alimentos, empaques, producto y/o equipos.
	\item El almacenamiento de todos los insecticidas y raticidas deben de hacerse en áreas especiales lejos de las áreas de alimentos o de químicos usados para otros fines.
\end{itemize}

\subsection{Controles del programa de control de plagas}
\index{Progrmas de prerequisitos!control de plagas, programa de!controles}
\begin{itemize}
	\item Los controles deben de incluir el pesticida empleado, así como la dosificación y el método de aplicación, lugar de aplicación, fecha y persona responsable.
	\item Se deben de especificar los resultados obtenidos de una inspección posterior a la aplicación del pesticida, así como las acciones correctivas que se tomen.
\end{itemize}

\subsection{Rastreo}
\index{Progrmas de prerequisitos!trazabilidad, programa de}
\index{Requisitos de seguridad!programa de control de plagas}
\begin{itemize}
	\item Se deben de establecer y mantener actualizados procedimientos documentados para controlar todos los documentos y datos generados por la empresa.
	\item Se debe establecer y mantener actualizados documentos para asegurar la prevención del uso no intencional o la instalación del producto no conforme con requisitos especificados. El control debe de estipular la identificación, la documentación, la evaluación, la disposición de producto no conforme y la notificación a las funciones interesadas.
	\item Se debe de definir la responsabilidad por la revisión y la autoridad para la disposición de producto no conforme.
	\item Se deben de establecer y mantener actualizados procedimientos documentados para aplicar acciones correctivas los cuales deben de incluir:
	      \begin{itemize}
		      \item El manejo efectivo de las quejas de los clientes y los informes de no conformidad del producto.
		      \item La investigación de la causa de no conformidad relacionadas con el producto, el proceso y el sistema de calidad, y el registro de resultados de la investigación.
		      \item La determinación de la acción correctiva necesaria para eliminar la causa de las no conformidades.
		      \item La aplicación de controles para asegurar que se aplique acción correctiva y que esta sea más eficaz.
	      \end{itemize}
\end{itemize}

\subsection{Codificación de productos}

\begin{itemize}
	\item Cada producto deberá contar con un código permanente y legible que indique el lote de producción.
	\item El código debe de identificar el establecimiento, el día, mes y año en que el producto caduca.
	\item El significado del código debe de estar por escrito y a la mano.
	\item Los códigos en el material de empaque, si se utilizan, deben de ser legibles y corresponder al código del producto del interior.
\end{itemize}

\subsection{Registros}

\begin{itemize}
	\item Todos los registros deben de ser claramente legibles.
	\item Se debe de \emph{usar tinta preferentemente negra} para llenado de registros, hojas de trabajo o cuadernos de notas. Cuando se cometa algún error se efectuara una cancelación mediante una línea diagonal, no se permiten borraduras o tachaduras.
	\item Todos los manuales (Técnicas, especificaciones, preparación de muestras y reactivos, formulaciones, etc.) deben de mantenerse identificados y de fácil acceso.
	\item Todos los manuales y garantías de los equipos y aparatos deben de mantenerse archivados, identificados y de fácil acceso.
	\item Se guardaran los datos y registros (Generados en el laboratorio) en disco flexible y una impresión de estos, los cuales serán almacenados en el archivo durante el lapso de 2 años.
	\item El escritorio y los archivos deben de mantenerse ordenados, identificados y limpios todo el tiempo
	\item Todas las actividades en la planta deberán observar en todo momento buenas prácticas de manufactura.
	\item Nunca mezcle un material que no cumpla especificaciones con otro en buenas condiciones.
\end{itemize}

\begin{changelog}[simple, sectioncmd=\subsection*,label=changelog-1.3]
	\begin{version}[v=2.1, date=2023--01, author=Pablo E. Alanis]
			\item Cambio de formato;
			\item Cambios en la serialización de versiones;
			\item Cambio de Codificación.
	\end{version}

	\begin{version}[v=1.8, date=2022-05, author=Alonso M.]
		\item No hubo cambios.
	\end{version}

	\shortversion{v=1.7, date=2021-05, changes=No hubo cambios}
\end{changelog}