\thispagestyle{formato-PI}
\renewcommand{\MayorVer}{2}
\renewcommand{\MenorVer}{1}
\renewcommand{\Codigo}{BPD-9-PRO}
\renewcommand{\FechaPub}{2023--01}
\renewcommand{\Titulo}{Adquisición de recursos}
\section{\Titulo}\index{Procedimiento!adquisición de recursos}
%\section{Compra de recursos}

\subsection{Objetivo}

\begin{itemize}
	\item \textbf{Establecer} el procedimiento para realizar compras de insumos requeridos por la organización.
\end{itemize}

\subsection{Alcance}

\begin{itemize}
	\item Este documento es aplicable a todo departamento que requiera hacer la compra de insumos.
\end{itemize}

\subsection{Términos y definiciones}

\begin{itemize}
	\item \textbf{orden de compra:} formulario por el cual se le da aviso al \emph{departamento de compras} sobre la necesidad de algun(os) insumos requeridos por la empresa;
	\item \textbf{lista de proveedores autorizados:} documento en donde se enumeran las compañias aprovadas por RDF para comprar materiales.
\end{itemize}

\subsection{Documentos y/o normas relacionadas}

\begin{itemize}
	\item PSA-L-2 - Lista de proveedores autorizados
\end{itemize}

\subsection{Procedimiento}

Cuando sea requerida la compra de varios insumos para la organización, el departamento interesado puede generar una orden de compra y solicitar su adquisición al \emph{departamento de compras.}

\subsubsection{Compra de insumos frecuentes}

Si los artículos que se compraran son frecuentemente adquiridos, estos generalmente se adquieren de la lista de proveedores autorizados (PSA-L-2 --- Lista de proveedores autorizados|PSA-L-2).

\begin{enumerate}
	\item llenar el formulario de \emph{orden de compra:} se debe especificar el producto y la cantidad requerida;
	\item se debe autorizar la solicitud por un representante de la alta dirección;
	\item entregar físicamente o por medios electrónicos éste documento firmado al \emph{área de compras.}
	\begin{itemize}
		\item Posteriormente el \emph{departamento de compras} asignará un numero de folio y se someterá a aprobación la compra por parte de la \emph{dirección adjunta;}
	\end{itemize}
\end{enumerate}

\subsubsection{Compra de recursos fuera de la lista de proveedores autorizados}

\begin{enumerate}
	\item Llenar el formato de orden de compra tal como en la compra de insumos frecuentes;
	\item anexar a la orden de compra la cotización del recurso requerido de por lo menos un vendedor;
	\begin{itemize}
		\item al recibir la orden de compra, el \emph{departamento de compras} se encargara de tratar de ubicar el proveedor que cuente con el recurso solicitado que sea la mejor opción para su adquisición;
		\item con estas cotizaciones anexadas, el \emph{departamento de compras} solicitará a la alta dirección la aprobación de estos insumos;
		\item cada vez que el \emph{departamento de compras} realice una orden de compra en nombre de cualquier departamento dentro de la organización, se le solicitará al \emph{departamento de administración} la entrega de facturas y medios de pagos para concretar la compra de los recursos.
	\end{itemize}
\end{enumerate}

\subsection{Frecuencia}

\begin{itemize}
	\item Cada vez que se requieran insumos para las tareas cotidianas dentro de la organización.
\end{itemize}

\begin{changelog}[simple, sectioncmd=\subsection*,label=changelog-1.10]
	\begin{version}[v=2.1, date=2023--01, author=Pablo E. Alanis]
		%\fixed
			\item Cambio de formato;
			\item se adoptaron versiones mayores y menores para revisiones;
			\item cambio en el código del documento.
	\end{version}

	\begin{version}[v=1.8, date=2022--05, author=Alonso M.]
		\item Primera versión.
	\end{version}
\end{changelog}
