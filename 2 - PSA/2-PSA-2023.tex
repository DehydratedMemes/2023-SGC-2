% % arara: xelatex: {
% % arara: --> interaction: nonstopmode,
% % arara: --> options: ['-file-line-error','-max-print-line=200'],
% % arara: --> shell: yes,
% % arara: --> synctex: yes,
% % arara: --> }
% arara: lualatex: {
% arara: --> interaction: nonstopmode,
% arara: --> options: ['-file-line-error'],
% arara: --> shell: yes,
% arara: --> synctex: yes,
% arara: --> }
% arara: makeglossaries
%% arara: makeindex

\documentclass[letterpaper,10pt,twoside,es-MX]{article}
%%%%%%%%%%%%%%%%%%%%%%%%%%% Definición de variables %%%%%%%%%%%%%%%%%%%%%%%%%%%%%%%%%%
%%%%%%% VARIABLES ESTÁTICAS %%%%%%%
% Aqui se definen las variables para diferentes procedimientos.
% Verificar que las unidades coincidan si se deciden cambiar.

\newcommand{\ie}{\emph{i.e.}}
\newcommand{\eg}{\emph{e.g.}}

%%%%%%%%%%%%%%%% Formularios %%%%%%%%%%%%%%%%
\newcommand{\IdFormAACC}{F-OP-40}

%%%%%%%%%%%%%%%% CÓDIGOS PSA %%%%%%%%%%%%%%%%
% \newcommand{\PSA}{PSA-3-PRO}                 % PSA-2.3 Inspección de producto durante su almacenamiento

%%%%%%%%%%%%%%%% PSA-2.3 | Inspección de producto durante su almacenamiento %%%%%%%%%%%%%%%%
\newcommand{\TiempoAveriaRefri}{4}              % Tiempo máximo de permanencia de alimento refrigerado en cámara averiada (en horas) 
\newcommand{\TiempoAveriaConge}{10}             % Tiempo máximo de permanencia de alimento congelado en cámara averiada (en horas)
\newcommand{\EspacioMinimoParedProducto}{30}    % Espacio entre tarima y pared (en centimetros)
\newcommand{\EspacioMinimoTechoProducto}{30}    % Espacio entre tarima y techo (en centimetros)
\newcommand{\VecesTempManual}{dos}              % Veces que se verifica la temperatura manualmente

%%%%%%%%%%%%%%%% PSA-2.3 | Política de aprobación de proveedores %%%%%%%%%%%%%%%%
\newcommand{\NotifAuditoriaAProveedor}{2 semanas}           % Plazo máximo para notificar a proveedor sobre auditoría.
\newcommand{\PlazoResultadoAProveedor}{2 semanas}           % Plazo máximo para notificar a proveedor sobre resultados de auditoría.
\newcommand{\PlazoPlanDeAccionAuditProveedor}{2 semanas}    % Plazo maximo para recibir plan de acción de parte del proveedor tras auditoría

%%%%%%%%%%%%%%%% PSA-2.15 | Almacenamiento de registros %%%%%%%%%%%%%%%%
\newcommand{\VigenciaAlmacRegistros}{2 años}
\newcommand{\VigenciaAlmacRegistrosElec}{3 años}

%%%%%%%%%%%%%%%%%%%%%%%%%%% Control de versiónes %%%%%%%%%%%%%%%%%%%%%%%%%%%%%%%%%%
\usepackage{changelog}

%%%%%%%%%%%%%%%%%%%%%%%%%%% ESPECIALES %%%%%%%%%%%%%%%%%%%%%%%%%%%%%%%%%%
\RequirePackage{expl3}    % \usepackage cannot be used before \documentclass
\ExplSyntaxOn             % Switch on expl3 syntax
\pdf_version_gset:n{1.7}  % Use provided expl3 function
\ExplSyntaxOff            % Switch off expl3 syntax
%%%%%%%%%%%%%%%%%%%%%%%%%%% PREAMBULO %%%%%%%%%%%%%%%%%%%%%%%%%%%%%%%%%%

\usepackage[sfdefault,t]{carlito}
% \usepackage{fontspec}
% 	\setmainfont{Carlito}

\usepackage{phonenumbers}

\usepackage{polyglossia}
	\setdefaultlanguage[variant=mexican,spanishoperators=all]{spanish}
	%\AtBeginDocument{\renewcommand\babelname{spanish}}
% \usepackage{tracklang}
% \SetCurrentTrackedDialect{es-MX}
% \CurrentTrackedDialectVariant{}

% \usepackage[record]{glossaries-extra}

% \GlsXtrLoadResources[
%   src={../glos.bib},% data in entries.bib
%   sort={en-GB},% sort according to 'en-GB' locale
% ]

\usepackage[toc=true,postpunc={;},showtargets=annoteleft,style=tree,xindy]{glossaries-extra}
	\makeglossaries
	\setabbreviationstyle[acronym]{long-short}
	\loadglsentries[main]{../glosario-def.tex}

\usepackage{imakeidx}
	\makeindex[columns=2]

\usepackage[framemethod=tikz]{mdframed}
% \usetikzlibrary{calc}

\makeatletter
	\newenvironment{nota}[1]{
	\protected@edef\@currentlabelname{#1}
	\protected@edef\@currentlabel{#1}
		\begin{mdframed}[
		innerlinewidth=0.5pt,
		innerleftmargin=10pt,
		innerrightmargin=10pt,
		innertopmargin = 10pt,
		innerbottommargin=10pt,
		skipabove=\dimexpr\topsep+\ht\strutbox\relax,
		roundcorner=5pt,
		frametitle={#1},
		frametitlerule=true,
		frametitlerulewidth=1pt]
	}{
	\end{mdframed}
	}
\makeatother

\usepackage[multiple]{footmisc}

% \usepackage{microtype}
\usepackage{tabularray}
\usepackage{graphicx}
\usepackage{hyperref}
\usepackage{cleveref}
\usepackage{array}
\usepackage{lastpage}
\usepackage{siunitx}
	\sisetup{detect-all}

\usepackage{color}

\usepackage{tabularray}
	\UseTblrLibrary{booktabs,siunitx}
	\definecolor{Gallery}{rgb}{0.929,0.929,0.929}
	\DefTblrTemplate{contfoot-text}{normal}{\textit{Continúa en la siguiente página}}
	\SetTblrTemplate{contfoot-text}{normal}
	\DefTblrTemplate{conthead-text}{normal}{\textit{(Continuación)}}
	\SetTblrTemplate{conthead-text}{normal}

\usepackage{xhfill}
%\usepackage{pdfpages}
%\usepackage{lipsum}

\usepackage[margin=2cm,bmargin=2cm,tmargin=5cm,headheight=97pt]{geometry}
%\addtolength{\topmargin}{-12pt}
	%\usepackage{geometry}
	%\setlength{\parindent}{0.95cm}
	%\renewcommand{\labelitemi}{\textemdash}

\usepackage{enumitem}
	\setlist[1]{labelindent=\parindent} % < Usually a good idea
	\setlist[itemize]{leftmargin=*}
	\setlist[itemize,1]{label=---}
	\setlist{noitemsep}

\usepackage{fancyhdr}
	\renewcommand{\headrulewidth}{0pt}
	\fancyhead[CE,CO,LE,LO,RE,RO]{}

	\newcommand{\Codigo}{}
	\newcommand{\FechaPub}{}
	\newcommand{\Titulo}{}
	\newcommand{\MayorVer}{}
	\newcommand{\MenorVer}{}
	\newcommand{\Edit}{\MayorVer.\MenorVer}

	\newcommand{\email}[1]{\href{mailto:#1}{#1}}

	\newcommand{\contacto}[2]{%
		\fbox{%
			\parbox[c]{0.5\textwidth}{
				\subsubsection*{#1}
				#2
			}%
		}
	}

	\newcommand{\cels}[1]{\qty{#1}{\celsius}}
\fancypagestyle{formato}{%
    \fancyhead[C]{%
	    \begin{tblr}{%	
            colspec = {|X[c,m]|X[c,m]|X[c,m]|X[c,m]|X[c,m]|X[c,m]|},
            }
            \hline
            \SetCell[c=2]{c} \includegraphics[height=1cm]{./RDF_Logo-eps-converted-to.pdf} & & \SetCell[c=3]{c} \large{Red de fríos S.A. de C.V.} & & & {\textbf{Código:}\\ \Codigo} \\
            \hline
            \SetCell[c=4]{c} {\Titulo} & & & & {\textbf{Fecha de publicación:}\\ \FechaPub} & {\textbf{Edición:}\\ \Edit} \\
            \hline
            \end{tblr}
	    }%
        \fancyfoot[CE,CO,LE,LO,RE,RO]{} %% clear out all footers
        \fancyfoot[C]{Página \thepage\ de \pageref{LastPage}}
        \fancyfoot[L]{\textbf{CONFIDENCIAL}}
        \fancyfoot[R]{\Codigo~v.\Edit}
}%

\setcounter{tocdepth}{4}
\setcounter{secnumdepth}{4}

\newcommand{\defglo}[1]{\item[\glsname{#1}] \glsdesc{#1}} % Define un comando para dar un item en un entorno de descripción para una palabra definida en el glosario

\includeonly{src/PSA-2.10.tex}

\begin{document}
\pagestyle{formato}
\include{src/PSA-ToC.tex}	
\thispagestyle{formato-PI}
\renewcommand{\MayorVer}{2}
\renewcommand{\MenorVer}{0}
\renewcommand{\Codigo}{PSA-1-PROG}
\renewcommand{\FechaPub}{2023--01}
%\renewcommand{\Edit}{2.1}
\renewcommand{\Titulo}{Inspección en la recepción de productos}

\section{\Titulo}\index{Inspección!en la recepción de productos}

\subsection{Objetivo}
\begin{itemize}
	\item \textbf{Establecer} un programa satisfactorio que asegure que la unidad de transporte se encuentra en condiciones adecuadas de mantenimiento, limpieza y temperatura apropiadas para el almacenamiento dentro de RDF.
\end{itemize}

\subsection{Alcance}
\begin{itemize}
	\item Este procedimiento aplica para todas las unidades con producto alimenticio recibidas en RDF;
	\item se extiende este documento al área de embarques, pero no se limita a otras áreas operativas.
\end{itemize}

\subsection{Términos y definiciones}


\begin{description}
	\defglo{producto}
	\defglo{producto-terminado}
	\defglo{alimento}
	\defglo{peligro-relacionado-con-la-inocuidad-de-los-alimentos}
\end{description}

\subsection{Documentos y/o normas relacionadas}
\begin{itemize}
	\item Programa de \glsfirst{BPD}.
\end{itemize}

\subsection{Procedimiento}\index{Procedimiento!inspección en la recepción de productos}

\subsubsection{Materiales}
\begin{itemize}
	\item Termómetro IR.
	\item Linterna.
\end{itemize}

\subsubsection{Precauciones de seguridad}

\begin{itemize}
	\item Usar \emph{uniforme completo} %TODO: Definición de uniforme completo.
\end{itemize}

\subsubsection{Instrucciones}

\paragraph{Inspección física del producto}\index{Instrucción!inspección fisica de productos}
\begin{enumerate}
	\item A la llegada del producto, se verifica la documentación del mismo. Si se descubre que el documento no cumple con los requisitos a la llegada, no se puede descargar el producto y el Supervisor de Almacén debe ser contactado inmediatamente para obtener instrucciones adicionales:
	      \begin{enumerate}
		      \item Documentos TIF \emph{(En caso de ser productos cárnicos);}
		      \item Listado de productos con cantidades;
		      \item Condiciones de almacenamiento.
	      \end{enumerate}
	\item Verificar la temperatura programada de la unidad con el producto entrante;
	\item Se revisa el número de sello que deberá coincidir contra la guía para asegurarse que cuadran. Si se descubre que el sello está roto a la llegada, no se puede descargar el producto y el Supervisor de Almacén debe ser contactado inmediatamente para obtener instrucciones adicionales;
	\item Se enrama la unidad para descargar el producto;
	\item La condición física del producto (caja, cubetas, etc.) debe ser revisado, cualquier daño debe ser documentado en el registro de entrada;
	\item Se verifica unidad interna. Cualquier evidencia de daño, suciedad, o plagas deben ser inmediatamente documento e informado al Supervisor de Almacén para obtener instrucciones adicionales antes de la descarga:
	      \begin{itemize}
		      \item Limpieza (suciedad, escombros, basura, etc.);
		      \item Evidencia de cualquier actividad de insectos (los excrementos, gomas de mascar, etc.);
		      \item Daños físicos (bolsas rotas, cajas dañadas, fugas, etc.);
		      \item Daño físico de la unidad (transporte);
		      \item Evidencia de violación de los materiales de empaque (caja o producto abierto);
		      \item presencia visible de agentes químicos.
	      \end{itemize}
	\item Verificar la temperatura del producto.\footnote{Todo producto refrigerado o congelado recibido en la instalación debe ser revisado para asegurar que fue enviado de origen a la temperatura adecuada, verificando la documentación correspondiente.}
	      \begin{itemize}
		      \item \textbf{Rangos de temperatura:}
		            \begin{itemize}
			            \item Los productos refrigerados DEBEN recibirse en un rango de temperatura de \qtyrange{2.7}{4.4}{\degreeCelsius}
			            \item Los productos congelados DEBEN recibirse a una temperatura \qty{<=-18}{\degreeCelsius}
		            \end{itemize}
		      \item De \textbf{no cumplirse} lo anterior se suspende el recibo por parte de personal de embarques y deberá notificarse inmediatamente al Supervisor del Almacén quien reportara al Gerente de Operaciones, quien a su vez informara de la situación al cliente y solicitara la ruta de acción.
	      \end{itemize}
	      % \begin{itemize}
	      % 	\item Se considerarán las especificaciones del producto para determinar si se aceptan o se rechazan los cargamentos.
	      % 	\item se comprobará que se cumple con la \gls{conformidad} de la temperatura 
	      % 	\item si es conforme la temperatura, se procederá a almacenar el producto en la camara designada según sus especificaciones.
	      % 	\begin{itemize}
	      % 	\item si el \gls{alimento} no requiere de condiciones de almacenamiento especiales\footnote{diferentes a las especificaciones genéricas de almacenamiento (ver \cref{esp:generica}).} se almacenará en la cámara adecuada. 
	      % 	\item si el \gls{alimento} requiere de condiciones de almacenamiento especiales y el cliente alquiló una camara exclusiva para su producto, entonces se almacenará ahí;
	      % 	\item de no ser así, se almacenará el \gls{alimento} en aquella camara compartida que coincida con el rango de las especificaciones del producto.
	      % 	\end{itemize}
	    %   \end{itemize}
	\item Todos los documentos de envío \emph{(números de lote, fechas, número de orden, etc.)} deberán ser registrados en el formulario de verificación de \gls{registro} y en la orden de entrada;
	\item Se almacena el \gls{alimento}.
\end{enumerate}

\paragraph{Identificación del producto}

\begin{enumerate}
	\item Al momento del envío de productos por parte de los proveedores se genera el documento de STOCK TRANFER el cual de manera inmediata es enviado a \gls{RDF} (control de Inventarios).
	\item Mesa de control recibe el documento \textit{STOCK TRANSFER} y genera el \emph{identificador de control de interno\footnote{Este es nombre que recibe el identificador de tarima.}} antes de la llegada del producto.
	\item La llegada del embarque a \gls{RDF} se tiene programada en fecha.
	\item Al momento de la llegada del embarque a \gls{RDF}, mesa de control coloca de manera inmediata a la descarga del producto la etiqueta identificador de control interno, al mismo tiempo que se realizará la verificación de cantidades de producto recibida
	\item El montacarguista debe trasladar el producto directamente del transporte a la cámara
\end{enumerate}

\paragraph{Reglamento de permanencia de producto en andén de carga}

\begin{itemize}
	\item Para \emph{\textbf{producto congelado:}} No más de \qty{1}{\hour};
	\item Para \emph{\textbf{productos refrigerados:}} No más de \qty{1}{\hour}.
\end{itemize}


\subsubsection{Responsables de la actividad}

\begin{itemize}
	\item \textbf{Ejecutado} por el personal de operaciones;
	\item \textbf{Monitoreado} por personal de aseguramiento de calidad;
	\item \textbf{Verificado} por personal de gerencia.
\end{itemize}

\subsubsection{Acciones preventivas}

\begin{itemize}
	\item Se llevará a cabo una inspección, registrando el indicador y medida correspondiente;
	\item si la desviación se repite, se capacitará al personal involucrado;
	\item Si después de la capacitación se repite una desviación, se tomarán acciones correctivas (\cref{sec:2.1:acc}).
\end{itemize}

\subsubsection{Acciones correctivas}
\label{sec:2.1:acc}
\begin{itemize}
	\item En caso contrario, la tarea se deberá volver a realizar como indica el procedimiento;
	\item En caso de no-conformidad, reportar en el formato de acciones correctivas.
\end{itemize}

\subsubsection{Frecuencia}

Cada recepción de \gls{alimento}

\begin{changelog}[simple, sectioncmd=\subsection*,label=changelog-1.2]
	\begin{version}[v=2.1, date=2023--01, author=Pablo E. Alanis]
		%\fixed
		\item Cambio de formato;
		\item Cambios en la serialización de versiones;
		\item Cambio de identificado, de PRO-OP-001 a OP-BPD-ESP-1.
		%\added
		\item Separación entre PPR y PPRO.
	\end{version}
\end{changelog}	% REVISADO
\thispagestyle{formato-PI}
\renewcommand{\MayorVer}{2}
\renewcommand{\MenorVer}{0}
\renewcommand{\Codigo}{PSA-2-PRO} 
\renewcommand{\FechaPub}{2023--01}
%\renewcommand{\Edit}{2.1}
\renewcommand{\Titulo}{Procedimiento de lotificación de productos para su almacenamiento}

\section{\Titulo}\index{Procedimiento!lotificación de productos}

% \section{Procedimiento de lotificación de productos para su almacenamiento}

\subsection{Objetivo}

\begin{itemize}
	\item \textbf{Asegurar} que las tarimas con producto a almacenar sean identificadas con la información requerida por el cliente.
\end{itemize}

\subsection{Alcance}

\begin{itemize}
	\item Este documento es aplicable a todas las tarimas con producto que almacenadas en RDF.
\end{itemize}

\subsection{Términos y definiciones}
%TODO definiciones

\subsection{Procedimiento}

\subsubsection{Fundamento}

Los establecimientos y equipos dedicados al proceso de alimentos para consumo humano deben mantener los registros y el control de los productos y materiales de empaque por medio de la notificación de los mismos, con el cual es posible la realización de un sistema de rastreo desde su producción hasta su distribución.

\subsection{Materiales}

\begin{itemize}
	\item Factura de entrada
\end{itemize}

\subsection{Instrucciones}

\begin{itemize}
	\item Cada vez que se reciba en el área de almacén, productos o materiales de empaque, se debe registrar el lote que el cliente designado a este producto.
	\item El personal de recepción debe de verificar que el lote de los registros coincidan con el lote marcado en las tarimas del producto;
	\item Estos se darán de alta en el sistema de la empresa con el lote asignado por el proveedor o cliente.
	\begin{itemize}
		\item Si el producto no tiene lote o este no coincide se dará aviso al Supervisor de almacén (el lote asignado será asignado con el número de entrada)
	\end{itemize}
	\item Las tarimas que contienen el producto son marcadas con \emph{etiqueta de recepción de producto,} en ella se coloca la fecha de recepción, nombre del cliente, nombre del producto, fecha de caducidad, orden de compra o lotificación, numero de entrada, cantidad por tarima.
\end{itemize}

\subsection{Responsables de la actividad}

\begin{itemize}
	\item \textbf{Ejecutado} por personal de operaciones;
	\item \textbf{Monitoreado} por personal de calidad;
	\item \textbf{Verificado} por personal de gerencia.
\end{itemize}

\subsection{Acciones preventivas}

\begin{itemize}
	\item Se llevara a cabo una inspección. Registrando el indicador y medida correspondiente;
	\item Si la desviación se repite frecuentemente se dará curso de capacitación al personal de almacén y limpieza para que realicen eficientemente su trabajo;
	\item Si después de haber capacitado al personal de almacén se siguen presentando desviaciones por causas injustificadas, se tomaran acciones más enérgicas con el personal por incumplimiento con sus deberes.
\end{itemize}

\subsection{Acciones correctivas}

\begin{itemize}
	\item En caso contrario la tarea se deberá volver a realizar como se indica en el procedimiento.
	\item En caso de no conformidad reportar en \RAC.
\end{itemize}

\subsection{Frecuencia}

\begin{itemize}
	\item Cada recepción de Producto
	\item Cada entrega de producto.
\end{itemize}

\begin{changelog}[simple, sectioncmd=\subsection*,label=changelog-2.2]
	
	\begin{version}[v=2.1, date=2023--01, author=Pablo E. Alanis]
			\item Cambio de formato;
			\item Cambios en la serialización de versiones;
	\end{version}

	\begin{version}[v=1.3, date=2022--05, author=Alonso M.]
		\item cambio de fecha;
		\item cambio de código.
	\end{version}

	\shortversion{v=1.2, date=2021--05, changes=No hubo cambios}
\end{changelog}	% REVISADO
\renewcommand{\MayorVer}{2}
\renewcommand{\MenorVer}{1}
\renewcommand{\Codigo}{PSA-1-PROG}
\renewcommand{\FechaPub}{2023--01}
%\renewcommand{\Edit}{2.1}
\renewcommand{\Titulo}{Inspección de producto durante su almacenamiento}

\section{\Titulo}
\index{Inspección!de productos durante su almacenamiento}

\subsection{Objetivo}

Establecer un programa de inspección que asegure que las condiciones de almacenamiento se encuentren en condiciones óptimas de acuerdo a un buen mantenimiento de instalaciones, limpieza y temperatura apropiada para la conservación de productos alimenticios.

\subsection{Alcance}

A todos las áreas de almacenamiento de producto

\subsection{Terminología y definiciones}

N/A

\subsection{Documentos y/o normas relacionados}

\begin{itemize}
	\item 1.0 - BPD - Indice|Programa de buenas prácticas de manufactura
\end{itemize}

\subsection{Procedimiento}

\subsubsection{Materiales}

\begin{itemize}
	\item N/A
\end{itemize}

\subsubsection{Precauciones de Seguridad}

\begin{itemize}
	\item Usar uniforme completo.
\end{itemize}

\subsection{Instrucciones}

\subsubsection{Temperatura de Almacenaje de productos Refrigerados y Congelados.}

\begin{enumerate}
	\item Para asegurar que todo producto refrigerado o congelado se mantenga a la temperatura adecuada durante el almacenamiento, todos los productos se deben colocar directamente después de ser recibido en la cámara de refrigeración o congelación según la asignación, una vez que se descarguen.
	\item Las cámaras de refrigeración y de congelación están equipadas con un termómetro calibrado (RACK), en el cual se puede verificar la temperatura de cada cámara.
	\item Las lecturas de temperaturas deben ser documentadas tres veces durante el turno los días laborados. El Supervisor de mantenimiento es responsable de la verificación de temperaturas.
	\item Las puertas del andén de embarques deben permanecer cerradas en todo momento a menos que la unidad este siendo utilizada para carga y descarga de productos.
	\item Las puertas del andén deben de ser abiertas cuando la unidad de transporte ya se encuentre colocada en la rampa, esto con el fin de evitar fuga de temperatura y / o entrada de polvo o plaga.
	\item Cualquier daño a las cámaras de refrigeración o congelación (techo, paredes, puertas, cortinas, etc.) debe ser reparadas en el menor tiempo posible. El Supervisor de mantenimiento es responsable de garantizar que todo el equipo este en buen estado y funcionando adecuadamente.
\end{enumerate}

\subsubsection{Almacenaje y manejo de Productos Secos / Ambiente, Refrigerados y Congelados}

\begin{enumerate}
	\item Para asegurarse de que los productos ambiente, refrigerados, y congelados son almacenados y protegidos de daños físicos, químicos, biológicos (contaminación), todos los productos deben ser almacenados 30 centímetros desde todas las paredes interiores (si el diseño del almacén lo permite), sobre atados limpios, 30 centímetros de las superficies del techo, así como lejos de todas las luces superiores y nunca almacenados directamente en contacto con los pisos (debe de estar sobre tarima)
	\item Todos los productos deben guardarse en su empaque secundario y etiquetados con las fecha de recepción necesarias.
	\item Todos los productos deben ser rotados de acuerdo en el principio de “First-in-First-Out” \textbf{(FIFO)} o primeras entradas primero a vencer (O según instrucciones de cada cliente).
	\item Un área de “retención” debe haber en el área de almacenamiento para almacenar cualquier artículo recuperado, muestras de productos, producto dañado.
	\item Todos los productos deben mantenerse libres de polvo y suciedad en todo momento.
	\item Cualquier o todos los productos químicos almacenados en la instalación deben estar completamente separados de todos los envases y productos alimenticios – SIN EXCEPCIONES. Todos los productos químicos ubicado en el centro debe tener una hoja de identificación (NO es el caso para este almacén).
\end{enumerate}

\subsubsection{Control y mantenimiento de temperatura}

\begin{enumerate}
	\item Es imperativo que las cámaras en refrigerado y congelado en nuestras instalaciones se mantengan a la temperatura adecuada en todo momento. Esto incluye horas regulares, después de horas regulares, fines de semana y días festivos.
	\item Existe un procedimiento para el seguimiento de temperaturas del refrigerador y del congelador durante el horario normal \%\%CUAL?\%\%las temperaturas son supervisadas y documentadas al día en tres tiempos (Inicio de turno, medio turno y fin de turno) de Lunes a Sábado. El Supervisor de Mantenimiento es inmediatamente informado de cualquier irregularidad detectada en la temperatura (incluyendo horas después del trabajo, fines de semana y días festivos), de modo que las acciones correctivas se puedan tomar.
\end{enumerate}


\subsubsection{Plan alterno en caso de avería de equipos de enfriamiento con productos}

Al fin de garantizar que todos los productos refrigerados y congelados estén protegidos contra cualquier peligro microbiológico asociado con la perdida de temperatura, las cámaras con refrigeración o congelación que presenten una falla y que no puedan ser reparadas en menos en el caso de refrigeración de 4 horas y en congelación de 10 hrs (siempre y cuando las puertas de cámara se encuentren cerradas y el producto mantenga la temperatura indicada), el producto DEBE ser trasladado inmediatamente a otra cámara (previamente enfriada).

\subsection{Responsables de la actividad}

\begin{itemize}
	\item \textbf{Ejecutado} por personal de operaciones y mantenimiento;
	\item \textbf{Monitoreado} por personal de calidad;
	\item \textbf{Verificado} por personal de gerencia.
\end{itemize}

\subsection{Acciones preventivas}

\begin{itemize}
	\item Se llevara a cabo una inspección. Registrando el indicador y medida correspondiente
	\item Si la desviación se repite frecuentemente se dará curso de capacitación al personal de almacén y limpieza para que realicen eficientemente su trabajo.
	\item Si después de haber capacitado al personal de almacén se siguen presentando desviaciones por causas injustificadas, se tomaran acciones más enérgicas con el personal por incumplimiento con sus deberes.
\end{itemize}

\subsection{Acciones correctivas}

\begin{itemize}
	\item En caso contrario la tarea se deberá volver a realizar como se indica en el procedimiento.
	\item En caso de no conformidad reportar en formato de acciones correctivas F-OP-40.xls
\end{itemize}

\subsection{Frecuencia}

\begin{itemize}
	\item \textbf{Verificación de temperatura:} tres veces al día.
\end{itemize}

\subsection{Historial de modificaciones}

\begin{itemize}
	\item \textbf{Cuarta edición:} cambio de fecha de 03 de noviembre 2017 a 28 de enero 2019, se realizó cambio de código de OA-P-IDA a PR-008 y no se realizaron cambios de la revisión 003 a la 004.
	\item \textbf{Quinta edición:} febrero 2020 se hizo cambio de formato y cambio de código de PR-008 a PRO-OP-005. Y no se realizaron cambios de la revisión 04 a la 05.
	\item \textbf{Sexta edición:} febrero 2021 no se realizaron cambios de la revisión 05 a la 06.
	\item \textbf{Séptima edición:} febrero 2022 no se realizaron cambios de la revisión 06 a la 07.
	\item \textbf{Octava edición:} 2023--01: cambios en serialización y en formato.
\end{itemize}
	% REVISADO
\renewcommand{\MayorVer}{2}
\renewcommand{\MenorVer}{1}
\renewcommand{\Codigo}{PSA-1-PROG}
\renewcommand{\FechaPub}{2023--01}
%\renewcommand{\Edit}{2.1}
\renewcommand{\Titulo}{PEPS o VENCER}

\section{\Titulo}
\index{Procedimiento!PEPS o VENCER}

% \section{PEPS o VENCER}

\subsection{Objetivo}
Asegurar que los productos, serán en sus totalidades lotificadas e identificados y verificados sus movimientos por medio de conteos programados de acuerdo a lo establecido en este procedimiento.

\subsection{Alcance}
Todo producto que ingrese a las instalaciones del almacén.

\subsection{Terminología y definiciones}
Inventario de suministros y materiales. La percepción de inventario asegura que la compañía tiene materiales a la mano para hacer productos y que los fondos no son desperdiciados en materiales innecesarios. Una cuenta precisa de inventario también permite que las compañías controlen y ordenen suficientes, materiales para la demanda de los consumidores.

\subsection{Documentos y/o normas relacionadas}
\begin{itemize}
	\item Registros de recepción
\end{itemize}

\subsection{Procedimiento}

\subsubsection{Materiales}

\begin{itemize}
	\item N/A
\end{itemize}

\subsubsection{Precauciones de seguridad}

\begin{itemize}
	\item Usar el uniforme completo.
\end{itemize}

\subsubsection{Instrucciones}

\paragraph{Primeras entradas, primeras salidas (PEPS) o VENCER}

\begin{enumerate}
	\item Hacer la revisión correspondiente en cada recepción de producto, la revisión abarca todos los identificadores únicos del envió (número de lote, fechas, cantidades) deben estar documentados.
\end{enumerate}


\begin{enumerate}
	\item Se etiqueta por tarima y por lote cada producto;
	\item La etiqueta de \emph{"Información de material"} contiene los siguientes datos:
	\begin{itemize}
		\item Nombre de producto
		\item Nombre del cliente
		\item Fecha de recepción
		\item Identificador de entrada
		\item Lote
		\item Cantidad
	\end{itemize}
\end{enumerate}

\subparagraph{En operación}

\begin{itemize}
	\item Cada insumo con fecha de caducidad más larga (Producto nuevo), se acomodara en la parte superior trasera de cada rack y los insumos que ya estaban presentes se recorren a la parte inferior frontal.
	\item Al surtir los insumos, el personal de almacén deberá entregar los que tengan la fecha de recepción más antigua o el producto que le cliente asigne para su distribución.
\end{itemize}

\subparagraph{En sistema}

\begin{itemize}
	\item En el sistema de inventarios se cargan las entradas de producto en general al momento de su llegada, en este sistema administra el orden de salida de los mismos y permite dar de baja únicamente en el orden de primeras entradas primeras salidas.
	\item El cliente también pudiera dar instrucciones de cual producto dar salida de forma primaria por estrategia comercial.
	\item En ambos casos el sistema de manera automática da de baja del sistema el producto que físicamente se da salida, actualizando las existencias.
\end{itemize}

\subsection{Responsables de la actividad}

\begin{itemize}
	\item \textbf{Ejecutado} por personal de operaciones;
	\item \textbf{Monitoreado} por personal de calidad;
	\item \textbf{Verificado} por personal de gerencia.
\end{itemize}

\subsection{Acciones preventivas}

\begin{itemize}
	\item Se llevara a cabo una inspección. Registrando el indicador y medida correspondiente
	\item Si la desviación se repite frecuentemente se dará curso de capacitación al personal de almacén y limpieza para que realicen eficientemente su trabajo.
	\item Si después de haber capacitado al personal de almacén se siguen presentando desviaciones por causas injustificadas, se tomaran acciones más enérgicas con el personal por incumplimiento con sus deberes.
\end{itemize}

\subsection{Acciones correctivas}

\begin{itemize}
	\item En caso contrario la tarea se deberá volver a realizar como se indica en el procedimiento.
	\item En caso de no conformidad reportar en formato de acciones correctivas F-OP-40.xls
\end{itemize}

\subsection{Frecuencia}

\begin{itemize}
	\item Diaria.
\end{itemize}

\subsection{Historial de modificaciones}

\begin{itemize}
	\item \textbf{Cuarta edición:} cambio de fecha de 03 de noviembre 2017 a 28 de enero 2019, se realizó cambio en el código de CA-P-PEPV a PR-009. Y no se realizaron modificaciones de la revisión 03 a la 04
	\item \textbf{Quinta edición:} febrero 2020 se hizo cambio de forma y cambio de código de PR-009 a PRO-OP-006. Y no se realizaron modificaciones de la revisión 04 a la 05.
	\item \textbf{Sexta edición:} febrero 2021 no se realizaron modificaciones de la revisión 05 a la 06.
	\item \textbf{Séptima edición:} febrero 2022 no se realizaron cambios de la revisión 06 a la 07.
\end{itemize}

\subsection{Listado de distribución}

N/A

\subsection{Anexos}

N/A
	% REVISADO
%\thispagestyle{formato-PI}
\renewcommand{\TipoID}{CI}
\section{Evaluación de proveedores}
\renewcommand{\Codigo}{\Prog--\thesection--\TipoID}
Estimado proveedor de Red de Fríos, S.A. de C.V., el presente formato de evaluación tiene por objetivo conocer los procesos involucrados en el surtimiento del insumo que nos provee y forma parte de nuestro Sistema de Calidad de los Alimentos. Con la información que nos proporcione realizaremos una evaluación y dependiendo de los resultados le calificaremos como un \emph{proveedor aprobado, aprobado condicionado o rechazado,} de los resultados de esta evaluación probablemente se desprendan acciones que le solicitaremos sean implementadas para mejorar la calidad del servicio que nos brinda. Agradecemos de antemano su apertura y las facilidades que nos brinda para realizar esta evaluación.

\subsection{Datos generales del proveedor}


\subsection{Cumplimiento de especificaciones}

Se han establecido con el proveedor las especificaciones del insumo requerido y el proveedor cumple con estas especificaciones en el 100\% de las entregas a RDF.

\subsection{Cumplimiento de fechas y cantidades comprometidas}

El proveedor \emph{cumple con las fechas comprometidas de entrega} de insumos así como con las cantidades requeridas en la orden de compra.

\subsection{Capacidad para surtir cantidades requeridas}

El proveedor cuenta con la \emph{capacidad para surtir las necesidades} de volumen de producto que RDF le solicita.

\subsection{Capacidad para surtir cantidades extraordinarias}

El proveedor tiene la capacidad de responder en tiempo razonable a demandas extraordinarias de volumen del producto que nos provee.

\subsection{Condiciones de producción y almacenamiento}

El proveedor mantiene un ambiente de producción y/o almacenamiento tal que no se observan condiciones que representen un riesgo físico, químico o biológico a la inocuidad de los alimentos almacenados en nuestras instalaciones.

\subsection{Presencia de plagas}

Durante la visita de las instalaciones no se observa infestación por algún tipo de plaga y preferentemente el proveedor cuenta con un sistema de control de plagas.

\subsection{Mantenimiento de equipos e instalaciones}

Durante la visita de las instalaciones se observan las instalaciones y maquinaria bien mantenidas de tal forma que se asegura su buen funcionamiento y por lo tanto el abasto del insumo.

Agradecemos nuevamente las facilidades brindadas para realizar esta evaluación, en los próximos días recibirá el resultado final. En este mismo reporte recibirá toda aquella solicitud de acción a implementar en sus instalaciones que haya sido mencionada durante el recorrido; le pedimos responder por escrito a estas solicitudes mencionando el responsable de implementarlas y la fecha compromiso para su finalización.


Nombre y firma del proveedor


Nombre y firma de red de fríos
	% TODO: VERIFICAR
\thispagestyle{formato-PI}
\renewcommand{\MayorVer}{2}
\renewcommand{\MenorVer}{0}
\renewcommand{\Codigo}{PSA-6-CE}
\renewcommand{\FechaPub}{2023--01}
\renewcommand{\TipoID}{PRO}
\renewcommand{\Titulo}{Procedimiento para solicitud de citas}

\section{\Titulo}\index{Procedimiento!solicitud de citas}
\renewcommand{\Codigo}{\Prog--\thesection--\TipoID}
Estimado cliente, a continuación le presentamos el proceso y lineamientos para la solicitud de citas tanto ingresos como salidas de material.

Le solicitamos que las citas sean solicitadas mínimo con 24 horas de anticipación mediante correo electrónico al equipo de mesa de control en los siguientes horarios:

\begin{itemize}
	\item \textbf{Lunes a viernes de 08:00 am a 16:00 pm}
	\item \textbf{Sábados de 08:00 am a 11:30 am.}
\end{itemize}

Quedando las solicitudes de la siguiente forma:

\definecolor{Gallery}{rgb}{0.929,0.929,0.929}
\begin{longtblr}[
	label = citas:solucitud,
	entry = Procedimiento de solicitud de citas.,
	caption = Horarios para solicitud de citas.
	]{%
	width = \linewidth,
	colspec = {Q[125]Q[129]Q[108]Q[577]},
	cells = {c},
	row{even} = {Gallery}
	}
	\toprule
	Día de solicitud & Hora límite para solicitud & Día de carga & Observación                                                                         \\
	\midrule
	Lunes            & 15:30 pm                   & martes       & La cita debe realizarse antes de las 15:30 pm para que se asigne cita el Martes.    \\
	Martes           & 15:30 pm                   & miércoles    & La cita debe realizarse antes de las 15:30 pm para que se asigne cita el miércoles. \\
	Miércoles        & 15:30 pm                   & jueves       & La cita debe realizarse antes de las 15:30 pm para que se asigne cita el jueves.    \\
	Jueves           & 15:30 pm                   & viernes      & La cita debe realizarse antes de las 15:30 pm para que se asigne cita el viernes.   \\
	Viernes          & 15:30 pm                   & sábado       & La cita debe realizarse antes de las 15:30 pm para que se asigne cita el sábado.    \\
	Sábado           & 11:30 am                   & lunes        & La cita debe realizarse antes de las 11:30 am para que se asigne cita el lunes.     \\
	\bottomrule
\end{longtblr}

\begin{itemize}
	\item \textbf{Los pedidos enviados fuera de estos horarios quedarán a disposición de andén;\footnote{En espera de que tengamos una ventana disponible.}}
	\item Es \emph{requisito indispensable} presentar el correo de confirmación impreso, que especifica el número de pedido a surtir y la hora asignada a su cita;
	\item Es importante que en la solicitud de la cita nos indiquen el \emph{nombre completo de la persona que realizará la recolección o entrega,} así como los datos de la unidad:
	      \begin{itemize}
		      \item Línea de transporte;
		      \item placas;
		      \item número de tractor y placas;
		      \item número de contenedor.
	      \end{itemize}
	\item Nuestro horario de atención sin cargo de tiempo extra es de 8:00 a 17:00 de lunes a viernes. El camión a carga o descarga debe presentarse una hora antes de las 17:00 para que no se extienda su servicio; lo mismo aplica para los sábados de 8:00 a 13:00; la última cita se otorga a las 16:00 de lunes a viernes y los sábados a las 12:00.
\end{itemize}

\newpage

\subsection{Contactos Red de Fríos}

\begin{contact}[Mesa de control] \label{contact:MesaDeControl}
	Magdalena Aguilar Robledo\\
	Teléfono: \textbf{(81) 8351-6685 ext. 105}\\
	\email{mesacontrolrayon@reddefrios.com}
\end{contact}

\begin{contact}[Jefe de almacén] \label{contact:JefeDeAlmacen}
	Luis Ángel Escareño Aguirre\\
	Teléfono: \textbf{(81) 8351-6685 ext. 105}\\
	Celular: \textbf{811-390-7099}\\
	\email{lescareno@reddefrios.com}
\end{contact}

\begin{contact}[Gerente de operaciones] \label{contact:Gerente}
	Gerardo Ruiz Torres, QFB\\
	Teléfono: \textbf{(81) 8351-6685 ext. 102}\\
	Celular: \textbf{818-029-5609}\\
	\email{gruiz@reddefrios.com}
\end{contact}

\begin{contact}[Médico TIF] \label{contact:MedicoTIF}
	M.V.Z. Rocío Luna\\
	Teléfono: \textbf{(81) 8351-6685 ext. 103}\\
	\email{luna\_mvz\_ceu@live.com.mx}
\end{contact}

Es necesario indicar:
\begin{enumerate}
	\item Nombre del producto a guardar o retirar;
	\item factura que lo ampara;
	\item código con el que usted lo maneja\footnote{Se dará de alta el mismo en nuestro sistema};
	\item lote;
	\item fecha de caducidad;
	\item cantidad de tarimas y/o cajas que se guardarán o retirarán;
	\item especificaciones de temperatura de almacenamiento.
	\item alérgenos presentes en el \gls{alimento} (ver \cref{nota:Alergenos})
\end{enumerate}

\begin{note}[Listado de alérgenos según la NOM-051-SCFI/SSA1-2010] \label{nota:Alergenos}
	Es de suma importancia que nos indique si el producto a resguardar es un Alérgeno o contiene algún ingrediente Alérgeno en su composición.
	\begin{itemize}
		\item Leche
		\item Trigo
		\item Frutas Secas
		\item Huevos
		\item Soya y derivados
		\item Cacahuates
		\item Pescados
		\item Crustáceos
		\item sulfitos en concentración \qty{\geq 10}{\milli\gram\per\kilo\gram}.
	\end{itemize}
\end{note}

Para carga o descarga requerimos contar con los datos de la unidad, placas, nombre del chofer y línea de transporte.

Para ingresos de mercancía es necesario presentar la siguiente documentación:

\begin{itemize}
	\item Factura, remisión y pedimento (para importados) del producto a recibir.
	\item Certificado de calidad de los productos a recibir.
	\item Certificado de fumigación de la unidad de transporte.
\end{itemize}

En caso de ser \emph{producto cárnico} es necesario presentar \emph{Aviso de Movilización TIF} con destino a nuestro almacén \emph{TIF 278.}

Para cuestión de \emph{Pescados y Mariscos} es necesario presentar la \emph{Guía de Pesca} del producto con destino a nuestro almacén TIF 278.

Para la \emph{salida de producto TIF} se requiere se nos informen los datos de la unidad que lo cargará:
\begin{itemize}
	\item Línea de Transporte;
	\item chofer;
	\item placas;
	\item marca del contenedor o caja;
	\item número de destino TIF.
\end{itemize}

\begin{note}[Datos de aviso TIF]\label{note:DatosAvisoTIF}
	TIF 278\\
	Red de Fríos S.A. de C.V.\\
	Av. I. López Rayón \# 2810.\\
	Col. Bella Vista.\\
	C.P. 64410, Monterrey N.L., México
\end{note}

\begin{changelog}[title=Registro de cambios,simple, sectioncmd=\subsection*,label=changelog-\thesection-\MayorVer.\MenorVer]
	\begin{version}[v=\MayorVer.\MenorVer, date=2023--01, author=Pablo E. Alanis]
		\item Cambio de formato;
		\item los contactos se agregaron a recuadros;
		\item cambio de código;
		\item se agregó recuadro sobre alérgenos.
	\end{version}

	\begin{version}[v=1.7, date=2022--05, author=Alonso M.]
		\item No se realizaron modificaciones de la revisión 05 a la 06.
	\end{version}
\end{changelog}	% REVISADO
\thispagestyle{formato-PI}
\renewcommand{\MayorVer}{2}
\renewcommand{\MenorVer}{0}
 %TODO
\renewcommand{\FechaPub}{2023--01}
\renewcommand{\TipoID}{POL}
\renewcommand{\Titulo}{Política de devoluciones}

\section{\Titulo}\index{Política!devoluciones, de}
\renewcommand{\Codigo}{\Prog--\thesection--\TipoID}
% \section{Política de devoluciones}

\subsection{Objetivo}

Establecer que no se realizan maniobras de recepción de devoluciones por problemas de calidad, inocuidad o mal estado de las condiciones físicas del producto como actividades comunes dentro de \gls{RDF}.

\subsection{Alcance}

Esta política es aplicable para los clientes y para los empleados de \gls{RDF}.

\subsection{Términos y definiciones}

\begin{description}
	\defglo{peligro-relacionado-con-la-inocuidad-de-los-alimentos} 
	\defglo{cliente}
	\defglo{cadena-alimentaria}
\end{description}

\subsection{Documentos y/o normas relacionadas}
\begin{itemize}
	\item Programa de \gls{BPD}.
\end{itemize}

\subsection{Política de devoluciones}
Es política de \gls{RDF} \textbf{NO} aceptar devoluciones como una actividad frecuente dentro de sus operaciones cotidianas, ya que esta práctica puede presentar un \gls{peligro-relacionado-con-la-inocuidad-de-los-alimentos}, a su vez esto permite resguardar la inocuidad de los otros \glspl{alimento} almacenados en este establecimiento.

\subsection{Historial de modificaciones}

\begin{changelog}[title=Registro de cambios,simple, sectioncmd=\subsection*,label=changelog-\thesection-\MayorVer.\MenorVer]

	\begin{version}[v=2.1, date=2023--01, author=Pablo E. Alanis]
		\item Cambio de formato;
		\item Cambios en la política de no devoluciones: se empleó el término \emph{peligro relacionado con la inocuidad de los alimentos} y {alimentos}, según los términos establecidos por la ISO 22000:2018;
	\end{version}

	\begin{version}[v=1.6, date=2022-02, author=Alonso M.]
		\item no se realizaron cambios de la revisión 05 a la 06.
	\end{version}
\end{changelog}	% REVISADO
\thispagestyle{formato-PI}
\renewcommand{\MayorVer}{2}
\renewcommand{\MenorVer}{1}
\renewcommand{\Codigo}{PSA-1-PROG} %TODO
\renewcommand{\FechaPub}{2023--01}
%\renewcommand{\Edit}{2.1}
\renewcommand{\Titulo}{Política de aprobación de proveedores}

\section{\Titulo}\index{Política!aprobación de proveedores, de}

% \section{Política de aprobación de proveedores}

\subsection{Objetivo}

Establecer el procedimiento para \emph{evaluar, seleccionar y monitorear} a los proveedores de insumos, para garantizar la capacidad de cada proveedor en lo referente la calidad del servicio proporcionado y de sus recursos brindados.

\subsection{Alcance}

Este procedimiento aplica para todos los empleados de \GLS{RDF} involucrados en la evaluación, selección y monitoreo de proveedores de insumos.

\subsection{Términos y definiciones}
\begin{description}
    \defglo{requisito}
    \defglo{proveedor}
    \defglo{proveedor-externo}
    \glspl{especificacion}
\end{description}
\subsection{Documentos y/o normas relacionadas}

\begin{itemize}
    \item Lista de proveedores autorizados
    \item Formato Evaluación de Proveedores
\end{itemize}

\subsection{Procedimiento}
\begin{enumerate}
    \item La selección de un nuevo proveedor se ejecuta bajo una evaluación en equipo; este Equipo está formado por miembros de Gerencia de Operaciones, Calidad, Almacenes, Embarques, Mantenimiento y Compras. Todos los miembros del equipo \textit{(vide infra)} son responsables de proveer retroalimentación para evaluar y determinar las capacidades del proveedor.
    \item La \Gls{LPA} es creada en conjunto por los miembros del equipo de Evaluación de Proveedores (ver \cref{sec:listaDeProveedores}).
    \item El comité de evaluación de proveedores es el encargado de programar las auditorías de evaluación anual a proveedores. Durante estas auditorías se verificará que el proveedor de insumos tiene la capacidad de proveer el insumo requerido y que éste es fabricado bajo condiciones tales que no representa un riesgo para la sanidad de los alimentos.
    \item Se debe enviar una notificación de la auditoría al proveedor al menos con \NotifAuditoriaAProveedor~de anticipación.
    \item Los resultados de las auditorías se envían al proveedor en un plazo no mayor a \PlazoResultadoAProveedor~posteriores a la auditoría.
    \item En caso de tener desviaciones, el proveedor debe enviar un Plan de Acción en un plazo máximo de \PlazoPlanDeAccionAuditProveedor posteriores a la recepción de los resultados de la auditoría.
    \item Los proveedores deberán cumplir con los siguientes puntos de acuerdo con el estatus en el que se encuentren:
          \begin{enumerate}
              \item \textbf{Proveedor Aprobado}
                    \begin{itemize}
                        \item Entrega de producto de acuerdo a las \glspl{especificacion} de \Gls{RDF}
                        \item En el caso de tener desviaciones presenta acciones correctivas en al menos un \qty{80}{\percent} de los hallazgos encontrados durante la auditoría.
                        \item Cumple con los tamaños de lote especificados y plazos de entrega.
                        \item Se mantiene calidad en el insumo entregado.
                    \end{itemize}
              \item \textbf{Proveedor Aprobado Condicionado}
                    \begin{itemize}
                        \item Entrega de producto de acuerdo a las \glspl{especificacion} de \Gls{RDF}
                        \item En el caso de tener desviaciones presenta acciones correctivas en al menos un \qty{60}{\percent} de los hallazgos encontrados durante la auditoría.
                        \item Cumple con los tamaños de lote especificados y plazos de entrega.
                        \item Se mantiene calidad en lo entregado.
                    \end{itemize}
              \item \textbf{Proveedor Oportunidad}
                    \begin{itemize}
                        \item El proveedor no ha sido evaluado, pero no existe otro material alterno u otro proveedor aprobado para cubrir con oportunidad la compra de un insumo determinado.
                    \end{itemize}
              \item \textbf{Proveedor Rechazado}
                    \begin{itemize}
                        \item No cumple con las especificaciones/orden de compra entregadas.
                        \item El proveedor no cumple con los estándares de calidad, precio, oportunidad de entrega, reacción a emergencias, etc.\ que la empresa requiere y el Equipo de Evaluación de Proveedores decide buscar otra alternativa.
                    \end{itemize}
          \end{enumerate}
    \item Auditoría a Proveedores: La auditoría se aplicará a todos los proveedores de insumos de Red de Fríos bajo los criterios que a continuación se enumeran, esto es con la finalidad de conocer el status de cada uno de ellos y trabajar en forma conjunta para obtener a corto plazo los beneficios de contar con proveedores aprobados.
\end{enumerate}

\begin{note}[Miembros del equipo de evaluación de proveedores] \label{miembrosEqProveedores}
Para la evaluación de proveedores, debe de contemplarse la presencia de:
    \begin{itemize}
        \item Jefe de Aseguramiento de Calidad
        \item Gerente de Operaciones
        \item Jefe de Almacén
        \item Gerente de Mantenimiento
        \item Encargado de Compras
    \end{itemize}
\end{note}

\begin{longtblr}[
    label = {tbl:criterios-proveedores},
    caption = {Criterios de evaluación hacia los proveedores.},
    entry = {Criterios de evaluación hacia los proveedores.},
    ]{%
    width = \linewidth,
    colspec = {Q[617]Q[163]Q[158]},
    cells = {c},
    row{even} = {Gallery},
    %hline{1,10} = {-}{0.08em},
    %        hline{2,9} = {-}{0.05em},
    }
    \toprule
    \textbf{Criterio}                                                                                                & \textbf{Calificación} & \textbf{Ponderación} \\
    \midrule
    Precio competitivo y cumplimiento de especificaciones                                                            & \num{30}              & \qty{15}{\percent}   \\
    Cumplimiento de fechas y cantidades comprometidas                                                                & \num{40}              & \qty{20}{\percent}   \\
    Capacidad para surtir los requerimientos                                                                         & \num{30}              & \qty{15}{\percent}   \\
    Capacidad para reaccionar a picos de demanda                                                                     & \num{30}              & \qty{15}{\percent}   \\
    El insumo se produce y/o almacena en condiciones que no representan un riesgo para la inocuidad de los alimentos & \num{40}              & \qty{20}{\percent}   \\
    No se observa presencia de plagas y preferentemente se cuenta con un programa de manejo de plagas.               & \num{20}              & \qty{10}{\percent}   \\
    Las condiciones generales de los edificios e instalaciones no ponen en riesgo el abasto del insumo               & \num{10}              & \qty{5}{\percent}    \\
    \textbf{PUNTUACIÓN TOTAL}                                                                                        & \num{200}             & \qty{100}{\percent}  \\
    \bottomrule
\end{longtblr}

De acuerdo a la siguiente tabulación, el proveedor recibe un dictamen de Aprobado, Aprobado Condicionado o Rechazo:

\begin{longtblr}[
    label = {tbl:Calificaciones-Proveedores},
    caption = {Posibles calificaciones obtenibles en la evaluación de proveedores.},
    entry = {Posibles calificaciones obtenibles en la evaluación de proveedores.},
    ]{%
    width = \linewidth,
    colspec = {Q[185]Q[181]Q[575]},
    cells = {c},
    row{even} = {Gallery},
    }
    \toprule
    \textbf{Estado}                                  & \textbf{Grado de cumplimiento}                & \textbf{Dictamen}                                                                                                                                                       \\
    \midrule
    \textbf{Aprobado}                                & \qtyrange{75}{100}{\percent}                  & Se mantendrá en la Lista de Proveedores Aprobados aquellos que logren un grado de cumplimiento mayor a \qty{75}{\percent}                                               \\
    \SetCell[r=2]{c}{\textbf{Aprobado condicionado}} & \SetCell[r=2]{c}{\qtyrange{60}{74}{\percent}} & Su ingreso en la Lista de Proveedores Aprobados queda condicionada a un incremento en su grado de cumplimiento de mínimo \qty{75}{\percent} para la siguiente auditoría \\
                                                     &                                               & Su permanencia en la Lista de Proveedores Aprobados se dará sólo si su grado de cumplimiento es de al menos \qty{60}{\percent}                                          \\
    \textbf{Rechazado}                               & \qty{59}{\percent}                            & Fuera de Lista de Proveedores                                                                                                                                           \\
    \bottomrule
\end{longtblr}

Cuando un proveedor obtenga una calificación que lo acredite como proveedor condicionado y/o cuando se señalen áreas de oportunidad durante la auditoría de evaluación, deberá presentar un plan de acciones de mejora a los puntos señalados y regresarlos a \GLS{RDF}. Este plan deberá contemplar el responsable de la actividad así como la fecha de cumplimiento y deberá dirigirse al Jefe de Control de Calidad de \GLS{RDF}.

\begin{changelog}[simple, sectioncmd=\subsection*,label=changelog-2.8]

    \begin{version}[v=2.1, date=2023--01, author=Pablo E. Alanis]
        \item Cambio de formato;
        \item adición de términos;
        \item creacción de apartado que establece los miembros del equipo de evaluación de proveedores
    \end{version}

    \begin{version}[v=1.4, date=2022-02, author=Alonso M.]
        \item no se realizaron cambios de la revisión 05 a la 06.
    \end{version}
\end{changelog}	% REVISADO
\thispagestyle{formato-PI}
\renewcommand{\MayorVer}{2}
\renewcommand{\MenorVer}{1}
\renewcommand{\Codigo}{PSA-9-ESP} %TODO
\renewcommand{\FechaPub}{2023--01}
%\renewcommand{\Edit}{2.1}
\renewcommand{\Titulo}{Especificación genérica de servicios de almacenamiento}

\section{\Titulo}\index{Especificación de servicios!almacenamiento}\label{esp:generica}
% \section{Especificación genérica de servicios de almacenamiento}

\subsection{Objetivos}

\begin{itemize}
	\item \textbf{Especificar} las condiciones de almacenamiento proporcionadas por \GLS{RDF};
	\item \textbf{Definir} los rangos de temperatura aplicable para cada tipo de especificación genérica de almacenamiento con la que opera \GLS{RDF};
\end{itemize}

\subsection{Alcance}

\begin{itemize}
	\item Las condiciones de almacenamiento genéricas establecidas en este documento son aplicables para acuerdos con clientes en el esquema de \emph{cámara compartida;}
	\item las especificaciones definidas en este documento aplican según el intervalo de temperatura definido por el fabricante del producto alimenticio que será almacenado;
	\item este documento no contempla las condiciones de almacenamiento \emph{específicas} en el esquema de \emph{cámara exclusiva;}
	\item las especificaciones de temperatura definidas en este documento no son aplicables a las de la ráfaga de congelación, ya que ahí se logran temperaturas ambientales inferiores a las del límite de especificación.
\end{itemize}

\subsection{Términos y definiciones}

\begin{description}
	\defglo{refrigeracion}
	\defglo{congelacion}
	\defglo{especificacion}
	\defglo{requisito}
\end{description}

\subsection{Disposiciones generales}

\begin{itemize}
	\item Con base en las necesidades de almacenamiento especificas de los productos alimenticios del \emph{cliente,} \GLS{RDF} acordará en que especificación de almacenamiento estos deberán ser almacenados;
	\item si los productos alimenticios del \emph{cliente} no se adecúan a las especificaciones genéricas manejadas por \GLS{RDF}, al cliente se le dará la opción, según la disponibilidad de las cámaras, de almacenar sus productos en cámaras de almacenamiento exclusivas, donde el cliente podrá establecer las necesidades de temperatura especificas de almacenamiento para su producto;
	\item en caso de que \emph{el cliente} no acepte la renta de una cámara de almacenamiento exclusiva, o por disponibilidad de espacio, no se cuente con ella, el cliente podrá decidir con base en las condiciones de almacenamiento genéricas, establecidas en este documento, en qué tipo de cámara será almacenado su producto.
	\begin{itemize}
		\item Este acuerdo debe de documentarse, ya sea en una comunicación externa \emph{i.e.} un e-mail, cliente de mensajería instantánea, etc.\ o bien en un documento en el que se establezca que el cliente está de acuerdo con que su producto se almacene en alguna de las condiciones genéricas de almacenamiento de \GLS{RDF}.
	\end{itemize}
	\item Para propósitos del control de las temperaturas con un enfoque hacia la inocuidad de los alimentos y las buenas prácticas de distribución, se definen los límites de control acordados de manera interna, con base en la NOM-001-SAGARPA/SCFI-2016 y la NOM-008-ZOO-1994. Cabe mencionar que las temperaturas aquí detalladas son las \emph{"óptimas",} por lo que se debe de establecer un rango de control adecuado con base en las especificaciones de temperatura aquí definidos.
\end{itemize}

\subsubsection{Almacenamiento en cámaras de refrigeración}

Con base en la circular № 20/2018 expedida por \Gls{SENASICA}, el intervalo definido para \emph{refrigeración} es el siguiente:

\begin{itemize}
	\item \textbf{Límite de especificación superior:} \cels{4}
	\item \textbf{Límite de control superior:} \cels{3.5}
	\item \textbf{Objetivo:} \cels{3}
	\item \textbf{Límite de control inferior:} \cels{1}
	\item \textbf{Límite de especificación inferior:} \cels{0}
\end{itemize}

\subsubsection{Almacenamiento en cámaras de congelación}

Para definir el intervalo de control, se tomó como base la circular № 20/2018 \GLS{SENASICA} en donde se menciona que la temperatura puede oscilar entre \qtyrange{-18}{-12}{\celsius} y según la NOM-001-SAGARPA/SCFI-2016, se define el límite inferior de control:

\begin{itemize}
	\item \textbf{Límite de especificación superior:} \cels{-12}
	\item \textbf{Límite de control superior:} \cels{-15}
	\item \textbf{Objetivo:} \cels{-18}
	\item \textbf{Límite de control inferior:} \cels{-25}
	\item \textbf{Límite de especificación inferior:} \cels{-30}
\end{itemize}

\subsection{Documentos relacionados}

\begin{itemize}
	\item \textbf{NOM-001-SAGARPA/SCFI-2016:} Prácticas comerciales—Especificaciones sobre el almacenamiento, guarda, conservación, manejo y control de bienes o mercancías bajo custodia de los almacenes generales de depósito. Incluyendo productos agropecuarios y pesqueros.
	\item \textbf{Circular № 20/2018 \GLS{SENASICA}}
	\item \textbf{NOM-008-ZOO-1994:} Especificaciones zoo-sanitarias para la construcción y equipamiento de establecimientos para el sacrificio de animales y los dedicados a la industrialización de productos cárnicos.
\end{itemize}

\subsection{Responsables}

\begin{itemize}
	\item \textbf{Ejecución:}
	\begin{itemize}
		\item El personal de mantenimiento es el que se encarga de procurar que las cámaras permanezcan dentro de los límites establecidos.
	\end{itemize}
	\item \textbf{Monitoreo:}
	\begin{itemize}
		\item Personal de \emph{aseguramiento de calidad:}
		\begin{itemize}
			\item realiza una inspección diaria de las temperaturas de las cámaras por medio de un termómetro IR;
			\item hace un informe semanal de las temperaturas registradas por el termómetro IR.
		\end{itemize}
		\item Personal de \emph{mantenimiento:}
		\begin{itemize}
			\item diariamente realiza inspección mediante termómetro IR de la temperatura superficial de las cámaras;
			\item semanalmente generan un histórico de temperaturas registradas por los equipos de enfriamiento;
			\item semanalmente generan un histórico de temperaturas registradas por un termoregistrador.
		\end{itemize}
		\item Personal de la \GLS{SADER}:
		\begin{itemize}
			\item aleatoriamente verifica las temperaturas de las cámaras en conjunto con \emph{aseguramiento de calidad}, aunque de esta actividad no deriva ningún registro para \GLS{RDF}.
		\end{itemize}
	\end{itemize}
	\item \textbf{Verificación:}
	\begin{itemize}
		\item Personal de \emph{aseguramiento de calidad:}
		\begin{itemize}
			\item recibe el histórico de temperaturas semanales registradas por los equipos y por los termoregistradores y los analiza;
		\end{itemize}
		\item \emph{Gerencia:}
		\begin{itemize}
			\item recibe la información de las temperaturas por parte de \emph{aseguramiento de calidad} y las verifica.
		\end{itemize}
	\end{itemize}
\end{itemize}

\begin{changelog}[simple, sectioncmd=\subsection*,label=changelog-2.9]
	\begin{version}[v=1.0, date=2023--01, author=Pablo E. Alanis]
		\item Primera edición.
	\end{version}
\end{changelog}	% REVISADO
\renewcommand{\MayorVer}{2}
\renewcommand{\MenorVer}{1}
\renewcommand{\Codigo}{PSA-1-PROG} %TODO
\renewcommand{\FechaPub}{2023--01}
%\renewcommand{\Edit}{2.1}
\renewcommand{\Titulo}{Inspección de transporte al embarque de producto}

\section{\Titulo}
\index{Inspección!transporte al embarque de producto, de}

% \section{Inspección de transporte al embarque de producto}

\subsection{Objetivo}

Establecer un procedimiento que asegure que la unidad a cargar se encuentra en condiciones adecuadas para poder ser ingresado el producto y que la unidad cumple con lo requerido para no generar alguna contaminación o daño al producto.

\subsection{Alcance}

Todas las unidades que se presenten a cargar producto en el almacén.

\subsection{Términos y definiciones}
%TODO

\subsection{Documentos y/o normas relacionadas}
%TODO
1.3 - MA-OP-001 - Manual de calidad y buenas prácticas de distribución|Manual de calidad y Buenas Prácticas de Distribución

\subsection{Procedimiento}

\subsubsection{Materiales}

\begin{itemize}
	\item Termómetro
\end{itemize}

\subsubsection{Precauciones de seguridad}

N/A

\subsubsection{Instrucciones}

\paragraph{Envío de productos}

\begin{enumerate}
	\item A fin de garantizar que todos los productos salen de las instalaciones libres de peligro física, química y/o los riesgos biológicos, todo el contenedor (caja del camión) deben ser inspeccionados antes de cargar y documentado usando la forma “Orden de Entrada o Salida”.
	\item La unidad debe de mostrar el termo encendido antes de ser colocado en la rampa, esta revisión es realizada por el personal de vigilancia
	\item El contenedor (caja del camión) deben ser revisado por el personal de vigilancia antes de que la unidad se coloque en la rampa. \textbf{Evidencia de daño, suciedad, o plagas deben ser reportado al Supervisor de Almacén y la unidad no podrá colocarse en rampa hasta no cumplir con los requerimientos de aceptación (Unidad limpia, sin daños que puedan causar una contaminación y libre de plaga).}
	\item El contenedor (caja del camión) deben ser revisado minuciosamente por los encargados de embarques de producto para confirmar la revisión realizada por el personal de vigilancia. \textbf{Evidencia de daño, suciedad, o plagas deben ser reportado al Supervisor de Almacén y la unidad no podrá colocarse en rampa hasta no cumplir con los requerimientos de aceptación (Unidad limpia, sin daños que puedan causar una contaminación y libre de plaga).}
	\item Todos los productos de salida \textbf{debe ser visualmente revisados} \textbf{para comprobar que no hay productos dañados} o en mal estado que se carguen en la unidad de carga. \textbf{NO debe de cargarse} producto abierto o dañado (cajas o contenedores abiertos o en mala condición). \textbf{El producto debe ser retenido para su verificación.}
	\item La temperatura del contenedor (caja del camión) deberá verificarse antes de la carga para asegurar que la unidad de refrigeración está funcionando correctamente. Los contenedores (caja de camión) deben estar programados a una temperatura \textbf{para los productos refrigerados No mayor a 4.0 °C y no inferior a 0.0 °C y para los productos congelados ≤ -15 °C (Ideal ≤ -18 °C).}
	\item \textbf{Si se tiene evidencia del NO funcionamiento del termo el producto NO debe ser cargado.}
	\item Los contenedores (caja del camión) deberán ser \textbf{sellados} antes de salir de las instalaciones cuando se trate de producto TIF o si así fuese requerido por el cliente.
	\item Requisitos pedidos por Medico TIF.
\end{enumerate}

\paragraph{Políticas de embarque}

\begin{enumerate}
	\item Para asegurarse de que todos los contenedores (caja del camión) estén limpios y cumplen con los requisitos de temperatura establecidos, todos los contenedores (caja del camión) tienen que ser inspeccionados y registrados en la forma “Orden de Entrada o Salida”. El contenedor (caja del camión) se analizara por la limpieza, daños, olores anormales, alguna evidencia de insectos o plaga, así como la temperatura adecuada.
	\item Antes de cargar o descargar el producto, este deberá ser revisado para verificar que no tenga daños en su empaque, evidencias de insectos / plagas.
\end{enumerate}


\begin{enumerate}
	\item Rango de temperaturas de cuarto donde se almacena el producto:
\end{enumerate}


\begin{enumerate}
	\item La temperatura del producto debe ser tomada según disposición del cliente:
	\begin{itemize}
		\item debe de ser tomada directamente en el producto con el termómetro de agua para el caso de cárnicos.
		\item Debe ser toma directamente del producto (Sin abrir empaque primario) con termómetro laser en el caso de productos compuestos ya sellados
	\end{itemize}
	\item El conductor deberá de enfriar la caja refrigerada de la unidad antes de enramparse en el andén de carga siguiendo el procedimiento de \textbf{pre-enfriado.}
\end{enumerate}




\paragraph{Procedimiento de pre enfriado de unidades}

Todas las unidades que transportan productos refrigerados o congelados para su distribución, deberán de ser pre enfriado antes de enrramparse para la carga siguiendo la siguiente regla:

\begin{itemize}
	\item Para él envió de productos refrigerados la caja de la unidad se deberá de pre enfriar hasta alcanzar la temperatura de 0.0 – 4.0°C.
	\item Para él envió de productos congelados la caja de la unidad se deberá programar hasta alcanzar la temperatura de ≤ -15°C.
\end{itemize}

\begin{enumerate}
	\item El conductor será responsable de verificar la temperatura de la caja del camión en la pantalla durante el traslado del producto.
	\item Cualquier inquietud o duda sobre el producto objeto del embarque deben ser comunicadas inmediatamente al Supervisor de Almacén antes de la carga.
\end{enumerate}

\subsection{Responsables de la actividad}

\begin{itemize}
	\item \textbf{Ejecutado} por personal de operaciones
	\item \textbf{Monitoreado} por personal de calidad
	\item \textbf{Verificado} por personal de gerencia.
\end{itemize}

\subsection{Acciones preventivas}

\begin{itemize}
	\item Se llevara a cabo una inspección. Registrando el indicador y medida correspondiente
	\item Si la desviación se repite frecuentemente se dará curso de capacitación al personal de almacén y limpieza para que realicen eficientemente su trabajo.
	\item Si después de haber capacitado al personal de almacén se siguen presentando desviaciones por causas injustificadas, se tomaran acciones más enérgicas con el personal por incumplimiento con sus deberes.
\end{itemize}

\subsection{Acciones correctivas}

\begin{itemize}
	\item En caso de no cumplimiento la tarea se deberá volver a realizar como se indica en el procedimiento.
	\item En caso de no conformidad reportar en formato de acciones correctivas F-OP-40.xls
\end{itemize}

\subsection{Frecuencia}

\begin{itemize}
	\item Cada embarque.
\end{itemize}

\subsection{Historial de modificaciones} %TODO

\begin{itemize}
	\item \textbf{Quinta edición:} cambio de fecha de 04 de noviembre 2017 a 28 de enero 2019, se realizó cambio de código de CT-P-ITEP a PR-010, se agregó punto 9. requisitos medico TIF y se anexo documento con los requisitos de dicho número y se realizaron cambios de revisión de 004 a 005.
	\item \textbf{Sexta edición:} febrero 2020 se hizo cambio de formato y cambio de código de PRO-010 a PRO-OP-007. Y no se realizaron cambio de la revisión 05 a la 06.
	\item \textbf{Séptima edición:} febrero 2021 no se realizaron cambios de la revisión 06 a la 07.
	\item \textbf{Octava edición:} febrero 2022 no se realizaron cambios de la revisión 07 a la 08.
\end{itemize}
	%
\thispagestyle{formato-PI}
\renewcommand{\MayorVer}{2}
\renewcommand{\MenorVer}{1}
\renewcommand{\Codigo}{PSA-11-PRO} %TODO
\renewcommand{\FechaPub}{2023--01}
%\renewcommand{\Edit}{2.1}
\renewcommand{\Titulo}{Acciones en caso de corte de energía o desastre}

\section{\Titulo}
\label{sec:AccEnCasoDeEmergencia}\index{Acciones inmediatas!en caso de corte de energía o desastre}

% \section{Acciones en caso de corte de energía o presencia de desastre o siniestro}

\subsection{Objetivo}

Definir las acciones a seguir en caso de haber un corte de energía eléctrica no controlado o presentarse un desastre natural o siniestro en las áreas de operación que requieren temperaturas controladas, para garantizar la protección del producto.

\subsection{Alcance}
Todas las áreas afectadas.

\subsection{Términos y definiciones}
\begin{description}
	\defglo{desastre-natural}
	\defglo{emergencia}
\end{description}

\subsection{Documentos y/o normas relacionadas}
\begin{itemize}
	\item Manual de mantenimiento
	\item Programa Interno de Protección Civil
\end{itemize}

\subsection{Procedimiento}
\subsubsection{Instrucciones}
\paragraph{Procedimiento de emergencia ante crisis y desastres naturales}\index{Acciones inmediatas!en caso de incendio en la periferia}
\begin{enumerate}
	\item La seguridad personal es lo principal en caso de crisis y emergencia. Por favor refiérase al equipo de Emergencia / Crisis.
	\item En el caso de una emergencia / crisis que suceda en una instalación de Red de Fríos, se deberá informar inmediatamente el Gerente General
	\item Si la emergencia es peligrosa para la vida, inmediatamente llame \textbf{al 065 para la Cruz Roja o al 066 para emergencias en Nuevo León y reporte el incidente o bien al 911.}
	\item Si la crisis se relaciona con un \emph{incendio en o alrededor de la instalación,} todos los empleados debe de salir inmediatamente de la instalación reunirse en el punto de reunión.
	\item Es importante mantener la calma, moverse con cautela y ayudar a cualquier persoba que requiera asistencia.
	\item \emph{Si el lugar acordado para reunirse se ha vuelto peligroso,} todos los empleados tienen que reunirse en la puerta principal (estacionamiento) y esperar instrucciones.
	\item Una vez que la seguridad de los empleados se ha resguardado y todos los empleados han sido contados, contacte inmediatamente con el organismo adecuado para reportar el incidente y de ser requerido, solicitar ayuda médica.\footnote{El Supervisor en turno debe liderar este esfuerzo.}
	\item Si la crisis se refiere a la seguridad personal, contacte inmediatamente con el Departamento de Emergencias, en el número arriba mencionado, si la emergencia es \textbf{potencialmente mortal, inmediatamente llame al 066 o 911} y reporte el incidente.
	\item Si la crisis se refiere a un terremoto, inundación o huracán todos los empleados deben salir inmediatamente de la instalación y juntar en el lugar acordado para reunirse siguiendo los pasos mencionados anteriormente
\end{enumerate}



\paragraph{Corte de energía eléctrica}\index{Acciones inmediatas!en caso de corte eléctrico}
\begin{itemize}
	\item En cuanto a cortes de energía o falla en los equipos que afecten directamente el funcionamiento y por lo tanto la temperatura de las cámaras, por favor, tenga en cuenta que los teléfonos de oficina pueden no funcionar, por lo que los teléfonos celulares deben ser utilizados y siga el siguiente procedimiento:
	\item Llama a la \gls{CFE} número de emergencia de falla (071) e inmediatamente reportar el apagón. \emph{(Indicar número de contrato).}
	\item Comunicarse con el Gerente de Operaciones, Supervisor de Almacén, Gerente y Supervisor de Mantenimiento informado de las actualizaciones de estado.
\end{itemize}

\begin{note}[Números de contrato] \label{note:NumerosDeContratoCFE}
	\index{Números de contrato (\gls{CFE})}
	Los números de contrato son los siguientes:
	\begin{enumerate}
		\item 999001000109
		\item 407210800266
	\end{enumerate}
\end{note}

\begin{itemize}
	\item Si \gls{CFE} informa de que no habrá electricidad por más de \qty{2}{\hour} \emph{(No representa un problema con el producto refrigerado o congelado),} se suspenderá el servicio a clientes y proveedores, cerrando todas las puertas de los almacenes que se encuentren en refrigeración o congelación evitando la perdida de temperatura.
\end{itemize}

\begin{note}[Tiempo máximo de permanencia de productos en cámaras apagadas] \label{note:TMaxCamAveriada}
	\index{Tiempo de permanencia máxima de alimentos en cámaras apagadas}
	\begin{itemize}
		\item \Gls{alimento} refrigerado: \qty{<=\TiempoAveriaRefri}{\hour};
		\item \Gls{alimento} congelado: \qty{<=\TiempoAveriaConge}{\hour}.
	\end{itemize}
\end{note}

\emph{Al re establecerse el servicio de energía, el personal de mantenimiento debe de realizar la revisión de los equipos para verificar que no se tuvo algún contratiempo con el evento y confirmar que todo se encuentra funcionando correctamente.}

\paragraph{Falla en los equipos de enfriamiento}\index{Acciones inmediatas!en caso de avería de equipos de enfriamiento}
En cuanto a falla en los equipos que afecten directamente el funcionamiento y por lo tanto la temperatura de las cámaras, por favor, tenga en cuenta que los teléfonos de oficina pueden no funcionar, por lo que los teléfonos celulares deben ser utilizados y siga el siguiente procedimiento:

\begin{itemize}
	\item Analice junto con el personal de mantenimiento la falla y determine el tiempo de respuesta para el restablecimiento de los equipos y por lo tanto de la temperatura.
	\item Si llegara a fallar uno de los compresores que dan energía a la planta, se cuenta con dos compresores extra los cuales están listos para funcionar si llegara a fallar algún otro en funcionamiento.
	\item Si mantenimiento informa de que puede resolver la falla en un periodo de no más de \qty{2}{\hour}, se suspenderá el servicio a clientes y proveedores, cerrando todas las puertas de los almacenes que se encuentren en refrigeración o congelación para evitar la perdida de temperatura.
\end{itemize}

Si la emergencia no es solucionada el Gerente General va a delegar responsabilidades (según corresponda) al personal considerando las siguientes acciones:

\paragraph{Seguridad y acceso a la instalación}
\begin{enumerate}
	\item Asesorar a los clientes de cualquier interrupción de servicio;
	\item Revisión de condiciones del producto;
	\item Revisión de condición del equipo de refrigeración;
	\item La principal preocupación en caso de emergencia o crisis radical en la pérdida de control de la temperatura, ya que el producto podría correr un riesgo microbiológico.
	\item Al re establecerse el servicio de energía, el personal de mantenimiento debe de realizar la revisión de los equipos para verificar que no se tuvo algún contratiempo con el evento y confirmar que todo se encuentra funcionando correctamente.
	\item Primeramente se debe verificar si no hay energía en toda la planta, esto debido a que el almacén es alimentado por 2 fuentes de luz diferentes, lo que se refiere, a que la mitad del almacén puede tener energía cuando la otra mitad no, siendo este el caso, se debe verificar la manera de mover el producto de las cámaras que no tienen energía, a las cámaras que siguen en funcionamiento.
	      \begin{itemize}
		      \item Si el producto requiere reubicación, la instalación (primaria) debe ser contactada. Si la primaria se ha visto comprometida, la instalación secundaria debe ser contactada.
	      \end{itemize}
\end{enumerate}

\begin{itemize}
	\item \textbf{Instalación primaria: Red de Fríos Suc. Ruiz Cortines}
	      Ruiz Cortines No. 2208, Col. Moderna. Monterrey, N.L. C.P. 64530, NL, México.
	      Tel. +52 (81) 1292--7272.
	\item \textbf{Instalación Secundaria: Mega Frio S.A de C.V.}
	      20 de Noviembre No. 2618 Col. Garza Nieto Monterrey, N.L. C.P. 64420, NL, México
	      Tel. +52 (81) 8040--7162 y +52 (81) 8040--7163

\end{itemize}

\begin{contact}[Números de emergencia] \label{contact:NumerosDeEmergencia}
	\index{Números de emergencia}
	Los números de contrato son los siguientes:\\
	\begin{tabular}{lc}
		\textbf{Cruz Roja:}               & 065 \\
		\textbf{Emergencias:}             & 066 \\
		\textbf{Cualquier incidente:}     & 911 \\
		\textbf{CFE (Reporte de fallas):} & 071
	\end{tabular}
\end{contact}

Además del área de vigilancia, cada instalación debe tener los números de emergencia de las autoridades locales, estatales y federales.

\begin{tabular}{cc}
	\noindent \textbf{Contacto primario:} & \cref{contact:Gerente}       \\
	\textbf{Contacto secundario:}         & \cref{contact:JefeDeAlmacen}
\end{tabular}

\begin{contact}[Números de transportes en caso de siniestros] \label{contact:TransportesEnCasoDeAveria}
	\index{Números de transportes en caso de siniestros}
	\begin{tabular}{lc}
		\textbf{Transportes Red de Fríos:} & +52 (81) 1636--8137 \\
		\textbf{Transportes Java:}         & +52 (81) 2681--0370 \\
		\textbf{Thermo Transportes:}       & +52 (81) 1387--0145
	\end{tabular}
\end{contact}
\renewcommand{\MayorVer}{2}
\renewcommand{\MenorVer}{1}
\renewcommand{\Codigo}{PSA-1-PROG} %TODO
\renewcommand{\FechaPub}{2023--01}
%\renewcommand{\Edit}{2.1}
\renewcommand{\Titulo}{Lista de contactos de emergencia}

\section{\Titulo}
\index{Lista!contactos de emergencia, de}

% \section{Lista de contactos de emergencia}

\subsection{Objetivo}

\begin{itemize}
	\item Crear el equipo para atacar casos de crisis por corte de energía eléctrica, desastres naturales o siniestros.
	\item Generar la coordinación del plan de contingencia a la etapa de crisis
	\item Garantizar la seguridad (Inocuidad) de alimentos para consumo humano.
\end{itemize}

\subsection{Alcance}

A todas las personas responsables para que se lleven a cabo las operaciones que se requieran para el buen cumplimiento de este procedimiento en el RED DE FRIOS S.A. de C.V.

\subsection{Términos y definiciones}

\begin{itemize}
	\item \textbf{Equipo de emergencia:} El grupo de personas responsables de desarrollar, implementar y verificar el cumplimiento del programa de emergencia.
\end{itemize}

\subsection{Documentos y/o normas relacionadas}

Procedimiento en caso de corte de energía o desastre natural o siniestro.

\subsection{Procedimiento}

\subsubsection{Instrucciones}

\paragraph{Equipo de Emergencias:}

\subparagraph{Coordinador 1}

Gerardo Ruiz, Gerente de Operaciones.
Teléfono: 01 (81) 83 51 05 12, 8180295609
Responsable de coordinar los esfuerzos según la emergencia y restablecer la operación, garantizando al mismo tiempo las políticas más estrictas de seguridad y protocolos.

\subparagraph{Coordinador 2}

Luis Escareño, Jefe de Almacén
Teléfono: +52 (81) 8351-0512,+52 (81) 1390-7099
Responsable de asegurar el entorno de almacén y asegurarse de que el personal de almacén salga del edificio inmediatamente y se congreguen en el exterior (punto de reunión) en caso de emergencia.

\subparagraph{Coordinador 3}

Salvador Verdín, Jefe de Mantenimiento
Teléfono: +52 (81) 8351-0512,+52 (81) 1706-1644.
Responsable del personal de oficinas salga del edificio inmediatamente y se congreguen en el exterior (punto de reunión) en caso de emergencia

\subparagraph{Coordinador 4}

Mario González, Gerente de Mantenimiento
Teléfono: +52 (81) 1292-7272, +52 (81) 1607-6035
Responsable de asegurar las condiciones internas de almacén y equipos, asegurar que el personal de almacén salga del edificio inmediatamente y se congreguen en el exterior (punto de reunión) en caso de emergencia.

\subparagraph{Vigilancia}

Lorenzo Hernández, Vigilancia
Teléfono: +52 (81) 2353-2346

\subparagraph{Números de Contacto de Emergencia}

Además del área de vigilancia, cada instalación debe de tener los números de emergencia de las autoridades locales, estatales y federales.


Responsable de verificar de primera instancia alguna anormalidad vista desde el punto externo del almacén y dar aviso al personal correspondiente según el hallazgo detectado.

\subsubsection{Historial de modificaciones}

\begin{itemize}
	\item \textbf{Quinta edición:} cambio de fecha de 04 de noviembre 2017 a 28 de enero 2019, se realizó cambio de código de DG-P-LCCE a PR-012, se dieron de alta contactos del personal de vigilancia y no se realizaron cambios en la revisión 004 a la 005.
	\item \textbf{Sexta edición:} febrero 2020 se hizo cambio de formato y cambio de código de PR-012 a L-OP-001. Y no se realizaron cambio en la revisión 05 a la 06.
	\item \textbf{Séptima edición:} febrero 2021 no se realizaron cambio en la revisión 06 a la 07.
	\item \textbf{Octava edición:} febrero 2022 se realizó un acomodo a la información y se agregaron los No. De contrato con CFE de la revisión 07 a la 08.
\end{itemize}

\subsection{Anexos}

\renewcommand{\MayorVer}{2}
\renewcommand{\MenorVer}{1}
\renewcommand{\Codigo}{PSA-1-PROG} %TODO
\renewcommand{\FechaPub}{2023--01}
%\renewcommand{\Edit}{2.1}
\renewcommand{\Titulo}{Manejo de producto retenido}

\section{\Titulo}
\index{Producto retenido!manejo}


\subsection{Objetivo}

Indicar el procedimiento adecuado del manejo del producto NO comestible que se encuentre dentro de los almacenes de RED DE FRIOS.

\subsection{Alcance}

Todos los productos dentro de almacén que sean no comestibles y se encuentren retenidos.

\subsection{Términos y definiciones}

\begin{itemize}
	\item TODO
\end{itemize}

\subsection{Documentos y/o normas relacionadas}

\begin{itemize}
	\item 1.0 - BPD - Indice|Programa de BPM
	\item 2.1 - PSA-OP-IT-3 - Inspección en la recepción de productos|Procedimiento de Inspección de transportes
\end{itemize}

\subsection{Procedimiento}

\subsubsection{Materiales}

\begin{itemize}
	\item Bitácora de registro
\end{itemize}

\subsubsection{Precauciones de seguridad}

\begin{itemize}
	\item Usar uniforme completo
\end{itemize}

\subsubsection{Instrucciones}

\paragraph{Procedimientos de Producto Detenido por parte de }

Con el fin de garantizar la eliminación de productos no conformes de la distribución y cadena de suministro, cualquier producto no conforme que ha sido identificado por parte de CALIDAD debe ser retenido y reportado al Supervisor de Almacén y/o Jefe de Almacén, que debe permanecer involucrado durante todo el proceso para garantizar el cumplimiento con los procedimientos y normas establecidas; Se debe dar aviso al cliente de la acción tomada por parte de CALIDAD y enviar el comunicado del motivo de esta retención.

\begin{itemize}
	\item Producto dañado
	\item Producto abierto (expuesto)
	\item Producto con presencia de suciedad
	\item Producto caducado
	\item Producto con evidencia de problema de inocuidad
	\item Entre otras
\end{itemize}

\begin{enumerate}
	\item Cuando CALIDAD durante la inspección al momento de la recepción o durante su almacenamiento detecta producto que se encuentra dañado, contaminado (visualmente), en mal estado debe de retener este producto para evitar que continúe la cadena de suministro y pueda ser utilizado para proceso o consumo y pueda generar un riesgo en el consumidor.
	\item CALIDAD genera la indicación del no recibo del producto si el producto no cumple con las especificaciones de inocuidad y que puede generar un riesgo para el resto de los productos en el almacén y al consumidos
	\item Si el producto es detectado durante su almacenamiento, CALIDAD genera la indicación de la separación de este producto y la identificación del mismo como producto rechazado o decomisado.
	\item Este producto es colocado en el área de producto retenido.
	\item CALIDAD genera la información del motivo del decomiso como evidencia de la acción tomada.
	\item CALIDAD genera la cita con la planta de rendimiento (autorizada por SADER) para realizar la eliminación del producto decomisado, con el fin de evitar que se pueda generar una contaminación cruzada.
	\item Al momento que la planta de rendimiento realiza la recolección, se genera la documentación con la información del producto en cuestión.
	\item Los registros de la recolección se archivan como evidencia de la operación realizada.
	\item El almacén tiene la obligación de generar de igual manera los registros de las acciones tomadas.
	\item El almacén tiene la obligación de dar aviso al cliente de las acciones tomadas de su producto por parte de CALIDAD y mantener informado de los pasos del proceso hasta el retiro del producto.
\end{enumerate}

\paragraph{Procedimientos de Producto Detenido por calidad del cliente}

Con el fin de garantizar la eliminación de productos no conformes de la distribución y cadena de suministro, cualquier producto no conforme que ha sido identificado debe ser reportado al Supervisor de Almacén y Coordinador de Calidad, que debe permanecer involucrado durante todo el proceso para garantizar el cumplimiento con los procedimientos y normas establecidas.

\begin{itemize}
	\item Producto dañado
	\item Producto abierto
	\item Producto con presencia de suciedad
	\item Producto caducado
	\item Entre otras
\end{itemize}

\begin{enumerate}
	\item Una vez que el producto ha sido identificado, se deberá documentar los detalles y avisar al Gerente de Operaciones así como el cliente en cuestión.
	\item Etiquetas de \textit{"Retenido"} deben ser generadas y se colocan identificando al producto señalado para indicar claramente que el producto en cuestión está \textit{RETENIDO.} Las etiquetas deben tener la siguiente información como referencia:
	\begin{enumerate}
		\item Fecha
		\item Producto
		\item Cantidad
		\item Número de lote (si aplica)
		\item Razón por la detención
	\end{enumerate}
	\item El Gerente de Operaciones de RDF contactara al Cliente para dar aviso y valorar el producto retenido.
	\item El Coordinador de Calidad del Cliente o el cliente emitirá la acción a realizar con el producto retenido.
	\item Todas las instrucciones recibidas del cliente debe ser remitida al Gerente de Operaciones, así como el Supervisor de Almacén para su revisión.
	\item Las instrucciones del cliente en cuestión definirá:
	\begin{enumerate}
		\item Si el producto va ser devuelto a ellos (inmediatamente será enviado al cliente)
		\item Si el producto va ser destruido (El Supervisor de Almacén seguirá los procedimientos establecidos de destrucción y el producto será destruido).
	\end{enumerate}
	\item El supervisor de almacén registrara el evento para generar la evidencia de la detención y el destino del producto.
	\item El supervisor de almacén es responsable de la realización de un inventario semanal de todos los productos en “Detenido” para asegurar el correcto manejo y disposición de dichos productos. Los Inventarios debe llevarse a cabo incluso si la zona detenida está vacante o vacía (que indica “cero” en el Reporte de Inventario de Producto Detenido).
\end{enumerate}

\paragraph{Procedimiento para destrucción de productos}

\begin{enumerate}
	\item Los departamentos que autoriza la destrucción del producto son el departamento de Calidad, departamento de Finanzas e Inventarios y cliente.
	\item Una vez que el cliente autoriza que el producto sea destruido, se procede a separar, etiquetar e identificar el producto no conforme. La autorización debe de ser de manera escrita.
	\item El producto no conforme será eliminado (Si el producto lo requiere) con un producto de tinta desnaturante y llevado al área asignada para almacenar producto retenido y programar su retiro y destrucción.
	\item Posterior a la destrucción RDF debe de enviar notificación y evidencia de la destrucción del producto.
	\item Los proveedores que pueden ser utilizados para la destrucción de los productos pudiera ser:
	\begin{itemize}
		\item \textbf{RENGRA:} para productos perecederos Congelados y Refrigerados.
		\item \textbf{SIMEPRODE:} para productos secos varios.
	\end{itemize}
\end{enumerate}


\subsection{Responsables de la actividad}

\begin{itemize}
	\item \textbf{Ejecutado} por personal de CALIDAD
	\item \textbf{Verificado} por personal de gerencia.
\end{itemize}

\subsection{Acciones preventivas}

\begin{itemize}
	\item Se llevara a cabo una inspección. Registrando el indicador y medida correspondiente
	\item Si la desviación se repite frecuentemente se dará curso de capacitación al personal de almacén y limpieza para que realicen eficientemente su trabajo.
	\item Si después de haber capacitado al personal de almacén se siguen presentando desviaciones por causas injustificadas, se tomaran acciones más enérgicas con el personal por incumplimiento con sus deberes.
\end{itemize}

\subsection{Acciones correctivas}

\begin{itemize}
	\item En caso de no cumplimiento la tarea se deberá volver a realizar como se indica en el procedimiento.
	\item En caso de no conformidad reportar en formato de acciones correctivas F-OP-40.xls|F-OP-40
\end{itemize}

\subsection{Frecuencia}

Cada eventualidad que amerite separación de producto.

\subsection{Historial de modificaciones}

\begin{itemize}
	\item Cuarta edición: cambio de fecha de 04 de noviembre 2017 a 28 de enero 2019, se realizó cambio de código de CA-P-MPR a PR-013. Y no se realizaron cambios de la revisión 003 a la 004
	\item Quinta edición: febrero 2020 se hizo cambio de formato y cambio de código de PR-013 a PRO-OP-009. Y no se realizaron cambios de la revisión 04 a la 05.
	\item Sexta edición: febrero 2021 no se realizaron cambios de la revisión 05 a la 06.
	\item Séptima edición: febrero 2022 no se realizaron cambios de la revisión 06 a la 07.
\end{itemize}

% \subsection{Anexos}

% \subsubsection{Contactos}
% \paragraph{GEN}
% Daniel González Treviño\\
% Asesor Ambiental\\
% GEN Industrial Monterrey\\
% \phonenumber{+52 81442200}[2245]\\
% \phonenumber{+52 812416373}

% \paragraph{RENGRA}

% Rendimientos Grasos
% Pedro Antonio Molina Herrera, C.P.
% Adquisición de Materia Prima
% Rengra, S.A. de C.V
% \phonenumber{+5281543210}
% \phonenumber{+5281543216}
% \url{http://www.rengra.com.mx}

% \paragraph{SIMEPRODE}
% Javier Jiménez Pacheco, Ing
% Director de Operaciones
% Emilio Carranza No. 730 sur, - 2° Piso, entre Padre Mier y Matamoros,
% Monterrey, N.L. C.P. 64000
% Tel. 2020-9500; 9500
% \url{mailto:simeprode@nuevoleon.gob.mx}

\renewcommand{\MayorVer}{2}
\renewcommand{\MenorVer}{1}
\renewcommand{\Codigo}{PSA-1-PROG} %TODO
\renewcommand{\FechaPub}{2023--01}
%\renewcommand{\Edit}{2.1}
\renewcommand{\Titulo}{Política de re-empaque, re-etiquetado, y uso de aditivos}

\section{\Titulo}
\index{Política!re-empaque, re-etiquetado y uso de aditivos, de}

%\section{Política de re-empaque, re-etiquetado, y uso de aditivos}

\subsection{Objetivo}

Dar a conocer que no se realizan maniobras de re empaque como actividades comunes dentro de RED DE FRIOS S.A de C.V.

\subsection{Alcance}

A todas las personas responsables para que se lleven a cabo las operaciones que se requieran para el buen cumplimiento de este procedimiento en las áreas de operaciones

\subsection{Términos y definiciones}

N/A

\subsection{Documentos y/o normas relacionadas}

\begin{itemize}
	\item 1.0 - BPD - Indice|Programa de BPM
\end{itemize}

\subsection{Procedimiento}

\subsubsection{Materiales}

\begin{itemize}
	\item Bitácora de registro
\end{itemize}

\subsubsection{Precauciones de seguridad}

\begin{itemize}
	\item Usar uniforme completo.
\end{itemize}

\subsubsection{Instrucciones}

\paragraph{Política de re-empacado.}

\begin{itemize}
	\item Red de Fríos S.A. de C.V. \textbf{NO} realiza actividades de \emph{RE-EMPAQUE} como práctica común dentro de sus procedimientos durante la operación.
	\item Red de Fríos S.A. de C.V. \textbf{NO} realiza actividades de \emph{RE-ETIQUETADO} como práctica común dentro de sus procedimientos durante la operación.
	\item Red de Fríos S.A. de C.V. \textbf{NO} hace \emph{uso de aditivos o ingredientes} como práctica común dentro de sus procedimientos durante la operación
\end{itemize}

\subsection{Frecuencia}

Cuando se presente una eventualidad.

\subsection{Historial de modificaciones}

\begin{itemize}
	\item \textbf{Tercera edición:} cambio de fecha de 04 de noviembre 2017 a 28 de enero 2019, se realizó cambio de código de CA-P-PRE a P-002. Y no se realizaron cambios de la revisión 002 a la 003.
	\item \textbf{Cuarta edición:} febrero 2020 se hizo cambio de formato y cambio de código de P-002 a PO-OP-002. Y no se realizaron cambios de la revisión 03 a la 04.
	\item \textbf{Quinta edición:} febrero 2021 no se realizaron cambios de la revisión 04 a la 05.
	\item \textbf{Sexta edición:} febrero 2022 no se realizaron cambios de la revisión 05 a la 06.
\end{itemize}

\thispagestyle{formato-PI}
\renewcommand{\MayorVer}{2}
\renewcommand{\MenorVer}{1}
\renewcommand{\Codigo}{PSA-1-PROG} %TODO
\renewcommand{\FechaPub}{2023--01}
%\renewcommand{\Edit}{2.1}
\renewcommand{\Titulo}{Control documental}

\section{\Titulo}
\index{Información documentada!control!almacenamiento de registros}

% \section{Archivo de registros: Disposición de documentación}

\subsection{Objetivo}
\begin{itemize}
	\item Establecer el \textit{tiempo de resguardo} de registros e información documentada miscelánea;
	\item establecer buenas prácticas de control documental.
\end{itemize}

\subsection{Alcance}
\begin{itemize}
	\item Este documento está destinado para el departamento que se asegurará de almacenar la información documentada.
\end{itemize}

\subsection{Términos y definiciones}
\begin{description}
	\defglo{informacion-documentada};
	\defglo{registro};
	\defglo{documento};
	\item[\glsfirst{SGC}] \glsdesc{SGC};
	\item[\glsfirst{SGA}] \glsdesc{SGA}.
\end{description}

\subsection{Documentos y/o normas relacionadas}
\begin{itemize}
	\item Toda aquella \gls{informacion-documentada} por \gls{RDF};
	\item toda aquella \gls{informacion} electrónica almacenada en el \gls{SGA}.
\end{itemize}

\subsection{Procedimiento}
\subsubsection{Control documental}

\begin{itemize}
	\item El Gerente de Operaciones es responsable del mantenimiento del \gls{PSA};
	\item Todos los cambios introducidos en el manual y los documentos que deben tenerse en cuenta deben identificar, seguido de la fecha de elaboración y la fecha de actualización;
	\item Los registros\footnote{Los registros se guardan principalmente en papel, sin embargo puede haber excepciones o pueden almacenarse de forma electrónica si están en proceso de elaboración.} de recepción, almacenamiento, embarque y distribución de producto, y todos los registros involucrados para rastreabilidad de producto, órdenes de mantenimiento serán resguardados por \VigenciaAlmacRegistros;
	\item Todos los registros del \gls{PSA} de igual manera son almacenados por un mínimo de \VigenciaAlmacRegistros;
	\item Estos documentos permanecerán almacenados en tres posibles puntos:
	\begin{itemize}
		\item En las oficinas de cada departamento;
		\item en oficinas administrativas;
		\item en área destinada en almacén para esta función.
	\end{itemize}
	\item Transcurridos los \VigenciaAlmacRegistros\ y dirección no los ha requerido, se enviarán al archivo muerto o destrucción de los mismos.
	\item Los registros en formato electrónico de recepción, almacenamiento, embarque y distribución de producto, y todos los registros involucrados para rastreabilidad de producto, órdenes de mantenimiento, estarán a la mano para su rápida revisión por \VigenciaAlmacRegistrosElec;
	\item Todos los registros del Programa de Seguridad Alimentaria que se encuentren de manera electrónica de igual manera son almacenados por un mínimo de \VigenciaAlmacRegistrosElec.
	\begin{itemize}
		\item Esta información es respaldada diariamente de manera automática al término de cada cierre de actividad;
		\item Estos archivos son grabados y almacenados en las oficinas administrativas de \gls{RDF};
		\item De esta manera se almacenarán por tiempo indefinido.
	\end{itemize}
	\item El \emph{Gerente Administrativo} es el responsable de garantizar que cada departamento maneje el documento como se describe en este procedimiento.
\end{itemize}

\subsection{Responsables de la actividad}
\begin{itemize}
	\item \textbf{Ejecutado} por personal de calidad;
	\item \textbf{Verificado} por personal de gerencia.
\end{itemize}

\subsection{Acciones preventivas}
\begin{itemize}
	\item Se llevara a cabo una inspección. Registrando el indicador y medida correspondiente;
	\item Si la desviación se repite frecuentemente se dará curso de capacitación al personal de almacén y limpieza para que realicen eficientemente su trabajo;
	\item Si después de haber capacitado al personal de almacén se siguen presentando desviaciones por causas injustificadas, se tomaran acciones más enérgicas con el personal por incumplimiento con sus deberes.
\end{itemize}

\subsection{Acciones correctivas}
\begin{itemize}
	\item En caso de no cumplimiento la tarea se deberá volver a realizar como se indica en el procedimiento;
	\item En caso de no conformidad reportar en formato de acciones correctivas (\IdFormAACC).
\end{itemize}

\begin{changelog}[simple, sectioncmd=\subsection*,label=changelog-2.15]
	\begin{version}[v=2.0, date=2023--01, author=Pablo E. Alanis]
		\item Cambios de formato;
		\item cambio de código;
		\item cambios en redacción.
	\end{version}

	\begin{version}[v=1.7, date=2022--01, author=Agustín M.]
		\item febrero 2022 no se realizaron cambios de la revisión 06 a la 07.
	\end{version}
\end{changelog}
\thispagestyle{formato-PI}
\renewcommand{\MayorVer}{2}
\renewcommand{\MenorVer}{0}
 %TODO
\renewcommand{\FechaPub}{2023--01}
\renewcommand{\TipoID}{PRO}
\renewcommand{\Titulo}{Aprobación de cambios a procedimientos}

\section{\Titulo}\index{Información documentada!aprobación de cambios a procedimientos}
\renewcommand{\Codigo}{\Prog--\thesection--\TipoID}
\subsection{Objetivo}
Asegurar que se aprueben las modificaciones realizadas a la \gls{informacion-documentada} con el propósito de tener un consenso en la revisión más reciente de los documentos. 

\subsection{Alcance}
\begin{itemize}
	\item Este documento está destinado para el departamento que se encargará de realizar modificaciones a la \gls{informacion-documentada}, \emph{i.e.\ el departamento de aseguramiento de calidad, la gerencia, etc.}
	\item las revisiones de información documentada no se limitan a procedimientos y/o instrucciones de trabajo, sino a todo tipo de documento que sea responsabilidad del departamento.
\end{itemize}

\subsection{Términos y definiciones}
\begin{description}
	\defglo{informacion-documentada}
	\defglo{registro}
	\defglo{documento}
	\item[\glsfirst{SGC}] \glsdesc{SGC}
	\item[\glsfirst{SGA}] \glsdesc{SGA}
\end{description}

\subsection{Procedimiento}
\subsubsection{Cambios en los procesos de Recepción, Almacenamiento y Embarque}
\begin{enumerate}
	\item Los cambios en los procesos que afectan a los productos \emph{(i.e.\ manejo de producto)} solo pueden ser solicitados y autorizados por \emph{requerimiento del cliente}.
	\item Este requerimiento se debe de hacer por escrito y firmado por parte del cliente.
\end{enumerate}

\subsubsection{Cambios de procedimientos y políticas}

\paragraph{Revisiones}
\begin{itemize}
	\item Se deben de hacer revisiones constantes toda información documentada con el objetivo de determinar que requerimientos adicionales deben incluirse en revisiones a dichos documentos. 
	\item El \emph{Gerente de Operaciones} es responsable del mantenimiento del \emph{Programa de Seguridad Alimentaria,} así como capacitación del personal en todos los cambios.
\end{itemize}

\paragraph{Solicitud de cambios}
\begin{enumerate}
	\item El equipo comunicará la política o procedimiento que requiera realizar cambios o mejoras.
	\item El equipo deberá justificar el cambio, exponiendo el motivo y los elementos a favor y los posibles elementos que estuvieran en contra.
	\item Autorización o rechazo de la propuesta de cambio por las áreas involucradas. Previamente se debe de realizar un análisis valorando minuciosamente la propuesta.
	\item Si se autoriza el cambio, este se debe de registrar por escrito en el programa de cambio de documentos por producto el cual debe de contener por qué se autorizó el cambio y debe de contener las firmas en el formato correspondiente de cambios de documentos.
\end{enumerate}
	\textbf{Los departamentos que intervienen en el análisis de la propuesta de cambio son:}
		\begin{itemize}
			\item Gerencia de Operaciones.
			\item Coordinador de logística.
			\item Áreas operativas.
			\item Dirección.
			\item Áreas administrativas.
			\item Mantenimiento.
			\item Calidad.
		\end{itemize}
\begin{enumerate}[resume*]
	\item Cuando se determina que los cambios en la especificación, la política o procedimiento es necesario en un esfuerzo por mantener la continuidad de la calidad y de procedimiento (así como el control de los sistemas), el Gerente de Operaciones actualizara el documento o sección en cuestión, lo que indica “actualización”, seguida del número de revisión en la esquina superior derecha del documento que ha sido revisado.
	\item El \emph{Gerente de Operaciones} difundirá el documento revisado a todas las partes afectadas. Esas actividades incluirán una breve explicación de la revisión y solicitud de preguntas para asegurar una línea clara, sólida y abierta de comunicación existente entre todas las partes implicadas.
	\item Se dará seguimiento para garantizar el buen funcionamiento del cambio.
\end{enumerate}

\paragraph{Administración de Cambios}

\begin{enumerate}
\item El cambio se produce en nuestra empresa como mejora continua, es una parte normal de nuestro negocio y puede referirse a los siguientes ámbitos, entre otros:
\begin{itemize}
	\item Cambios de personal
	\item Modificaciones a los contratistas
	\item Administración de equipos
	\item Reubicación de instalaciones
	\item Nuevos clientes y cambios en las relaciones con un cliente.
	\item Cambios en las rutas de distribución.
\end{itemize}
\item Como tal, el \emph{Gerente de Operaciones} se encargara de difundir la información relativa a la modificación de todo el personal correspondiente. Esta comunicación se hará por escrito en forma de correo electrónico, e incluirá la información pertinente, tales como:
\begin{itemize}
	\item Los nombres, títulos, etc.\ de los empleados recién agregados;
	\item Los nombres de los empleados despedidos;
	\item El nombre, dirección, número de teléfono, información de contacto, etc.\ de contratistas o terceras partes recién agregados, así como si este cambio reemplaza a un contratista que anteriormente se aplicaba;
	\item Información pertinente en relación de la administración de equipos;
	\item La información pertinente relativa a la reubicación de instalación;
	\item Nombre, dirección, información de contacto, etc.\ para los clientes nuevos, así como información pertinente sobre cualquier cambio en relaciones con los clientes existentes.
\end{itemize}
	\item Cualquier pregunta  relacionada con la administración de cambios debe ser puesto en conocimiento del Gerente de Operaciones.
\end{enumerate}

\subsection{Acciones correctivas}
\begin{itemize}
	\item Cuando se presente una no conformidad en la realización del procedimiento, detectada por el supervisor o encargado del área se deberá de reportar a su superior inmediato.
	\item En caso de que la desviación sea mayor, se deberá de registrar la acción correctiva en el formato correspondiente.
\end{itemize}

\subsection{Frecuencia}

\subsubsection{Revisión}
Se revisarán los procedimientos anualmente.

\subsubsection{Actualización}
Es posible realizar una actualización a algún procedimiento previo al periodo de vigencia del documento, solo debe de indicarse la nueva fecha de actualización.

\begin{changelog}[title=Registro de cambios,simple, sectioncmd=\subsection*,label=changelog-\thesection-\MayorVer.\MenorVer6]
	\begin{version}[v=\MayorVer.\MenorVer, date=2023--01, author=Pablo E. Alanis]
		\item Cambios de formato;
		\item cambio de código;
		\item cambios en redacción.
	\end{version}

	\begin{version}[v=1.5, date=2022--01, author=Agustín M.]
		\item cambio de fecha de 04 de noviembre 2017 a 28 de enero 2019, se realizó cambio de código de DG-P-ACP a PR-015. Y no se realizaron cambios de la revisión 003 a la 004
	\end{version}
\end{changelog}
\thispagestyle{formato-PI}
\renewcommand{\MayorVer}{2}
\renewcommand{\MenorVer}{1}
\renewcommand{\Codigo}{PSA-1-PROG} %TODO
\renewcommand{\FechaPub}{2023--01}
%\renewcommand{\Edit}{2.1}
\renewcommand{\Titulo}{Manejo de proteína cruda fresca}

\section{\Titulo}\label{PRO-ManejoDeProtFresca}\index{Información documentada!tipo!procedimiento!Manejo de proteína cruda fresca}

\subsection{Objetivo}
Asegurar que los alimentos que clasifiquen como \emph{proteína fresca cruda} se almacenan de manera que no puedan causar contaminación cruzada.

\subsection{Alcance}
\begin{itemize}
	\item Personal de \emph{embarques} que se encarga de almacenar los \glspl{alimento} al ingresar a \gls{RDF};
	\item personal de \emph{aseguramiento de calidad} que se encarga de vigilar el cumplimiento de este documento y marcar áreas de oportunidad en este tema.
\end{itemize}

\subsection{Términos y definiciones}
\begin{description}
	\defglo{proteina-cruda}
	\defglo{alergeno}
	\item[\glsfirst{alimento-RTE}] \glsdesc{alimento-RTE}
	\defglo{tarima-compuesta}
\end{description}

\subsection{Documentos y/o normas relacionadas}
\begin{itemize}
	\item Programa de Sanidad;
	\item Programa de control de alérgenos.
\end{itemize}

\subsection{Procedimiento}
\subsubsection{Precauciones de seguridad}
	\begin{itemize}
		\item Usar uniforme completo.
	\end{itemize}

\subsubsection{Instrucciones}
\paragraph{Almacenaje de proteína cruda}
A continuación se detallan las consideraciones que deben de tomarse en cuenta para el almacenamiento de alimentos considerados como \emph{proteína cruda fresca:}
	\begin{enumerate}
		\item Todas las \emph{proteínas crudas frescas} deben de ser almacenadas en congelación o refrigeración\footnote{Según las especificaciones del alimento.} estas deben de almacenarse de manera que no sean causa de \emph{contaminación cruzada} con otros productos o materiales de empaque;
		\item No deben almacenarse sobre \gls{alimento-RTE};
		\item No deben de almacenarse sobre materiales de empaque u otros alimentos que se consideren \glspl{alergeno};
		\item Los productos que contengan proteína cruda fresca deben de almacenarse de forma tal que en caso de que ocurriera un derrame no contaminen otros productos, esto quiere decir que solo se deben de almacenar:
		\begin{itemize}
			\item El producto con proteína cruda fresca se debe de almacenar en la parte más baja del \textit{rack} y sobre él, los productos que no contengan proteína cruda.
			\item Por ningún motivo debe de almacenarse alimento \gls{alimento-RTE} debajo de productos que contengan proteína cruda fresca.
			\item El alimento con proteína cruda fresca se debe de almacenar sobre producto con proteína cruda fresca y no se debe de combinar con alimento \gls{alimento-RTE}
			\item Se debe de almacenar productos con proteína cruda fresca sobre producto del mismo tipo de proteína cruda \textit{e.g.\ pollo sobre pollo, pescado sobre pescado.}
			\item Todo alimento debe estar paletizado con el fin de mantener una barrera física con los alimentos almacenados lateralmente.
			\item[Nota:] En caso de derrames, revisar procedimiento de limpieza para proteínas crudas frescas.
		\end{itemize}
	\end{enumerate}

\paragraph{Embarque de proteína cruda}

\begin{enumerate}
	\item Todo alimento debe estar paletizado con el fin de mantener una barrera física con los alimentos almacenados lateralmente;\label{sec1s}
	\item En caso de que hubiera varias tarimas con proteína cruda fresca, estas deben de colocarse juntas al momento del embarque para minimizar la cercanía con otros productos;
	\item Cuando se tienen \glspl{tarima-compuesta}, las proteínas crudas frescas deben de ser acomodadas en la parte inferior de la tarima y estos alimentos deben de ser paletizados de tal manera que exista una barrera física sea colocada, separando los múltiples alimentos que no pertenezcan a la misma categoría;
	\item Por ningún motivo debe de almacenarse alimento \gls{alimento-RTE} debajo de alimentos que contengan proteína cruda fresca;
	\item El producto con proteína cruda fresca se debe de almacenar sobre producto con proteína cruda fresca.
\end{enumerate}

Ver \cref{sec1s}

\paragraph{Alérgenos}

\begin{itemize}
	\item Algunos de los alimentos que contienen proteínas crudas frescas también son consideradas \gls{alergeno} como ejemplos se pueden mencionar el huevo, mariscos, los cuales se deben de manejar con base en lo indicado a este procedimiento y al procedimiento de manejo de productos Alérgenos.
\end{itemize}

\subsection{Responsables de la actividad}

\begin{itemize}
	\item \textbf{Ejecutado} por personal de operaciones
	\item \textbf{Monitoreado} por personal de calidad
	\item \textbf{Verificado} por personal de gerencia.
\end{itemize}

\subsection{Acciones preventivas}

\begin{itemize}
	\item Se llevará a cabo una inspección. Registrando el indicador y medida correspondiente
	\item Si la desviación se repite frecuentemente se dará curso de capacitación al personal de almacén y limpieza para que realicen eficientemente su trabajo.
	\item Si después de haber capacitado al personal de almacén se siguen presentando desviaciones por causas injustificadas, se tomaran acciones más enérgicas con el personal por incumplimiento con sus deberes.
\end{itemize}

\subsection{Acciones correctivas}

\begin{itemize}
	\item En caso de no cumplimiento la tarea se deberá volver a realizar como se indica en el procedimiento.
	\item En caso de no conformidad reportar en formato de acciones correctivas (\IdFormAACC)
\end{itemize}

\subsection{Frecuencia}

\begin{itemize}
	\item Cada recepción de producto.
	\item Cada entrega de producto.
\end{itemize}


\begin{changelog}[simple, sectioncmd=\subsection*,label=changelog-2.16]
	\begin{version}[v=2.0, date=2023--01, author=Pablo E. Alanis]
		\item Cambios de formato;
		\item cambio de código;
		\item cambios en redacción.
	\end{version}

	\begin{version}[v=1.7, date=2022--01, author=Agustín M.]
		\item Cambio de fecha;
		\item No se realizaron cambios de la revisión 003 a la 004
	\end{version}
\end{changelog}


\thispagestyle{formato-PI}
\renewcommand{\MayorVer}{2}
\renewcommand{\MenorVer}{1}
\renewcommand{\Codigo}{PSA-1-PROG} %TODO
\renewcommand{\FechaPub}{2023--01}
%\renewcommand{\Edit}{2.1}
\renewcommand{\Titulo}{Lista de proveedores autorizados}

\section{\Titulo}
\label{sec:listaDeProveedores}
\index{Información documentada!tipo!lista!Proveedores autorizados, de}

% Lista De Proveedores Autorizados
\definecolor{Gallery}{rgb}{0.929,0.929,0.929}
\begin{longtblr}[
  label = {tbl:listaDeProveedores},
  entry = {Lista de proveedores},
  caption = {Lista de proveedores aprovados.}
  ]{%
  width = \linewidth,
  colspec = {Q[281]Q[210]Q[137]Q[312]},
  cells = {c},
  row{even} = {Gallery},
  }
  \toprule
  \textbf{Proveedor}                          & \textbf{Servicio}                       & \textbf{Representante}    & \textbf{Contacto }          \\
  \midrule
  Ecolab                                      & Producto De Limpieza Para Almacen       & Alfredo Montes De Oca     & {Oficina: 4739-5900         \\ \email{Alfredo.montesdeoca@ecolab.com}}\\
  HT Lavanderia Industriales                  & Lavanderia De Equipo De Proteccion Frio & Karla Tamez               & {Oficina: 8372-3141         \\ \email{Lavanely@prodigy.net.mx}}\\
  ARCE Control De Plagas                      & Control De Plagas                       & Mayra Rivera              & {Oficina: 8044-0548         \\ \email{Programacion@plagasarce.com.mx}}\\
  Centro De Capacitación En Calidad Sanitaria & Laboratorio De Agua                     & Lorena Trejo              & {Oficina: 8397-3577         \\ \email{Cotizaciones@calidadsanitaria.com}}\\
  Laboratorio Clínico Integral                & Laboratorio Clinico De Personal         & Salvador Garcia           & {Oficina: 8374-0509         \\ \email{Labintegral@yahoo.com.mx}}\\
  Tyvavy                                      & Emplaye Para Tarima                     & Carlos Martinez           & {Celular: 8115891706        \\ \email{contadormtz@hotmail.com}}\\
  MYM                                         & Equipo De Protección Personal           & ---                       & {Oficina: 8121390634}       \\
  Faxy Copy                                   & Impresora Y Copiadora                   & Rosa Isela Tello Alvarado & {Oficina: 83476274          \\ \email{Isela Tello Iselatello@faxycopy.com}}\\
  Ferreteria GM                               & Materiales De Ferreteria                & ---                       & {Oficina: 8122611964        \\ \email{ventas@ferreteriagm.com}}\\
  GEN                                         & Recoleccion De Basura                   & ---                       & {Call Center: 8001800436}   \\
  Alarmas Y Proyectos De Seguridad Del Norte  & Alarmas                                 & Humberto Castro           & {Oficina: 86472460 Ext.1009 \\ \email{Hcastro@alarmasyproyectos.com}}\\
  \bottomrule
\end{longtblr}
\thispagestyle{formato-PI}
\renewcommand{\MayorVer}{2}
\renewcommand{\MenorVer}{0}
\renewcommand{\FechaPub}{2023--01}
\renewcommand{\TipoID}{ESP}
\renewcommand{\Titulo}{Control de estibas y cuidado de producto}

\section{\Titulo}\label{ESP-ControlDeEstibas}\index{Información documentada!tipo!especificación!Control de estibas y cuidado de producto}
\renewcommand{\Codigo}{\Prog--\thesection--\TipoID} %TODO

\subsection{Objetivo}

\begin{itemize}
	\item Estandarizar los tamaños de estibas para tener un buen acomodo en el almacén y evitar cualquier tipo de riesgo hacia el personal de su mal entarimado.
	\item Evitar algún riesgo de contaminación cruzada.
\end{itemize}

\subsection{Alcance}

A todas las personas responsables para que se lleven a cabo las operaciones que se requieran para el buen cumplimiento de este procedimiento en las áreas operativas.

\subsection{Términos y definiciones}

N/A

\subsection{Documentos y/o normas relacionadas}

\begin{itemize}
	\item Programa de \gls{BPD}.
\end{itemize}

\subsection{Procedimiento}

\begin{itemize}
	\item Con esta estimación de altura de estibas se controla de manera más eficiente el almacenamiento de los productos, mas ordenados y reducir riesgos hacia todo personal, evitando así algún accidente por alguna caída de estiba.
	\item Las tarimas en rack deben de estar separadas de la pared y no estar en contacto la pared con la tarima.
	\item Respetando la regla de estiba en las tarimas se pondrá en la parte de abajo lo más pesado y lo más liviano en la parte de arriba,
	\item Cuidar que no se estiben productos alérgenos arriba de productos no alérgenos.
	\item No se puede almacenar alguna tarima con producto alimenticio junto con alguna estiba con limpiadores o químicos ya pude haber derrame y puede contaminar el producto.
	\item Se deberán manejar en tarimas separadas productos, secos, refrigerados y congelados, en caso que por necesidad de cupo en el transporte se tenga que entarimar o estibar productos refrigerados con secos, se tendrá que asegurar que estas tarimas se almacenen en ambiente refrigerado, y deberá ser documentado en la hoja de salida o documento de envío para que se cuiden las temperaturas de cada producto para prevenir riesgos o posibles daños a los productos.
	\item No se puede almacenar las cajas de empaque primario sin identificarlas o juntarlas con otras que no lo son para poder evitar la contaminación por algún derrame del empaque.
	\item Cualquiera de los productos químicos deben ser cargados en un atado independiente y cubierto con envoltura e identificado para que se le dé el manejo adecuado así como estipulado.
	\item No se permite colocar tarimas con producto arriba de otro producto al menos que se coloque un separador entre ambas tarimas y que esté autorizado por el cliente.
	\item No se puede almacenar algún producto con tarimas dañadas (uso de tarima en buen estado).
	\item No se permite pisar producto.
	\item No se permite colocar encima del producto materiales ajenos al mismo.
\end{itemize}

\subsection{Responsables de la actividad}

\begin{itemize}
	\item \textbf{Ejecutado} por personal de operaciones
	\item \textbf{Monitoreado} por personal de calidad
	\item \textbf{Verificado} por personal de gerencia.
\end{itemize}

\subsection{Acciones preventivas}

\begin{itemize}
	\item Se llevará a cabo una inspección. Registrando el indicador y medida correspondiente
	\item Si la desviación se repite frecuentemente se dará curso de capacitación al personal de almacén y limpieza para que realicen eficientemente su trabajo.
	\item Si después de haber capacitado al personal de almacén se siguen presentando desviaciones por causas injustificadas, se tomaran acciones más enérgicas con el personal por incumplimiento con sus deberes.
\end{itemize}

\subsection{Acciones correctivas}

\begin{itemize}
	\item En caso de no cumplimiento la tarea se deberá volver a realizar como se indica en el procedimiento.
	\item En caso de no conformidad reportar en \RAC.
\end{itemize}

\begin{changelog}[simple, sectioncmd=\subsection*,label=changelog-\thesection-\MayorVer.\MenorVer9]
	\begin{version}[v=\MayorVer.\MenorVer, date=2023--01, author=Pablo E. Alanis]
		% \fixed
		\item Cambio de formato;
		\item Cambios en la serialización de versiones;
		\item Correcciones ortográficas y de estilo.
	\end{version}

	\begin{version}[v=1.7, date=2022--05, author=Alonso M.]
		\item cambio de fecha;
	\end{version}	

	\shortversion{v=1.6, date=2021--05, changes=No hubo cambios}
\end{changelog}

\thispagestyle{formato-PI}
\renewcommand{\MayorVer}{2}
\renewcommand{\MenorVer}{0}
 %TODO
\renewcommand{\FechaPub}{2023--01}
\renewcommand{\TipoID}{IT}
\renewcommand{\Titulo}{Instructivo para selección de tarimas de madera}

\section{\Titulo}\index{Información documentada!tipo!especificación!Instructivo para selección de tarimas de madera}
\renewcommand{\Codigo}{\Prog--\thesection--\TipoID}

\subsection{Objetivo}

Seleccionar tarimas en buenas condiciones para evitar riesgo de contaminación de producto o un posible colapso que genere daño al mismo y al mismo tiempo minimizar la posibilidad de algún accidente dentro del almacén.

\subsection{Alcance}

Áreas operativas del almacén.

\subsection{Términos y definiciones}

\subsection{Actividades}
\subsubsection{Descartar tarimas}
Las tarimas que presenten alguno de estos daños deben ser separadas de las tarimas de uso y ser desechadas del almacén para evitar ser reutilizadas.

\begin{itemize}
	\item Tarimas con restos de material orgánico que genere una posible contaminación del producto.
	\item Tarimas con presencia de plaga.
	\item Tarimas dañadas, con desprendimiento de tablas, falta de tablas o tablas partidas.
	\item Tarimas con clavos expuestos.
\end{itemize}

\begin{changelog}[simple, sectioncmd=\subsection*,label=changelog-\thesection-\MayorVer.\MenorVer9]
	\begin{version}[v=\MayorVer.\MenorVer, date=2023--01, author=Pablo E. Alanis]
		% \fixed
		\item Cambio de formato;
		\item Cambios en la serialización de versiones;
	\end{version}

	\begin{version}[v=1.7, date=2022--05, author=Alonso M.]
		\item cambio de fecha;
	\end{version}

	\shortversion{v=1.6, date=2021--05, changes=No hubo cambios}
\end{changelog}

\renewcommand{\MayorVer}{2}
\renewcommand{\MenorVer}{1}
\renewcommand{\Codigo}{PSA-1-PROG} %TODO
\renewcommand{\FechaPub}{2023--01}
%\renewcommand{\Edit}{2.1}
\renewcommand{\Titulo}{Eliminación de humedad y escarcha}

\section{\Titulo}
\index{Información documentada!tipo!procedimiento!Eliminación de humedad y escarcha}
%\section{Eliminación de humedad y escarcha en caso de posible presencia}

\subsection{Objetivo}

Establecer un programa eficaz que determine las acciones a seguir en caso de existir Humedad o escarcha dentro de una cámara de refrigeración y/o congelación, para garantizar la protección de los productos.

\subsection{Alcance}

A todas las personas responsables para que se lleven a cabo las operaciones que se requieran para el buen cumplimiento de este procedimiento en las áreas de operaciones y mantenimiento.

\subsection{Términos y definiciones}

\begin{itemize}
	\item \textbf{Humedad:} Cantidad de agua, vapor de agua o cualquier otro líquido que está presente en la superficie o el interior de un cuerpo o en el aire.
	\item \textbf{Escarcha:} Roció o Vapor de agua condensado que se congela en una superficie o cuerpos expuestos a un enfriamiento.
\end{itemize}

\subsection{Documentos y/o normas relacionadas}

Manual de mantenimiento %TODO

\subsection{Procedimiento}

\subsubsection{Materiales}

\begin{itemize}
	\item Limpiador largo para quitar humedad
	\item Cepillo largo de cerda gruesa
	\item Cepillo para barrido (color Rojo)
	\item Recogedor (color Rojo)
\end{itemize}

\subsubsection{Precauciones de seguridad}

\begin{itemize}
	\item Usar uniforme completo.
\end{itemize}

\subsubsection{Instrucciones}

Procedimiento de Eliminación de Humedad y/o Escarcha en caso de posible presencia

\paragraph{Humedad}

Una vez que se detecte presencia de humedad, se procede a eliminar dicha humedad.

\begin{enumerate}
	\item Se quitara el producto que pudiera estar en riesgo.
	\item Con la ayuda del mango telescópico (limpiador de humedad) se procederá a quitar la humedad de los techos, y/o paredes (según donde se requiera).
	\item El departamento de mantenimiento hará una evaluación y tomara las acciones necesarias para eliminar la fuente que provoco dicha humedad.
	\item Se secara perfectamente el área afectada.
	\item Mantenimiento informara al departamento de calidad y almacén, para liberar la desviación y asegurar que el área esta lista y funcional antes de acomodar el producto retirado
\end{enumerate}


\paragraph{Escarcha}

En caso de Existencia de Escarcha:

\begin{enumerate}
	\item Se retirara el producto que pudiera estar en riesgo del área afectada.
	\item Con la ayuda del mango telescópico y un cepillo de cerdas duras, se tallara las paredes y/o los techos (según sea el caso) para retirar la escarcha de la superficie.
	\item Se retirara el exceso de humedad que quede con el limpiador de humedad
	\item El departamento de mantenimiento, evaluara el área afectada y hará las acciones correctivas necesarias para eliminar la fuente del problema.
	\item Una vez corregido el problema, mantenimiento dará aviso al departamento de calidad y almacén, para verificar y liberar el área.
	\item Se procederá hacer el acomodo del producto que fue retirado para eliminar la desviación.
\end{enumerate}

\subsection{Responsables de la actividad}

\begin{itemize}
	\item \textbf{Ejecutado} por personal de operaciones
	\item \textbf{Monitoreado} por personal de calidad
	\item \textbf{Verificado} por personal de gerencia.
\end{itemize}

\subsection{Acciones preventivas}

\begin{itemize}
	\item Se llevara a cabo una inspección. Registrando el indicador y medida correspondiente
	\item Si la desviación se repite frecuentemente se dará curso de capacitación al personal de almacén y limpieza para que realicen eficientemente su trabajo.
	\item Si después de haber capacitado al personal de almacén se siguen presentando desviaciones por causas injustificadas, se tomaran acciones más enérgicas con el personal por incumplimiento con sus deberes.
\end{itemize}

\subsection{Acciones correctivas}

\begin{itemize}
	\item En caso de no cumplimiento la tarea se deberá volver a realizar como se indica en el procedimiento.
	\item En caso de no conformidad reportar en formato de acciones correctivas F-OP-40
\end{itemize}

\subsection{Frecuencia}

Cada eventualidad.

\subsection{Historial de modificaciones}

\begin{itemize}
	\item \textbf{Tercera edición:} cambio de fecha de 04 de noviembre 2017 a 28 de enero 2019, se realizó cambio de código de M-P-EHECPP a PR-041. Y no se realizaron cambios de la revisión 003 a la 004.
	\item \textbf{Quinta edición:} febrero 2021 no se realizaron modificaciones de la revisión 04 a la 05.
	\item \textbf{Sexta edición:} febrero 2022 no se realizaron cambios de la revisión 05 a la 06.
\end{itemize}

\subsection{Listado de distribución}


\thispagestyle{formato-PI}
\renewcommand{\MayorVer}{2}
\renewcommand{\MenorVer}{0}
 %TODO
\renewcommand{\FechaPub}{2023--01}
\renewcommand{\TipoID}{PRO}

\renewcommand{\Titulo}{Atención de quejas}
\section{\Titulo}\index{Información documentada!tipo!procedimiento!Atención de quejas}\label{adm:quejas}
\renewcommand{\Codigo}{\Prog--\thesection--\TipoID}

\subsection{Objetivos}

\begin{itemize}
	\item Establecer la metodología para atender y tomar acciones correctivas de las quejas, reclamaciones y devoluciones de los clientes y consumidores.
	\item Definir los lineamientos para asegurar la correcta identificación, documentación, evaluación y notificación cuando se generen daños, faltantes o sobrantes en los productos embarcados de exportación.
\end{itemize}

\subsection{Alcance}

Este procedimiento aplica a las quejas, reclamaciones y devoluciones de clientes recibidas de los productos almacenados en Red de Fríos S.A. de C.V.

\subsection{Términos y definiciones}

\begin{itemize}
	\item \textbf{Cliente:} Toda persona que realiza actos de comercio con la Empresa llámese proveedor, intermediario o consumidor final de los productos.
	\item \textbf{Empresa:} Se refiere a \gls{RDF}
	\item \textbf{Queja:} Malestar externado por el cliente al no poder darle al producto el uso original, es decir recibir un producto fuera de especificación, atención incorrecta, cantidades incorrectas.
	\item \textbf{Devolución:} Producto que regresa el cliente, el cual ha sido evaluado y aceptado como producto no conforme.
	\item \textbf{Queja de cliente:} Son conocidos como reportes de falla de calidad que emiten el área de Aseguramiento de Calidad de nuestra empresa través del área Servicio al cliente.
	\item \textbf{Fallas de calidad por manufactura} Son defectos detectados en el producto y se derivan de alguna desviación durante el proceso de elaboración.
	\item \textbf{Fallas de calidad por mal manejo} Son defectos detectados en el producto y se derivan un mal manejo del producto durante su almacenamiento, distribución y venta del mismo.
	\item \textbf{Fallas de riesgo a la inocuidad del producto} Son riesgos detectados con efecto nocivo para la salud y de la gravedad de dicho efecto, como consecuencia de un peligro o peligros presentes en los alimentos.
	\item \textbf{Casos con riesgo de crisis} Son reportes recibidos por Gerencia de Operaciones en conjunto con Aseguramiento de Calidad a través del área de calidad del CLIENTE y que requieren una pronta respuesta.
	\item \textbf{Retiro de producto} Es la solicitud que hace el cliente de retirar un producto de su distribución o venta y regresarlo a la al almacén.
\end{itemize}

\subsection{Documentos y/o normas relacionadas}

\begin{itemize}
	\item Procedimiento de Acciones Correctivas y/o Preventivas.
	\item Procedimiento Trazabilidad
	\item Procedimiento de Food Defense
\end{itemize}

\subsection{Procedimiento}

\subsubsection{Recepción de Quejas de Calidad y/o Inocuidad o Servicio}

\begin{enumerate}
	\item Todas las quejas de producto proveniente del cliente son de la más alta prioridad y deben ser tratados de inmediato para garantizar que el cliente recibe una solución satisfactoria.
	\item Diferenciar si el punto señalado se cataloga como queja o como punto de vista.
	\item Las Quejas reportadas son aceptadas cuando estas tengan la evidencia necesaria de la falla de calidad y/o inocuidad o de servicio prestado.
	\item El Gerente de Operaciones asigna como responsable del seguimiento de las Acciones Correctivas o desviaciones al responsable de Aseguramiento de Calidad quien debe dar respuesta al cliente definiendo las acciones correctivas y/o el plan de acción en un tiempo máximo de 3 días hábiles.
	\item Aseguramiento de calidad según lo indicado en el Procedimiento de Acciones Correctivas/Preventivas; realiza la validación de las acciones correctivas y/o preventivas posterior a la fecha declarada como conclusión, esta se considera como cerrada hasta que se tenga cerrado el ciclo de cada uno de los diferentes puntos que se refieren en el Formato Solicitud de Acción Correctiva / Acción Preventiva.
\end{enumerate}

\subsubsection{Reportes de Casos con Riesgo de Crisis}

\begin{enumerate}
	\item El Gerente de Operaciones recibe los reportes de casos con riesgo de crisis a través del área de Servicios a Clientes del CLIENTE, e inmediatamente este turna al caso al área que corresponda mediante el planteamiento del Manual de Defensa del producto.
	\item El Gerente de Planta y/o el Gerente de Aseguramiento de Calidad, definen si es necesario trazabilidad de producto, de ser así se procede de acuerdo a lo indicado en el procedimiento de trazabilidad de Producto.
	\item Se lleva a cabo una investigación y se toman las acciones correctivas y preventivas necesarias a la brevedad posible. Solicitud de Acción Correctiva /Acción Preventiva.
\end{enumerate}

\subsubsection{Daños, faltantes o sobrantes en los productos embarcados}

\begin{enumerate}
	\item El Departamento de Logística recibe el reporte de reclamación de parte de los clientes vía correo electrónico, formato encuesta de servicio al cliente o llamada telefónica y se encarga de recabar la información necesaria para que el Departamento de Aseguramiento de Calidad identifique el historial del embarque en cuestión
	\item Departamento de Logística debe sondear que es lo que desea el cliente: nota de crédito, reposición del material, etc.
	\item El Departamento de Calidad, debe analizar si la reclamación es por un problema de Calidad y si procede, debe darle seguimiento por medio del procedimiento de Producto No Conforme.
	\item Si el análisis efectuado por Calidad, determina que es un problema de Logística (Faltante o sobrante de Mercancía), el responsable del departamento de logística y almacén deben darle seguimiento hasta su solución, en coordinación con la Gerencia de Operaciones y Aseguramiento de Calidad.
	\item Si la reclamación procede, se debe recabar la siguiente información:
	\begin{itemize}
		\item Numero de Orden de Venta
		\item Numero de Remisión (para establecer la rastreabilidad del producto)
		\item Información adicional que pueda ayudar a la solución del problema, como fotos.
		\item Documentos de recepción
		\item Folio de entrada
		\item Folio de salida
		\item Reporte de trazabilidad
	\end{itemize}
	\item Una vez que se haya esclarecido el problema Aseguramiento de Calidad convocará a una junta con los responsables de los departamentos involucrados.
	\item Se establece el origen del problema con las acciones correctivas correspondientes. En la junta se toma la decisión sobre la resolución que se le dará al cliente y el Departamento de Logística dará aviso al cliente en un plazo menor a 5 días hábiles.
	\item El seguimiento de las acciones tomadas sobre la reclamación y el archivo de los registros generados lo realiza el departamento de Aseguramiento de Calidad.
\end{enumerate}

\subsection{Frecuencia}

Al presentar una queja.

\subsection{Historial de modificaciones}

\begin{itemize}
	\item \textbf{Cuarta edición:} cambio de fecha de 04 de noviembre 2017 a 28 de enero 2019, se realizó cambio de código de QP-RMQ a PR-042. Y no se realizaron cambios de la revisión 003 a la 004
	\item \textbf{Quinta edición:} febrero 2020 se hizo cambio de formato y cambio de código de PR-042 a PRO-OP-016. Y no se realizaron cambios de la revisión 04 a la 05.
	\item \textbf{Sexta edición:} febrero 2021 no se realizaron cambios de la revisión 05 a la 06.
	\item \textbf{Séptima edición:} febrero 2022 no se realizaron cambios de la revisión 06 a la 07.
\end{itemize}

\subsection{Listado de distribución}


\subsection{Anexos}

F-OP-41-Solicitud de acción correctiva - Quejas.xls

\thispagestyle{formato-PI}
\renewcommand{\MayorVer}{2}
\renewcommand{\MenorVer}{1}
\renewcommand{\Codigo}{PSA-1-PROG} %TODO
\renewcommand{\FechaPub}{2023--01}
%\renewcommand{\Edit}{2.1}

\renewcommand{\Titulo}{Básculas en el almacén}
\section{\Titulo}
\index{Información documentada!tipo!especificación!Básculas en el almacén}
% \section{Básculas de almacén}

\subsection{Objetivo}

Establecer un método para el correcto uso de básculas.

\subsection{Alcance}

Aplica a las básculas en el área.

\subsection{Procedimiento}

\begin{itemize}
	\item Red de Fríos S.A. de C.V. NO CUENTA CON BASCULA dentro de sus operaciones;
	\item Red de Fríos S.A. de C.V. NO realiza actividades de PESAJE DE PRODUCTO EN BASCULA como práctica común dentro de sus procedimientos durante la operación;
	\item Red de Fríos S.A. de C.V. NO cuenta con NINGUN tipo de BASCULA PARA PESAJE dentro de las instalaciones del almacén.
\end{itemize}

\thispagestyle{formato-PI}
\renewcommand{\MayorVer}{2}
\renewcommand{\MenorVer}{1}
\renewcommand{\Codigo}{PSA-1-PROG} %TODO
\renewcommand{\FechaPub}{2023--01}
%\renewcommand{\Edit}{2.1}

\renewcommand{\Titulo}{Acciones correctivas y preventivas}
\section{\Titulo}
\index{Información documentada!tipo!procedimiento!Acciones correctivas y preventivas}
\section{Acciones correctivas y preventivas}

\subsection{Objetivo}

Definir los lineamientos para investigar las causas raíz de las No Conformidades, cerrar y verificar la efectividad de las Acciones Preventivas y Correctivas, además establecer los lineamientos para establecer las acciones necesarias para prevenir su recurrencia.

\subsection{Alcance}

Este procedimiento aplica a todas las áreas de Red de Fríos S.A de C.V. desde la recepción hasta la entrega de producto al consumidor final en unidades propias.

\subsection{Términos y definiciones}

\begin{itemize}
	\item \textbf{Observación:} Se entiende como observación a un aspecto de un requisito que podría mejorarse y que no se requiere que se haga de manera inmediata.
	\item \textbf{Acción Correctiva:} Es aquella que llevamos a cabo para eliminar la causa de un problema. Las correcciones atacan los problemas y las acciones correctivas sus causas.
	\item \textbf{Acción Preventiva:} Se anticipan a la causa, y pretenden eliminarla antes de su existencia. Evitan los problemas identificando los riesgos. Cualquier acción que disminuya un riesgo es una acción preventiva.
	\item \textbf{Corrección:} Significa acción para eliminar un defecto o una no conformidad.
	\item \textbf{Desviación interna:} No satisfacción de un límite crítico que puede llevar a la pérdida de control en un PCC (Punto Crítico de Control).
	\item \textbf{No Conformidad:} Incumplimiento de las normas o requisitos establecidos.
	\item \textbf{Requisito:} Necesidad establecida, generalmente implícita u obligatoria.
	\item \textbf{No conformidad Potencial:} Es un incumplimiento menor que no ha ocurrido aún, pero si no se hace algo al respecto, terminará ocurriendo convirtiéndose en incumplimiento real.
	\item \textbf{No Conformidad Mayor (NCM):} Es un incumplimiento que ya ocurrió en el sistema (Incumplimiento real) que afecta a un punto completo de la norma aplicable.
	\item \textbf{No Conformidad menor (NCm):} Es un incumplimiento que puede ya haber ocurrido (Real) o no haber ocurrido aún (Potencial) en el sistema de calidad y que solo afecta parcialmente a un punto de la norma.
	\item \textbf{No conformidad real:} Es un incumplimiento mayor o menor que ya ocurrió
\end{itemize}

\subsection{Documentos y/o normas relacionadas}

\begin{itemize}
	\item Procedimiento de Generación y Control de documentos y registros %TODO
	\item 1.3 - MA-OP-001 - Manual de calidad y buenas prácticas de distribución|Manual de Calidad e Inocuidad Alimentaría
	\item Procedimiento Trazabilidad. %TODO
	\item 2.22 - PRO-OP-016 - Atención a quejas|Procedimiento de Manejo de Quejas
\end{itemize}

\subsection{Procedimiento}

\subsubsection{Acción Preventiva}

1. Se deberá tomar una Acción Preventiva cuando se detecte algún problema o situación que pueda causar una desviación en el proceso y así afectar la calidad o inocuidad del producto. Se llama Preventiva porque se actúa anticipadamente para evitar que ocurra.
2. Para determinar la aplicación de acciones oportunas y efectivas se genera la siguiente clasificación de acciones:
- Acción Preventiva: Cuando se genera una acción para evitar desviaciones en el proceso y/o sistema.
- Acciones de Mejora: Cuando se generan acciones para mejorar el proceso y/o sistema.
3. Las acciones preventivas para la eliminación de no conformidades potenciales son generadas de las siguientes fuentes:
- Auditorías Internas
- Desviaciones en el Proceso de Recepción hasta la entrega final
- Desviaciones en la recepción de Materia Prima e Insumos
- Encuestas de satisfacción del cliente y reclamaciones.
4. Una vez detectadas las No conformidades potenciales, Aseguramiento de Calidad llena los datos solicitados en el Formato Solicitud de Acción Correctiva / Acción Preventiva F-OP-40 y registra la descripción de la No Conformidad Potencial en la sección
5. El Jefe de Calidad entrega el Formato Solicitud de Acción Correctiva / Acción Preventiva F-OP-40 al Responsable del Área a la que se le va aplicar la Acción Preventiva para el análisis de la causa potencial y el establecimiento de la acción preventiva.
6. El Responsable de Área realiza el análisis de la causa potencial que originó la No Conformidad Potencial, utilizando la metodología Análisis de Causa Raíz, especificado en el formato F-0016 y la registra en la sección III del Formato Solicitud de Acción Correctiva / Acción Preventiva F-OP-40; así como la acción Preventiva y las fechas compromiso de cierre y revisión de efectividad.

\begin{enumerate}
	\item El responsable del área, implementa las acciones preventivas definidas en el Formato Solicitud de Acción Correctiva / Acción Preventiva F-OP-40 antes de la fecha compromiso.
	\item Aseguramiento de Calidad verifica el cumplimiento a la implementación de la acción preventiva acordada.
	\item Si la acción preventiva fue cerrada en la fecha acordada marca SI en el Formato Solicitud de Acción Correctiva / Acción Preventiva F-OP-40.
	\item Si la acción preventiva no fue cerrada en la fecha acordada, marca NO y solicita al responsable de área registrar la causa de incumplimiento, y se define nueva fecha compromiso.
	\item El responsable de Aseguramiento de Calidad revisa la efectividad de la acción preventiva implementada en la fecha acordada registrando las acciones que realizó para medir la efectividad y registra las evidencias para la verificación de la efectividad en el Formato de Revisión de la efectividad de las Acciones Correctivas/Preventivas FC-AC-04, posteriormente firma en el campo auditor y solicita firma al Responsable de Área.
	\item Dar folio a una solicitud de Acción Correctiva / Acción Preventiva F-OP-40
\end{enumerate}

\subsubsection{Acción Correctiva}

Las acciones correctivas son generadas a partir de las siguientes fuentes:

\begin{itemize}
	\item Auditorías Internas
	\item Auditorías Externas
	\item Queja de Cliente.
\end{itemize}

\begin{enumerate}
	\item Las Acciones Correctivas generadas de las Auditorías Internas y Externas son registradas en el Formato Solicitud de Acción Correctiva / Acción Preventiva F-OP-40 se especifica la descripción de la No conformidad, la causa raíz que la originó, la acción correctiva a realizar; además del responsable de llevarla a cabo y la fecha de cierre.
	\item Aseguramiento de Calidad en conjunto con los responsables de cada acción correctiva; revisa el cumplimiento de la (s) acción (es) de acuerdo a la fecha de cierre, a fin de que se revise su efectividad.
	\item Cuando se tiene una queja de cliente, se registra en el Formato de Recepción de Quejas F-0015, de acuerdo al Procedimiento de Manejo de Quejas QP-RMQ y se genera un análisis para determinar la causa raíz y registrar la acción en el Formato Solicitud de Acción Correctiva / Acción Preventiva F-OP-40 para corregir la No conformidad y si es necesario realizar un Retiro del Producto especificado en el Procedimiento P-T-T Trazabilidad. El control de las quejas se lleva en el formato F-0021 Formato Concentrado de Quejas.
	\item Ejemplos de No Conformidades reportadas por el cliente:
	\begin{itemize}
		\item Mala calidad en los productos.
		\item Daño en almacén, proceso o en embarque.
		\item Problemas de inocuidad del producto.
	\end{itemize}
\end{enumerate}

\subsubsection{Acciones correctivas generadas de Auditorias Interna o Externas}

\begin{enumerate}
	\item En el Caso de la Auditoría Interna el Auditor es quien registra las No Conformidades en el Formato Solicitud de Acción Correctiva / Acción Preventiva F-OP-40 y la entrega al responsable de Aseguramiento de Calidad para que lo entregue al área correspondiente y se generen los análisis de causas raíz y se establezcan las acciones correctivas necesarias.
	\item En el Caso de la Auditoría Interna el Auditor es quien registra las No Conformidades en el Formato Solicitud de Acción Correctiva / Acción Preventiva F-OP-40 y la entrega al responsable de Aseguramiento de Calidad para que lo entregue al área correspondiente y se generen los análisis de causas raíz y se establezcan las acciones correctivas necesarias.
	\item En el Caso de Auditoría externas el responsable de Aseguramiento de Calidad es quien registra las No Conformidades en el Formato Solicitud de Acción Correctiva / Acción Preventiva F-OP-40 y lo entrega al responsable de área para el análisis de causa raíz y establecimiento de acciones correctivas.
	\item El responsable del área auditada, firma el Formato Solicitud de Acción Correctiva / Acción Preventiva F-0013, realiza una investigación para determinar la causa raíz de la desviación y registra el resultado obtenido y posteriormente regresa el Formato al responsable de Aseguramiento de Calidad.
	\item El Responsable de Área realiza el análisis de la causa potencial que originó la No Conformidad Potencial, utilizando la metodología Análisis de Causa Raíz, especificado en el formato F-0016 y la registra en la sección II del formato F-0016; así como la acción Correctiva y las fechas compromiso de cierre y revisión de efectividad.
\end{enumerate}

\subsubsection{Seguimiento y Validación de la Acción Correctiva de Auditoría Interna}

\begin{enumerate}
	\item Aseguramiento de verifica el cumplimiento y la implementación de la acción correctiva acordada y la efectividad de la misma:
	\item Si la acción correctiva fue cerrada en la fecha acordada se marca SI en el Formato Solicitud de Acción Correctiva / Acción Preventiva \IdFormAACC.
	\item Si la Acción Correctiva no fue cerrada en la fecha acordada marca NO en el Formato Solicitud de Acción Correctiva / Acción Preventiva \IdFormAACC, se debe registrar la causa del porque no se cumplió, en este caso Aseguramiento de Calidad informa al Responsable del área, para que validen la causa y acuerden nueva fecha de cumplimiento.
	\item El responsable de Aseguramiento de Calidad revisa la efectividad de la acción preventiva implementada en la fecha acordada registrando las acciones que realizó para medir la efectividad y registra las evidencias para la verificación de la efectividad en el Formato de Revisión de la efectividad de las Acciones Correctivas/Preventivas F-0014, posteriormente firma en el campo auditor y solicita firma al Responsable de Área.
	\item Cuando la Acción Correctiva es efectiva se envía copia a todo el personal involucrado con asunto de cerrada. La efectividad de la acción correctiva la revisa el encargado del departamento al que corresponda.
	\item En caso de necesitar recursos como por ejemplo: herramientas, materiales o equipos para llevar acabo las Acciones Correctivas o Preventivas se notifica a la Gerencia para su autorización.
\end{enumerate}

\subsubsection{Registros}

\begin{enumerate}
	\item Los registros de Acción Correctiva/Acción Preventiva son almacenados y mantenidos por la Gerencia de acuerdo al área de responsabilidad.
\end{enumerate}

\subsection{Frecuencia}

Al presentarse una desviación cuando se detecte algún problema o situación que pueda causar una desviación en el proceso y así afectar la calidad o inocuidad del producto.

\subsection{Historial de modificaciones}

\begin{itemize}
	\item \textbf{Cuarta edición:} Agosto 2021 se hizo cambio de formato y cambio de código de D-P-AC/AP a PRO-OP-003. Y no se realizaron cambios de la revisión 03 a la 04.
	\item Quinta edición: febrero 2022 no se realizaron cambios de la revisión 04 a la 05.
\end{itemize}

\subsection{Listado de distribución}


\thispagestyle{formato-PI}
\renewcommand{\MayorVer}{2}
\renewcommand{\MenorVer}{0}
 %TODO
\renewcommand{\FechaPub}{2023--01}
\renewcommand{\TipoID}{PRO}

\renewcommand{\Titulo}{Elaboración y control de documentos}
\section{\Titulo}\index{Información documentada!tipo!procedimiento!Elaboración y control de documentos}
\renewcommand{\Codigo}{\Prog--\thesection--\TipoID}

\subsection{Objetivo}

Asegurar que los documentos del Sistema de Calidad se preparan, revisan, aprueban, publican, distribuyen, implementan y administran de acuerdo a lo especificado en este procedimiento.

\subsection{Alcance}

Este procedimiento aplica a todos los documentos generados internamente por cada uno de los departamentos o de fuentes externas y personal involucrado en la elaboración, consulta y actualización de documentos en el Sistema de Calidad.

\subsection{Términos y definiciones}

\begin{itemize}
	\item \textbf{Copia Controlada:} Documento que se emite físicamente por el Sistema de Calidad el cual incluye esta leyenda en el mismo, para poder mantener el control de documentos y asegurar que el documento empleado es la versión vigente del documento en cuestión.
	\item \textbf{Copia no controlada:} Documento (por diferentes razones) del cual se emite una copa física pero que no es actualizado al sufrir cambios.
	\item \textbf{Documento Interno:} Documento que forma parte del sistema de calidad de la compañía, los documentos que incluye el sistema de calidad son los siguientes:
	\begin{itemize}
		\item Listado Maestro
		\item Procedimiento Estándar
		\item POE
		\item POES
		\item Instrucciones de Trabajo
		\item Reglamentos
		\item Formatos
		\item Registros
	\end{itemize}
	\item \textbf{Control de Índices:} Documento en el que se enumeran los documentos del sistema de calidad; incluye código, titulo, edición vigente y fecha de revisión.
	\item \textbf{Procedimiento Estándar De Operación (PRO) Y POE:} Es un término que hace referencia a la acción de\textbf{proceder}, que significa actuar de una forma determinada. El concepto, por otra parte, está vinculado a un o una manera de ejecutar algo. Consiste en\textbf{seguir ciertos pasos predefinidos y de manera secuencial}para desarrollar una labor de manera eficaz.
	\item \textbf{Procedimiento Estándar de Sanitización (POES):} Es un término que hace referencia a la acción de\textbf{proceder}, que significa actuar de una forma determinada. El concepto, por otra parte, está vinculado a un o una manera de ejecutar algo. Consiste en\textbf{seguir ciertos pasos predefinidos y de manera secuencial}para desarrollar una labor de limpieza de manera más eficaz.
	\item \textbf{Instrucción de Trabajo:} Desarrollan secuencialmente los pasos a seguir para la correcta realización de un trabajo específico y que normalmente involucra a una sola persona o área de responsabilidad.
	\item \textbf{Especificaciones:} Documento que describe en forma detallada las características o requisitos técnicos de un servicio o producto y que deben cumplirse para lograr un propósito determinado. Pueden ser documentos internos y/o externos.
	\item \textbf{Anexos:} Documentos que complementan lo descrito en un procedimiento. Estos pueden ser referencias bibliográficas, normas, formatos, esquemas, gráficos etc.
	\item \textbf{Formato:} Documento en el que se plasman los resultados obtenidos y actividades realizadas y que representara evidencia de cumplimiento de la o las actividades.
	\item \textbf{Registros:} Es un formato que ha sido completado y que se resguarda como evidencia de la realización de actividades específicas.
\end{itemize}

\subsection{Responsables de la actividad}
\subsubsection{Aseguramiento de Calidad}
\begin{itemize}
	\item Coordina y verifica que se lleve a cabo la actualización en tiempo y forma de los documentos y registros del Sistema de Calidad.
	\item Administra la totalidad de documentos del Sistema de Calidad.
	\item Asigna la numeración correspondiente a los documentos elaborados.
	\item Responsable de la aprobación, revisión y emisión de documentos de acuerdo con su área de responsabilidad.
\end{itemize}

\subsubsection{Gerente de Mantenimiento}
\begin{itemize}
	\item Responsable de la elaboración, actualización, revisión, emisión y aplicación de los documentos generados en su área, así como de los documentos del Sistema de Calidad donde tenga injerencia.
\end{itemize}

\subsubsection{Gerente de Operaciones}
\begin{itemize}
	\item Responsable de la elaboración, actualización, revisión, emisión y aplicación de los documentos generados en su área, así como de los documentos del Sistema de Calidad donde tenga injerencia.
\end{itemize}

\subsection{Procedimiento}
\subsubsection{Generación de Documentos}

Determinar la necesidad de documentar y validar con el jefe inmediato del área. Identificar el tipo de documento necesario a generar y proceder a documentar y/o generar el documento en base a los siguientes lineamientos:

\paragraph{Estructura}

Todos los documentos deben contenerlo y debe incluir:

\begin{itemize}
	\item Encabezado
	\item Logotipo
	\item Título del documento
	\item Código de identificación del documento
	\item Revisión actual del documento
	\item Fecha de emisión
\end{itemize}

\paragraph{Edición}

Indica el número de veces que el documento ha sido modificado y/o adecuado, se inicia con el número que corresponde a la primera emisión.

\paragraph{Fecha de Publicación}

Corresponde a la fecha en que el documento se elaboró.

\paragraph{Contenido}

Corresponde a la información que contiene el documento y como debe ser presentada.

\begin{enumerate}
	\item \textbf{Objetivo:} Finalidad para la cual fue creado el documento.
	\item \textbf{Alcance:} Áreas o puestos para los cuales es aplicable el documento.
	\item \textbf{Términos y definiciones:} Conjunto de términos o palabras propias utilizadas en un procedimiento.
	\item \textbf{Responsabilidades:} Indica los compromisos de los participantes en el desarrollo de un documento.
	\item \textbf{Procedimiento:} Desarrollo de la actividad o proceso a seguir paso a paso.
	\item \textbf{Frecuencia:} Es la frecuencia con la que se debe realizar cada actividad.
	\item \textbf{Documentos Relacionados:} Todos los documentos con relación al procedimiento.
	\item \textbf{Anexos:} Agregados de un trabajo que se incluyen al final del documento y ofrecen información adicional.
	\item \textbf{Formatos:} Todos los formatos con relación al procedimiento.
	\item \textbf{Historial de modificaciones:} Muestra cual ha sido el histórico de las modificaciones o adecuaciones que ha tenido el documento. Se establece la revisión anterior, revisión actual, fecha de las revisiones y una descripción de las modificaciones realizadas.
	\item \textbf{Listado de Distribución:} Lista donde se menciona a quien se le ha compartido el procedimiento o documento y cuenta con copia.
\end{enumerate}

\paragraph{Tipo de letra}

Los documentos se deben desarrollar con base a la estructura del presente documento y deben ser escritos en letra Calibri 12 pt. Los títulos deben ser en negrilla, combinando mayúsculas y minúsculas.

\paragraph{Numeración}

\begin{enumerate}
	\item Inicie el esquema de numeración partiendo del número 1 y según se requiera, desglose el mismo agregando un punto y un decimal. Por ejemplo: 1, 1.1, 1.1.1. De ser necesario en cada punto utilice incisos y /o viñetas.
	\item Revisar el documento elaborado, para asegurar que se cumplen todos los puntos para la estandarización; que incluye el formato, estructura y contenido.
	\item Llevar a cabo la ruta de aprobación del documento.
	\item Si el documento es aprobado proceder a la difusión e implementación del mismo. 
	\item Los formatos tienen como mínimo (cuando aplique):
	\begin{itemize}
		\item Código
		\item Revisión
		\item Titulo
		\item Numero de hojas
		\item Firma de la(s) persona(s) responsable(s) de llenar el registro
		\item Firma de autorización y/o verificación
	\end{itemize}
	\item Los registros son llenados en cada uno de sus espacios, cuando un espacio no es utilizado, por no ser necesario se coloca una línea horizontal o diagonal para cancelarlo.
	\item Todos los documentos y registros deben resguardarse de manera que no sufran daño y/o deterioro.
\end{enumerate}

\paragraph{Revisión y aprobación de Documentos}

\begin{itemize}
	\item Si el documento presenta faltas de cumplimiento al presente procedimiento, la persona responsable de la revisión notifica al puesto que generó el documento para que proceda a realizar las modificaciones señaladas y repita las actividades anteriores.
	\item Si el documento fue aprobado por los involucrados en la ruta de aprobación se procede a su difusión e implementación.
\end{itemize}

\paragraph{Distribución, implementación y control de Documentos}

\begin{enumerate}
	\item Los documentos aprobados serán incluidos con la codificación correspondiente en el Listado Maestro de documentos para su control.
	\item Ya liberado el documento (emisión inicial o cambios) el Jefe de Aseguramiento de Calidad realizará las copias y distribuirá el documento de acuerdo a 12 Listado de Distribución.
	\item Cuando se requiera una copia física de los documentos aprobados y vigentes, se realiza una solicitud a Aseguramiento de Calidad, cada copia deberá ser sellada como se indica en el Anexo A según corresponda.
	\item Para solicitar documentos con copia no controlada, se tiene que mandar una solicitud por escrito al Jefe de Aseguramiento de Calidad especificando el motivo de la solicitud.
\end{enumerate}

\paragraph{Cambios, alta o baja de documentos}

\begin{enumerate}
	\item Para realizar cambios, alta o baja de documentos se debe seguir y cumplir los pasos del 5.1 al 5.2. Se debe informar a Aseguramiento de Calidad a través del formato O-F-RCP Solicitud de Alta, Baja o Cambio de documentos.
	\item Aseguramiento de Calidad revisa con cada una de las bases de la solicitud O-F-RCP y determinar si procede, posteriormente se hace el proceso de revisión y autorización de documentos y se actualiza el listado maestro de documentos. Aseguramiento de Calidad es responsable de:
	\begin{enumerate}
		\item Cambiar el contenido del documento según los cambios necesarios para adecuarlo al proceso y/o revisar la propuesta de cambio que haga el área solicitante.
		\item Cambiar la revisión de los documentos (Procedimientos, Instructivos o registros).
		\item Llenar el punto 10 del procedimiento, Historial de Cambios, con las especificaciones generales del cambio.
		\item Actualizar el Listado Maestro.
		\item Para finalizar, el Jefe de Aseguramiento de Calidad imprime el documento autorizado y comienza el proceso de distribución y difusión.
	\end{enumerate}
	\item Cuando se realice algún cambio, se debe actualizar el punto 10.0 Historial de Modificaciones del Documento, este punto muestra cual ha sido el histórico de las modificaciones o adecuaciones que ha tenido el documento.
\end{enumerate}

\paragraph{Control de documentos obsoletos:}

\begin{enumerate}
	\item Todo documento tiene 1 año de vigencia. Posterior a esta fecha se debe realizar una revisión, para asegurar que la información es actual y corresponde al proceso y/o actividad que se ejecutan.
	\item Retire de los puntos de uso los documentos obsoletos que haya distribuido físicamente de acuerdo al listado de distribución y remplace el documento por la revisión vigente para que el personal involucrado siempre tenga la versión actualizada para ejecutar sus actividades; asegure que todos los documentos obsoletos sean retirados y remplazados.
	\item Destruya las copias de los documentos obsoletos y destrúyalos conservando un ejemplar que se identificará con la leyenda de \textbf{DOCUMENTO OBSOLETO} y consérvelos con base en los lineamientos de Control de Registros.
\end{enumerate}

\subsubsection{Respaldo de la información}
\begin{enumerate}
	\item La documentos electrónicos del Sistema de Calidad es respaldada por el Jefe de Aseguramiento de Calidad con una frecuencia mensual.
\end{enumerate}

\subsection{Frecuencia}
Cada vez que sea necesaria la publicación o modificación de algún documento del sistema de calidad.

\begin{changelog}[simple, sectioncmd=\subsection*,label=changelog-\thesection-\MayorVer.\MenorVer5]
	\begin{version}[v=\MayorVer.\MenorVer, date=2023--01, author=Pablo E. Alanis]
		% \fixed
		\item Cambio de formato;
		\item Cambios en la serialización de versiones;
	\end{version}

	\begin{version}[v=1.6, date=2022--05, author=Alonso M.]
		\item cambio de fecha;
	\end{version}

	\shortversion{v=1.5, date=2021--05, changes=No hubo cambios}
\end{changelog}

\thispagestyle{formato-PI}
\renewcommand{\MayorVer}{2}
\renewcommand{\MenorVer}{0}
 %TODO
\renewcommand{\FechaPub}{2023--01}
\renewcommand{\TipoID}{PRO}

\renewcommand{\Titulo}{Entrega de pequeñas cantidades}
\section{\Titulo}\index{Información documentada!tipo!política!Entrega de pequeñas cantidades}
\renewcommand{\Codigo}{\Prog--\thesection--\TipoID}

\subsection{Objetivos}

\begin{itemize}
	\item \emph{\textbf{Establecer}} una política de despacho de pequeñas cantidades de productos del cliente que se encuentren almacenados en RDF;
	\item \emph{\textbf{Establecer}} un protocolo para la entrega de pequeñas cantidades de productos;
	\item \emph{\textbf{Establecer}} las medidas de seguridad que se deben de tomar al hacer este procedimiento.
\end{itemize}

\subsection{Alcance}

\begin{itemize}
	\item Cliente;
	\item Personal de embarques;
	\item Personal de aseguramiento de calidad;
	\item Personal de mesa de control.
\end{itemize}

\subsection{Términos y definiciones}

\begin{description}
	\defglo{cadena-alimentaria}
	\defglo{cantidades-pequeñas}
\end{description}

\subsection{Documentos y/o normas relacionadas}
\begin{itemize}
	\item \nameref{PRO-ManejoDeProtFresca}
	\item \nameref{ESP-ControlDeEstibas}
\end{itemize}

\subsection{Procedimiento}

\subsubsection{Precauciones de seguridad}
\begin{itemize}
	\item \emph{El cliente} será el responsable, una vez que el producto almacenado en RDF salga de la instalación, de que se procure la cadena de frío;
	\item \emph{RDF} no se hace responsable de el mal almacenamiento después de haber sido despachadado el producto;
	\item Para la entrega de cantidades pequeñas no se requiere de el uso de unidades climatizadas, pero se incentiva.
	\item Debido a que no todas las unidades de transporte son adecuadas para \emph{enrampar,} en caso de que la unidad del cliente no cuente con esta especificación, se entregaran los productos pasando por \emph{mesa de control} y posteriormente se disponen en la unidad  \emph{del cliente.}
	\item Para procurar la inocuidad del producto, no se puede emplear este procedimiento para entregar mercancía a granel;
	\item En el caso de ciertos clientes, RDF funciona como el punto final de la cadena de frío.
\end{itemize}

\subsubsection{Instrucciones}
\begin{enumerate}
	\item Se elabora la orden de salida;
	\item Se ubica el producto que va a ser despachado y se actualiza el inventario en WMS;
	\item Se extraen la cantidad requerida por \emph{el cliente} de forma manual por el \emph{personal de embarques u operaciones;}
	\item Se transfieren a la unidad climatizada o no climatizada \emph{del cliente,} pasando por \emph{mesa de control} y posteriormente se disponen en la unidad del cliente.
\end{enumerate}

\subsection{Responsables de la actividad}
\begin{itemize}
	\item \textbf{Aseguramiento de calidad:} es el responsable de establecer este procedimiento así como sus futuras actualizaciones, así como el monitoreo del cumplimiento de las instrucciones de trabajo establecidas;
	\item \textbf{Personal de embarques:} es el responsable de llevare acabo este procedimiento;
	\item \textbf{Personal de mesa de control:} es el responsable del control de inventario en el almacén.
\end{itemize}

\subsection{Acciones preventivas}
De forma aleatoria, cuando se presente el caso, el \emph{personal de aseguramiento de calidad} verificara que se sigan las instrucciones de trabajo establecidas en este documento;

\subsection{Acciones correctivas}
\begin{itemize}
	\item De no cumplirse con el procedimiento establecido, se procederá a capacitar a los responsables de llevar esta actividad \emph{(vide supra).}
	\item En el caso de que se repita una no conformidad en éste procedimiento se procederá a llenar el formulario de acciones correctivas y se escalará con \emph{el gerente de operaciones.}
\end{itemize}

\subsection{Frecuencia}

Cuando se presente la necesidad.

\begin{changelog}[simple, sectioncmd=\subsection*,label=changelog-\thesection-\MayorVer.\MenorVer6]
	\begin{version}[v=\MayorVer.\MenorVer, date=2023--01, author=Pablo E. Alanis]
		% \fixed
		\item Cambio de formato;
		\item Cambios en la serialización de versiones;
	\end{version}

	\begin{version}[v=1.0, date=2022--05, author=Alonso M.]
		\item Primera versión.
	\end{version}

\end{changelog}

\thispagestyle{formato-PI}
\renewcommand{\MayorVer}{1}
\renewcommand{\MenorVer}{0}
\renewcommand{\FechaPub}{2023--01}
\renewcommand{\TipoID}{G}
\renewcommand{\Titulo}{Programa de Seguridad Alimenticia --- Glosario}
\renewcommand{\Codigo}{\Prog--\TipoID}
% \printunsrtglossary[type=acronym,nonumberlist,style=long]
% \printunsrtglossary[type=\glsdefaulttype]
\printglossary
% \printunsrtglossary[type=acronym,nonumberlist]
% \printunsrtglossary[type=\glsdefaulttype,nonumberlist]
\thispagestyle{formato-PI}
\renewcommand{\MayorVer}{1}
\renewcommand{\MenorVer}{0}
\renewcommand{\FechaPub}{2023--01}
%\renewcommand{\Edit}{01}
\renewcommand{\TipoID}{I}
\renewcommand{\Titulo}{Programa de Buenas Prácticas de Distribución --- Indice alfabético}
\renewcommand{\Codigo}{\Prog--\TipoID}
\printindex
\end{document}