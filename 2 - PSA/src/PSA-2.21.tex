\thispagestyle{formato-PI}
\renewcommand{\MayorVer}{2}
\renewcommand{\MenorVer}{0}
 %TODO
\renewcommand{\FechaPub}{2023--01}
\renewcommand{\TipoID}{PRO}
\renewcommand{\Titulo}{Eliminación de humedad y escarcha}

\section{\Titulo}\index{Información documentada!tipo!procedimiento!Eliminación de humedad y escarcha}
\renewcommand{\Codigo}{\Prog--\thesection--\TipoID}

\subsection{Objetivo}

Establecer un programa eficaz que determine las acciones a seguir en caso de existir Humedad o escarcha dentro de una cámara de refrigeración y/o congelación, para garantizar la protección de los productos.

\subsection{Alcance}

A todas las personas responsables para que se lleven a cabo las operaciones que se requieran para el buen cumplimiento de este procedimiento en las áreas de operaciones y mantenimiento.

\subsection{Términos y definiciones}

\begin{itemize}
	\item \textbf{Humedad:} Cantidad de agua, vapor de agua o cualquier otro líquido que está presente en la superficie o el interior de un cuerpo o en el aire.
	\item \textbf{Escarcha:} Roció o Vapor de agua condensado que se congela en una superficie o cuerpos expuestos a un enfriamiento.
\end{itemize}

\subsection{Documentos y/o normas relacionadas}

Manual de mantenimiento %TODO

\subsection{Procedimiento}

\subsubsection{Materiales}

\begin{itemize}
	\item Limpiador largo para quitar humedad
	\item Cepillo largo de cerda gruesa
	\item Cepillo para barrido (color Rojo)
	\item Recogedor (color Rojo)
\end{itemize}

\subsubsection{Precauciones de seguridad}

\begin{itemize}
	\item Usar uniforme completo.
\end{itemize}

\subsubsection{Instrucciones}

Procedimiento de Eliminación de Humedad y/o Escarcha en caso de posible presencia

\paragraph{Humedad}

Una vez que se detecte presencia de humedad, se procede a eliminar dicha humedad.

\begin{enumerate}
	\item Se quitara el producto que pudiera estar en riesgo.
	\item Con la ayuda del mango telescópico (limpiador de humedad) se procederá a quitar la humedad de los techos, y/o paredes (según donde se requiera).
	\item El departamento de mantenimiento hará una evaluación y tomara las acciones necesarias para eliminar la fuente que provoco dicha humedad.
	\item Se secara perfectamente el área afectada.
	\item Mantenimiento informara al departamento de calidad y almacén, para liberar la desviación y asegurar que el área esta lista y funcional antes de acomodar el producto retirado
\end{enumerate}


\paragraph{Escarcha}

En caso de Existencia de Escarcha:

\begin{enumerate}
	\item Se retirara el producto que pudiera estar en riesgo del área afectada.
	\item Con la ayuda del mango telescópico y un cepillo de cerdas duras, se tallara las paredes y/o los techos (según sea el caso) para retirar la escarcha de la superficie.
	\item Se retirara el exceso de humedad que quede con el limpiador de humedad
	\item El departamento de mantenimiento, evaluara el área afectada y hará las acciones correctivas necesarias para eliminar la fuente del problema.
	\item Una vez corregido el problema, mantenimiento dará aviso al departamento de calidad y almacén, para verificar y liberar el área.
	\item Se procederá hacer el acomodo del producto que fue retirado para eliminar la desviación.
\end{enumerate}

\subsection{Responsables de la actividad}

\begin{itemize}
	\item \textbf{Ejecutado} por personal de operaciones
	\item \textbf{Monitoreado} por personal de calidad
	\item \textbf{Verificado} por personal de gerencia.
\end{itemize}

\subsection{Acciones preventivas}

\begin{itemize}
	\item Se llevará a cabo una inspección. Registrando el indicador y medida correspondiente
	\item Si la desviación se repite frecuentemente se dará curso de capacitación al personal de almacén y limpieza para que realicen eficientemente su trabajo.
	\item Si después de haber capacitado al personal de almacén se siguen presentando desviaciones por causas injustificadas, se tomaran acciones más enérgicas con el personal por incumplimiento con sus deberes.
\end{itemize}

\subsection{Acciones correctivas}

\begin{itemize}
	\item En caso de no cumplimiento la tarea se deberá volver a realizar como se indica en el procedimiento.
	\item En caso de no conformidad reportar en \RAC.
\end{itemize}

\subsection{Frecuencia}

Cada eventualidad.

\subsection{Historial de modificaciones}

\begin{itemize}
	\item \textbf{Tercera edición:} cambio de fecha de 04 de noviembre 2017 a 28 de enero 2019, se realizó cambio de código de M-P-EHECPP a PR-041. Y no se realizaron cambios de la revisión 003 a la 004.
	\item \textbf{Quinta edición:} febrero 2021 no se realizaron modificaciones de la revisión 04 a la 05.
	\item \textbf{Sexta edición:} febrero 2022 no se realizaron cambios de la revisión 05 a la 06.
\end{itemize}

\subsection{Listado de distribución}

