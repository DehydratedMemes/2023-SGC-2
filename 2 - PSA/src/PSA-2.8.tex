\thispagestyle{formato-PI}
\renewcommand{\MayorVer}{2}
\renewcommand{\MenorVer}{1}
\renewcommand{\Codigo}{PSA-1-PROG} %TODO
\renewcommand{\FechaPub}{2023--01}
%\renewcommand{\Edit}{2.1}
\renewcommand{\Titulo}{Política de aprobación de proveedores}

\section{\Titulo}
\index{Política!aprobación de proveedores, de}

% \section{Política de aprobación de proveedores}

\subsection{Objetivo}

Establecer el procedimiento para \emph{evaluar, seleccionar y monitorear} a los proveedores de insumos, para garantizar la capacidad de cada proveedor en lo referente la calidad del servicio proporcionado y de sus recursos brindados.

\subsection{Alcance}

Este procedimiento aplica para todos los empleados de \GLS{RDF} involucrados en la evaluación, selección y monitoreo de proveedores de insumos.

\subsection{Términos y definiciones}
\begin{description}
    \defglo{requisito}
    \defglo{proveedor}
    \defglo{proveedor-externo}
    \glspl{especificacion}
\end{description}
\subsection{Documentos y/o normas relacionadas}

\begin{itemize}
    \item Lista de proveedores autorizados
    \item Formato Evaluación de Proveedores
\end{itemize}

\subsection{Procedimiento}
\begin{enumerate}
    \item La selección de un nuevo proveedor se ejecuta bajo una evaluación en equipo; este Equipo está formado por miembros de Gerencia de Operaciones, Calidad, Almacenes, Embarques, Mantenimiento y Compras. Todos los miembros del equipo \textit{(vide infra)} son responsables de proveer retroalimentación para evaluar y determinar las capacidades del proveedor.
    \item La \Gls{LPA} es creada en conjunto por los miembros del equipo de Evaluación de Proveedores (ver \cref{sec:listaDeProveedores}).
    \item El comité de evaluación de proveedores es el encargado de programar las auditorías de evaluación anual a proveedores. Durante estas auditorías se verificará que el proveedor de insumos tiene la capacidad de proveer el insumo requerido y que éste es fabricado bajo condiciones tales que no representa un riesgo para la sanidad de los alimentos.
    \item Se debe enviar una notificación de la auditoría al proveedor al menos con \NotifAuditoriaAProveedor~de anticipación.
    \item Los resultados de las auditorías se envían al proveedor en un plazo no mayor a \PlazoResultadoAProveedor~posteriores a la auditoría.
    \item En caso de tener desviaciones, el proveedor debe enviar un Plan de Acción en un plazo máximo de \PlazoPlanDeAccionAuditProveedor posteriores a la recepción de los resultados de la auditoría.
    \item Los proveedores deberán cumplir con los siguientes puntos de acuerdo con el estatus en el que se encuentren:
          \begin{enumerate}
              \item \textbf{Proveedor Aprobado}
                    \begin{itemize}
                        \item Entrega de producto de acuerdo a las \glspl{especificacion} de \Gls{RDF}
                        \item En el caso de tener desviaciones presenta acciones correctivas en al menos un \qty{80}{\percent} de los hallazgos encontrados durante la auditoría.
                        \item Cumple con los tamaños de lote especificados y plazos de entrega.
                        \item Se mantiene calidad en el insumo entregado.
                    \end{itemize}
              \item \textbf{Proveedor Aprobado Condicionado}
                    \begin{itemize}
                        \item Entrega de producto de acuerdo a las \glspl{especificacion} de \Gls{RDF}
                        \item En el caso de tener desviaciones presenta acciones correctivas en al menos un \qty{60}{\percent} de los hallazgos encontrados durante la auditoría.
                        \item Cumple con los tamaños de lote especificados y plazos de entrega.
                        \item Se mantiene calidad en lo entregado.
                    \end{itemize}
              \item \textbf{Proveedor Oportunidad}
                    \begin{itemize}
                        \item El proveedor no ha sido evaluado, pero no existe otro material alterno u otro proveedor aprobado para cubrir con oportunidad la compra de un insumo determinado.
                    \end{itemize}
              \item \textbf{Proveedor Rechazado}
                    \begin{itemize}
                        \item No cumple con las especificaciones/orden de compra entregadas.
                        \item El proveedor no cumple con los estándares de calidad, precio, oportunidad de entrega, reacción a emergencias, etc.\ que la empresa requiere y el Equipo de Evaluación de Proveedores decide buscar otra alternativa.
                    \end{itemize}
          \end{enumerate}
    \item Auditoría a Proveedores: La auditoría se aplicará a todos los proveedores de insumos de Red de Fríos bajo los criterios que a continuación se enumeran, esto es con la finalidad de conocer el status de cada uno de ellos y trabajar en forma conjunta para obtener a corto plazo los beneficios de contar con proveedores aprobados.
\end{enumerate}

\begin{note}[Miembros del equipo de evaluación de proveedores] \label{miembrosEqProveedores}
Para la evaluación de proveedores, debe de contemplarse la presencia de:
    \begin{itemize}
        \item Jefe de Aseguramiento de Calidad
        \item Gerente de Operaciones
        \item Jefe de Almacén
        \item Gerente de Mantenimiento
        \item Encargado de Compras
    \end{itemize}
\end{note}

\begin{longtblr}[
    label = {tbl:criterios-proveedores},
    caption = {Criterios de evaluación hacia los proveedores.},
    entry = {Criterios de evaluación hacia los proveedores.},
    ]{%
    width = \linewidth,
    colspec = {Q[617]Q[163]Q[158]},
    cells = {c},
    row{even} = {Gallery},
    %hline{1,10} = {-}{0.08em},
    %        hline{2,9} = {-}{0.05em},
    }
    \toprule
    \textbf{Criterio}                                                                                                & \textbf{Calificación} & \textbf{Ponderación} \\
    \midrule
    Precio competitivo y cumplimiento de especificaciones                                                            & \num{30}              & \qty{15}{\percent}   \\
    Cumplimiento de fechas y cantidades comprometidas                                                                & \num{40}              & \qty{20}{\percent}   \\
    Capacidad para surtir los requerimientos                                                                         & \num{30}              & \qty{15}{\percent}   \\
    Capacidad para reaccionar a picos de demanda                                                                     & \num{30}              & \qty{15}{\percent}   \\
    El insumo se produce y/o almacena en condiciones que no representan un riesgo para la inocuidad de los alimentos & \num{40}              & \qty{20}{\percent}   \\
    No se observa presencia de plagas y preferentemente se cuenta con un programa de manejo de plagas.               & \num{20}              & \qty{10}{\percent}   \\
    Las condiciones generales de los edificios e instalaciones no ponen en riesgo el abasto del insumo               & \num{10}              & \qty{5}{\percent}    \\
    \textbf{PUNTUACIÓN TOTAL}                                                                                        & \num{200}             & \qty{100}{\percent}  \\
    \bottomrule
\end{longtblr}

De acuerdo a la siguiente tabulación, el proveedor recibe un dictamen de Aprobado, Aprobado Condicionado o Rechazo:

\begin{longtblr}[
    label = {tbl:Calificaciones-Proveedores},
    caption = {Posibles calificaciones obtenibles en la evaluación de proveedores.},
    entry = {Posibles calificaciones obtenibles en la evaluación de proveedores.},
    ]{%
    width = \linewidth,
    colspec = {Q[185]Q[181]Q[575]},
    cells = {c},
    row{even} = {Gallery},
    }
    \toprule
    \textbf{Estado}                                  & \textbf{Grado de cumplimiento}                & \textbf{Dictamen}                                                                                                                                                       \\
    \midrule
    \textbf{Aprobado}                                & \qtyrange{75}{100}{\percent}                  & Se mantendrá en la Lista de Proveedores Aprobados aquellos que logren un grado de cumplimiento mayor a \qty{75}{\percent}                                               \\
    \SetCell[r=2]{c}{\textbf{Aprobado condicionado}} & \SetCell[r=2]{c}{\qtyrange{60}{74}{\percent}} & Su ingreso en la Lista de Proveedores Aprobados queda condicionada a un incremento en su grado de cumplimiento de mínimo \qty{75}{\percent} para la siguiente auditoría \\
                                                     &                                               & Su permanencia en la Lista de Proveedores Aprobados se dará sólo si su grado de cumplimiento es de al menos \qty{60}{\percent}                                          \\
    \textbf{Rechazado}                               & \qty{59}{\percent}                            & Fuera de Lista de Proveedores                                                                                                                                           \\
    \bottomrule
\end{longtblr}

Cuando un proveedor obtenga una calificación que lo acredite como proveedor condicionado y/o cuando se señalen áreas de oportunidad durante la auditoría de evaluación, deberá presentar un plan de acciones de mejora a los puntos señalados y regresarlos a \GLS{RDF}. Este plan deberá contemplar el responsable de la actividad así como la fecha de cumplimiento y deberá dirigirse al Jefe de Control de Calidad de \GLS{RDF}.

\begin{changelog}[simple, sectioncmd=\subsection*,label=changelog-2.8]

    \begin{version}[v=2.1, date=2023--01, author=Pablo E. Alanis]
        \item Cambio de formato;
        \item adición de términos;
        \item creacción de apartado que establece los miembros del equipo de evaluación de proveedores
    \end{version}

    \begin{version}[v=1.4, date=2022-02, author=Alonso M.]
        \item no se realizaron cambios de la revisión 05 a la 06.
    \end{version}
\end{changelog}