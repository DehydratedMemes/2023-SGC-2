\thispagestyle{formato-PI}
\renewcommand{\MayorVer}{2}
\renewcommand{\MenorVer}{0}
 %TODO
\renewcommand{\FechaPub}{2023--01}
\renewcommand{\TipoID}{POL}
\renewcommand{\Titulo}{Política de devoluciones}

\section{\Titulo}\index{Política!devoluciones, de}
\renewcommand{\Codigo}{\Prog--\thesection--\TipoID}
% \section{Política de devoluciones}

\subsection{Objetivo}

Establecer que no se realizan maniobras de recepción de devoluciones por problemas de calidad, inocuidad o mal estado de las condiciones físicas del producto como actividades comunes dentro de \gls{RDF}.

\subsection{Alcance}

Esta política es aplicable para los clientes y para los empleados de \gls{RDF}.

\subsection{Términos y definiciones}

\begin{description}
	\defglo{peligro-relacionado-con-la-inocuidad-de-los-alimentos} 
	\defglo{cliente}
	\defglo{cadena-alimentaria}
\end{description}

\subsection{Documentos y/o normas relacionadas}
\begin{itemize}
	\item Programa de \gls{BPD}.
\end{itemize}

\subsection{Política de devoluciones}
Es política de \gls{RDF} \textbf{NO} aceptar devoluciones como una actividad frecuente dentro de sus operaciones cotidianas, ya que esta práctica puede presentar un \gls{peligro-relacionado-con-la-inocuidad-de-los-alimentos}, a su vez esto permite resguardar la inocuidad de los otros \glspl{alimento} almacenados en este establecimiento.

\subsection{Historial de modificaciones}

\begin{changelog}[title=Registro de cambios,simple, sectioncmd=\subsection*,label=changelog-\thesection-\MayorVer.\MenorVer]

	\begin{version}[v=2.1, date=2023--01, author=Pablo E. Alanis]
		\item Cambio de formato;
		\item Cambios en la política de no devoluciones: se empleó el término \emph{peligro relacionado con la inocuidad de los alimentos} y {alimentos}, según los términos establecidos por la ISO 22000:2018;
	\end{version}

	\begin{version}[v=1.6, date=2022-02, author=Alonso M.]
		\item no se realizaron cambios de la revisión 05 a la 06.
	\end{version}
\end{changelog}