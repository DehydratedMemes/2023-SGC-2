\thispagestyle{formato-PI}
\renewcommand{\MayorVer}{2}
\renewcommand{\MenorVer}{0}
\renewcommand{\Codigo}{PSA-2-PRO} 
\renewcommand{\FechaPub}{2023--01}
%\renewcommand{\Edit}{2.1}
\renewcommand{\Titulo}{Procedimiento de lotificación de productos para su almacenamiento}

\section{\Titulo}\index{Procedimiento!lotificación de productos}

% \section{Procedimiento de lotificación de productos para su almacenamiento}

\subsection{Objetivo}

\begin{itemize}
	\item \textbf{Asegurar} que las tarimas con producto a almacenar sean identificadas con la información requerida por el cliente.
\end{itemize}

\subsection{Alcance}

\begin{itemize}
	\item Este documento es aplicable a todas las tarimas con producto que almacenadas en RDF.
\end{itemize}

\subsection{Términos y definiciones}
%TODO definiciones

\subsection{Procedimiento}

\subsubsection{Fundamento}

Los establecimientos y equipos dedicados al proceso de alimentos para consumo humano deben mantener los registros y el control de los productos y materiales de empaque por medio de la notificación de los mismos, con el cual es posible la realización de un sistema de rastreo desde su producción hasta su distribución.

\subsection{Materiales}

\begin{itemize}
	\item Factura de entrada
\end{itemize}

\subsection{Instrucciones}

\begin{itemize}
	\item Cada vez que se reciba en el área de almacén, productos o materiales de empaque, se debe registrar el lote que el cliente designado a este producto.
	\item El personal de recepción debe de verificar que el lote de los registros coincidan con el lote marcado en las tarimas del producto;
	\item Estos se darán de alta en el sistema de la empresa con el lote asignado por el proveedor o cliente.
	\begin{itemize}
		\item Si el producto no tiene lote o este no coincide se dará aviso al Supervisor de almacén (el lote asignado será asignado con el número de entrada)
	\end{itemize}
	\item Las tarimas que contienen el producto son marcadas con \emph{etiqueta de recepción de producto,} en ella se coloca la fecha de recepción, nombre del cliente, nombre del producto, fecha de caducidad, orden de compra o lotificación, numero de entrada, cantidad por tarima.
\end{itemize}

\subsection{Responsables de la actividad}

\begin{itemize}
	\item \textbf{Ejecutado} por personal de operaciones;
	\item \textbf{Monitoreado} por personal de calidad;
	\item \textbf{Verificado} por personal de gerencia.
\end{itemize}

\subsection{Acciones preventivas}

\begin{itemize}
	\item Se llevara a cabo una inspección. Registrando el indicador y medida correspondiente;
	\item Si la desviación se repite frecuentemente se dará curso de capacitación al personal de almacén y limpieza para que realicen eficientemente su trabajo;
	\item Si después de haber capacitado al personal de almacén se siguen presentando desviaciones por causas injustificadas, se tomaran acciones más enérgicas con el personal por incumplimiento con sus deberes.
\end{itemize}

\subsection{Acciones correctivas}

\begin{itemize}
	\item En caso contrario la tarea se deberá volver a realizar como se indica en el procedimiento.
	\item En caso de no conformidad reportar en \RAC.
\end{itemize}

\subsection{Frecuencia}

\begin{itemize}
	\item Cada recepción de Producto
	\item Cada entrega de producto.
\end{itemize}

\begin{changelog}[simple, sectioncmd=\subsection*,label=changelog-2.2]
	
	\begin{version}[v=2.1, date=2023--01, author=Pablo E. Alanis]
			\item Cambio de formato;
			\item Cambios en la serialización de versiones;
	\end{version}

	\begin{version}[v=1.3, date=2022--05, author=Alonso M.]
		\item cambio de fecha;
		\item cambio de código.
	\end{version}

	\shortversion{v=1.2, date=2021--05, changes=No hubo cambios}
\end{changelog}