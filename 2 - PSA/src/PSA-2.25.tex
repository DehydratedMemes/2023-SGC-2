\thispagestyle{formato-PI}
\renewcommand{\MayorVer}{2}
\renewcommand{\MenorVer}{0}
 %TODO
\renewcommand{\FechaPub}{2023--01}
\renewcommand{\TipoID}{PRO}

\renewcommand{\Titulo}{Elaboración y control de documentos}
\section{\Titulo}\index{Información documentada!tipo!procedimiento!Elaboración y control de documentos}
\renewcommand{\Codigo}{\Prog--\thesection--\TipoID}

\subsection{Objetivo}

Asegurar que los documentos del Sistema de Calidad se preparan, revisan, aprueban, publican, distribuyen, implementan y administran de acuerdo a lo especificado en este procedimiento.

\subsection{Alcance}

Este procedimiento aplica a todos los documentos generados internamente por cada uno de los departamentos o de fuentes externas y personal involucrado en la elaboración, consulta y actualización de documentos en el Sistema de Calidad.

\subsection{Términos y definiciones}

\begin{itemize}
	\item \textbf{Copia Controlada:} Documento que se emite físicamente por el Sistema de Calidad el cual incluye esta leyenda en el mismo, para poder mantener el control de documentos y asegurar que el documento empleado es la versión vigente del documento en cuestión.
	\item \textbf{Copia no controlada:} Documento (por diferentes razones) del cual se emite una copa física pero que no es actualizado al sufrir cambios.
	\item \textbf{Documento Interno:} Documento que forma parte del sistema de calidad de la compañía, los documentos que incluye el sistema de calidad son los siguientes:
	\begin{itemize}
		\item Listado Maestro
		\item Procedimiento Estándar
		\item POE
		\item POES
		\item Instrucciones de Trabajo
		\item Reglamentos
		\item Formatos
		\item Registros
	\end{itemize}
	\item \textbf{Control de Índices:} Documento en el que se enumeran los documentos del sistema de calidad; incluye código, titulo, edición vigente y fecha de revisión.
	\item \textbf{Procedimiento Estándar De Operación (PRO) Y POE:} Es un término que hace referencia a la acción de\textbf{proceder}, que significa actuar de una forma determinada. El concepto, por otra parte, está vinculado a un o una manera de ejecutar algo. Consiste en\textbf{seguir ciertos pasos predefinidos y de manera secuencial}para desarrollar una labor de manera eficaz.
	\item \textbf{Procedimiento Estándar de Sanitización (POES):} Es un término que hace referencia a la acción de\textbf{proceder}, que significa actuar de una forma determinada. El concepto, por otra parte, está vinculado a un o una manera de ejecutar algo. Consiste en\textbf{seguir ciertos pasos predefinidos y de manera secuencial}para desarrollar una labor de limpieza de manera más eficaz.
	\item \textbf{Instrucción de Trabajo:} Desarrollan secuencialmente los pasos a seguir para la correcta realización de un trabajo específico y que normalmente involucra a una sola persona o área de responsabilidad.
	\item \textbf{Especificaciones:} Documento que describe en forma detallada las características o requisitos técnicos de un servicio o producto y que deben cumplirse para lograr un propósito determinado. Pueden ser documentos internos y/o externos.
	\item \textbf{Anexos:} Documentos que complementan lo descrito en un procedimiento. Estos pueden ser referencias bibliográficas, normas, formatos, esquemas, gráficos etc.
	\item \textbf{Formato:} Documento en el que se plasman los resultados obtenidos y actividades realizadas y que representara evidencia de cumplimiento de la o las actividades.
	\item \textbf{Registros:} Es un formato que ha sido completado y que se resguarda como evidencia de la realización de actividades específicas.
\end{itemize}

\subsection{Responsables de la actividad}
\subsubsection{Aseguramiento de Calidad}
\begin{itemize}
	\item Coordina y verifica que se lleve a cabo la actualización en tiempo y forma de los documentos y registros del Sistema de Calidad.
	\item Administra la totalidad de documentos del Sistema de Calidad.
	\item Asigna la numeración correspondiente a los documentos elaborados.
	\item Responsable de la aprobación, revisión y emisión de documentos de acuerdo con su área de responsabilidad.
\end{itemize}

\subsubsection{Gerente de Mantenimiento}
\begin{itemize}
	\item Responsable de la elaboración, actualización, revisión, emisión y aplicación de los documentos generados en su área, así como de los documentos del Sistema de Calidad donde tenga injerencia.
\end{itemize}

\subsubsection{Gerente de Operaciones}
\begin{itemize}
	\item Responsable de la elaboración, actualización, revisión, emisión y aplicación de los documentos generados en su área, así como de los documentos del Sistema de Calidad donde tenga injerencia.
\end{itemize}

\subsection{Procedimiento}
\subsubsection{Generación de Documentos}

Determinar la necesidad de documentar y validar con el jefe inmediato del área. Identificar el tipo de documento necesario a generar y proceder a documentar y/o generar el documento en base a los siguientes lineamientos:

\paragraph{Estructura}

Todos los documentos deben contenerlo y debe incluir:

\begin{itemize}
	\item Encabezado
	\item Logotipo
	\item Título del documento
	\item Código de identificación del documento
	\item Revisión actual del documento
	\item Fecha de emisión
\end{itemize}

\paragraph{Edición}

Indica el número de veces que el documento ha sido modificado y/o adecuado, se inicia con el número que corresponde a la primera emisión.

\paragraph{Fecha de Publicación}

Corresponde a la fecha en que el documento se elaboró.

\paragraph{Contenido}

Corresponde a la información que contiene el documento y como debe ser presentada.

\begin{enumerate}
	\item \textbf{Objetivo:} Finalidad para la cual fue creado el documento.
	\item \textbf{Alcance:} Áreas o puestos para los cuales es aplicable el documento.
	\item \textbf{Términos y definiciones:} Conjunto de términos o palabras propias utilizadas en un procedimiento.
	\item \textbf{Responsabilidades:} Indica los compromisos de los participantes en el desarrollo de un documento.
	\item \textbf{Procedimiento:} Desarrollo de la actividad o proceso a seguir paso a paso.
	\item \textbf{Frecuencia:} Es la frecuencia con la que se debe realizar cada actividad.
	\item \textbf{Documentos Relacionados:} Todos los documentos con relación al procedimiento.
	\item \textbf{Anexos:} Agregados de un trabajo que se incluyen al final del documento y ofrecen información adicional.
	\item \textbf{Formatos:} Todos los formatos con relación al procedimiento.
	\item \textbf{Historial de modificaciones:} Muestra cual ha sido el histórico de las modificaciones o adecuaciones que ha tenido el documento. Se establece la revisión anterior, revisión actual, fecha de las revisiones y una descripción de las modificaciones realizadas.
	\item \textbf{Listado de Distribución:} Lista donde se menciona a quien se le ha compartido el procedimiento o documento y cuenta con copia.
\end{enumerate}

\paragraph{Tipo de letra}

Los documentos se deben desarrollar con base a la estructura del presente documento y deben ser escritos en letra Calibri 12 pt. Los títulos deben ser en negrilla, combinando mayúsculas y minúsculas.

\paragraph{Numeración}

\begin{enumerate}
	\item Inicie el esquema de numeración partiendo del número 1 y según se requiera, desglose el mismo agregando un punto y un decimal. Por ejemplo: 1, 1.1, 1.1.1. De ser necesario en cada punto utilice incisos y /o viñetas.
	\item Revisar el documento elaborado, para asegurar que se cumplen todos los puntos para la estandarización; que incluye el formato, estructura y contenido.
	\item Llevar a cabo la ruta de aprobación del documento.
	\item Si el documento es aprobado proceder a la difusión e implementación del mismo. 
	\item Los formatos tienen como mínimo (cuando aplique):
	\begin{itemize}
		\item Código
		\item Revisión
		\item Titulo
		\item Numero de hojas
		\item Firma de la(s) persona(s) responsable(s) de llenar el registro
		\item Firma de autorización y/o verificación
	\end{itemize}
	\item Los registros son llenados en cada uno de sus espacios, cuando un espacio no es utilizado, por no ser necesario se coloca una línea horizontal o diagonal para cancelarlo.
	\item Todos los documentos y registros deben resguardarse de manera que no sufran daño y/o deterioro.
\end{enumerate}

\paragraph{Revisión y aprobación de Documentos}

\begin{itemize}
	\item Si el documento presenta faltas de cumplimiento al presente procedimiento, la persona responsable de la revisión notifica al puesto que generó el documento para que proceda a realizar las modificaciones señaladas y repita las actividades anteriores.
	\item Si el documento fue aprobado por los involucrados en la ruta de aprobación se procede a su difusión e implementación.
\end{itemize}

\paragraph{Distribución, implementación y control de Documentos}

\begin{enumerate}
	\item Los documentos aprobados serán incluidos con la codificación correspondiente en el Listado Maestro de documentos para su control.
	\item Ya liberado el documento (emisión inicial o cambios) el Jefe de Aseguramiento de Calidad realizará las copias y distribuirá el documento de acuerdo a 12 Listado de Distribución.
	\item Cuando se requiera una copia física de los documentos aprobados y vigentes, se realiza una solicitud a Aseguramiento de Calidad, cada copia deberá ser sellada como se indica en el Anexo A según corresponda.
	\item Para solicitar documentos con copia no controlada, se tiene que mandar una solicitud por escrito al Jefe de Aseguramiento de Calidad especificando el motivo de la solicitud.
\end{enumerate}

\paragraph{Cambios, alta o baja de documentos}

\begin{enumerate}
	\item Para realizar cambios, alta o baja de documentos se debe seguir y cumplir los pasos del 5.1 al 5.2. Se debe informar a Aseguramiento de Calidad a través del formato O-F-RCP Solicitud de Alta, Baja o Cambio de documentos.
	\item Aseguramiento de Calidad revisa con cada una de las bases de la solicitud O-F-RCP y determinar si procede, posteriormente se hace el proceso de revisión y autorización de documentos y se actualiza el listado maestro de documentos. Aseguramiento de Calidad es responsable de:
	\begin{enumerate}
		\item Cambiar el contenido del documento según los cambios necesarios para adecuarlo al proceso y/o revisar la propuesta de cambio que haga el área solicitante.
		\item Cambiar la revisión de los documentos (Procedimientos, Instructivos o registros).
		\item Llenar el punto 10 del procedimiento, Historial de Cambios, con las especificaciones generales del cambio.
		\item Actualizar el Listado Maestro.
		\item Para finalizar, el Jefe de Aseguramiento de Calidad imprime el documento autorizado y comienza el proceso de distribución y difusión.
	\end{enumerate}
	\item Cuando se realice algún cambio, se debe actualizar el punto 10.0 Historial de Modificaciones del Documento, este punto muestra cual ha sido el histórico de las modificaciones o adecuaciones que ha tenido el documento.
\end{enumerate}

\paragraph{Control de documentos obsoletos:}

\begin{enumerate}
	\item Todo documento tiene 1 año de vigencia. Posterior a esta fecha se debe realizar una revisión, para asegurar que la información es actual y corresponde al proceso y/o actividad que se ejecutan.
	\item Retire de los puntos de uso los documentos obsoletos que haya distribuido físicamente de acuerdo al listado de distribución y remplace el documento por la revisión vigente para que el personal involucrado siempre tenga la versión actualizada para ejecutar sus actividades; asegure que todos los documentos obsoletos sean retirados y remplazados.
	\item Destruya las copias de los documentos obsoletos y destrúyalos conservando un ejemplar que se identificará con la leyenda de \textbf{DOCUMENTO OBSOLETO} y consérvelos con base en los lineamientos de Control de Registros.
\end{enumerate}

\subsubsection{Respaldo de la información}
\begin{enumerate}
	\item La documentos electrónicos del Sistema de Calidad es respaldada por el Jefe de Aseguramiento de Calidad con una frecuencia mensual.
\end{enumerate}

\subsection{Frecuencia}
Cada vez que sea necesaria la publicación o modificación de algún documento del sistema de calidad.

\begin{changelog}[simple, sectioncmd=\subsection*,label=changelog-\thesection-\MayorVer.\MenorVer5]
	\begin{version}[v=\MayorVer.\MenorVer, date=2023--01, author=Pablo E. Alanis]
		% \fixed
		\item Cambio de formato;
		\item Cambios en la serialización de versiones;
	\end{version}

	\begin{version}[v=1.6, date=2022--05, author=Alonso M.]
		\item cambio de fecha;
	\end{version}

	\shortversion{v=1.5, date=2021--05, changes=No hubo cambios}
\end{changelog}
