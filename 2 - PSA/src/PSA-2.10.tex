\renewcommand{\MayorVer}{2}
\renewcommand{\MenorVer}{1}
\renewcommand{\Codigo}{PSA-1-PROG} %TODO
\renewcommand{\FechaPub}{2023--01}
%\renewcommand{\Edit}{2.1}
\renewcommand{\Titulo}{Inspección de transporte al embarque de producto}

\section{\Titulo}
\index{Inspección!transporte al embarque de producto, de}

% \section{Inspección de transporte al embarque de producto}

\subsection{Objetivo}

Establecer un procedimiento que asegure que la unidad a cargar se encuentra en condiciones adecuadas para poder ser ingresado el producto y que la unidad cumple con lo requerido para no generar alguna contaminación o daño al producto.

\subsection{Alcance}

Todas las unidades que se presenten a cargar producto en el almacén.

\subsection{Términos y definiciones}
%TODO

\subsection{Documentos y/o normas relacionadas}
%TODO
1.3 - MA-OP-001 - Manual de calidad y buenas prácticas de distribución|Manual de calidad y Buenas Prácticas de Distribución

\subsection{Procedimiento}

\subsubsection{Materiales}

\begin{itemize}
	\item Termómetro
\end{itemize}

\subsubsection{Precauciones de seguridad}

N/A

\subsubsection{Instrucciones}

\paragraph{Envío de productos}

\begin{enumerate}
	\item A fin de garantizar que todos los productos salen de las instalaciones libres de peligro física, química y/o los riesgos biológicos, todo el contenedor (caja del camión) deben ser inspeccionados antes de cargar y documentado usando la forma “Orden de Entrada o Salida”.
	\item La unidad debe de mostrar el termo encendido antes de ser colocado en la rampa, esta revisión es realizada por el personal de vigilancia
	\item El contenedor (caja del camión) deben ser revisado por el personal de vigilancia antes de que la unidad se coloque en la rampa. \textbf{Evidencia de daño, suciedad, o plagas deben ser reportado al Supervisor de Almacén y la unidad no podrá colocarse en rampa hasta no cumplir con los requerimientos de aceptación (Unidad limpia, sin daños que puedan causar una contaminación y libre de plaga).}
	\item El contenedor (caja del camión) deben ser revisado minuciosamente por los encargados de embarques de producto para confirmar la revisión realizada por el personal de vigilancia. \textbf{Evidencia de daño, suciedad, o plagas deben ser reportado al Supervisor de Almacén y la unidad no podrá colocarse en rampa hasta no cumplir con los requerimientos de aceptación (Unidad limpia, sin daños que puedan causar una contaminación y libre de plaga).}
	\item Todos los productos de salida \textbf{debe ser visualmente revisados} \textbf{para comprobar que no hay productos dañados} o en mal estado que se carguen en la unidad de carga. \textbf{NO debe de cargarse} producto abierto o dañado (cajas o contenedores abiertos o en mala condición). \textbf{El producto debe ser retenido para su verificación.}
	\item La temperatura del contenedor (caja del camión) deberá verificarse antes de la carga para asegurar que la unidad de refrigeración está funcionando correctamente. Los contenedores (caja de camión) deben estar programados a una temperatura \textbf{para los productos refrigerados No mayor a 4.0 °C y no inferior a 0.0 °C y para los productos congelados ≤ -15 °C (Ideal ≤ -18 °C).}
	\item \textbf{Si se tiene evidencia del NO funcionamiento del termo el producto NO debe ser cargado.}
	\item Los contenedores (caja del camión) deberán ser \textbf{sellados} antes de salir de las instalaciones cuando se trate de producto TIF o si así fuese requerido por el cliente.
	\item Requisitos pedidos por Medico TIF.
\end{enumerate}

\paragraph{Políticas de embarque}

\begin{enumerate}
	\item Para asegurarse de que todos los contenedores (caja del camión) estén limpios y cumplen con los requisitos de temperatura establecidos, todos los contenedores (caja del camión) tienen que ser inspeccionados y registrados en la forma “Orden de Entrada o Salida”. El contenedor (caja del camión) se analizara por la limpieza, daños, olores anormales, alguna evidencia de insectos o plaga, así como la temperatura adecuada.
	\item Antes de cargar o descargar el producto, este deberá ser revisado para verificar que no tenga daños en su empaque, evidencias de insectos / plagas.
\end{enumerate}


\begin{enumerate}
	\item Rango de temperaturas de cuarto donde se almacena el producto:
\end{enumerate}


\begin{enumerate}
	\item La temperatura del producto debe ser tomada según disposición del cliente:
	\begin{itemize}
		\item debe de ser tomada directamente en el producto con el termómetro de agua para el caso de cárnicos.
		\item Debe ser toma directamente del producto (Sin abrir empaque primario) con termómetro laser en el caso de productos compuestos ya sellados
	\end{itemize}
	\item El conductor deberá de enfriar la caja refrigerada de la unidad antes de enramparse en el andén de carga siguiendo el procedimiento de \textbf{pre-enfriado.}
\end{enumerate}




\paragraph{Procedimiento de pre enfriado de unidades}

Todas las unidades que transportan productos refrigerados o congelados para su distribución, deberán de ser pre enfriado antes de enrramparse para la carga siguiendo la siguiente regla:

\begin{itemize}
	\item Para él envió de productos refrigerados la caja de la unidad se deberá de pre enfriar hasta alcanzar la temperatura de 0.0 – 4.0°C.
	\item Para él envió de productos congelados la caja de la unidad se deberá programar hasta alcanzar la temperatura de ≤ -15°C.
\end{itemize}

\begin{enumerate}
	\item El conductor será responsable de verificar la temperatura de la caja del camión en la pantalla durante el traslado del producto.
	\item Cualquier inquietud o duda sobre el producto objeto del embarque deben ser comunicadas inmediatamente al Supervisor de Almacén antes de la carga.
\end{enumerate}

\subsection{Responsables de la actividad}

\begin{itemize}
	\item \textbf{Ejecutado} por personal de operaciones
	\item \textbf{Monitoreado} por personal de calidad
	\item \textbf{Verificado} por personal de gerencia.
\end{itemize}

\subsection{Acciones preventivas}

\begin{itemize}
	\item Se llevara a cabo una inspección. Registrando el indicador y medida correspondiente
	\item Si la desviación se repite frecuentemente se dará curso de capacitación al personal de almacén y limpieza para que realicen eficientemente su trabajo.
	\item Si después de haber capacitado al personal de almacén se siguen presentando desviaciones por causas injustificadas, se tomaran acciones más enérgicas con el personal por incumplimiento con sus deberes.
\end{itemize}

\subsection{Acciones correctivas}

\begin{itemize}
	\item En caso de no cumplimiento la tarea se deberá volver a realizar como se indica en el procedimiento.
	\item En caso de no conformidad reportar en formato de acciones correctivas F-OP-40.xls
\end{itemize}

\subsection{Frecuencia}

\begin{itemize}
	\item Cada embarque.
\end{itemize}

\subsection{Historial de modificaciones} %TODO

\begin{itemize}
	\item \textbf{Quinta edición:} cambio de fecha de 04 de noviembre 2017 a 28 de enero 2019, se realizó cambio de código de CT-P-ITEP a PR-010, se agregó punto 9. requisitos medico TIF y se anexo documento con los requisitos de dicho número y se realizaron cambios de revisión de 004 a 005.
	\item \textbf{Sexta edición:} febrero 2020 se hizo cambio de formato y cambio de código de PRO-010 a PRO-OP-007. Y no se realizaron cambio de la revisión 05 a la 06.
	\item \textbf{Séptima edición:} febrero 2021 no se realizaron cambios de la revisión 06 a la 07.
	\item \textbf{Octava edición:} febrero 2022 no se realizaron cambios de la revisión 07 a la 08.
\end{itemize}
