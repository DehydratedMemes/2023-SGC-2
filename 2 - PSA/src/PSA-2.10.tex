\thispagestyle{formato-PI}
\renewcommand{\MayorVer}{2}
\renewcommand{\MenorVer}{0}
\renewcommand{\Codigo}{PSA-10-PRO} %TODO
\renewcommand{\FechaPub}{2023--01}
\renewcommand{\TipoID}{PRO}
\renewcommand{\Titulo}{Inspección de transporte al embarque de producto}

\section{\Titulo}\index{Inspección!transporte al embarque de producto, de}\index{Procedimiento!Inspección de transporte al embarque de producto}
\renewcommand{\Codigo}{\Prog--\thesection--\TipoID}
\subsection{Objetivo}
Establecer un procedimiento que asegure que la unidad a cargar se encuentra en condiciones adecuadas para poder ser ingresado el producto y que la unidad cumple con lo requerido para no generar alguna contaminación o daño al producto.

\subsection{Alcance}
Todas las unidades que se presenten a cargar producto en el almacén.

\subsection{Términos y definiciones}
\begin{description}
	\defglo{requisito}
	\defglo{producto-terminado}
	\defglo{producto}
\end{description}

\subsection{Documentos y/o normas relacionadas}
	\begin{itemize}
		\item Manual de \gls{BPD}.
	\end{itemize}

\subsection{Procedimiento}

\subsubsection{Materiales}

\begin{itemize}
	\item Termómetro
\end{itemize}

\subsubsection{Precauciones de seguridad}
\begin{itemize}
	\item Usar uniforme completo.
\end{itemize}

\subsubsection{Instrucciones}
\paragraph{Envío de productos}
\begin{enumerate}
	\item Con el fin de garantizar que todos los productos salen de las instalaciones libres de peligro físico, químico y/o los riesgos biológicos, todo el contenedor debe ser inspeccionado antes de cargar y los hallazgos ser documentados usando el formulario \emph{orden de entrada} u \emph{orden de salida.}
	\item La unidad debe de presentarse con el termo encendido antes de ser colocado en la rampa, \emph{esta revisión es realizada por el personal de vigilancia}
	\item El contenedor debe ser revisado por el personal de vigilancia antes de que la unidad se coloque en la rampa.
	Cualquier Evidencia de daño, suciedad, o plagas deben ser reportados al \emph{Supervisor de Almacén} y la unidad no podrá colocarse en rampa hasta no cumplir con los \emph{requerimientos de aceptación.}
	\item El contenedor debe ser revisado minuciosamente por los encargados de embarques de producto para confirmar la revisión realizada por el personal de vigilancia.
	\item Todos los productos de salida \emph{debe ser visualmente revisados para comprobar que no hay productos dañados o en mal estado} que se carguen en la unidad de carga. \textbf{NO} debe de cargarse producto abierto o dañado. \emph{El producto debe ser retenido para su verificación.}
	\item La temperatura del contenedor deberá verificarse antes de la carga para asegurar que la unidad de refrigeración funciona correctamente. Los contenedores deben estar programados a una temperatura que se encuentre dentro del rango de la especificación del \gls{alimento}.\footnote{para los productos \emph{refrigerados:} No mayor a \cels{4} y no inferior a \cels{\leq 0} y para los productos \emph{congelados} \cels{\leq -15} (Ideal \cels{\leq -18}).}
	\item \textbf{Si se tiene evidencia del NO funcionamiento del termo el producto NO debe ser cargado.}
	\item Los contenedores deberán ser sellados \emph{antes} de salir de las instalaciones cuando se trate de producto TIF o si así fuese requerido por el cliente.
	\item Requisitos pedidos por Medico TIF (En caso de producto cárnico).
\end{enumerate}

\paragraph{Políticas de embarque}
\begin{enumerate}
	\item Para asegurarse de que todos los contenedores estén limpios y cumplen con los requisitos de temperatura establecidos, todos los contenedores tienen que ser inspeccionados y registrados en la formulario \emph{orden de entrada} u \emph{orden de salida;}
	\item se buscara evidencia de limpieza, daños, olores anormales, alguna evidencia de insectos o plaga, así como la temperatura adecuada en el contenedor;
	\item antes de cargar o descargar el producto, este deberá ser revisado para verificar que no tenga daños en su empaque, evidencias de fauna nociva.
\end{enumerate}


\paragraph{Rango de temperaturas de cuarto donde se almacena el producto}
\begin{enumerate}
	\item La temperatura del producto debe ser tomada \emph{según disposición del cliente:}
	\begin{itemize}
		\item debe de ser tomada directamente en el producto con el termómetro de vástago para el caso de cárnicos;
		\item debe ser toma directamente del producto \emph{(sin abrir empaque primario)} con termómetro infrarrojo en el caso de \gls{producto-terminado} sellado.
	\end{itemize}
	\item El conductor deberá de enfriar la caja refrigerada de la unidad antes de \textit{enrramparse} en el andén de carga siguiendo el procedimiento de pre-enfriamiento \textit{(vide infra)}.
\end{enumerate}

\paragraph{Procedimiento de pre-enfriamiento de unidades}
Todas las unidades que transportan productos refrigerados o congelados para su distribución, deberán de ser pre enfriado antes de \textit{enrramparse} para la carga siguiendo la siguiente regla:
\begin{itemize}
	\item Para el envío de productos refrigerados la caja de la unidad se deberá de pre enfriar hasta alcanzar la temperatura de \qtyrange{0}{4}{\celsius}
	\item Para el envío de productos congelados la caja de la unidad se deberá programar hasta alcanzar la temperatura de \cels{\leq -15}.
\end{itemize}

\begin{enumerate}
	\item El conductor será responsable de verificar la temperatura de la caja del camión en la pantalla durante el traslado del producto;
	\item cualquier inquietud o duda sobre el producto objeto del embarque deben ser comunicadas inmediatamente al Supervisor de Almacén antes de la carga.
\end{enumerate}

\subsection{Responsables de la actividad}

\begin{itemize}
	\item \textbf{Ejecutado} por personal de operaciones;
	\item \textbf{Monitoreado} por personal de calidad;
	\item \textbf{Verificado} por personal de gerencia.
\end{itemize}

\subsection{Acciones preventivas}

\begin{itemize}
	\item Se llevará a cabo una inspección. Registrando el indicador y medida correspondiente
	\item Si la desviación se repite frecuentemente se dará curso de capacitación al personal de almacén y limpieza para que realicen eficientemente su trabajo.
	\item Si después de haber capacitado al personal de almacén se siguen presentando desviaciones por causas injustificadas, se tomaran acciones más enérgicas con el personal por incumplimiento con sus deberes.
\end{itemize}

\subsection{Acciones correctivas}

\begin{itemize}
	\item En caso de no cumplimiento la tarea se deberá volver a realizar como se indica en el procedimiento.
	\item En caso de no conformidad reportar en formato de acciones correctivas \RAC.
\end{itemize}

\subsection{Frecuencia}

\begin{itemize}
	\item Cada embarque.
\end{itemize}

\begin{changelog}[title=Registro de cambios,simple, sectioncmd=\subsection*,label=changelog-\thesection-\MayorVer.\MenorVer0]
	\begin{version}[v=\MayorVer.\MenorVer, date=2023--01, author=Pablo E. Alanis]
		\item Cambios de formato y serialización
	\end{version}
	
	\begin{version}[v=1.8, date=2022-02, author=Alonso M.]
	\item no se realizaron cambios de la revisión 7 a la 8.
	\end{version}
\end{changelog}