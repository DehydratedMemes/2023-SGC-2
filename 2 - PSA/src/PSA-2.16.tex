\renewcommand{\MayorVer}{2}
\renewcommand{\MenorVer}{1}
\renewcommand{\Codigo}{PSA-1-PROG} %TODO
\renewcommand{\FechaPub}{2023--01}
%\renewcommand{\Edit}{2.1}
\renewcommand{\Titulo}{Aprobación de cambios a procedimientos}

\section{\Titulo}
\index{Información documentada!control!aprobación de cambios a procedimientos}

% \section{Aprobación de cambios de procedimientos}

\subsection{Objetivo}

Asegurar que todos los cambios y/o modificaciones dentro de la operación serán iniciados al tener la autorización previa de los departamentos, operativos, mantenimiento, administrativos y gerencia.

\subsection{Alcance}

Todos los procedimientos, formatos, guías y programas.

\subsection{Términos y definiciones}

N/A

\subsection{Documentos y/o normas relacionadas}

N/A

\subsection{Procedimiento}

\subsubsection{Instrucciones}

\paragraph{Cambios en los procesos de Recepción, Almacenamiento y Embarque}

\begin{enumerate}
	\item Los cambios en los procesos que afectan a los productos (manejo de producto) solo pueden ser solicitados y autorizados por requerimiento del cliente.
	\item Este requerimiento se debe de hacer por escrito y firmado por parte del cliente.
\end{enumerate}

\paragraph{Cambios de procedimientos y políticas}

\begin{enumerate}
	\item El Gerente de Operaciones es responsable del mantenimiento del Programa de Seguridad Alimentaria, así como capacitación del personal en todos los cambios.
	\item Solicitud de cambios:
	\begin{itemize}
		\item El equipo comunicara la política o procedimiento que requiera realizar cambios o mejoras.
		\item El equipo deberá justificar el cambio, exponiendo el motivo y los elementos a favor y los posibles elementos que estuvieran en contra.
		\item Autorización o rechazo de la propuesta de cambio por las áreas involucradas. Previamente se debe de realizar un análisis valorando minuciosamente la propuesta.
	\end{itemize}
	\item Si se autoriza el cambio, este se debe de registrar por escrito en el programa de cambio de documentos por producto el cual debe de contener por qué se autorizó el cambio y debe de contener las firmas en el formato correspondiente de cambios de documentos.
	\item Los departamentos que intervienen en el análisis de la propuesta de cambio son:
	\begin{itemize}
		\item Gerencia de Operaciones.
		\item Coordinador de logística.
		\item Áreas operativas.
		\item Dirección.
		\item Áreas administrativas.
		\item Mantenimiento.
		\item Calidad.
	\end{itemize}
	\item Cuando se determina que los cambios en la especificación, la política o procedimiento es necesario en un esfuerzo por mantener la continuidad de la calidad y de procedimiento (así como el control de los sistemas), el Gerente de Operaciones actualizara el documento o sección en cuestión, lo que indica “actualización”, seguida del número de revisión en la esquina superior derecha del documento que ha sido revisado.
	\item El Gerente de Operaciones difundirá el documento revisado a todas las partes afectadas. Esas actividades incluirán una breve explicación de la revisión y solicitud de preguntas para asegurar una línea clara, sólida y abierta de comunicación existente entre todas las partes implicadas.
	\item Se dará seguimiento para garantizar el buen funcionamiento del cambio.
\end{enumerate}

\paragraph{Administración de Cambios - General}

\begin{enumerate}
	\item El cambio se produce en nuestra empresa como mejora continua, es una parte normal de nuestro negocio y puede referirse a los siguientes ámbitos, entre otros:
	\begin{itemize}
		\item Cambios de personal
		\item Modificaciones a los contratistas
		\item Administración de equipos
		\item Reubicación de instalaciones
		\item Nuevos clientes y cambios en las relaciones con un cliente.
		\item Cambios en las rutas de distribución.
	\end{itemize}
	\item Como tal, el Gerente de Operaciones se encargara de difundir la información relativa a la modificación de todo el personal correspondiente. Esta comunicación se hará por escrito en forma de correo electrónico, e incluirá la información pertinente, tales como:
	\begin{itemize}
		\item Los nombres, títulos, etc.\ de los empleados recién agregados;
		\item Los nombres de los empleados despedidos;
		\item El nombre, dirección, número de teléfono, información de contacto, etc.\ de contratistas o 3ra partes recién agregados, así como si este cambio reemplaza a un contratista que anteriormente se aplicaba;
		\item Información pertinente en relación de la administración de equipos;
		\item La información pertinente relativa a la reubicación de instalación;
		\item Nombre, dirección, información de contacto, etc.\ para los clientes nuevos, así como información pertinente sobre cualquier cambio en relaciones con los clientes existentes.
	\end{itemize}
	\item Cualquier pregunta  relacionada con la administración de cambios debe ser puesto en conocimiento del Gerente de Operaciones.
\end{enumerate}

\paragraph{Revisiones}

Las revisiones se realizaran cada año o cuando el procedentito requiera de que se realice algún cabio, estas revisiones ycambios se registran en la parte final de cada documento; con ello se mantendrá el historial de los cambios del documento.

\subsection{Acciones correctivas}

\begin{itemize}
	\item Cuando se presente una no conformidad en la realización del procedimiento, detectada por el supervisor o encargado del área se deberá de reportar a su superior inmediato.
	\item En caso de que la desviación sea mayor, se deberá de registrar la acción correctiva en el formato correspondiente.
\end{itemize}

\subsection{Frecuencia}

\begin{itemize}
	\item \textbf{Revisión:} anual
	\item Cada vez que se realicen cambios en las áreas operativas, procedimientos o políticas.
\end{itemize}

\subsection{Historial de modificaciones}

\begin{itemize}
	\item \textbf{Cuarta edición:} cambio de fecha de 04 de noviembre 2017 a 28 de enero 2019, se realizó cambio de código de DG-P-ACP a PR-015. Y no se realizaron cambios de la revisión 003 a la 004.
	\item \textbf{Quinta edición:} febrero 2020 cambio de formato y cambio de código de PR-015 a PRO-OP-011. Y no se realizaron cambios de la revisión 04 a la 05.
	\item \textbf{Sexta edición:} febrero 2021 no se realizaron cambios de la revisión 05 a la 06.
	\item \textbf{Séptima edición:} febrero 2022 no se realizaron cambios de la revisión 06 a la 07.
\end{itemize}

\subsection{Listado de distribución}

