\thispagestyle{formato-PI}
\renewcommand{\MayorVer}{2}
\renewcommand{\MenorVer}{1}
\renewcommand{\Codigo}{PSA-1-PROG} %TODO
\renewcommand{\FechaPub}{2023--01}
%\renewcommand{\Edit}{2.1}
\renewcommand{\Titulo}{Control de estibas y cuidado de producto}

\section{\Titulo}
\label{ESP-ControlDeEstibas}\index{Información documentada!tipo!especificación!Control de estibas y cuidado de producto}

% \section{Control de estibas y cuidado de producto}

\subsection{Objetivo}

\begin{itemize}
	\item Estandarizar los tamaños de estibas para tener un buen acomodo en el almacén y evitar cualquier tipo de riesgo hacia el personal de su mal entarimado.
	\item Evitar algún riesgo de contaminación cruzada.
\end{itemize}

\subsection{Alcance}

A todas las personas responsables para que se lleven a cabo las operaciones que se requieran para el buen cumplimiento de este procedimiento en las áreas operativas.

\subsection{Términos y definiciones}

N/A

\subsection{Documentos y/o normas relacionadas}

\begin{itemize}
	\item Programa de \gls{BPD}.
\end{itemize}

\subsection{Procedimiento}

\begin{itemize}
	\item Con esta estimación de altura de estibas se controla de manera más eficiente el almacenamiento de los productos, mas ordenados y reducir riesgos hacia todo personal, evitando así algún accidente por alguna caída de estiba.
	\item Las tarimas en rack deben de estar separadas de la pared y no estar en contacto la pared con la tarima.
	\item Respetando la regla de estiba en las tarimas se pondrá en la parte de abajo lo más pesado y lo más liviano en la parte de arriba,
	\item Cuidar que no se estiben productos alérgenos arriba de productos no alérgenos.
	\item No se puede almacenar alguna tarima con producto alimenticio junto con alguna estiba con limpiadores o químicos ya pude haber derrame y puede contaminar el producto.
	\item Se deberán manejar en tarimas separadas productos, secos, refrigerados y congelados, en caso que por necesidad de cupo en el transporte se tenga que entarimar o estibar productos refrigerados con secos, se tendrá que asegurar que estas tarimas se almacenen en ambiente refrigerado, y deberá ser documentado en la hoja de salida o documento de envío para que se cuiden las temperaturas de cada producto para prevenir riesgos o posibles daños a los productos.
	\item No se puede almacenar las cajas de empaque primario sin identificarlas o juntarlas con otras que no lo son para poder evitar la contaminación por algún derrame del empaque.
	\item Cualquiera de los productos químicos deben ser cargados en un atado independiente y cubierto con envoltura e identificado para que se le dé el manejo adecuado así como estipulado.
	\item No se permite colocar tarimas con producto arriba de otro producto al menos que se coloque un separador entre ambas tarimas y que esté autorizado por el cliente.
	\item No se puede almacenar algún producto con tarimas dañadas (uso de tarima en buen estado).
	\item No se permite pisar producto.
	\item No se permite colocar encima del producto materiales ajenos al mismo.
\end{itemize}

\subsection{Responsables de la actividad}

\begin{itemize}
	\item \textbf{Ejecutado} por personal de operaciones
	\item \textbf{Monitoreado} por personal de calidad
	\item \textbf{Verificado} por personal de gerencia.
\end{itemize}

\subsection{Acciones preventivas}

\begin{itemize}
	\item Se llevara a cabo una inspección. Registrando el indicador y medida correspondiente
	\item Si la desviación se repite frecuentemente se dará curso de capacitación al personal de almacén y limpieza para que realicen eficientemente su trabajo.
	\item Si después de haber capacitado al personal de almacén se siguen presentando desviaciones por causas injustificadas, se tomaran acciones más enérgicas con el personal por incumplimiento con sus deberes.
\end{itemize}

\subsection{Acciones correctivas}

\begin{itemize}
	\item En caso de no cumplimiento la tarea se deberá volver a realizar como se indica en el procedimiento.
	\item En caso de no conformidad reportar en formato de acciones correctivas F-OP-40
\end{itemize}

\subsection{Frecuencia}

\subsection{Historial de modificaciones}

\begin{itemize}
	\item \textbf{Cuarta edición:} cambio de fecha de 04 de noviembre 2017 a 28 de enero 2019, se realizó cambio de código de DG-P-CE a PR-019. Y no se realizaron cambios de la revisión 003 a la 004.
	\item \textbf{Quinta edición:} febrero 2020 se hizo cambio de formato y cambio de código de PR-019 a PRO-OP-014. Y no se realizaron cambio de la revisión 04 a la 05.
	\item \textbf{Sexta edición:} febrero 2021 no se realizaron cambios de la revisión 05 a la 06.
	\item \textbf{Séptima edición:} febrero 2022 no se realizaron cambios de la revisión 06 a la 07.
\end{itemize}

\subsection{Listado de distribución}

