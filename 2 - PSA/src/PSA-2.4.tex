\renewcommand{\MayorVer}{2}
\renewcommand{\MenorVer}{1}
\renewcommand{\Codigo}{PSA-1-PROG}
\renewcommand{\FechaPub}{2023--01}
%\renewcommand{\Edit}{2.1}
\renewcommand{\Titulo}{PEPS o VENCER}

\section{\Titulo}
\index{Procedimiento!PEPS o VENCER}

% \section{PEPS o VENCER}

\subsection{Objetivo}
Asegurar que los \glspl{alimento}, serán en sus completamente lotificados, identificados y verificados sus movimientos por medio de conteos programados de acuerdo a lo establecido en este procedimiento.

\subsection{Alcance}
Este documento será de utilidad para el personal del área de operaciones o el que se designe por la alta dirección como el encargado de identificar las tarimas de \glspl{alimento}

\subsection{Términos y definiciones}
\begin{description}
	\defglo{inventario-de-suministros-y-materiales} 
\end{description}
	
\subsection{Documentos y/o normas relacionadas}
\begin{itemize}
	\item Registros de recepción %TODO:Hacer Identificador
\end{itemize}

\subsection{Procedimiento}

\subsubsection{Instrucciones}
\paragraph{Primeras entradas, primeras salidas (PEPS) o VENCER}

\subparagraph{Ingreso de alimentos}
\begin{enumerate}
	\item Hacer la revisión correspondiente en cada recepción de producto (ver \cref{PRO:InspeccionDeAlimentosAlmacenados});
	\item revisar que se tengan todos los identificadores únicos del envío (número de lote, fechas, cantidades);
	\item regístralos en los formularios correspondientes para que se puedan ingresar en el \GLS{SGA}.
\end{enumerate}

\subparagraph{Etiquetado de alimentos}
\begin{enumerate}
	\item Se etiqueta por tarima y por lote cada producto;
	\item La etiqueta de \emph{Información de material} contiene los siguientes datos:
	\begin{itemize}
		\item Nombre de producto;
		\item Nombre del cliente;
		\item Fecha de recepción;
		\item Identificador de entrada;
		\item Lote;
		\item Cantidad.
	\end{itemize}
\end{enumerate}

\subparagraph{Consideraciones para el almacenaje de alimentos}
\begin{itemize}
	\item Cada \gls{alimento} con fecha de caducidad más tardía \emph{---Producto nuevo---}, se acomodará en la parte superior trasera de cada \textit{rack} y los \glspl{alimento} que ya estaban presentes se recorren a la parte inferior frontal.
	\item Al surtir los \glspl{alimento}, el personal de almacén deberá entregar los que tengan la \emph{fecha de recepción más antigua} o el producto que el cliente requiera para su operacion.
\end{itemize}

\begin{nota}{Nota sobre uso de \emph{PEPS}}
	\label{nota:PEPS}
\begin{itemize}
		\item El cliente puede decidir si se le surtirá \gls{alimento} en el esquema \emph{PEPS} o si requiere de un lote en específico;
		\item En caso de que el cliente prefiera que la salida de su producto sea por lote, y no por fecha de caducidad, \gls{RDF} no se responsabiliza si el cliente solicita lotes más nuevos aún cuando sigan existencias de lotes viejos o a punto de caducar almacenados en las instalaciones.\footnote{\GLS{RDF} puede dar aviso a los clientes sobre mercancía a punto de caducar, sin embargo, \GLS{RDF} no se responsabiliza de la destrucción de productos caducos, el cliente tiene que retirar \glspl{alimento} caducos.}
\end{itemize}
\end{nota}

% \begin{lamp}[frametitle={Nota sobre uso de \emph{PEPS}}] \label{nota:PEPS}
% 	\begin{itemize}
% 		\item El cliente puede decidir si se le surtirá \gls{alimento} en el esquema \emph{PEPS} o si requiere de un lote en específico;
% 		\item En caso de que el cliente prefiera que la salida de su producto sea por lote, y no por fecha de caducidad, \gls{RDF} no se responsabiliza si el cliente solicita lotes más nuevos aún cuando sigan existencias de lotes viejos o a punto de caducar almacenados en las instalaciones.\footnote{\GLS{RDF} puede dar aviso a los clientes sobre mercancía a punto de caducar, sin embargo, \GLS{RDF} no se responsabiliza de la destrucción de productos caducos, el cliente tiene que retirar \glspl{alimento} caducos.}
% 	\end{itemize}
% \end{lamp}

\subparagraph{En el \glsfirst{SGA}}

\begin{itemize}
	\item En el \GLS{SGA} se cargan las entradas de producto en general al momento de su llegada.
	\item En este sistema administra la orden de salida de los mismos y permite dar de baja únicamente en el esquema de primeras entradas primeras salidas.
	\item El cliente también pudiera dar instrucciones del cual producto dar salida de forma prioritaria por estrategia comercial.
	\item En ambos casos el sistema de manera automática da de baja del sistema el producto que físicamente se da salida, actualizando las existencias.
\end{itemize}

\subsection{Responsables de la actividad}

\begin{itemize}
	\item \textbf{Ejecutado} por personal de operaciones;
	\item \textbf{Monitoreado} por personal de calidad;
	\item \textbf{Verificado} por personal de gerencia.
\end{itemize}

\subsection{Acciones preventivas}
\begin{itemize}
	\item Se llevará a cabo una inspección, registrando el indicador y medida correspondiente;
	\item Si la desviación se repite frecuentemente se dará curso de capacitación al personal de almacén y limpieza para que realicen eficientemente su trabajo.
	\item Si después de haber capacitado al personal de almacén se siguen presentando desviaciones por causas injustificadas, se tomaran acciones más enérgicas con el personal por incumplimiento con sus deberes.
\end{itemize}

\subsection{Acciones correctivas}

\begin{itemize}
	\item En caso contrario la tarea se deberá volver a realizar como se indica en el procedimiento.
	\item En caso de no conformidad reportar en formato de acciones correctivas (\IdFormAACC).
\end{itemize}

\subsection{Frecuencia}

\begin{itemize}
	\item Diaria.
\end{itemize}

\begin{changelog}[simple, sectioncmd=\subsection*,label=changelog-2.4]

	\begin{version}[v=2.1, date=2023--01, author=Pablo E. Alanis]
		\item Cambio de formato;
		\item Cambios en la serialización de versiones;
	\end{version}

	\begin{version}[v=1.8, date=2022-02, author=Alonso M.]
		\item no se realizaron cambios de la revisión 06 a la 07.
	\end{version}

	\shortversion{v=1.7, date=2021-05, changes=No se realizaron modificaciones de la revisión 05 a la 06.}
\end{changelog}