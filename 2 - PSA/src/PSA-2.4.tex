\renewcommand{\MayorVer}{2}
\renewcommand{\MenorVer}{1}
\renewcommand{\Codigo}{PSA-1-PROG}
\renewcommand{\FechaPub}{2023--01}
%\renewcommand{\Edit}{2.1}
\renewcommand{\Titulo}{PEPS o VENCER}

\section{\Titulo}
\index{Procedimiento!PEPS o VENCER}

% \section{PEPS o VENCER}

\subsection{Objetivo}
Asegurar que los productos, serán en sus totalidades lotificadas e identificados y verificados sus movimientos por medio de conteos programados de acuerdo a lo establecido en este procedimiento.

\subsection{Alcance}
Todo producto que ingrese a las instalaciones del almacén.

\subsection{Terminología y definiciones}
Inventario de suministros y materiales. La percepción de inventario asegura que la compañía tiene materiales a la mano para hacer productos y que los fondos no son desperdiciados en materiales innecesarios. Una cuenta precisa de inventario también permite que las compañías controlen y ordenen suficientes, materiales para la demanda de los consumidores.

\subsection{Documentos y/o normas relacionadas}
\begin{itemize}
	\item Registros de recepción
\end{itemize}

\subsection{Procedimiento}

\subsubsection{Materiales}

\begin{itemize}
	\item N/A
\end{itemize}

\subsubsection{Precauciones de seguridad}

\begin{itemize}
	\item Usar el uniforme completo.
\end{itemize}

\subsubsection{Instrucciones}

\paragraph{Primeras entradas, primeras salidas (PEPS) o VENCER}

\begin{enumerate}
	\item Hacer la revisión correspondiente en cada recepción de producto, la revisión abarca todos los identificadores únicos del envió (número de lote, fechas, cantidades) deben estar documentados.
\end{enumerate}


\begin{enumerate}
	\item Se etiqueta por tarima y por lote cada producto;
	\item La etiqueta de \emph{"Información de material"} contiene los siguientes datos:
	\begin{itemize}
		\item Nombre de producto
		\item Nombre del cliente
		\item Fecha de recepción
		\item Identificador de entrada
		\item Lote
		\item Cantidad
	\end{itemize}
\end{enumerate}

\subparagraph{En operación}

\begin{itemize}
	\item Cada insumo con fecha de caducidad más larga (Producto nuevo), se acomodara en la parte superior trasera de cada rack y los insumos que ya estaban presentes se recorren a la parte inferior frontal.
	\item Al surtir los insumos, el personal de almacén deberá entregar los que tengan la fecha de recepción más antigua o el producto que le cliente asigne para su distribución.
\end{itemize}

\subparagraph{En sistema}

\begin{itemize}
	\item En el sistema de inventarios se cargan las entradas de producto en general al momento de su llegada, en este sistema administra el orden de salida de los mismos y permite dar de baja únicamente en el orden de primeras entradas primeras salidas.
	\item El cliente también pudiera dar instrucciones de cual producto dar salida de forma primaria por estrategia comercial.
	\item En ambos casos el sistema de manera automática da de baja del sistema el producto que físicamente se da salida, actualizando las existencias.
\end{itemize}

\subsection{Responsables de la actividad}

\begin{itemize}
	\item \textbf{Ejecutado} por personal de operaciones;
	\item \textbf{Monitoreado} por personal de calidad;
	\item \textbf{Verificado} por personal de gerencia.
\end{itemize}

\subsection{Acciones preventivas}

\begin{itemize}
	\item Se llevara a cabo una inspección. Registrando el indicador y medida correspondiente
	\item Si la desviación se repite frecuentemente se dará curso de capacitación al personal de almacén y limpieza para que realicen eficientemente su trabajo.
	\item Si después de haber capacitado al personal de almacén se siguen presentando desviaciones por causas injustificadas, se tomaran acciones más enérgicas con el personal por incumplimiento con sus deberes.
\end{itemize}

\subsection{Acciones correctivas}

\begin{itemize}
	\item En caso contrario la tarea se deberá volver a realizar como se indica en el procedimiento.
	\item En caso de no conformidad reportar en formato de acciones correctivas F-OP-40.xls
\end{itemize}

\subsection{Frecuencia}

\begin{itemize}
	\item Diaria.
\end{itemize}

\subsection{Historial de modificaciones}

\begin{itemize}
	\item \textbf{Cuarta edición:} cambio de fecha de 03 de noviembre 2017 a 28 de enero 2019, se realizó cambio en el código de CA-P-PEPV a PR-009. Y no se realizaron modificaciones de la revisión 03 a la 04
	\item \textbf{Quinta edición:} febrero 2020 se hizo cambio de forma y cambio de código de PR-009 a PRO-OP-006. Y no se realizaron modificaciones de la revisión 04 a la 05.
	\item \textbf{Sexta edición:} febrero 2021 no se realizaron modificaciones de la revisión 05 a la 06.
	\item \textbf{Séptima edición:} febrero 2022 no se realizaron cambios de la revisión 06 a la 07.
\end{itemize}

\subsection{Listado de distribución}

N/A

\subsection{Anexos}

N/A
