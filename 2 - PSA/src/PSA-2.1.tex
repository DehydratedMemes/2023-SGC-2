\thispagestyle{formato-PI}
\renewcommand{\MayorVer}{2}
\renewcommand{\MenorVer}{1}
\renewcommand{\FechaPub}{2023--01}
\renewcommand{\TipoID}{PRO}
\renewcommand{\Titulo}{Inspección en la recepción de alimentos}

\section{\Titulo}\index{Inspección!en la recepción de alimentos}
\renewcommand{\Codigo}{\Prog--\thesection--\TipoID}

\subsection{Objetivo}
\begin{itemize}
	\item \textbf{Establecer} un programa satisfactorio que asegure que la unidad de transporte se encuentra en condiciones adecuadas de mantenimiento, limpieza y temperatura apropiadas para el almacenamiento dentro de \gls{RDF}
\end{itemize}

\subsection{Alcance}
\begin{itemize}
	\item Este procedimiento aplica para todas las unidades con alimento alimenticio recibidas en RDF;
	\item se extiende este documento al área de embarques, pero no se limita a otras áreas operativas.
\end{itemize}

\subsection{Términos y definiciones}
\begin{description}
	\defglo{alimento}
	\defglo{criterio-de-acción}
	\defglo{peligro-relacionado-con-la-inocuidad-de-los-alimentos}
\end{description}

\subsection{Documentos y/o normas relacionadas}
\begin{itemize}
	\item Programa de \glsfirst{BPD}.
\end{itemize}

\subsection{Procedimiento}\index{Procedimiento!inspección en la recepción de alimentos}

\subsubsection{Materiales}
\begin{itemize}
	\item Termómetro IR;
	\item Linterna;
	\item Orden de entrada --- \Oent.
\end{itemize}

\subsubsection{Precauciones de seguridad}

\begin{itemize}
	\item Usar \emph{uniforme completo.}
\end{itemize}

\subsubsection{Instrucciones}
\paragraph{Inspección física del alimento}\index{Instrucción!inspección fisica de los alimentos}

\begin{enumerate}
	\item A la llegada del alimento, se verifica la documentación del mismo. \emph{Si se descubre la documentación no es la adecuada,} no se puede descargar el alimento y el \emph{Supervisor de Almacén} debe ser contactado inmediatamente para obtener instrucciones adicionales.
		\begin{enumerate}
			\item Documentos TIF;\footnote{En caso de ser alimentos de procedencia TIF.}
			\item Listado de alimentos con cantidades;
			\item Condiciones de almacenamiento.
		\end{enumerate}
	\item Se verifica la temperatura \emph{programada} de la unidad;
	\item Se revisa que el número de sello coincida con la guía. \emph{Si se descubre que el sello está roto a la llegada,} no se puede descargar el alimento y el \emph{Supervisor de Almacén} debe ser contactado inmediatamente para obtener instrucciones adicionales;
	\item Se \textit{enrampa} la unidad para descargar el alimento;
	\item La condición física del alimento (caja, cubetas, etc.) debe ser inspeccionada, cualquier daño debe ser documentado en el registro de entrada (\Oent);
	\item Se inspecciona la unidad. \emph{Cualquier evidencia de daño, suciedad, o presencia de plagas} deben ser inmediatamente documento e informado al \emph{Supervisor de Almacén} para obtener instrucciones adicionales antes de la descarga. Se debe de revisar:
		\begin{itemize}
			\item Limpieza (presencia de suciedad, escombros, basura, etc.);
			\item Evidencia de cualquier actividad de insectos (excrementos, gomas de mascar, etc.);
			\item Daños físicos (bolsas rotas, cajas dañadas, fugas, etc.);
			\item Daño físico a la unidad que comprometa la inocuidad del alimento;
			\item Evidencia de violación de los materiales de empaque (caja o alimento abierto);
			\item Presencia visible de agentes químicos.\footnote{También es importante reportar olores químicos, como olor a solvente o a plaguicida excesivos.}
		\end{itemize}
	\item Verificar la temperatura del alimento.\footnote{Todo alimento refrigerado o congelado recibido en la instalación debe ser revisado para asegurar que fue enviado de origen a la temperatura adecuada, verificando la documentación correspondiente.}
	\item Se revisarán todos los documentos de envío \emph{(números de lote, fechas, número de orden, etc.)} y la información requerida deberá ser registrada en el formulario de orden de entrada (\Oent);
	\item Se comprobará que si cumple con los criterios de acción para la aceptación (ver \cref{criterios:aceptacion})
	\item si es conforme la temperatura, se procederá a almacenar el alimento en la camara designada según sus especificaciones.
	      \begin{itemize}
		      \item si el \gls{alimento} no requiere de condiciones de almacenamiento especiales\footnote{diferentes a las especificaciones genéricas de almacenamiento (ver \cref{esp:generica}).} se almacenará en la cámara adecuada.
		      \item si el \gls{alimento} requiere de condiciones de almacenamiento especiales y el cliente alquiló una camara exclusiva para su alimento, entonces se almacenará ahí;
		      \item de no ser así, se almacenará el \gls{alimento} en aquella camara compartida que coincida con el rango de las especificaciones del alimento.
	      \end{itemize}
\end{enumerate}

\begin{note}[Sobre las temperaturas de los alimentos en la recepción]
	Es importante mencionar que las temperaturas obtenidas al llegar los alimentos al almacén son registradas con un \emph{termómetro infrarrojo (IR),} este en realidad nos arroja la \emph{temperatura superficial del objeto} que vayamos a medir, por lo que \emph{la temperatura \textbf{puede ser mayor} a la de los criterios de aceptación} para la recepción de alimentos, los rangos de temperatura de almacenamiento estan descritos en \cref{esp:generica}, a continuación se detallan de forma resumida:
	\begin{itemize}
		\item Para \emph{alimentos refrigerados,} \gls{RDF} ha establecido con base en regulaciones gubernamentales el rango de temperatura establecido para el \emph{almacenamiento} de alimentos es de \qtyrange{0}{4}{\degreeCelsius}.
		\item Para \emph{alimentos congelados,} se estableció que la temperatura de almacenamiento debe de ser \qty{<= -12}{\degreeCelsius}.
	\end{itemize}
\end{note}
\begin{note}[Criterios de acción para la aceptación de alimentos]\label{criterios:aceptacion}
	Si la temperatura de los alimentos al realizar la inspección de la mercancía para su \emph{ingreso} está fuera del rango establecido según los rangos establecidos en el \cref{esp:generica}, se deben de tomar en cuenta las siguientes consideraciones para su aceptación:

	\begin{longtblr}[%
		label={esp.crit.acep},
		caption={Criterios de acción para la descarga de alimentos al almacén.},
		note{$\dagger$} = Se tiene que contactar al cliente previo a cualquier procedimiento de acondicionamiento. El cliente es responsable de la preservación del alimento en caso de que rechace el servicio de ráfaga.,
		note{$\ddagger$} = La aceptación de esta carga está a disposición de la disponibilidad de la ráfaga.
		]{%
		colspec={X[c]X[2,c]},
		width = 0.6\linewidth,
		rowhead = 1
		}
		% \toprule
		\textbf{Intervalo}                 & \textbf{Acción}                                                                                                                                                                                                                                                          \\
		% \midrule
		\qty{>=-3}{\degreeCelsius}         & Se rechaza la carga                                                                                                                                                                                                                                                      \\
		\qtyrange{-7}{-4}{\degreeCelsius}  & Se tiene que inspeccionar la carga de manera profunda para descartar cualquier \gls{peligro-relacionado-con-la-inocuidad-de-los-alimentos} y posteriormente se tiene que enviar a la ráfaga hasta lograr congelar el alimento~\TblrNote{$\dagger$}~\TblrNote{$\ddagger$} \\
		\qtyrange{-11}{-8}{\degreeCelsius} & Se acepta la carga con cobro adicional por acondicionamiento                                                                                                                                                                                                             \\
		\qty{<=-12}{\degreeCelsius}        & Se acepta la carga sin cobro adicional por acondicionamiento
	\end{longtblr}
\end{note}

% \begin{itemize}
% 	\item Se considerarán las especificaciones del alimento para determinar si se aceptan o se rechazan los cargamentos.
% 	\item se comprobará que se cumple con la \gls{conformidad} de la temperatura
% 	\item si es conforme la temperatura, se procederá a almacenar el alimento en la camara designada según sus especificaciones.
% 	      \begin{itemize}
% 		      \item si el \gls{alimento} no requiere de condiciones de almacenamiento especiales\footnote{diferentes a las especificaciones genéricas de almacenamiento (ver \cref{esp:generica}).} se almacenará en la cámara adecuada.
% 		      \item si el \gls{alimento} requiere de condiciones de almacenamiento especiales y el cliente alquiló una camara exclusiva para su alimento, entonces se almacenará ahí;
% 		      \item de no ser así, se almacenará el \gls{alimento} en aquella camara compartida que coincida con el rango de las especificaciones del alimento.
% 	      \end{itemize}
% \end{itemize}

\paragraph{Identificación del alimento}
\begin{enumerate}
	\item Al momento del envío de alimentos por parte de los proveedores se genera el documento de STOCK TRANFER el cual de manera inmediata es enviado a \gls{RDF} (control de Inventarios).
	\item Mesa de control recibe el documento \textit{STOCK TRANSFER} y genera el \emph{identificador de control de interno\footnote{Este es nombre que recibe el identificador de tarima.}} antes de la llegada del alimento.
	\item La llegada del embarque a \gls{RDF} se tiene programada en fecha.
	\item Al momento de la llegada del embarque a \gls{RDF}, mesa de control coloca de manera inmediata a la descarga del alimento la etiqueta identificador de control interno, al mismo tiempo que se realizará la verificación de cantidades de alimento recibida
	\item El montacarguista debe trasladar el alimento directamente del transporte a la cámara
\end{enumerate}

\paragraph{Reglamento de permanencia de alimento en andén de carga}
\begin{itemize}
	\item \emph{\textbf{Para alimentos congelados:}} \TiempoAndenCong;
	\item \emph{\textbf{Para alimentos refrigerados:}} \TiempoAndenRefri.
	\item[\textbf{NOTA:}] Este tiempo de permanencia en el andén no aplica para \emph{descargas a granel.}
\end{itemize}

\subsubsection{Responsables de la actividad}
\begin{itemize}
	\item \textbf{Ejecutado} por el personal de operaciones;
	\item \textbf{Monitoreado} por personal de aseguramiento de calidad;
	\item \textbf{Verificado} por personal de gerencia.
\end{itemize}

\subsubsection{Acciones preventivas}
\begin{itemize}
	\item Se llevará a cabo una inspección, registrando el indicador y medida correspondiente;
	\item si la desviación se repite, se capacitará al personal involucrado;
	\item Si después de la capacitación se repite una desviación, se tomarán acciones correctivas (\cref{sec:2.1:acc}).
\end{itemize}

\subsubsection{Acciones correctivas}
\label{sec:2.1:acc}
\begin{itemize}
	\item En caso contrario, la tarea se deberá volver a realizar como indica el procedimiento;
	\item En caso de no-conformidad, reportar en el formato de acciones correctivas.
\end{itemize}

\subsubsection{Frecuencia}

Cada recepción de \gls{alimento}

\begin{changelog}[title=Registro de cambios,simple, sectioncmd=\subsection*]
	\begin{version}[v=2.0, date=2023--01, author=Pablo E. Alanis]
			\item Cambio de formato;
			\item Cambios en la serialización de versiones;
			\item Se cambió "producto" por "alimento";
			\item Separación entre PPR y PPRO.
		\end{version}

	\begin{version}[v=2.1, date=2023--07, author=Pablo E. Alanis]
		\item Correcciones a rangos de temperatura.
	\end{version}
\end{changelog}