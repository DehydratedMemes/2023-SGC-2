\renewcommand{\MayorVer}{2}
\renewcommand{\MenorVer}{1}
\renewcommand{\Codigo}{PSA-1-PROG}
\renewcommand{\FechaPub}{2023--01}
%\renewcommand{\Edit}{2.1}
\renewcommand{\Titulo}{Inspección de producto durante su almacenamiento}

\section{\Titulo}
\index{Inspección!de productos durante su almacenamiento}

\subsection{Objetivo}

Establecer un programa de inspección que asegure que las condiciones de almacenamiento se encuentren en condiciones óptimas de acuerdo a un buen mantenimiento de instalaciones, limpieza y temperatura apropiada para la conservación de productos alimenticios.

\subsection{Alcance}

A todos las áreas de almacenamiento de producto

\subsection{Terminología y definiciones}

N/A

\subsection{Documentos y/o normas relacionados}

\begin{itemize}
	\item 1.0 - BPD - Indice|Programa de buenas prácticas de manufactura
\end{itemize}

\subsection{Procedimiento}

\subsubsection{Materiales}

\begin{itemize}
	\item N/A
\end{itemize}

\subsubsection{Precauciones de Seguridad}

\begin{itemize}
	\item Usar uniforme completo.
\end{itemize}

\subsection{Instrucciones}

\subsubsection{Temperatura de Almacenaje de productos Refrigerados y Congelados.}

\begin{enumerate}
	\item Para asegurar que todo producto refrigerado o congelado se mantenga a la temperatura adecuada durante el almacenamiento, todos los productos se deben colocar directamente después de ser recibido en la cámara de refrigeración o congelación según la asignación, una vez que se descarguen.
	\item Las cámaras de refrigeración y de congelación están equipadas con un termómetro calibrado (RACK), en el cual se puede verificar la temperatura de cada cámara.
	\item Las lecturas de temperaturas deben ser documentadas tres veces durante el turno los días laborados. El Supervisor de mantenimiento es responsable de la verificación de temperaturas.
	\item Las puertas del andén de embarques deben permanecer cerradas en todo momento a menos que la unidad este siendo utilizada para carga y descarga de productos.
	\item Las puertas del andén deben de ser abiertas cuando la unidad de transporte ya se encuentre colocada en la rampa, esto con el fin de evitar fuga de temperatura y / o entrada de polvo o plaga.
	\item Cualquier daño a las cámaras de refrigeración o congelación (techo, paredes, puertas, cortinas, etc.) debe ser reparadas en el menor tiempo posible. El Supervisor de mantenimiento es responsable de garantizar que todo el equipo este en buen estado y funcionando adecuadamente.
\end{enumerate}

\subsubsection{Almacenaje y manejo de Productos Secos / Ambiente, Refrigerados y Congelados}

\begin{enumerate}
	\item Para asegurarse de que los productos ambiente, refrigerados, y congelados son almacenados y protegidos de daños físicos, químicos, biológicos (contaminación), todos los productos deben ser almacenados 30 centímetros desde todas las paredes interiores (si el diseño del almacén lo permite), sobre atados limpios, 30 centímetros de las superficies del techo, así como lejos de todas las luces superiores y nunca almacenados directamente en contacto con los pisos (debe de estar sobre tarima)
	\item Todos los productos deben guardarse en su empaque secundario y etiquetados con las fecha de recepción necesarias.
	\item Todos los productos deben ser rotados de acuerdo en el principio de “First-in-First-Out” \textbf{(FIFO)} o primeras entradas primero a vencer (O según instrucciones de cada cliente).
	\item Un área de “retención” debe haber en el área de almacenamiento para almacenar cualquier artículo recuperado, muestras de productos, producto dañado.
	\item Todos los productos deben mantenerse libres de polvo y suciedad en todo momento.
	\item Cualquier o todos los productos químicos almacenados en la instalación deben estar completamente separados de todos los envases y productos alimenticios – SIN EXCEPCIONES. Todos los productos químicos ubicado en el centro debe tener una hoja de identificación (NO es el caso para este almacén).
\end{enumerate}

\subsubsection{Control y mantenimiento de temperatura}

\begin{enumerate}
	\item Es imperativo que las cámaras en refrigerado y congelado en nuestras instalaciones se mantengan a la temperatura adecuada en todo momento. Esto incluye horas regulares, después de horas regulares, fines de semana y días festivos.
	\item Existe un procedimiento para el seguimiento de temperaturas del refrigerador y del congelador durante el horario normal \%\%CUAL?\%\%las temperaturas son supervisadas y documentadas al día en tres tiempos (Inicio de turno, medio turno y fin de turno) de Lunes a Sábado. El Supervisor de Mantenimiento es inmediatamente informado de cualquier irregularidad detectada en la temperatura (incluyendo horas después del trabajo, fines de semana y días festivos), de modo que las acciones correctivas se puedan tomar.
\end{enumerate}


\subsubsection{Plan alterno en caso de avería de equipos de enfriamiento con productos}

Al fin de garantizar que todos los productos refrigerados y congelados estén protegidos contra cualquier peligro microbiológico asociado con la perdida de temperatura, las cámaras con refrigeración o congelación que presenten una falla y que no puedan ser reparadas en menos en el caso de refrigeración de 4 horas y en congelación de 10 hrs (siempre y cuando las puertas de cámara se encuentren cerradas y el producto mantenga la temperatura indicada), el producto DEBE ser trasladado inmediatamente a otra cámara (previamente enfriada).

\subsection{Responsables de la actividad}

\begin{itemize}
	\item \textbf{Ejecutado} por personal de operaciones y mantenimiento;
	\item \textbf{Monitoreado} por personal de calidad;
	\item \textbf{Verificado} por personal de gerencia.
\end{itemize}

\subsection{Acciones preventivas}

\begin{itemize}
	\item Se llevara a cabo una inspección. Registrando el indicador y medida correspondiente
	\item Si la desviación se repite frecuentemente se dará curso de capacitación al personal de almacén y limpieza para que realicen eficientemente su trabajo.
	\item Si después de haber capacitado al personal de almacén se siguen presentando desviaciones por causas injustificadas, se tomaran acciones más enérgicas con el personal por incumplimiento con sus deberes.
\end{itemize}

\subsection{Acciones correctivas}

\begin{itemize}
	\item En caso contrario la tarea se deberá volver a realizar como se indica en el procedimiento.
	\item En caso de no conformidad reportar en formato de acciones correctivas F-OP-40.xls
\end{itemize}

\subsection{Frecuencia}

\begin{itemize}
	\item \textbf{Verificación de temperatura:} tres veces al día.
\end{itemize}

\subsection{Historial de modificaciones}

\begin{itemize}
	\item \textbf{Cuarta edición:} cambio de fecha de 03 de noviembre 2017 a 28 de enero 2019, se realizó cambio de código de OA-P-IDA a PR-008 y no se realizaron cambios de la revisión 003 a la 004.
	\item \textbf{Quinta edición:} febrero 2020 se hizo cambio de formato y cambio de código de PR-008 a PRO-OP-005. Y no se realizaron cambios de la revisión 04 a la 05.
	\item \textbf{Sexta edición:} febrero 2021 no se realizaron cambios de la revisión 05 a la 06.
	\item \textbf{Séptima edición:} febrero 2022 no se realizaron cambios de la revisión 06 a la 07.
	\item \textbf{Octava edición:} 2023--01: cambios en serialización y en formato.
\end{itemize}
