\thispagestyle{formato-PI}
\renewcommand{\MayorVer}{2}
\renewcommand{\MenorVer}{0}
 %TODO
\renewcommand{\FechaPub}{2023--01}
\renewcommand{\TipoID}{PRO}

\renewcommand{\Titulo}{Acciones correctivas y preventivas}
\section{\Titulo}\index{Información documentada!tipo!procedimiento!Acciones correctivas y preventivas}
\renewcommand{\Codigo}{\Prog--\thesection--\TipoID}

\subsection{Objetivo}

Definir los lineamientos para investigar las causas raíz de las No Conformidades, cerrar y verificar la efectividad de las Acciones Preventivas y Correctivas, además establecer los lineamientos para establecer las acciones necesarias para prevenir su recurrencia.

\subsection{Alcance}

Este procedimiento aplica a todas las áreas de \gls{RDF} desde la recepción hasta la entrega de producto al consumidor final en unidades propias.

\subsection{Términos y definiciones}

\begin{itemize}
	\item \textbf{Observación:} Se entiende como observación a un aspecto de un requisito que podría mejorarse y que no se requiere que se haga de manera inmediata.
	\item \textbf{Acción Correctiva:} Es aquella que llevamos a cabo para eliminar la causa de un problema. Las correcciones atacan los problemas y las acciones correctivas sus causas.
	\item \textbf{Acción Preventiva:} Se anticipan a la causa, y pretenden eliminarla antes de su existencia. Evitan los problemas identificando los riesgos. Cualquier acción que disminuya un riesgo es una acción preventiva.
	\item \textbf{Corrección:} Significa acción para eliminar un defecto o una no conformidad.
	\item \textbf{Desviación interna:} No satisfacción de un límite crítico que puede llevar a la pérdida de control en un PCC (Punto Crítico de Control).
	\item \textbf{No Conformidad:} Incumplimiento de las normas o requisitos establecidos.
	\item \textbf{Requisito:} Necesidad establecida, generalmente implícita u obligatoria.
	\item \textbf{No conformidad Potencial:} Es un incumplimiento menor que no ha ocurrido aún, pero si no se hace algo al respecto, terminará ocurriendo convirtiéndose en incumplimiento real.
	\item \textbf{No Conformidad Mayor (NCM):} Es un incumplimiento que ya ocurrió en el sistema (Incumplimiento real) que afecta a un punto completo de la norma aplicable.
	\item \textbf{No Conformidad menor (NCm):} Es un incumplimiento que puede ya haber ocurrido (Real) o no haber ocurrido aún (Potencial) en el sistema de calidad y que solo afecta parcialmente a un punto de la norma.
	\item \textbf{No conformidad real:} Es un incumplimiento mayor o menor que ya ocurrió
	\item \gls{alimento-RTE}: \glsdesc{alimento-RTE}
\end{itemize}

\subsection{Documentos y/o normas relacionadas}

\begin{itemize}
	\item Procedimiento de Generación y Control de documentos y registros %TODO
	\item 1--BPD --- Programa de buenas prácticas de distribución
	\item \cref{adm:quejas}
\end{itemize}

\subsection{Procedimiento}

\subsubsection{Acción Preventiva}
\begin{enumerate}
\item Se deberá tomar una Acción Preventiva cuando se detecte algún problema o situación que pueda causar una desviación en el proceso y así afectar la calidad o inocuidad del producto.\footnote{Se llama Preventiva porque se actúa anticipadamente para evitar que ocurra.}
\item Para determinar la aplicación de acciones oportunas y efectivas se genera la siguiente clasificación de acciones:
\begin{itemize}
\item \textbf{Acción Preventiva:} Cuando se genera una acción para evitar desviaciones en el proceso y/o sistema.
\item \textbf{Acciones de Mejora:} Cuando se generan acciones para mejorar el proceso y/o sistema.
\end{itemize}
\item Las acciones preventivas para la eliminación de no conformidades potenciales son generadas de las siguientes fuentes:
\begin{enumerate}
	\item Auditorías Internas
	\item Desviaciones en el Proceso de Recepción hasta la entrega final
	\item Desviaciones en la recepción de Materia Prima e Insumos
	\item Encuestas de satisfacción del cliente y reclamaciones.
	\end{enumerate}
\item Una vez detectadas las No conformidades potenciales, Aseguramiento de Calidad llena los datos solicitados en el \RAC~y registra la descripción de la No Conformidad Potencial en la sección
\item El Jefe de Calidad entrega el \RAC~al Responsable del Área a la que se le va a aplicar la Acción Preventiva para el análisis de la causa potencial y el establecimiento de la acción preventiva.
\item El Responsable de Área realiza el análisis de la causa potencial que originó la No Conformidad Potencial, utilizando la metodología Análisis de Causa Raíz, y la registra en la sección III del \RAC; así como la acción Preventiva y las fechas compromiso de cierre y revisión de efectividad.
\item El responsable del área implementa las acciones preventivas definidas en el \RAC~antes de la fecha compromiso.
\item Aseguramiento de Calidad verifica el cumplimiento a la implementación de la acción preventiva acordada.
\item Si la acción preventiva fue cerrada en la fecha acordada marca SI en el \RAC.
\item Si la acción preventiva no fue cerrada en la fecha acordada, marca NO y solicita al responsable de área registrar la causa de incumplimiento, y se define nueva fecha compromiso.
\item El responsable de Aseguramiento de Calidad revisa la efectividad de la acción preventiva implementada en la fecha acordada registrando las acciones que realizó para medir la efectividad y registra las evidencias para la verificación de la efectividad en el Formato de Revisión de la efectividad de las Acciones Correctivas/Preventivas FC-AC-04, posteriormente firma en el campo auditor y solicita firma al Responsable de Área.
\item Dar folio a una \RAC.
\end{enumerate}

\subsubsection{Acción Correctiva}
Las acciones correctivas son generadas a partir de las siguientes fuentes:

\begin{itemize}
	\item Auditorías Internas
	\item Auditorías Externas
	\item Queja de Cliente.
\end{itemize}

\begin{enumerate}
	\item Las Acciones Correctivas generadas con base en \emph{Auditorías Internas y Externas} son registradas en el \RAC~se especifica la descripción de la No conformidad, la causa raíz que la originó, la acción correctiva a realizar; además del responsable de llevarla a cabo y la fecha de cierre.
	\item Aseguramiento de Calidad en conjunto con los responsables de cada acción correctiva; revisa el cumplimiento de la(s) acción(es) de acuerdo a la fecha de cierre, a fin de que se revise su efectividad.
	\item Cuando se tiene una queja de cliente, se registra en el \RACQ, de acuerdo al Procedimiento de Manejo de Quejas y se genera un análisis para determinar la causa raíz y registrar la acción en el \RAC~para corregir la No conformidad y si es necesario realizar un Retiro del Producto especificado en el procedimiento de Trazabilidad.
	\end{enumerate}

\paragraph*{Ejemplos de No Conformidades reportadas por el cliente}
	\begin{itemize}
		\item Mala calidad en los productos.
		\item Daño del alimento dentro del almacén, proceso o en el embarque.
		\item Problemas de inocuidad del producto.
	\end{itemize}


\begin{note}[Herramientas para el análisis de causa raíz] \label{nota:RAC}
	Para lograr llegar a una solución efectiva para erradicar la no-conformidad hallada, se debe de hacer un analisis de causa raíz.
	\begin{itemize}
		\item Se debe de anexar en el formulario de acciones correctivas la evidencia de que se condujo el anteriormente mencionado estudio.
		\item La metodología que se empleará para la elaboración de este analisis es la \emph{lluvia de ideas} \textit{(vide infra) \cref{sec:lluvia}.}
	\end{itemize}
\end{note}

\paragraph{Instrucciones para hacer una lluvia de ideas}\footnote{Vease Hicks, M.J. (1991). Brainstorming. In: Problem Solving in Business and Management. Springer, Boston, MA. para obtener más Información.}\label{sec:lluvia}  
\begin{enumerate}
	\item \textbf{Se debe de establecer el objetivo:} En el caso de una acion correctiva, el objetivo debe de enfocarse en por qué ocurrió la inconformidad en primer lugar;
	\item \textbf{designar a un moderador:} en caso de que la lluvia de ideas se haga en colaboración con multiples departamentos, se tiene que definir a un moderador para evitar que la lluvia de ideas se salga de control;
	\item \textbf{Establece un tiempo concreto:} se debe de establecer un tiempo para concluir la lluvia de ideas. Se recomienda de 60--90 minutos;
	\item \textbf{Toda idea es bienvenida:} en una lluvia de ideas, no se debe de censurar ninguna idea, aun y cuando no se relacione del todo con el tema;
	\item \textbf{Puntuar las ideas obtenidas:} despues del tiempo establecido, se puede establecer un puntaje para cada idea y de tal manera filtrar las que son más adecuadas para determinar la(s) causa(s) de la insatisfacción.
	\item \textbf{Reportar el resultado.}
\end{enumerate}

\subsubsection{Acciones correctivas generadas de Auditorias Interna o Externas}
\begin{enumerate}
	\item En el Caso de la Auditoría Interna el Auditor es quien registra las No Conformidades en el \RAC~y la entrega al responsable de Aseguramiento de Calidad para que lo entregue al área correspondiente y se generen los análisis de causas raíz y se establezcan las acciones correctivas necesarias.
	\item En el Caso de la Auditoría Interna el Auditor es quien registra las No Conformidades en el \RAC~y la entrega al responsable de Aseguramiento de Calidad para que lo entregue al área correspondiente y se generen los análisis de causas raíz y se establezcan las acciones correctivas necesarias.
	\item En el Caso de Auditoría externas el responsable de Aseguramiento de Calidad es quien registra las No Conformidades en el \RAC~y lo entrega al responsable de área para el análisis de causa raíz y establecimiento de acciones correctivas.
	\item El responsable del área auditada, firma el Formato Solicitud de Acción Correctiva / Acción Preventiva F-0013, realiza una investigación para determinar la causa raíz de la desviación y registra el resultado obtenido y posteriormente regresa el Formato al responsable de Aseguramiento de Calidad.
	\item El Responsable de Área realiza el análisis de la causa potencial que originó la No Conformidad Potencial, utilizando la metodología Análisis de Causa Raíz, especificado en el formato F-0016 y la registra en la sección II del formato F-0016; así como la acción Correctiva y las fechas compromiso de cierre y revisión de efectividad.
\end{enumerate}

\subsubsection{Seguimiento y Validación de la Acción Correctiva de Auditoría Interna}
\begin{enumerate}
	\item Aseguramiento de calidad verifica el cumplimiento y la implementación de la acción correctiva acordada y la efectividad de la misma:
	\item Si la acción correctiva fue cerrada en la fecha acordada se marca SI en el Formato Solicitud de Acción Correctiva / Acción Preventiva \RAC.
	\item Si la Acción Correctiva no fue cerrada en la fecha acordada marca NO en el Formato Solicitud de Acción Correctiva / Acción Preventiva \RAC.se debe registrar la causa del porque no se cumplió, en este caso Aseguramiento de Calidad informa al Responsable del área, para que validen la causa y acuerden nueva fecha de cumplimiento.
	\item El responsable de Aseguramiento de Calidad revisa la efectividad de la acción preventiva implementada en la fecha acordada registrando las acciones que realizó para medir la efectividad y registra las evidencias para la verificación de la efectividad en el Formato de Revisión de la efectividad de las Acciones Correctivas/Preventivas F-0014, posteriormente firma en el campo auditor y solicita firma al Responsable de Área.
	\item Cuando la Acción Correctiva es efectiva se envía copia a todo el personal involucrado con asunto de cerrada. La efectividad de la acción correctiva la revisa el encargado del departamento al que corresponda.
	\item En caso de necesitar recursos como por ejemplo: herramientas, materiales o equipos para llevar acabo las Acciones Correctivas o Preventivas se notifica a la Gerencia para su autorización.
\end{enumerate}

\subsubsection{Registros}
\begin{enumerate}
	\item Los registros de \RAC~son almacenados y mantenidos por la Gerencia de acuerdo al área de responsabilidad.
\end{enumerate}

\subsection{Frecuencia}
Al presentarse una desviación cuando se detecte algún problema o situación que pueda causar una desviación en el proceso y así afectar la calidad o inocuidad del producto.

\begin{changelog}[simple, sectioncmd=\subsection*,label=changelog-\thesection-\MayorVer.\MenorVer4]
	\begin{version}[v=\MayorVer.\MenorVer, date=2023--01, author=Pablo E. Alanis]
		% \fixed
		\item Cambio de formato;
		\item Cambios en la serialización de versiones;
	\end{version}

	\begin{version}[v=1.6, date=2022--05, author=Alonso M.]
		\item cambio de fecha;
	\end{version}

	\shortversion{v=1.5, date=2021--05, changes=No hubo cambios}
\end{changelog}

