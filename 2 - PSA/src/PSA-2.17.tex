\thispagestyle{formato-PI}
\renewcommand{\MayorVer}{2}
\renewcommand{\MenorVer}{0}
\renewcommand{\FechaPub}{2023--01}
\renewcommand{\TipoID}{PRO}
\renewcommand{\Titulo}{Manejo de proteína cruda fresca}

\section{\Titulo}\label{PRO-ManejoDeProtFresca}\index{Información documentada!tipo!procedimiento!Manejo de proteína cruda fresca}
\renewcommand{\Codigo}{\Prog--\thesection--\TipoID}
\subsection{Objetivo}
Asegurar que los alimentos que clasifiquen como \emph{proteína fresca cruda} se almacenan de manera que no puedan causar contaminación cruzada.

\subsection{Alcance}
\begin{itemize}
	\item Personal de \emph{embarques} que se encarga de almacenar los \glspl{alimento} al ingresar a \gls{RDF};
	\item personal de \emph{aseguramiento de calidad} que se encarga de vigilar el cumplimiento de este documento y marcar áreas de oportunidad en este tema.
\end{itemize}

\subsection{Términos y definiciones}
\begin{description}
	\defglo{proteina-cruda}
	\defglo{alergeno}
	\item[\glsfirst{alimento-RTE}] \glsdesc{alimento-RTE}
	\defglo{tarima-compuesta}
\end{description}

\subsection{Documentos y/o normas relacionadas}
\begin{itemize}
	\item Programa de Sanidad;
	\item Programa de control de alérgenos.
\end{itemize}

\subsection{Procedimiento}
\subsubsection{Precauciones de seguridad}
	\begin{itemize}
		\item Usar uniforme completo.
	\end{itemize}

\subsubsection{Instrucciones}
\paragraph{Almacenaje de proteína cruda}
A continuación se detallan las consideraciones que deben de tomarse en cuenta para el almacenamiento de alimentos considerados como \emph{proteína cruda fresca:}
	\begin{enumerate}
		\item Todas las \emph{proteínas crudas frescas} deben de ser almacenadas en congelación o refrigeración\footnote{Según las especificaciones del alimento.} estas deben de almacenarse de manera que no sean causa de \emph{contaminación cruzada} con otros productos o materiales de empaque;
		\item No deben almacenarse sobre \gls{alimento-RTE};
		\item No deben de almacenarse sobre materiales de empaque u otros alimentos que se consideren \glspl{alergeno};
		\item Los productos que contengan proteína cruda fresca deben de almacenarse de forma tal que en caso de que ocurriera un derrame no contaminen otros productos, esto quiere decir que solo se deben de almacenar:
		\begin{itemize}
			\item El producto con proteína cruda fresca se debe de almacenar en la parte más baja del \textit{rack} y sobre él, los productos que no contengan proteína cruda.
			\item Por ningún motivo debe de almacenarse alimento \gls{alimento-RTE} debajo de productos que contengan proteína cruda fresca.
			\item El alimento con proteína cruda fresca se debe de almacenar sobre producto con proteína cruda fresca y no se debe de combinar con alimento \gls{alimento-RTE}
			\item Se debe de almacenar productos con proteína cruda fresca sobre producto del mismo tipo de proteína cruda \textit{e.g.\ pollo sobre pollo, pescado sobre pescado.}
			\item Todo alimento debe estar paletizado con el fin de mantener una barrera física con los alimentos almacenados lateralmente.
			\item[Nota:] En caso de derrames, revisar procedimiento de limpieza para proteínas crudas frescas.
		\end{itemize}
	\end{enumerate}

\paragraph{Embarque de proteína cruda}

\begin{enumerate}
	\item Todo alimento debe estar paletizado con el fin de mantener una barrera física con los alimentos almacenados lateralmente;\label{sec1s}
	\item En caso de que hubiera varias tarimas con proteína cruda fresca, estas deben de colocarse juntas al momento del embarque para minimizar la cercanía con otros productos;
	\item Cuando se tienen \glspl{tarima-compuesta}, las proteínas crudas frescas deben de ser acomodadas en la parte inferior de la tarima y estos alimentos deben de ser paletizados de tal manera que exista una barrera física sea colocada, separando los múltiples alimentos que no pertenezcan a la misma categoría;
	\item Por ningún motivo debe de almacenarse alimento \gls{alimento-RTE} debajo de alimentos que contengan proteína cruda fresca;
	\item El producto con proteína cruda fresca se debe de almacenar sobre producto con proteína cruda fresca.
\end{enumerate}

Ver \cref{sec1s}

\paragraph{Alérgenos}

\begin{itemize}
	\item Algunos de los alimentos que contienen proteínas crudas frescas también son consideradas \gls{alergeno} como ejemplos se pueden mencionar el huevo, mariscos, los cuales se deben de manejar con base en lo indicado a este procedimiento y al procedimiento de manejo de productos Alérgenos.
\end{itemize}

\subsection{Responsables de la actividad}

\begin{itemize}
	\item \textbf{Ejecutado} por personal de operaciones
	\item \textbf{Monitoreado} por personal de calidad
	\item \textbf{Verificado} por personal de gerencia.
\end{itemize}

\subsection{Acciones preventivas}

\begin{itemize}
	\item Se llevará a cabo una inspección. Registrando el indicador y medida correspondiente
	\item Si la desviación se repite frecuentemente se dará curso de capacitación al personal de almacén y limpieza para que realicen eficientemente su trabajo.
	\item Si después de haber capacitado al personal de almacén se siguen presentando desviaciones por causas injustificadas, se tomaran acciones más enérgicas con el personal por incumplimiento con sus deberes.
\end{itemize}

\subsection{Acciones correctivas}

\begin{itemize}
	\item En caso de no cumplimiento la tarea se deberá volver a realizar como se indica en el procedimiento.
	\item En caso de no conformidad reportar en formato de acciones correctivas (\RAC)
\end{itemize}

\subsection{Frecuencia}

\begin{itemize}
	\item Cada recepción de producto.
	\item Cada entrega de producto.
\end{itemize}


\begin{changelog}[simple, sectioncmd=\subsection*,label=changelog-\thesection-\MayorVer.\MenorVer7]
	\begin{version}[v=\MayorVer.\MenorVer, date=2023--01, author=Pablo E. Alanis]
		\item Cambios de formato;
		\item cambio de código;
		\item cambios en redacción.
	\end{version}

	\begin{version}[v=1.7, date=2022--01, author=Agustín M.]
		\item Cambio de fecha;
		\item No se realizaron cambios de la revisión 003 a la 004
	\end{version}
\end{changelog}

