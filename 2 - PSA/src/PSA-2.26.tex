\thispagestyle{formato-PI}
\renewcommand{\MayorVer}{2}
\renewcommand{\MenorVer}{0}
 %TODO
\renewcommand{\FechaPub}{2023--01}
\renewcommand{\TipoID}{PRO}

\renewcommand{\Titulo}{Entrega de pequeñas cantidades}
\section{\Titulo}\index{Información documentada!tipo!política!Entrega de pequeñas cantidades}
\renewcommand{\Codigo}{\Prog--\thesection--\TipoID}

\subsection{Objetivos}

\begin{itemize}
	\item \emph{\textbf{Establecer}} una política de despacho de pequeñas cantidades de productos del cliente que se encuentren almacenados en RDF;
	\item \emph{\textbf{Establecer}} un protocolo para la entrega de pequeñas cantidades de productos;
	\item \emph{\textbf{Establecer}} las medidas de seguridad que se deben de tomar al hacer este procedimiento.
\end{itemize}

\subsection{Alcance}

\begin{itemize}
	\item Cliente;
	\item Personal de embarques;
	\item Personal de aseguramiento de calidad;
	\item Personal de mesa de control.
\end{itemize}

\subsection{Términos y definiciones}

\begin{description}
	\defglo{cadena-alimentaria}
	\defglo{cantidades-pequeñas}
\end{description}

\subsection{Documentos y/o normas relacionadas}
\begin{itemize}
	\item \nameref{PRO-ManejoDeProtFresca}
	\item \nameref{ESP-ControlDeEstibas}
\end{itemize}

\subsection{Procedimiento}

\subsubsection{Precauciones de seguridad}
\begin{itemize}
	\item \emph{El cliente} será el responsable, una vez que el producto almacenado en RDF salga de la instalación, de que se procure la cadena de frío;
	\item \emph{RDF} no se hace responsable de el mal almacenamiento después de haber sido despachadado el producto;
	\item Para la entrega de cantidades pequeñas no se requiere de el uso de unidades climatizadas, pero se incentiva.
	\item Debido a que no todas las unidades de transporte son adecuadas para \emph{enrampar,} en caso de que la unidad del cliente no cuente con esta especificación, se entregaran los productos pasando por \emph{mesa de control} y posteriormente se disponen en la unidad  \emph{del cliente.}
	\item Para procurar la inocuidad del producto, no se puede emplear este procedimiento para entregar mercancía a granel;
	\item En el caso de ciertos clientes, RDF funciona como el punto final de la cadena de frío.
\end{itemize}

\subsubsection{Instrucciones}
\begin{enumerate}
	\item Se elabora la orden de salida;
	\item Se ubica el producto que va a ser despachado y se actualiza el inventario en WMS;
	\item Se extraen la cantidad requerida por \emph{el cliente} de forma manual por el \emph{personal de embarques u operaciones;}
	\item Se transfieren a la unidad climatizada o no climatizada \emph{del cliente,} pasando por \emph{mesa de control} y posteriormente se disponen en la unidad del cliente.
\end{enumerate}

\subsection{Responsables de la actividad}
\begin{itemize}
	\item \textbf{Aseguramiento de calidad:} es el responsable de establecer este procedimiento así como sus futuras actualizaciones, así como el monitoreo del cumplimiento de las instrucciones de trabajo establecidas;
	\item \textbf{Personal de embarques:} es el responsable de llevare acabo este procedimiento;
	\item \textbf{Personal de mesa de control:} es el responsable del control de inventario en el almacén.
\end{itemize}

\subsection{Acciones preventivas}
De forma aleatoria, cuando se presente el caso, el \emph{personal de aseguramiento de calidad} verificara que se sigan las instrucciones de trabajo establecidas en este documento;

\subsection{Acciones correctivas}
\begin{itemize}
	\item De no cumplirse con el procedimiento establecido, se procederá a capacitar a los responsables de llevar esta actividad \emph{(vide supra).}
	\item En el caso de que se repita una no conformidad en éste procedimiento se procederá a llenar el formulario de acciones correctivas y se escalará con \emph{el gerente de operaciones.}
\end{itemize}

\subsection{Frecuencia}

Cuando se presente la necesidad.

\begin{changelog}[simple, sectioncmd=\subsection*,label=changelog-\thesection-\MayorVer.\MenorVer6]
	\begin{version}[v=\MayorVer.\MenorVer, date=2023--01, author=Pablo E. Alanis]
		% \fixed
		\item Cambio de formato;
		\item Cambios en la serialización de versiones;
	\end{version}

	\begin{version}[v=1.0, date=2022--05, author=Alonso M.]
		\item Primera versión.
	\end{version}

\end{changelog}
