\thispagestyle{formato-PI}
\renewcommand{\MayorVer}{2}
\renewcommand{\MenorVer}{0}
\renewcommand{\Codigo}{PSA-6-CE}
\renewcommand{\FechaPub}{2023--01}
\renewcommand{\TipoID}{PRO}
\renewcommand{\Titulo}{Procedimiento para solicitud de citas}

\section{\Titulo}\index{Procedimiento!solicitud de citas}
\renewcommand{\Codigo}{\Prog--\thesection--\TipoID}
Estimado cliente, a continuación le presentamos el proceso y lineamientos para la solicitud de citas tanto ingresos como salidas de material.

Le solicitamos que las citas sean solicitadas mínimo con 24 horas de anticipación mediante correo electrónico al equipo de mesa de control en los siguientes horarios:

\begin{itemize}
	\item \textbf{Lunes a viernes de 08:00 am a 16:00 pm}
	\item \textbf{Sábados de 08:00 am a 11:30 am.}
\end{itemize}

Quedando las solicitudes de la siguiente forma:

\definecolor{Gallery}{rgb}{0.929,0.929,0.929}
\begin{longtblr}[
	label = citas:solucitud,
	entry = Procedimiento de solicitud de citas.,
	caption = Horarios para solicitud de citas.
	]{%
	width = \linewidth,
	colspec = {Q[125]Q[129]Q[108]Q[577]},
	cells = {c},
	row{even} = {Gallery}
	}
	\toprule
	Día de solicitud & Hora límite para solicitud & Día de carga & Observación                                                                         \\
	\midrule
	Lunes            & 15:30 pm                   & martes       & La cita debe realizarse antes de las 15:30 pm para que se asigne cita el Martes.    \\
	Martes           & 15:30 pm                   & miércoles    & La cita debe realizarse antes de las 15:30 pm para que se asigne cita el miércoles. \\
	Miércoles        & 15:30 pm                   & jueves       & La cita debe realizarse antes de las 15:30 pm para que se asigne cita el jueves.    \\
	Jueves           & 15:30 pm                   & viernes      & La cita debe realizarse antes de las 15:30 pm para que se asigne cita el viernes.   \\
	Viernes          & 15:30 pm                   & sábado       & La cita debe realizarse antes de las 15:30 pm para que se asigne cita el sábado.    \\
	Sábado           & 11:30 am                   & lunes        & La cita debe realizarse antes de las 11:30 am para que se asigne cita el lunes.     \\
	\bottomrule
\end{longtblr}

\begin{itemize}
	\item \textbf{Los pedidos enviados fuera de estos horarios quedarán a disposición de andén;\footnote{En espera de que tengamos una ventana disponible.}}
	\item Es \emph{requisito indispensable} presentar el correo de confirmación impreso, que especifica el número de pedido a surtir y la hora asignada a su cita;
	\item Es importante que en la solicitud de la cita nos indiquen el \emph{nombre completo de la persona que realizará la recolección o entrega,} así como los datos de la unidad:
	      \begin{itemize}
		      \item Línea de transporte;
		      \item placas;
		      \item número de tractor y placas;
		      \item número de contenedor.
	      \end{itemize}
	\item Nuestro horario de atención sin cargo de tiempo extra es de 8:00 a 17:00 de lunes a viernes. El camión a carga o descarga debe presentarse una hora antes de las 17:00 para que no se extienda su servicio; lo mismo aplica para los sábados de 8:00 a 13:00; la última cita se otorga a las 16:00 de lunes a viernes y los sábados a las 12:00.
\end{itemize}

\newpage

\subsection{Contactos Red de Fríos}

\begin{contact}[Mesa de control] \label{contact:MesaDeControl}
	Magdalena Aguilar Robledo\\
	Teléfono: \textbf{(81) 8351-6685 ext. 105}\\
	\email{mesacontrolrayon@reddefrios.com}
\end{contact}

\begin{contact}[Jefe de almacén] \label{contact:JefeDeAlmacen}
	Luis Ángel Escareño Aguirre\\
	Teléfono: \textbf{(81) 8351-6685 ext. 105}\\
	Celular: \textbf{811-390-7099}\\
	\email{lescareno@reddefrios.com}
\end{contact}

\begin{contact}[Gerente de operaciones] \label{contact:Gerente}
	Gerardo Ruiz Torres, QFB\\
	Teléfono: \textbf{(81) 8351-6685 ext. 102}\\
	Celular: \textbf{818-029-5609}\\
	\email{gruiz@reddefrios.com}
\end{contact}

\begin{contact}[Médico TIF] \label{contact:MedicoTIF}
	M.V.Z. Rocío Luna\\
	Teléfono: \textbf{(81) 8351-6685 ext. 103}\\
	\email{luna\_mvz\_ceu@live.com.mx}
\end{contact}

Es necesario indicar:
\begin{enumerate}
	\item Nombre del producto a guardar o retirar;
	\item factura que lo ampara;
	\item código con el que usted lo maneja\footnote{Se dará de alta el mismo en nuestro sistema};
	\item lote;
	\item fecha de caducidad;
	\item cantidad de tarimas y/o cajas que se guardarán o retirarán;
	\item especificaciones de temperatura de almacenamiento.
	\item alérgenos presentes en el \gls{alimento} (ver \cref{nota:Alergenos})
\end{enumerate}

\begin{note}[Listado de alérgenos según la NOM-051-SCFI/SSA1-2010] \label{nota:Alergenos}
	Es de suma importancia que nos indique si el producto a resguardar es un Alérgeno o contiene algún ingrediente Alérgeno en su composición.
	\begin{itemize}
		\item Leche
		\item Trigo
		\item Frutas Secas
		\item Huevos
		\item Soya y derivados
		\item Cacahuates
		\item Pescados
		\item Crustáceos
		\item sulfitos en concentración \qty{\geq 10}{\milli\gram\per\kilo\gram}.
	\end{itemize}
\end{note}

Para carga o descarga requerimos contar con los datos de la unidad, placas, nombre del chofer y línea de transporte.

Para ingresos de mercancía es necesario presentar la siguiente documentación:

\begin{itemize}
	\item Factura, remisión y pedimento (para importados) del producto a recibir.
	\item Certificado de calidad de los productos a recibir.
	\item Certificado de fumigación de la unidad de transporte.
\end{itemize}

En caso de ser \emph{producto cárnico} es necesario presentar \emph{Aviso de Movilización TIF} con destino a nuestro almacén \emph{TIF 278.}

Para cuestión de \emph{Pescados y Mariscos} es necesario presentar la \emph{Guía de Pesca} del producto con destino a nuestro almacén TIF 278.

Para la \emph{salida de producto TIF} se requiere se nos informen los datos de la unidad que lo cargará:
\begin{itemize}
	\item Línea de Transporte;
	\item chofer;
	\item placas;
	\item marca del contenedor o caja;
	\item número de destino TIF.
\end{itemize}

\begin{note}[Datos de aviso TIF]\label{note:DatosAvisoTIF}
	TIF 278\\
	Red de Fríos S.A. de C.V.\\
	Av. I. López Rayón \# 2810.\\
	Col. Bella Vista.\\
	C.P. 64410, Monterrey N.L., México
\end{note}

\begin{changelog}[title=Registro de cambios,simple, sectioncmd=\subsection*,label=changelog-\thesection-\MayorVer.\MenorVer]
	\begin{version}[v=\MayorVer.\MenorVer, date=2023--01, author=Pablo E. Alanis]
		\item Cambio de formato;
		\item los contactos se agregaron a recuadros;
		\item cambio de código;
		\item se agregó recuadro sobre alérgenos.
	\end{version}

	\begin{version}[v=1.7, date=2022--05, author=Alonso M.]
		\item No se realizaron modificaciones de la revisión 05 a la 06.
	\end{version}
\end{changelog}