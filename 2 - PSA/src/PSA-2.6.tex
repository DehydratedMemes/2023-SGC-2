\renewcommand{\MayorVer}{2}
\renewcommand{\MenorVer}{1}
\renewcommand{\Codigo}{PSA-1-PROG}
\renewcommand{\FechaPub}{2023--01}
%\renewcommand{\Edit}{2.1}
\renewcommand{\Titulo}{Procedimiento para solicitud de citas}

\section{\Titulo}
\index{Procedimiento!PEPS o VENCER}

Estimado cliente, a continuación le presentamos el proceso y lineamientos para la solicitud de citas tanto ingresos como salidas de material.

Le solicitamos que las citas sean solicitadas mínimo con 24 horas de anticipación mediante correo electrónico al equipo de mesa de control en los siguientes horarios:

\textbf{Lunes a viernes de 08:00 am a 16:00 pm}
\textbf{Sábados de 08:00 am a 11:30 am}.

Quedando las solicitudes de la siguiente forma:

\definecolor{Gallery}{rgb}{0.929,0.929,0.929}
\begin{longtblr}[
  label = citas:solucitud,
  entry = Procedimiento de solicitud de citas.,
]{
  width = \linewidth,
  colspec = {Q[125]Q[129]Q[108]Q[577]},
  cells = {c},
  row{even} = {Gallery},
    hline{1,8} = {-}{0.08em},
  hline{2} = {-}{0.05em},
}
Día de solicitud & { Hora límite para solicitud } & Día de carga & Observación\\
Lunes & 15:30   pm & martes & La cita debe   realizarse antes de las 15:30 pm para que se asigne cita el Martes.\\
Martes & 15:30   pm & miércoles & La cita debe realizarse antes de las   15:30 pm para que se asigne cita el miércoles.\\
Miércoles & 15:30   pm & jueves & La cita debe realizarse antes de las   15:30 pm para que se asigne cita el jueves.\\
Jueves & 15:30   pm & viernes & La cita debe realizarse antes de las   15:30 pm para que se asigne cita el viernes.\\
Viernes & 15:30   pm & sábado & La cita debe realizarse antes de las   15:30 pm para que se asigne cita el sábado.\\
Sábado & 11:30   am & lunes & La cita debe realizarse antes de las   11:30 am para que se asigne cita el lunes.
\end{longtblr}
\textbf{Los pedidos enviados fuera de estos horarios quedarán a disposición de andén (en espera de que tengamos una ventana disponible).}

Es requisito indispensable presentar el correo de confirmación impreso el cual consta del número de pedido a surtir y la hora asignada a su cita.

Es importante que en la solicitud de la cita nos indiquen el nombre completo de la persona que realizará la recolección o entrega, así como los datos de la unidad: Línea de transporte, placas y número de tractor y placas y número de contenedor.

Nuestro horario de atención sin cargo de tiempo extra es de 8:00 a.m. a 17:00 p.m. de lunes a viernes, el camión a carga o descarga debe presentarse 1 hora antes de las 5 p.m. para que no se extienda su servicio después de las 5 p.m.; lo mismo aplica para los sábados de 8:00 a.m. a 1:00 p.m. La última cita se otorga a las 4:00 p.m. de lunes a viernes y los sábados a las 12:00 p.m.

\subsection{Contactos Red de Fríos}

\subsubsection*{Mesa de Control} %TODO actualizar datos
ILSE BERENICE BECERRA VELEZ\\
Teléfono: \textbf{(81) 8351-6685 ext. 105}\\
\email{mesacontrolrayon@reddefrios.com}

\subsubsection*{Jefe de Almacén.}
Luis Angel Escareño Aguirre\\
Teléfono: \textbf{(81) 8351-6685 ext. 105}\\
Celular: \textbf{811-390-7099}\\
\email{lescareno@reddefrios.com}

\subsubsection*{Gerente de Operaciones}
QFB Gerardo Ruiz Torres\\
Teléfono: \textbf{(81) 8351-6685 ext. 102}\\
Celular: \textbf{818-029-5609}\\
\email{gruiz@reddefrios.com}

Es necesario indicar nombre del producto a guardar o retirar, factura que lo ampara, código con el que usted lo maneja (se dará de alta el mismo en nuestro sistema), lote, fecha de caducidad, cantidad de tarimas o cajas que requiere guardar o retirar y temperatura de resguardo.

También es de suma importancia que nos indique si el producto a resguardar es un Alérgeno o contiene algún ingrediente Alérgeno en su composición.

\subsubsection*{Listado de Productos Alérgenos}
\begin{itemize}
	\item Leche
	\item Trigo
	\item Frutas Secas
	\item Huevos
	\item Soya
	\item Cacahuates
	\item Pescados
	\item Crustáceos
\end{itemize}

Para carga o descarga requerimos contar con los datos de la unidad, placas, nombre del chofer y línea de transporte.

Para ingresos de mercancía es necesario presentar la siguiente documentación:

\begin{itemize}
	\item Factura, remisión y pedimento (para importados) del producto a recibir.
	\item Certificado de calidad de los productos a recibir.
	\item Certificado de fumigación de la unidad de transporte.
\end{itemize}

En caso de ser producto Cárnico es necesario presentar Aviso de Movilización TIF con destino a nuestro almacén TIF 278.

Para cuestión de Pescados y Mariscos es necesario presentar la Guía de Pesca del producto con destino a nuestro almacén TIF 278.

Para la salida de producto TIF se requiere nos informe los datos de la unidad que lo cargará: Línea de Transporte, Chofer, Placas y Marca del Contenedor o Caja y Número de Destino TIF; Favor de incluir en la solicitud de cita al Médico TIF Rocío Luna.

\subsubsection*{Médico TIF}
M.V.Z. Rocío Luna\\
Teléfono: \textbf{(81) 8351-6685 ext. 103}\\
\email{luna\_mvz\_ceu@live.com.mx}

\subsubsection*{DATOS PARA AVISO TIF:}
TIF 278\\
Red de Fríos S.A. de C.V.\\
Av. I. Lopez Rayón \# 2810.\\
Col. Bella Vista.\\
C.P. 64410, Monterrey N.L., México
