\renewcommand{\MayorVer}{2}
\renewcommand{\MenorVer}{1}
\renewcommand{\Codigo}{PSA-1-PROG} %TODO
\renewcommand{\FechaPub}{2023--01}
%\renewcommand{\Edit}{2.1}
\renewcommand{\Titulo}{Manejo de producto retenido}

\section{\Titulo}
\index{Producto retenido!manejo}


\subsection{Objetivo}

Indicar el procedimiento adecuado del manejo del producto NO comestible que se encuentre dentro de los almacenes de RED DE FRIOS.

\subsection{Alcance}

Todos los productos dentro de almacén que sean no comestibles y se encuentren retenidos.

\subsection{Términos y definiciones}

\begin{itemize}
	\item TODO
\end{itemize}

\subsection{Documentos y/o normas relacionadas}

\begin{itemize}
	\item 1.0 - BPD - Indice|Programa de BPM
	\item 2.1 - PSA-OP-IT-3 - Inspección en la recepción de productos|Procedimiento de Inspección de transportes
\end{itemize}

\subsection{Procedimiento}

\subsubsection{Materiales}

\begin{itemize}
	\item Bitácora de registro
\end{itemize}

\subsubsection{Precauciones de seguridad}

\begin{itemize}
	\item Usar uniforme completo
\end{itemize}

\subsubsection{Instrucciones}

\paragraph{Procedimientos de Producto Detenido por parte de }

Con el fin de garantizar la eliminación de productos no conformes de la distribución y cadena de suministro, cualquier producto no conforme que ha sido identificado por parte de CALIDAD debe ser retenido y reportado al Supervisor de Almacén y/o Jefe de Almacén, que debe permanecer involucrado durante todo el proceso para garantizar el cumplimiento con los procedimientos y normas establecidas; Se debe dar aviso al cliente de la acción tomada por parte de CALIDAD y enviar el comunicado del motivo de esta retención.

\begin{itemize}
	\item Producto dañado
	\item Producto abierto (expuesto)
	\item Producto con presencia de suciedad
	\item Producto caducado
	\item Producto con evidencia de problema de inocuidad
	\item Entre otras
\end{itemize}

\begin{enumerate}
	\item Cuando CALIDAD durante la inspección al momento de la recepción o durante su almacenamiento detecta producto que se encuentra dañado, contaminado (visualmente), en mal estado debe de retener este producto para evitar que continúe la cadena de suministro y pueda ser utilizado para proceso o consumo y pueda generar un riesgo en el consumidor.
	\item CALIDAD genera la indicación del no recibo del producto si el producto no cumple con las especificaciones de inocuidad y que puede generar un riesgo para el resto de los productos en el almacén y al consumidos
	\item Si el producto es detectado durante su almacenamiento, CALIDAD genera la indicación de la separación de este producto y la identificación del mismo como producto rechazado o decomisado.
	\item Este producto es colocado en el área de producto retenido.
	\item CALIDAD genera la información del motivo del decomiso como evidencia de la acción tomada.
	\item CALIDAD genera la cita con la planta de rendimiento (autorizada por SADER) para realizar la eliminación del producto decomisado, con el fin de evitar que se pueda generar una contaminación cruzada.
	\item Al momento que la planta de rendimiento realiza la recolección, se genera la documentación con la información del producto en cuestión.
	\item Los registros de la recolección se archivan como evidencia de la operación realizada.
	\item El almacén tiene la obligación de generar de igual manera los registros de las acciones tomadas.
	\item El almacén tiene la obligación de dar aviso al cliente de las acciones tomadas de su producto por parte de CALIDAD y mantener informado de los pasos del proceso hasta el retiro del producto.
\end{enumerate}

\paragraph{Procedimientos de Producto Detenido por calidad del cliente}

Con el fin de garantizar la eliminación de productos no conformes de la distribución y cadena de suministro, cualquier producto no conforme que ha sido identificado debe ser reportado al Supervisor de Almacén y Coordinador de Calidad, que debe permanecer involucrado durante todo el proceso para garantizar el cumplimiento con los procedimientos y normas establecidas.

\begin{itemize}
	\item Producto dañado
	\item Producto abierto
	\item Producto con presencia de suciedad
	\item Producto caducado
	\item Entre otras
\end{itemize}

\begin{enumerate}
	\item Una vez que el producto ha sido identificado, se deberá documentar los detalles y avisar al Gerente de Operaciones así como el cliente en cuestión.
	\item Etiquetas de \textit{"Retenido"} deben ser generadas y se colocan identificando al producto señalado para indicar claramente que el producto en cuestión está \textit{RETENIDO.} Las etiquetas deben tener la siguiente información como referencia:
	\begin{enumerate}
		\item Fecha
		\item Producto
		\item Cantidad
		\item Número de lote (si aplica)
		\item Razón por la detención
	\end{enumerate}
	\item El Gerente de Operaciones de RDF contactara al Cliente para dar aviso y valorar el producto retenido.
	\item El Coordinador de Calidad del Cliente o el cliente emitirá la acción a realizar con el producto retenido.
	\item Todas las instrucciones recibidas del cliente debe ser remitida al Gerente de Operaciones, así como el Supervisor de Almacén para su revisión.
	\item Las instrucciones del cliente en cuestión definirá:
	\begin{enumerate}
		\item Si el producto va ser devuelto a ellos (inmediatamente será enviado al cliente)
		\item Si el producto va ser destruido (El Supervisor de Almacén seguirá los procedimientos establecidos de destrucción y el producto será destruido).
	\end{enumerate}
	\item El supervisor de almacén registrara el evento para generar la evidencia de la detención y el destino del producto.
	\item El supervisor de almacén es responsable de la realización de un inventario semanal de todos los productos en “Detenido” para asegurar el correcto manejo y disposición de dichos productos. Los Inventarios debe llevarse a cabo incluso si la zona detenida está vacante o vacía (que indica “cero” en el Reporte de Inventario de Producto Detenido).
\end{enumerate}

\paragraph{Procedimiento para destrucción de productos}

\begin{enumerate}
	\item Los departamentos que autoriza la destrucción del producto son el departamento de Calidad, departamento de Finanzas e Inventarios y cliente.
	\item Una vez que el cliente autoriza que el producto sea destruido, se procede a separar, etiquetar e identificar el producto no conforme. La autorización debe de ser de manera escrita.
	\item El producto no conforme será eliminado (Si el producto lo requiere) con un producto de tinta desnaturante y llevado al área asignada para almacenar producto retenido y programar su retiro y destrucción.
	\item Posterior a la destrucción RDF debe de enviar notificación y evidencia de la destrucción del producto.
	\item Los proveedores que pueden ser utilizados para la destrucción de los productos pudiera ser:
	\begin{itemize}
		\item \textbf{RENGRA:} para productos perecederos Congelados y Refrigerados.
		\item \textbf{SIMEPRODE:} para productos secos varios.
	\end{itemize}
\end{enumerate}


\subsection{Responsables de la actividad}

\begin{itemize}
	\item \textbf{Ejecutado} por personal de CALIDAD
	\item \textbf{Verificado} por personal de gerencia.
\end{itemize}

\subsection{Acciones preventivas}

\begin{itemize}
	\item Se llevara a cabo una inspección. Registrando el indicador y medida correspondiente
	\item Si la desviación se repite frecuentemente se dará curso de capacitación al personal de almacén y limpieza para que realicen eficientemente su trabajo.
	\item Si después de haber capacitado al personal de almacén se siguen presentando desviaciones por causas injustificadas, se tomaran acciones más enérgicas con el personal por incumplimiento con sus deberes.
\end{itemize}

\subsection{Acciones correctivas}

\begin{itemize}
	\item En caso de no cumplimiento la tarea se deberá volver a realizar como se indica en el procedimiento.
	\item En caso de no conformidad reportar en formato de acciones correctivas F-OP-40.xls|F-OP-40
\end{itemize}

\subsection{Frecuencia}

Cada eventualidad que amerite separación de producto.

\subsection{Historial de modificaciones}

\begin{itemize}
	\item Cuarta edición: cambio de fecha de 04 de noviembre 2017 a 28 de enero 2019, se realizó cambio de código de CA-P-MPR a PR-013. Y no se realizaron cambios de la revisión 003 a la 004
	\item Quinta edición: febrero 2020 se hizo cambio de formato y cambio de código de PR-013 a PRO-OP-009. Y no se realizaron cambios de la revisión 04 a la 05.
	\item Sexta edición: febrero 2021 no se realizaron cambios de la revisión 05 a la 06.
	\item Séptima edición: febrero 2022 no se realizaron cambios de la revisión 06 a la 07.
\end{itemize}

% \subsection{Anexos}

% \subsubsection{Contactos}
% \paragraph{GEN}
% Daniel González Treviño\\
% Asesor Ambiental\\
% GEN Industrial Monterrey\\
% \phonenumber{+52 81442200}[2245]\\
% \phonenumber{+52 812416373}

% \paragraph{RENGRA}

% Rendimientos Grasos
% Pedro Antonio Molina Herrera, C.P.
% Adquisición de Materia Prima
% Rengra, S.A. de C.V
% \phonenumber{+5281543210}
% \phonenumber{+5281543216}
% \url{http://www.rengra.com.mx}

% \paragraph{SIMEPRODE}
% Javier Jiménez Pacheco, Ing
% Director de Operaciones
% Emilio Carranza No. 730 sur, - 2° Piso, entre Padre Mier y Matamoros,
% Monterrey, N.L. C.P. 64000
% Tel. 2020-9500; 9500
% \url{mailto:simeprode@nuevoleon.gob.mx}
