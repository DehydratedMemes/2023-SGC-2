\thispagestyle{formato-PI}
\renewcommand{\MayorVer}{2}
\renewcommand{\MenorVer}{0}
\renewcommand{\Codigo}{PSA-1-PROG} %TODO
\renewcommand{\FechaPub}{2023--01}
%\renewcommand{\Edit}{2.1}
\renewcommand{\Titulo}{Manejo de producto retenido}
\section{\Titulo}
\index{Producto retenido!manejo de}

\subsection{Objetivo}
Establecer el procedimiento adecuado para el manejo de \emph{aliemntos no comestibles} que se encuentren dentro de los almacenes de \gls{RDF}.

\subsection{Alcance}
Todos los productos dentro de almacén que sean no comestibles y se encuentren retenidos.

\subsection{Términos y definiciones}
\begin{description}
	\defglo{retiro-de-producto}
	\defglo{recuperación-de-producto}
	\defglo{idoneidad-de-los-alimentos}
	\defglo{inocuidad-de-los-alimentos}
\end{description}

\subsection{Documentos y/o normas relacionadas}
\begin{itemize}
	\item Programa de \gls{BPD}
\end{itemize}

\subsection{Procedimiento}
\subsubsection{Manejo de producto retenido por cuestiones de idoneaidad (por parte de RDF)}
\index{Manejo de producto retenido!por cuestiones de calidad (por parte de RDF)}
Con el fin de garantizar la eliminación de productos no conformes de la distribución y cadena de suministro, cualquier producto no conforme que ha sido identificado por parte de \emph{aseguramiento de calidad} debe ser retenido y reportado al \emph{Jefe de Almacén,} que debe permanecer involucrado durante todo el proceso para garantizar el cumplimiento con los procedimientos y normas establecidas; Se debe dar aviso al cliente de la acción tomada por parte de aseguramiento de calidad y enviar el comunicado del motivo de esta retención.

\begin{itemize}
	\item Producto dañado
	\item Producto abierto (expuesto)
	\item Producto con presencia de suciedad
	\item Producto caducado
	\item Producto con evidencia de problema de inocuidad
	\item Entre otras
\end{itemize}

\begin{enumerate}
	\item Cuando aseguramiento de calidad durante la inspección al momento de la recepción o durante su almacenamiento detecta producto que se encuentra dañado, contaminado (visualmente), en mal estado debe de retener este producto para evitar que continúe la cadena de suministro y pueda ser utilizado para proceso o consumo y pueda generar un riesgo en el consumidor.
	\item aseguramiento de calidad genera la indicación del no recibo del producto si el producto no cumple con las especificaciones de inocuidad y que puede generar un riesgo para el resto de los productos en el almacén y al consumidor
	\item Si el producto es detectado durante su almacenamiento, aseguramiento de calidad genera la indicación de la separación de este producto y la identificación del mismo como producto rechazado o decomisado.
	\item Este producto es colocado en el área de producto retenido.
	\item aseguramiento de calidad genera la información del motivo del decomiso como evidencia de la acción tomada.
	\item aseguramiento de calidad genera la cita con la planta de rendimiento (autorizada por SADER) para realizar la eliminación del producto decomisado, con el fin de evitar que se pueda generar una contaminación cruzada.
	\item Al momento que la planta de rendimiento realiza la recolección, se genera la documentación con la información del producto en cuestión.
	\item Los registros de la recolección se archivan como evidencia de la operación realizada.
	\item El almacén tiene la obligación de generar de igual manera los registros de las acciones tomadas.
	\item El almacén tiene la obligación de dar aviso al cliente de las acciones tomadas de su producto por parte de aseguramiento de calidad y mantener informado de los pasos del proceso hasta el retiro del producto.
\end{enumerate}


\subsubsection{Manejo de producto retenido por cuestiones de idoneaidad (por parte del cliente)}
\index{Manejo de producto retenido!por cuestiones de calidad (por parte del cliente)}
El cliente puede decidir, por razones de calidad, la retención de producto. Esto es cuando el alimento sigue considerándose inocuo, sin embargo que no cumpla con alguna especificación del cliente. \textit{i.e.} exceso de sal, etc. 

\begin{enumerate}
	\item Se localizan las tarimas del producto que el cliente solicitó retener;
	\item se identifican tarimas con un letrero rojo que indique la retención, esta debe de contener:
		\begin{itemize}
		\item Fecha de retención
		\item Producto
		\item Cantidad
		\item Número de lote (si aplica)
		\item Razón por la detención
		\end{itemize}
	\item si el espacio dentro de la cámara lo permite, se reubican las tarimas retenidas en una zona establecida dentro de la cámara;
	\item el \emph{gerente de almacén} notifica al cliente que los lotes han sido retenidos
	\item cuando se puedan liberar estos productos, se retirarán las etiquetas.
	\item Las instrucciones del cliente en cuestión definirá:
	\begin{enumerate}
		\item Si el producto va a ser devuelto al cliente;
		\item Si el producto va a ser destruido.\footnote{El Supervisor de Almacén seguirá los procedimientos establecidos de destrucción y el producto será destruido.}
	\end{enumerate}
	\item El supervisor de almacén registrará el evento para generar la evidencia de la detención y el destino del producto.
\end{enumerate}

\subsubsection{Procedimiento para retiro de producto}
\index{Manejo de producto retenido!por cuestiones de inocuidad!retiro de productos}
En caso de que se tenga que hacer un \gls{retiro-de-producto}, el producto retenido por razones de inocuidad tendrá que comunicarse con el cliente para determinar si ellos serán los que se encargarán de la destrucción del producto o si \GLS{RDF} llevará a cabo la destrucción, contratando servicios de las compañías descritas en \cref{sec:destprod}.

\subsubsection{Destrucción de producto}
\index{Manejo de producto retenido!por cuestiones de inocuidad!destrucción de producto}
\label{sec:destprod}
\begin{enumerate}
	\item Los departamentos que autoriza la destrucción del producto son el departamento de aseguramiento de calidad, departamento de Finanzas e Inventarios y cliente.
	\item Una vez que el cliente autoriza que el producto sea destruido, se procede a separar, etiquetar e identificar el producto no conforme. La autorización debe de ser de manera escrita.
	\item El producto no conforme será eliminado (Si el producto lo requiere) con un producto de tinta desnaturante y llevado al área asignada para almacenar producto retenido y programar su retiro y destrucción.
	\item Posterior a la destrucción RDF debe de enviar notificación y evidencia de la destrucción del producto.
	\item Los proveedores que pueden ser utilizados para la destrucción de los productos pudiera ser:
	\begin{itemize}
		\item \textbf{RENGRA:} para productos perecederos Congelados y Refrigerados (ver \cref{contacto:RENGRA});
		\item \textbf{SIMEPRODE:} para productos secos varios (ver \cref{contacto:SIMEPRODE}).
	\end{itemize}
\end{enumerate}


\subsection{Responsables de la actividad}
\begin{itemize}
	\item \textbf{Ejecutado} por personal de aseguramiento de calidad
	\item \textbf{Verificado} por personal de gerencia.
\end{itemize}

\subsection{Acciones preventivas}
\begin{itemize}
	\item Se llevará a cabo una inspección. Registrando el indicador y medida correspondiente
	\item Si la desviación se repite frecuentemente se dará curso de capacitación al personal de almacén y limpieza para que realicen eficientemente su trabajo.
	\item Si después de haber capacitado al personal de almacén se siguen presentando desviaciones por causas injustificadas, se tomaran acciones más enérgicas con el personal por incumplimiento con sus deberes.
\end{itemize}

\subsection{Acciones correctivas}
\begin{itemize}
	\item En caso de no cumplimiento la tarea se deberá volver a realizar como se indica en el procedimiento;
	\item en caso de no conformidad reportar en formato de acciones correctivas (\IdFormAACC).
\end{itemize}

\subsection{Frecuencia}

Cada eventualidad que amerite separación de producto.

\begin{changelog}[simple, sectioncmd=\subsection*,label=changelog-2.13]
	\begin{version}[v=2.0, date=2023--01, author=Pablo E. Alanis]
		\item Cambios de formato;
		\item cambio de código;
		\item cambios en redacción.
	\end{version}

	\begin{version}[v=1.9, date=2022--01, author=Agustín M.]
		\item febrero 2022 no se realizaron cambios de la revisión 06 a la 07.
	\end{version}
\end{changelog}

\newpage

\section*{Anexos}
\subsection*{Contactos}
\index{Manejo de producto retenido!por cuestiones de inocuidad!destrucción de producto!contactos}

\begin{contact}[GEN]\label{contacto:GEN}
\noindent Daniel González Treviño\\
Asesor Ambiental\\
GEN Industrial Monterrey\\
+52 81442200\\
+52 812416373
\end{contact}

\begin{contact}[RENGRA]\label{contacto:RENGRA}
\noindent Rendimientos Grasos\\
Pedro Antonio Molina Herrera, C.P.\\
Adquisición de Materia Prima\\
Rengra, S.A. de C.V\\
+52 81543210\\
+52 81543216\\
\url{http://www.rengra.com.mx}
\end{contact}

\begin{contact}[SIMEPRODE]\label{contacto:SIMEPRODE}
\noindent Javier Jiménez Pacheco, Ing.\\
Director de Operaciones\\
Emilio Carranza No. 730 sur, - 2° Piso, entre Padre Mier y Matamoros,\\
Monterrey, N.L. C.P. 64000\\
Tel. 2020-9500; 9500\\
\url{mailto:simeprode@nuevoleon.gob.mx}
\end{contact}