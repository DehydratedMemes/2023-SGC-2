\renewcommand{\MayorVer}{2}
\renewcommand{\MenorVer}{1}
\renewcommand{\Codigo}{PSA-1-PROG} %TODO
\renewcommand{\FechaPub}{2023--01}
%\renewcommand{\Edit}{2.1}
\renewcommand{\Titulo}{Resguardo de registros}

\section{\Titulo}
\index{Información documentada!control!resguardo de registros}

% \section{Archivo de registros: Disposición de documentación}

\subsection{Objetivo}

Establecer tiempos de resguardo de archivos y generar la guía que manejaran de los mismos para mantenerlos disponibles en caso de ser requeridos.

\subsection{Alcance}

Toda la documentación generada dentro del almacén que tenga relevancia en la operación y administración.

\subsection{Términos y definiciones}

N/A

\subsection{Documentos y/o normas relacionadas}

\begin{itemize}
	\item Todos los documentos en uso dentro RED DE FRIOS.
	\item Información eléctrica PROTHEUS.
\end{itemize}

\subsection{Procedimiento}

\subsubsection{Instrucciones}

\paragraph{Registros y manejo de Documentos}

El Gerente de Operaciones es responsable del mantenimiento del Programa de Seguridad Alimentaria, todos los cambios introducidos en el manual y los documentos que deben tenerse en cuenta deben identificar “elaboro”, seguido de la fecha de elaboración y la fecha de actualización.

\begin{itemize}
	\item Los registros (papel) de recepción, almacenamiento, embarque y distribución de producto, y todos los registros involucrados para rastreabilidad de producto, órdenes de mantenimiento, estarán a la mano para su rápida revisión por NO MENOS DE DOS AÑOS.
	\item Todos los registros del Programa de Seguridad Alimentaria de igual manera son almacenados por un mínimo de \emph{dos años.}
	\item Estos documentos permanecerán almacenados en tres posibles puntos:
	\begin{itemize}
		\item En las oficinas de cada departamento.
		\item En oficinas administrativas
		\item En área destinada en almacén para esta función
	\end{itemize}
	\item Transcurridos los \emph{dos años} y dirección no los ha requerido, se enviaran al archivo muerto o destrucción de los mismos.
	\item Los registros en formato electrónico de recepción, almacenamiento, embarque y distribución de producto, y todos los registros involucrados para rastreabilidad de producto, órdenes de mantenimiento, estarán a la mano para su rápida revisión por NO MENOS DE TRES AÑOS.
	\item Todos los registros del Programa de Seguridad Alimentaria que se encuentren de manera electrónica de igual manera son almacenados por un mínimo de 3 años.
	\begin{itemize}
		\item Esta información es respaldada diariamente de manera automática al término de cada cierre de actividad.
		\item Estos archivos son grabados y almacenados en las oficinas administrativas de RED DE FRIOS S.A. de C.V.
		\item De esta manera se almacenaran por tiempo indefinido.
	\end{itemize}
	\item El Gerente Administrativo es responsable de garantizar que cada departamento maneje el documento como se describe en este procedimiento.
\end{itemize}

\subsection{Responsables de la actividad}

\begin{itemize}
	\item \textbf{Ejecutado} por personal de calidad
	\item \textbf{Verificado} por personal de gerencia.
\end{itemize}

\subsection{Acciones preventivas}

\begin{itemize}
	\item Se llevara a cabo una inspección. Registrando el indicador y medida correspondiente
	\item Si la desviación se repite frecuentemente se dará curso de capacitación al personal de almacén y limpieza para que realicen eficientemente su trabajo.
	\item Si después de haber capacitado al personal de almacén se siguen presentando desviaciones por causas injustificadas, se tomaran acciones más enérgicas con el personal por incumplimiento con sus deberes.
\end{itemize}

\subsection{Acciones correctivas}

\begin{itemize}
	\item En caso de no cumplimiento la tarea se deberá volver a realizar como se indica en el procedimiento.
	\item En caso de no conformidad reportar en formato de acciones correctivas F-OP-40.xls|F-OP-40
\end{itemize}

\subsection{Frecuencia}

Continua

\subsection{Historial de modificaciones}

\begin{itemize}
	\item \textbf{Cuarta edición:} cambio de fecha de 04 de noviembre 2017 a 28 de enero 2019, se realizó cambio de código de CA-P-ARDD a PR-014. Y no se realizaron cambios de la revisión 003 a la 004.
	\item \textbf{Quinta edición:} febrero 2020 se hizo cambio de formato y cambio de código de PR-014 a PRO-OP-010. Y no se realizaron cambios de revisión de 04 a 05.
	\item \textbf{Sexta edición:} febrero 2021 no se realizaron cambios de revisión de 05 a 06.
	\item \textbf{Séptima edición:} febrero 2022 no se realizaron cambios de revisión de 06 a 07.
\end{itemize}
