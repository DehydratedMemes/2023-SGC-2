\thispagestyle{formato-PI}
\renewcommand{\MayorVer}{2}
\renewcommand{\MenorVer}{1}
\renewcommand{\Codigo}{PSA-1-PROG} %TODO
\renewcommand{\FechaPub}{2023--01}
%\renewcommand{\Edit}{2.1}
\renewcommand{\Titulo}{Control documental}

\section{\Titulo}
\index{Información documentada!control!almacenamiento de registros}

% \section{Archivo de registros: Disposición de documentación}

\subsection{Objetivo}
\begin{itemize}
	\item Establecer el \textit{tiempo de resguardo} de registros e información documentada miscelánea;
	\item establecer buenas prácticas de control documental.
\end{itemize}

\subsection{Alcance}
\begin{itemize}
	\item Este documento está destinado para el departamento que se asegurará de almacenar la información documentada.
\end{itemize}

\subsection{Términos y definiciones}
\begin{description}
	\defglo{informacion-documentada};
	\defglo{registro};
	\defglo{documento};
	\item[\glsfirst{SGC}] \glsdesc{SGC};
	\item[\glsfirst{SGA}] \glsdesc{SGA}.
\end{description}

\subsection{Documentos y/o normas relacionadas}
\begin{itemize}
	\item Toda aquella \gls{informacion-documentada} por \gls{RDF};
	\item toda aquella \gls{informacion} electrónica almacenada en el \gls{SGA}.
\end{itemize}

\subsection{Procedimiento}
\subsubsection{Control documental}

\begin{itemize}
	\item El Gerente de Operaciones es responsable del mantenimiento del \gls{PSA};
	\item Todos los cambios introducidos en el manual y los documentos que deben tenerse en cuenta deben identificar, seguido de la fecha de elaboración y la fecha de actualización;
	\item Los registros\footnote{Los registros se guardan principalmente en papel, sin embargo puede haber excepciones o pueden almacenarse de forma electrónica si están en proceso de elaboración.} de recepción, almacenamiento, embarque y distribución de producto, y todos los registros involucrados para rastreabilidad de producto, órdenes de mantenimiento serán resguardados por \VigenciaAlmacRegistros;
	\item Todos los registros del \gls{PSA} de igual manera son almacenados por un mínimo de \VigenciaAlmacRegistros;
	\item Estos documentos permanecerán almacenados en tres posibles puntos:
	\begin{itemize}
		\item En las oficinas de cada departamento;
		\item en oficinas administrativas;
		\item en área destinada en almacén para esta función.
	\end{itemize}
	\item Transcurridos los \VigenciaAlmacRegistros\ y dirección no los ha requerido, se enviarán al archivo muerto o destrucción de los mismos.
	\item Los registros en formato electrónico de recepción, almacenamiento, embarque y distribución de producto, y todos los registros involucrados para rastreabilidad de producto, órdenes de mantenimiento, estarán a la mano para su rápida revisión por \VigenciaAlmacRegistrosElec;
	\item Todos los registros del Programa de Seguridad Alimentaria que se encuentren de manera electrónica de igual manera son almacenados por un mínimo de \VigenciaAlmacRegistrosElec.
	\begin{itemize}
		\item Esta información es respaldada diariamente de manera automática al término de cada cierre de actividad;
		\item Estos archivos son grabados y almacenados en las oficinas administrativas de \gls{RDF};
		\item De esta manera se almacenarán por tiempo indefinido.
	\end{itemize}
	\item El \emph{Gerente Administrativo} es el responsable de garantizar que cada departamento maneje el documento como se describe en este procedimiento.
\end{itemize}

\subsection{Responsables de la actividad}
\begin{itemize}
	\item \textbf{Ejecutado} por personal de calidad;
	\item \textbf{Verificado} por personal de gerencia.
\end{itemize}

\subsection{Acciones preventivas}
\begin{itemize}
	\item Se llevara a cabo una inspección. Registrando el indicador y medida correspondiente;
	\item Si la desviación se repite frecuentemente se dará curso de capacitación al personal de almacén y limpieza para que realicen eficientemente su trabajo;
	\item Si después de haber capacitado al personal de almacén se siguen presentando desviaciones por causas injustificadas, se tomaran acciones más enérgicas con el personal por incumplimiento con sus deberes.
\end{itemize}

\subsection{Acciones correctivas}
\begin{itemize}
	\item En caso de no cumplimiento la tarea se deberá volver a realizar como se indica en el procedimiento;
	\item En caso de no conformidad reportar en formato de acciones correctivas (\IdFormAACC).
\end{itemize}

\begin{changelog}[simple, sectioncmd=\subsection*,label=changelog-2.15]
	\begin{version}[v=2.0, date=2023--01, author=Pablo E. Alanis]
		\item Cambios de formato;
		\item cambio de código;
		\item cambios en redacción.
	\end{version}

	\begin{version}[v=1.7, date=2022--01, author=Agustín M.]
		\item febrero 2022 no se realizaron cambios de la revisión 06 a la 07.
	\end{version}
\end{changelog}