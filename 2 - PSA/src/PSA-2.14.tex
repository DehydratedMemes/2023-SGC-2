\thispagestyle{formato-PI}
\renewcommand{\MayorVer}{2}
\renewcommand{\MenorVer}{0}
 %TODO
\renewcommand{\FechaPub}{2023--01}
\renewcommand{\TipoID}{POL}
\renewcommand{\Titulo}{Política de re-empaque, re-etiquetado, y uso de aditivos}

\section{\Titulo}\index{Política!re-empaque, re-etiquetado y uso de aditivos, de}
\renewcommand{\Codigo}{\Prog--\thesection--\TipoID}
\subsection{Objetivo}
Dar a conocer que no se realizan maniobras de re empaque como actividades comunes dentro de \gls{RDF}.

\subsection{Alcance}
A todas las personas responsables para que se lleven a cabo las operaciones que se requieran para el buen cumplimiento de este procedimiento en las áreas de operaciones.

\subsection{Términos y definiciones}
\begin{description}
	\defglo{politica}
\end{description}

\subsection{Documentos y/o normas relacionadas}
\begin{itemize}
	\item Programa de \glsfirst{BPD};
	\item Porgrama de \glsfirst{BPH}.
\end{itemize}

\subsection{Política de re-empacado}
\begin{itemize}\label{pol:reempaquetado}
	\item \gls{RDF} \textbf{NO} realiza actividades de \emph{re-empaquetado} como práctica común dentro de sus procedimientos durante la operación.
	\item \gls{RDF} \textbf{NO} realiza actividades de \emph{re-etiquetado} como práctica común dentro de sus procedimientos durante la operación.
	\item \gls{RDF} \textbf{NO} hace \emph{uso de aditivos o ingredientes} como práctica común dentro de sus procedimientos durante la operación
\end{itemize}

\begin{changelog}[simple, sectioncmd=\subsection*,label=changelog-\thesection-\MayorVer.\MenorVer4]
	\begin{version}[v=\MayorVer.\MenorVer, date=2023--01, author=Pablo E. Alanis]
		\item Cambios de formato;
		\item cambio de código;
		\item cambios en redacción.
	\end{version}

	\begin{version}[v=1.6, date=2022--01, author=Agustín M.]
		\item febrero 2022 no se realizaron cambios de la revisión 06 a la 07.
	\end{version}
\end{changelog}