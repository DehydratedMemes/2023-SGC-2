\renewcommand{\MayorVer}{2}
\renewcommand{\MenorVer}{1}
\renewcommand{\Codigo}{PSA-1-PROG} %TODO
\renewcommand{\FechaPub}{2023--01}
%\renewcommand{\Edit}{2.1}
\renewcommand{\Titulo}{Lista de contactos de emergencia}

\section{\Titulo}
\index{Lista!contactos de emergencia, de}

% \section{Lista de contactos de emergencia}

\subsection{Objetivo}

\begin{itemize}
	\item Crear el equipo para atacar casos de crisis por corte de energía eléctrica, desastres naturales o siniestros.
	\item Generar la coordinación del plan de contingencia a la etapa de crisis
	\item Garantizar la seguridad (Inocuidad) de alimentos para consumo humano.
\end{itemize}

\subsection{Alcance}

A todas las personas responsables para que se lleven a cabo las operaciones que se requieran para el buen cumplimiento de este procedimiento en el RED DE FRIOS S.A. de C.V.

\subsection{Términos y definiciones}

\begin{itemize}
	\item \textbf{Equipo de emergencia:} El grupo de personas responsables de desarrollar, implementar y verificar el cumplimiento del programa de emergencia.
\end{itemize}

\subsection{Documentos y/o normas relacionadas}

Procedimiento en caso de corte de energía o desastre natural o siniestro.

\subsection{Procedimiento}

\subsubsection{Instrucciones}

\paragraph{Equipo de Emergencias:}

\subparagraph{Coordinador 1}

Gerardo Ruiz, Gerente de Operaciones.
Teléfono: 01 (81) 83 51 05 12, 8180295609
Responsable de coordinar los esfuerzos según la emergencia y restablecer la operación, garantizando al mismo tiempo las políticas más estrictas de seguridad y protocolos.

\subparagraph{Coordinador 2}

Luis Escareño, Jefe de Almacén
Teléfono: +52 (81) 8351-0512,+52 (81) 1390-7099
Responsable de asegurar el entorno de almacén y asegurarse de que el personal de almacén salga del edificio inmediatamente y se congreguen en el exterior (punto de reunión) en caso de emergencia.

\subparagraph{Coordinador 3}

Salvador Verdín, Jefe de Mantenimiento
Teléfono: +52 (81) 8351-0512,+52 (81) 1706-1644.
Responsable del personal de oficinas salga del edificio inmediatamente y se congreguen en el exterior (punto de reunión) en caso de emergencia

\subparagraph{Coordinador 4}

Mario González, Gerente de Mantenimiento
Teléfono: +52 (81) 1292-7272, +52 (81) 1607-6035
Responsable de asegurar las condiciones internas de almacén y equipos, asegurar que el personal de almacén salga del edificio inmediatamente y se congreguen en el exterior (punto de reunión) en caso de emergencia.

\subparagraph{Vigilancia}

Lorenzo Hernández, Vigilancia
Teléfono: +52 (81) 2353-2346

\subparagraph{Números de Contacto de Emergencia}

Además del área de vigilancia, cada instalación debe de tener los números de emergencia de las autoridades locales, estatales y federales.


Responsable de verificar de primera instancia alguna anormalidad vista desde el punto externo del almacén y dar aviso al personal correspondiente según el hallazgo detectado.

\subsubsection{Historial de modificaciones}

\begin{itemize}
	\item \textbf{Quinta edición:} cambio de fecha de 04 de noviembre 2017 a 28 de enero 2019, se realizó cambio de código de DG-P-LCCE a PR-012, se dieron de alta contactos del personal de vigilancia y no se realizaron cambios en la revisión 004 a la 005.
	\item \textbf{Sexta edición:} febrero 2020 se hizo cambio de formato y cambio de código de PR-012 a L-OP-001. Y no se realizaron cambio en la revisión 05 a la 06.
	\item \textbf{Séptima edición:} febrero 2021 no se realizaron cambio en la revisión 06 a la 07.
	\item \textbf{Octava edición:} febrero 2022 se realizó un acomodo a la información y se agregaron los No. De contrato con CFE de la revisión 07 a la 08.
\end{itemize}

\subsection{Anexos}
