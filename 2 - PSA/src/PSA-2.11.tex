\thispagestyle{formato-PI}
\renewcommand{\MayorVer}{2}
\renewcommand{\MenorVer}{1}
\renewcommand{\Codigo}{PSA-11-PRO} %TODO
\renewcommand{\FechaPub}{2023--01}
%\renewcommand{\Edit}{2.1}
\renewcommand{\Titulo}{Acciones en caso de corte de energía o desastre}

\section{\Titulo}
\label{sec:AccEnCasoDeEmergencia}\index{Acciones inmediatas!en caso de corte de energía o desastre}

% \section{Acciones en caso de corte de energía o presencia de desastre o siniestro}

\subsection{Objetivo}

Definir las acciones a seguir en caso de haber un corte de energía eléctrica no controlado o presentarse un desastre natural o siniestro en las áreas de operación que requieren temperaturas controladas, para garantizar la protección del producto.

\subsection{Alcance}
Todas las áreas afectadas.

\subsection{Términos y definiciones}
\begin{description}
	\defglo{desastre-natural}
	\defglo{emergencia}
\end{description}

\subsection{Documentos y/o normas relacionadas}
\begin{itemize}
	\item Manual de mantenimiento
	\item Programa Interno de Protección Civil
\end{itemize}

\subsection{Procedimiento}
\subsubsection{Instrucciones}
\paragraph{Procedimiento de emergencia ante crisis y desastres naturales}\index{Acciones inmediatas!en caso de incendio en la periferia}
\begin{enumerate}
	\item La seguridad personal es lo principal en caso de crisis y emergencia. Por favor refiérase al equipo de Emergencia / Crisis.
	\item En el caso de una emergencia / crisis que suceda en una instalación de Red de Fríos, se deberá informar inmediatamente el Gerente General
	\item Si la emergencia es peligrosa para la vida, inmediatamente llame \textbf{al 065 para la Cruz Roja o al 066 para emergencias en Nuevo León y reporte el incidente o bien al 911.}
	\item Si la crisis se relaciona con un \emph{incendio en o alrededor de la instalación,} todos los empleados debe de salir inmediatamente de la instalación reunirse en el punto de reunión.
	\item Es importante mantener la calma, moverse con cautela y ayudar a cualquier persoba que requiera asistencia.
	\item \emph{Si el lugar acordado para reunirse se ha vuelto peligroso,} todos los empleados tienen que reunirse en la puerta principal (estacionamiento) y esperar instrucciones.
	\item Una vez que la seguridad de los empleados se ha resguardado y todos los empleados han sido contados, contacte inmediatamente con el organismo adecuado para reportar el incidente y de ser requerido, solicitar ayuda médica.\footnote{El Supervisor en turno debe liderar este esfuerzo.}
	\item Si la crisis se refiere a la seguridad personal, contacte inmediatamente con el Departamento de Emergencias, en el número arriba mencionado, si la emergencia es \textbf{potencialmente mortal, inmediatamente llame al 066 o 911} y reporte el incidente.
	\item Si la crisis se refiere a un terremoto, inundación o huracán todos los empleados deben salir inmediatamente de la instalación y juntar en el lugar acordado para reunirse siguiendo los pasos mencionados anteriormente
\end{enumerate}



\paragraph{Corte de energía eléctrica}\index{Acciones inmediatas!en caso de corte eléctrico}
\begin{itemize}
	\item En cuanto a cortes de energía o falla en los equipos que afecten directamente el funcionamiento y por lo tanto la temperatura de las cámaras, por favor, tenga en cuenta que los teléfonos de oficina pueden no funcionar, por lo que los teléfonos celulares deben ser utilizados y siga el siguiente procedimiento:
	\item Llama a la \gls{CFE} número de emergencia de falla (071) e inmediatamente reportar el apagón. \emph{(Indicar número de contrato).}
	\item Comunicarse con el Gerente de Operaciones, Supervisor de Almacén, Gerente y Supervisor de Mantenimiento informado de las actualizaciones de estado.
\end{itemize}

\begin{note}[Números de contrato] \label{note:NumerosDeContratoCFE}
	\index{Números de contrato (\gls{CFE})}
	Los números de contrato son los siguientes:
	\begin{enumerate}
		\item 999001000109
		\item 407210800266
	\end{enumerate}
\end{note}

\begin{itemize}
	\item Si \gls{CFE} informa de que no habrá electricidad por más de \qty{2}{\hour} \emph{(No representa un problema con el producto refrigerado o congelado),} se suspenderá el servicio a clientes y proveedores, cerrando todas las puertas de los almacenes que se encuentren en refrigeración o congelación evitando la perdida de temperatura.
\end{itemize}

\begin{note}[Tiempo máximo de permanencia de productos en cámaras apagadas] \label{note:TMaxCamAveriada}
	\index{Tiempo de permanencia máxima de alimentos en cámaras apagadas}
	\begin{itemize}
		\item \Gls{alimento} refrigerado: \qty{<=\TiempoAveriaRefri}{\hour};
		\item \Gls{alimento} congelado: \qty{<=\TiempoAveriaConge}{\hour}.
	\end{itemize}
\end{note}

\emph{Al re establecerse el servicio de energía, el personal de mantenimiento debe de realizar la revisión de los equipos para verificar que no se tuvo algún contratiempo con el evento y confirmar que todo se encuentra funcionando correctamente.}

\paragraph{Falla en los equipos de enfriamiento}\index{Acciones inmediatas!en caso de avería de equipos de enfriamiento}
En cuanto a falla en los equipos que afecten directamente el funcionamiento y por lo tanto la temperatura de las cámaras, por favor, tenga en cuenta que los teléfonos de oficina pueden no funcionar, por lo que los teléfonos celulares deben ser utilizados y siga el siguiente procedimiento:

\begin{itemize}
	\item Analice junto con el personal de mantenimiento la falla y determine el tiempo de respuesta para el restablecimiento de los equipos y por lo tanto de la temperatura.
	\item Si llegara a fallar uno de los compresores que dan energía a la planta, se cuenta con dos compresores extra los cuales están listos para funcionar si llegara a fallar algún otro en funcionamiento.
	\item Si mantenimiento informa de que puede resolver la falla en un periodo de no más de \qty{2}{\hour}, se suspenderá el servicio a clientes y proveedores, cerrando todas las puertas de los almacenes que se encuentren en refrigeración o congelación para evitar la perdida de temperatura.
\end{itemize}

Si la emergencia no es solucionada el Gerente General va a delegar responsabilidades (según corresponda) al personal considerando las siguientes acciones:

\paragraph{Seguridad y acceso a la instalación}
\begin{enumerate}
	\item Asesorar a los clientes de cualquier interrupción de servicio;
	\item Revisión de condiciones del producto;
	\item Revisión de condición del equipo de refrigeración;
	\item La principal preocupación en caso de emergencia o crisis radical en la pérdida de control de la temperatura, ya que el producto podría correr un riesgo microbiológico.
	\item Al re establecerse el servicio de energía, el personal de mantenimiento debe de realizar la revisión de los equipos para verificar que no se tuvo algún contratiempo con el evento y confirmar que todo se encuentra funcionando correctamente.
	\item Primeramente se debe verificar si no hay energía en toda la planta, esto debido a que el almacén es alimentado por 2 fuentes de luz diferentes, lo que se refiere, a que la mitad del almacén puede tener energía cuando la otra mitad no, siendo este el caso, se debe verificar la manera de mover el producto de las cámaras que no tienen energía, a las cámaras que siguen en funcionamiento.
	      \begin{itemize}
		      \item Si el producto requiere reubicación, la instalación (primaria) debe ser contactada. Si la primaria se ha visto comprometida, la instalación secundaria debe ser contactada.
	      \end{itemize}
\end{enumerate}

\begin{itemize}
	\item \textbf{Instalación primaria: Red de Fríos Suc. Ruiz Cortines}
	      Ruiz Cortines No. 2208, Col. Moderna. Monterrey, N.L. C.P. 64530, NL, México.
	      Tel. +52 (81) 1292--7272.
	\item \textbf{Instalación Secundaria: Mega Frio S.A de C.V.}
	      20 de Noviembre No. 2618 Col. Garza Nieto Monterrey, N.L. C.P. 64420, NL, México
	      Tel. +52 (81) 8040--7162 y +52 (81) 8040--7163

\end{itemize}

\begin{contact}[Números de emergencia] \label{contact:NumerosDeEmergencia}
	\index{Números de emergencia}
	Los números de contrato son los siguientes:\\
	\begin{tabular}{lc}
		\textbf{Cruz Roja:}               & 065 \\
		\textbf{Emergencias:}             & 066 \\
		\textbf{Cualquier incidente:}     & 911 \\
		\textbf{CFE (Reporte de fallas):} & 071
	\end{tabular}
\end{contact}

Además del área de vigilancia, cada instalación debe tener los números de emergencia de las autoridades locales, estatales y federales.

\begin{tabular}{cc}
	\noindent \textbf{Contacto primario:} & \cref{contact:Gerente}       \\
	\textbf{Contacto secundario:}         & \cref{contact:JefeDeAlmacen}
\end{tabular}

\begin{contact}[Números de transportes en caso de siniestros] \label{contact:TransportesEnCasoDeAveria}
	\index{Números de transportes en caso de siniestros}
	\begin{tabular}{lc}
		\textbf{Transportes Red de Fríos:} & +52 (81) 1636--8137 \\
		\textbf{Transportes Java:}         & +52 (81) 2681--0370 \\
		\textbf{Thermo Transportes:}       & +52 (81) 1387--0145
	\end{tabular}
\end{contact}