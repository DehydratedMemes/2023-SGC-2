\thispagestyle{formato-PI}
\renewcommand{\MayorVer}{2}
\renewcommand{\MenorVer}{1}
\renewcommand{\Codigo}{PSA-1-PROG} %TODO
\renewcommand{\FechaPub}{2023--01}
%\renewcommand{\Edit}{2.1}
\renewcommand{\Titulo}{Instructivo para selección de tarimas de madera}

\section{\Titulo}
\index{Información documentada!tipo!especificación!Instructivo para selección de tarimas de madera}

\subsection{Objetivo}

Seleccionar tarimas en buenas condiciones para evitar riesgo de contaminación de producto o un posible colapso que genere daño al mismo y al mismo tiempo minimizar la posibilidad de algún accidente dentro del almacén.

\subsection{Alcance}

Áreas operativas del almacén.

\subsection{Términos y definiciones}

N/A

\subsection{Documentos y/o normas relacionadas}

N/A

\subsection{Actividades}

\subsubsection{Descartar tarimas que presenten:}

\begin{itemize}
	\item Tarimas con restos de material orgánico que genere una posible contaminación del producto.
	\item Tarimas con presencia de plaga.
	\item Tarimas dañadas, con desprendimiento de tablas, falta de tablas o tablas partidas.
	\item Tarimas con clavos expuestos.
\end{itemize}

Las tarimas que presenten alguno de estos daños deben ser separadas de las tarimas de uso y ser desechadas del almacén para evitar ser reutilizadas.

\subsection{Listado de distribución}

