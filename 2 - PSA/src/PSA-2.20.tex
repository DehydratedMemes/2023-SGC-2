\thispagestyle{formato-PI}
\renewcommand{\MayorVer}{2}
\renewcommand{\MenorVer}{0}
 %TODO
\renewcommand{\FechaPub}{2023--01}
\renewcommand{\TipoID}{IT}
\renewcommand{\Titulo}{Instructivo para selección de tarimas de madera}

\section{\Titulo}\index{Información documentada!tipo!especificación!Instructivo para selección de tarimas de madera}
\renewcommand{\Codigo}{\Prog--\thesection--\TipoID}

\subsection{Objetivo}

Seleccionar tarimas en buenas condiciones para evitar riesgo de contaminación de producto o un posible colapso que genere daño al mismo y al mismo tiempo minimizar la posibilidad de algún accidente dentro del almacén.

\subsection{Alcance}

Áreas operativas del almacén.

\subsection{Términos y definiciones}

\subsection{Actividades}
\subsubsection{Descartar tarimas}
Las tarimas que presenten alguno de estos daños deben ser separadas de las tarimas de uso y ser desechadas del almacén para evitar ser reutilizadas.

\begin{itemize}
	\item Tarimas con restos de material orgánico que genere una posible contaminación del producto.
	\item Tarimas con presencia de plaga.
	\item Tarimas dañadas, con desprendimiento de tablas, falta de tablas o tablas partidas.
	\item Tarimas con clavos expuestos.
\end{itemize}

\begin{changelog}[simple, sectioncmd=\subsection*,label=changelog-\thesection-\MayorVer.\MenorVer9]
	\begin{version}[v=\MayorVer.\MenorVer, date=2023--01, author=Pablo E. Alanis]
		% \fixed
		\item Cambio de formato;
		\item Cambios en la serialización de versiones;
	\end{version}

	\begin{version}[v=1.7, date=2022--05, author=Alonso M.]
		\item cambio de fecha;
	\end{version}

	\shortversion{v=1.6, date=2021--05, changes=No hubo cambios}
\end{changelog}
