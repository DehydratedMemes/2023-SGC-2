\thispagestyle{formato-PI}
\renewcommand{\MayorVer}{2}
\renewcommand{\MenorVer}{0}
\renewcommand{\Codigo}{PSA-1-PROG} %TODO
\renewcommand{\FechaPub}{2023--01}
%\renewcommand{\Edit}{2.1}

\renewcommand{\Titulo}{Atención de quejas}
\section{\Titulo}\index{Información documentada!tipo!procedimiento!Atención de quejas}

\subsection{Objetivos}

\begin{itemize}
	\item Establecer la metodología para atender y tomar acciones correctivas de las quejas, reclamaciones y devoluciones de los clientes y consumidores.
	\item Definir los lineamientos para asegurar la correcta identificación, documentación, evaluación y notificación cuando se generen daños, faltantes o sobrantes en los productos embarcados de exportación.
\end{itemize}

\subsection{Alcance}

Este procedimiento aplica a las quejas, reclamaciones y devoluciones de clientes recibidas de los productos almacenados en Red de Fríos S.A. de C.V.

\subsection{Términos y definiciones}

\begin{itemize}
	\item \textbf{Cliente:} Toda persona que realiza actos de comercio con la Empresa llámese proveedor, intermediario o consumidor final de los productos.
	\item \textbf{Empresa:} Se refiere a \gls{RDF}
	\item \textbf{Queja:} Malestar externado por el cliente al no poder darle al producto el uso original, es decir recibir un producto fuera de especificación, atención incorrecta, cantidades incorrectas.
	\item \textbf{Devolución:} Producto que regresa el cliente, el cual ha sido evaluado y aceptado como producto no conforme.
	\item \textbf{Queja de cliente:} Son conocidos como reportes de falla de calidad que emiten el área de Aseguramiento de Calidad de nuestra empresa través del área Servicio al cliente.
	\item \textbf{Fallas de calidad por manufactura} Son defectos detectados en el producto y se derivan de alguna desviación durante el proceso de elaboración.
	\item \textbf{Fallas de calidad por mal manejo} Son defectos detectados en el producto y se derivan un mal manejo del producto durante su almacenamiento, distribución y venta del mismo.
	\item \textbf{Fallas de riesgo a la inocuidad del producto} Son riesgos detectados con efecto nocivo para la salud y de la gravedad de dicho efecto, como consecuencia de un peligro o peligros presentes en los alimentos.
	\item \textbf{Casos con riesgo de crisis} Son reportes recibidos por Gerencia de Operaciones en conjunto con Aseguramiento de Calidad a través del área de calidad del CLIENTE y que requieren una pronta respuesta.
	\item \textbf{Retiro de producto} Es la solicitud que hace el cliente de retirar un producto de su distribución o venta y regresarlo a la al almacén.
\end{itemize}

\subsection{Documentos y/o normas relacionadas}

\begin{itemize}
	\item Procedimiento de Acciones Correctivas y/o Preventivas.
	\item Procedimiento Trazabilidad
	\item Procedimiento de Food Defense
\end{itemize}

\subsection{Procedimiento}

\subsubsection{Recepción de Quejas de Calidad y/o Inocuidad o Servicio}

\begin{enumerate}
	\item Todas las quejas de producto proveniente del cliente son de la más alta prioridad y deben ser tratados de inmediato para garantizar que el cliente recibe una solución satisfactoria.
	\item Diferenciar si el punto señalado se cataloga como queja o como punto de vista.
	\item Las Quejas reportadas son aceptadas cuando estas tengan la evidencia necesaria de la falla de calidad y/o inocuidad o de servicio prestado.
	\item El Gerente de Operaciones asigna como responsable del seguimiento de las Acciones Correctivas o desviaciones al responsable de Aseguramiento de Calidad quien debe dar respuesta al cliente definiendo las acciones correctivas y/o el plan de acción en un tiempo máximo de 3 días hábiles.
	\item Aseguramiento de calidad según lo indicado en el Procedimiento de Acciones Correctivas/Preventivas; realiza la validación de las acciones correctivas y/o preventivas posterior a la fecha declarada como conclusión, esta se considera como cerrada hasta que se tenga cerrado el ciclo de cada uno de los diferentes puntos que se refieren en el Formato Solicitud de Acción Correctiva / Acción Preventiva.
\end{enumerate}

\subsubsection{Reportes de Casos con Riesgo de Crisis}

\begin{enumerate}
	\item El Gerente de Operaciones recibe los reportes de casos con riesgo de crisis a través del área de Servicios a Clientes del CLIENTE, e inmediatamente este turna al caso al área que corresponda mediante el planteamiento del Manual de Defensa del producto.
	\item El Gerente de Planta y/o el Gerente de Aseguramiento de Calidad, definen si es necesario trazabilidad de producto, de ser así se procede de acuerdo a lo indicado en el procedimiento de trazabilidad de Producto.
	\item Se lleva a cabo una investigación y se toman las acciones correctivas y preventivas necesarias a la brevedad posible. Solicitud de Acción Correctiva /Acción Preventiva.
\end{enumerate}

\subsubsection{Daños, faltantes o sobrantes en los productos embarcados}

\begin{enumerate}
	\item El Departamento de Logística recibe el reporte de reclamación de parte de los clientes vía correo electrónico, formato encuesta de servicio al cliente o llamada telefónica y se encarga de recabar la información necesaria para que el Departamento de Aseguramiento de Calidad identifique el historial del embarque en cuestión
	\item Departamento de Logística debe sondear que es lo que desea el cliente: nota de crédito, reposición del material, etc.
	\item El Departamento de Calidad, debe analizar si la reclamación es por un problema de Calidad y si procede, debe darle seguimiento por medio del procedimiento de Producto No Conforme.
	\item Si el análisis efectuado por Calidad, determina que es un problema de Logística (Faltante o sobrante de Mercancía), el responsable del departamento de logística y almacén deben darle seguimiento hasta su solución, en coordinación con la Gerencia de Operaciones y Aseguramiento de Calidad.
	\item Si la reclamación procede, se debe recabar la siguiente información:
	\begin{itemize}
		\item Numero de Orden de Venta
		\item Numero de Remisión (para establecer la rastreabilidad del producto)
		\item Información adicional que pueda ayudar a la solución del problema, como fotos.
		\item Documentos de recepción
		\item Folio de entrada
		\item Folio de salida
		\item Reporte de trazabilidad
	\end{itemize}
	\item Una vez que se haya esclarecido el problema Aseguramiento de Calidad convocará a una junta con los responsables de los departamentos involucrados.
	\item Se establece el origen del problema con las acciones correctivas correspondientes. En la junta se toma la decisión sobre la resolución que se le dará al cliente y el Departamento de Logística dará aviso al cliente en un plazo menor a 5 días hábiles.
	\item El seguimiento de las acciones tomadas sobre la reclamación y el archivo de los registros generados lo realiza el departamento de Aseguramiento de Calidad.
\end{enumerate}

\subsection{Frecuencia}

Al presentar una queja.

\subsection{Historial de modificaciones}

\begin{itemize}
	\item \textbf{Cuarta edición:} cambio de fecha de 04 de noviembre 2017 a 28 de enero 2019, se realizó cambio de código de QP-RMQ a PR-042. Y no se realizaron cambios de la revisión 003 a la 004
	\item \textbf{Quinta edición:} febrero 2020 se hizo cambio de formato y cambio de código de PR-042 a PRO-OP-016. Y no se realizaron cambios de la revisión 04 a la 05.
	\item \textbf{Sexta edición:} febrero 2021 no se realizaron cambios de la revisión 05 a la 06.
	\item \textbf{Séptima edición:} febrero 2022 no se realizaron cambios de la revisión 06 a la 07.
\end{itemize}

\subsection{Listado de distribución}


\subsection{Anexos}

F-OP-41-Solicitud de acción correctiva - Quejas.xls
