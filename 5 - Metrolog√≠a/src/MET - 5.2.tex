\thispagestyle{formato-PI}
\renewcommand{\MayorVer}{2}
\renewcommand{\MenorVer}{0}
\renewcommand{\Titulo}{Mantenimiento de equipos de medición}
\renewcommand{\TipoID}{PRO}
\renewcommand{\FechaPub}{2023--01}

\section{\Titulo}\index{Programa!metrología, de}\index{Procedimiento!mantenimiento de termómetros}
\renewcommand{\Codigo}{\Prog--\thesection--\TipoID}

\subsection{Objetivo}

\begin{itemize}
	\item Garantizar que los valores de temperatura señalados por los equipos de medición indiquen valores reales y confiables para garantizar la calidad e inocuidad de los productos alimenticios recibidos, almacenados y distribuidos.
	\item Corrección de aquellos equipos que no cumplen los requerimientos establecidos.
\end{itemize}

\subsection{Alcance}
Aplica a los equipos de medición de las áreas operativas, termómetros de mano.

\subsection{Terminología y definiciones}
\begin{description}
    \defglo{metrologia}
    \defglo{proveedor}
\end{description}

\subsection{Procedimiento}
\subsubsection{Verificación externa}
\begin{itemize}
	\item \Gls{RDF} implementa un programa de calibración de sus equipos de medición por parte de una empresa externa calificada, la frecuencia se encuentra programada de manera anual.
	\item El \gls{proveedor} externo deberá presentar las certificaciones que lo acreditan como un proveedor certificado y contar con el suficiente entrenamiento para poder realizar los trabajos de calibración de los equipos.
	\item El servicio se llevará a cabo en las instalaciones del proveedor calificado.
	\item El prestador de servicios deberá entregar un informe de calibración (o certificado) por equipo y este a su vez será archivado como evidencia en la carpeta de registros.
	\item Si es necesaria una reparación, verificación o calibración intermedia a este programa, se llamará al proveedor de servicio para realizar el mantenimiento correspondiente.
    \item[NOTA:] Para revisar cualquier duda técnica, consulte el manual del usuario.
\item[NOTA:] Los termómetros colocados en cámaras solo se utilizan como referencia
\end{itemize}

\subsection{Responsables de la actividad}
\begin{itemize}
	\item \textbf{Ejecutado} por personal de mantenimiento;
	\item \textbf{Monitoreado} por personal de calidad;
	\item \textbf{Verificado} por personal de calidad.
\end{itemize}

\subsection{Frecuencia}
Anual

\begin{changelog}[simple, sectioncmd=\subsection*,label=changelog-\thesection-\MayorVer.\MenorVer]
	\begin{version}[v=\MayorVer.\MenorVer, date=2023--01, author=Pablo E. Alanis]
		\item Cambio de formato;
		\item Cambios en la serialización de versiones;
		\item Correcciones ortográficas y de estilo.
	\end{version}

	\begin{version}[v=1.7, date=2022--05, author=Alonso M.]
		\item cambio de fecha;
	\end{version}

	\shortversion{v=1.6, date=2021--05, changes=No hubo cambios}
\end{changelog}