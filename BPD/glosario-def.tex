% DEFINICIONES
\newglossaryentry{validacion-documental}
{
    name=validación documental,
    description={proceso en el que se verifica que la información documentada contemple los requerimientos actuales del SGC. De ser así, no se requerirá de una revisión}
}

\newglossaryentry{informacion-documental}
{
    name=revisión documental,
    plural=revisiónes documentales,
    description={proceso de actualización de la información documentada contenida en algun programa al determinarse que se necesita algún requerimiento adicional o hubo algún cambio significativo.}
}

\newglossaryentry{revision-documental}
{
    name=revisión documental,
    plural=revisiónes documentales,
    description={proceso de actualización de la información documentada contenida en algun programa al determinarse que se necesita algún requerimiento adicional o hubo algún cambio significativo.}
}

\newglossaryentry{personal-interno}
{
    name=personal interno,
    description={persona contratada por RDF}
}

\newglossaryentry{visitante}
{
    name=visitante,
    plural=visitantes,
    description={persona que accederá a las instalaciones para brindar un servicio, o bien, personal contratado por un \emph{cliente} que laborará constantemente dentro de las instalaciones de RDF y tiene que someterse al reglamento interno}
}

\newglossaryentry{uniforme-completo}
{
    name=uniforme completo,
    description={se considera como \textit{uniforme completo} al uso de zapatos cerrados, cubrebocas, cofia y a un conjunto de chamarra y pantalones para protección contra el freio o en su defecto un overlo. Los guantes son opcionales pero recomendados}
}

\newglossaryentry{area-operativa}
{
    name=área operativa,
    description={zona destinada a que ocurran las actividades operativas; en nuestro caso, el almacenamiento de producto y maniobras asociadas.}
}

\newglossaryentry{aduana}
{
    name=aduana sanitaria,
    description={zona destinada a que las personas que entrarán a áreas operativas puedan lavarse y desinfectarse las manos, así como ponerse cubrebocas y cofia con el proposito de evitar riesgos de inocuidad al entrar a dicha área.}
}

\newglossaryentry{cliente}
{
    name=cliente,
    plural=clientes,
    description={persona ajena a la operación cotidiana en RDF, que tiene un acuerdo con RDF para almacenar productos en sus instalaciones.}
}

\newglossaryentry{alimento}{
    name=alimento,
    plural=alimentos,
    description={sustancia (ingrediente), ya sea procesada, semi-procesada o cruda, que se destina para consumo, e incluye bebidas, goma de mascar y cualquier sustancia que se haya utilizado en la fabricación, preparación o tratamiento de “alimentos”, pero no incluye cosméticos ni tabaco o sustancias (ingredientes) usados solamente como fármacos}
}

\newglossaryentry{organizacion}{
    name=organización,
    description={persona o grupo de personas que tienen sus propias funciones con responsabilidades, autoridades y relaciones para lograr sus objetivos.\footnote{El concepto de organización incluye, pero no se limita a, un operador individual, compañía, corporación, firma, empresa, autoridad, sociedad, institución benéfica o con otros fines, o parte o combinación de los mismos, ya sea incorporada o no, pública o privada.}}
}

\newglossaryentry{producto}{
    name=producto,
    description={salida que es el resultado de un \glsname{proceso}\footnote{Un producto puede ser un servicio.}}
}

\newglossaryentry{proceso}{
    name=proceso,
    description={conjunto de actividades interrelacionadas o que interactúan que transforman las entradas en salidas}
}

\newglossaryentry{producto-terminado}{
    name=producto terminado,
    description={\glsname{producto} que no se someterá a procesamiento o transformación posterior por parte de la \glsname{organizacion}\footnote{Un producto que es sometido a un procesamiento o transformación posterior por otra organización, es un producto terminado, en el contexto de la primera organización, y una materia prima o un ingrediente, en el contexto de la segunda organización}}
}

\newglossaryentry{inocuidad-de-los-alimentos}{
    name=peligro relacionado con la inocuidad de los alimentos,
    description={agente biológico, químico o físico en el \glsname{alimento} con potencial de causar un efecto adverso en la salud\footnote{El término “peligro” no se debe confundir con el término “riesgo” el cual, en el contexto de la inocuidad de los alimentos, significa una función de la probabilidad de un efecto adverso en la salud (por ejemplo, enfermar) y la gravedad de ese efecto (por ejemplo, muerte, hospitalización) cuando se expone a un peligro especificado.}\footnote{Peligros para la inocuidad de los alimentos incluye alérgenos y sustancias radiológicas.}}
}

\newglossaryentry{nivel-aceptable}{
    name=nivel aceptable,
    description={nivel de un peligro relacionado con la \glsname{inocuidad-de-los-alimentos} que no se debe exceder en el \glsname{producto-terminado} proporcionado por la \glsname{organizacion}}
}

\newglossaryentry{medida-de-control}{
    name=medida de control,
    description={acción o actividad que es esencial para prevenir un peligro relacionado con la inocuidad de los \glsname{alimento}  significativo o reducirlo a un \glsname{nivel-aceptable}}
}

\newglossaryentry{seguimiento}{
    name=seguimiento,
    description={determinación del estado de un sistema, un \glsname{proceso} o una actividad\footnote{Para determinar el estado, puede ser necesario verificar, supervisar u observar críticamente.}\footnote{En el contexto de la inocuidad de los alimentos, el seguimiento se lleva a cabo con una secuencia planificada de observaciones o mediciones para evaluar si un proceso está funcionando según lo previsto.}}
}

\newglossaryentry{criterio-de-acción}{
    name={criterio de acción},
    plural={criterios de acción},
    description={especificación medible u observable para el seguimiento de un \glsname{PPRO}\footnote{Un criterio de acción se establece para determinar si un PPRO permanece bajo control, y distingue entre lo que es aceptable (que el criterio cumpla o logre significa que el PPRO está operando como está previsto) e inaceptable (que el criterio no se cumple ni se logre significa que el PPRO no está operando como está previsto).}}
}

% ACRÓNIMOS

\newacronym{RDF}{RDF}{Red de Fríos S.A. de C.V.}
\newglossaryentry{PPRO}{
    name=PPRO,
    first=programa de prerrequisito operativo (PPRO),
    description={\glsname{medida-de-control} o combinación de medidas de control aplicadas para prevenir o reducir un peligro significativo relacionado con la inocuidad de los \glsname{alimento} a un \glsname{nivel-aceptable}, y donde el \glsname{criterio-de-acción} y medición u observación permite el control efectivo del \glsname{proceso} y/o \glsname{producto}.}
}

\newglossaryentry{PPR}{
    name={programa de prerrequisito},
    first=programa de prerrequisito (PPR),
    description={condiciones y actividades básicas que son necesarias dentro de la \glsname{organizacion} y a lo largo de la cadena alimentaria para mantener la inocuidad de los alimentos\footnote{Los PPR necesarios dependen del segmento de la cadena alimentaria en el que opera la organización y del tipo de organización . Son ejemplos de términos equivalentes: Buenas Prácticas Agrícolas (BPA), Buenas Prácticas Veterinarias (BPV), Buenas Prácticas de Fabricación/Manufactura (BPF, BPM), Buenas Prácticas de Higiene (BPH), Buenas Prácticas de Producción (BPP), Buenas Prácticas de Distribución (BPD), y Buenas Prácticas de Comercialización (BPC).}}
    }

\newglossaryentry{HACCP}{
    name={HACCP},
    first={Analisis de Peligros y Puntos Criticos de Control (APPC), del inglés \textit{Hazard Analysis and Critical Control Points (HACCP).}},
    description={},
}