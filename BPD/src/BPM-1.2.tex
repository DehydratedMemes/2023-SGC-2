\renewcommand{\Codigo}{BPD-PROG}
\renewcommand{\FechaPub}{2023-01}
\renewcommand{\Edit}{03}
\renewcommand{\Titulo}{Programa de buenas prácticas de distribución (BPD)}

\section{\Titulo}

\subsection{Objetivo}

\begin{itemize}
	\item \textbf{Garantizar} la calidad de los productos alimenticios recibidos, almacenados y distribuidos por medio del cumplimiento del programa de calidad establecido por RED DE FRIOS S.A de C.V.
	\item \textbf{Garantizar} la eficiencia de las operaciones durante la recepción, mantenimiento y distribución de sus productos.
	\item \textbf{Garantizar} que las condiciones de operación del almacén y del personal son adecuadas para asegurar que los productos son seguros para su consumo.
	\item \textbf{Establecer} guías para la administración e implementación de reglas que se refieran a la apariencia, higiene, sanidad y prácticas de manejo de los alimentos por parte de las personas que laboran en RED DE FRIOS.
\end{itemize}

\subsection{Alcance}

\begin{itemize}
	\item A todas las personas responsables para que se lleven a cabo las operaciones que se requieran para el buen cumplimiento de este procedimiento en las áreas de operaciones.
	\item Todo personal que labora en RED DE FRIOS.
\end{itemize}

\subsection{Responsables}

\begin{itemize}
	\item Es responsabilidad del Gerente de Operaciones el control y distribución de este documento.
	\item Es responsabilidad del Gerente de Operaciones la aprobación de este documento.
	\item Es responsabilidad de todos los que laboran en esta área operativa el cumplir con los requerimientos que se señalan en este.
\end{itemize}

\subsection{Metodología}

\subsubsection{Introducción}

El programa de calidad y buenas prácticas de distribución de RED DE FRIOS, S.A de C.V. está conformado por los siguientes manuales de procedimientos y programas.

\subsubsection{Programas}

El sistema de gestión de calidad (SGC) de RDF comprende los siguientes programas de pre-requisitos y de pre-requisitos operacionales:

\begin{itemize}
	\item Programa de calidad y buenas prácticas de distribución (BPD);
	\item Programa de seguridad alimentaria;
	\item Programa de higiene y sanidad;
	\item Programa de metrología;
	\item Control de control de alérgenos;
	\item Control de vidrio y plástico duro;
	\item Control de plagas:
	\begin{itemize}
		\item Verificación del servicio de control de plagas;
		\item Control de plagas transporte;
		\item Monitoreo captura de insectos.
	\end{itemize}
	\item Programa de trazabilidad;
	\item \emph{Food Defense;}
	\item Programa de auditorías internas;
	\item Mantenimiento;
	\item Programa de metrología;
	\begin{itemize}
		\item Programa de verificación interna de termómetros.
	\end{itemize}
	\item Programa de gestión de quejas y no-conformidades;
	\item Programa de control documental;
	\item Programa de capacitaciones;
	\item Programa de HACCP;
	\item Programa de gestión de recursos;
	\item Análisis de Agua.
\end{itemize}

\subsection{Historial de modificaciones}

\begin{itemize}
	\item Cuarta edición: Cambio de fecha 24 de octubre 2017 a 24 de octubre 2018, se realizó cambio de código de M-PCBPD a PL-001 y no se registraron cambios en el documentos de la versión 03 a la 04
	\item Quinta edición: Cambio de fecha 24 de octubre 2018 a 24 de octubre 2019 y no se registraron cambios en el documentos de la versión 04 a la 05
	\item Sexta edición: Cambio de formato y cambio de código de PL-001 a PRO-OP-001 el 06 de enero 2020. Y no se realizaron cambio de la versión 05 a la 06.
	\item Séptima edición: febrero 2021 no se realizaron cambios de la versión 06 a la 07.
	\item Octava edición: febrero 2022 no se realizaron cambios de la versión 07 a la 08.
	\item Novena edición: abril 2022 se realizó el cambio de lista de programas.
	\item Décima edición: 2023-01: cambios en el formato (PRO-OP-001 a OP-BPD-ESP-1) y en la serialización.
\end{itemize}
