\renewcommand{\Codigo}{BPD-PROG}
\renewcommand{\FechaPub}{2023-01}
\renewcommand{\Edit}{03}
\renewcommand{\Titulo}{Sanción por no cumplir el reglamento de BPD}
\section{\Titulo}

%\section{Sanción por no cumplir el reglamento de BPD}

\subsection{Objetivo}

Establecer amonestación por incumplimiento del reglamento de buenas prácticas de distribución con el fin de hacer cumplir los reglamentos y procedimientos establecidos por la empresa.

\subsection{Alcance}

A todas las personas que incumplan los reglamentos y/o procedimientos establecidos por la empresa.

\subsection{Documentos y/o normas relacionados}

\begin{itemize}
	\item 1.3 - MA-OP-001 - Manual de calidad y buenas prácticas de distribución
	\item Tríptico de buenas prácticas de distribución.
\end{itemize}

\subsection{Procedimiento}

\subsubsection{Instrucciones}

\begin{itemize}
	\item Toda persona que ingrese a las áreas operativas deberá apegarse a los reglamentos de buenas prácticas de distribución establecidas por RED DE FRIOS S.A de C.V.
	\item El no cumplimiento a los puntos establecidos será causa a una sanción administrativa.
	\item Se aplicara una amonestación por escrito a la persona que no cumpla con las buenas prácticas de manufactura.
	\item El no cumplimiento a las buenas prácticas de distribución, se le aplicara una amonestación dependiendo de la falta, estipulada en el capítulo IX de sanciones del Reglamento Interno de Trabajo.
	\item En caso de no pertenecer a la empresa (Visitante) se podrá revocar su permiso de ingreso.
	\item En caso de ser proveedor se podrá llegar el caso de cancelación de contratos.
\end{itemize}

\subsection{Responsable de la actividad}

\begin{itemize}
	\item \textbf{Ejecutado} por personal de RRHH;
	\item \textbf{Monitoreado} por personal de calidad;
	\item \textbf{Verificado} por personal de gerencia.
\end{itemize}

\subsection{Acciones preventivas}

\begin{itemize}
	\item Se llevara a cabo una inspección. Registrando el indicador y medida correspondiente
	\item Si la desviación se repite frecuentemente se dará curso de capacitación al personal.
	\item Si después de haber capacitado al personal de almacén se siguen presentando desviaciones por causas injustificadas, se tomaran acciones más enérgicas con el personal por incumplimiento con sus deberes.
\end{itemize}

\subsection{Acciones correctivas}

\begin{itemize}
	\item En caso contrario la tarea se deberá volver a realizar como se indica en el procedimiento.
	\item En caso de no conformidad reportar en formato de acciones correctivas.
\end{itemize}

\subsection{Frecuencia}

\begin{itemize}
	\item Cuando el personal falte a la política de Buenas Prácticas de Distribución.
\end{itemize}

\subsection{Historial de modificaciones}

\begin{itemize}
	\item \textbf{Segunda edición:} Se anexa formato de amonestaciones
	\item \textbf{Tercera edición:} cambio de fecha de 02 de noviembre 2017 a 02 de noviembre 2018, se realizó cambio de código de M-RBPD a PR-002 y no se realizaron cambio en el documento de la revisión 002 a 003.
	\item \textbf{Cuarta edición:} enero 2020 se hizo cambio de formato y cambio de código del PR-002 al SA-OP-001. Y no se realizaron cambios de la revisión 03 a la 04.
	\item \textbf{Quinta edición:} febrero 2021 no se realizaron cambios de la revisión 04 a la 05.
	\item \textbf{Sexta edición:} febrero 2022 no se realizaron cambios de la revisión 05 a la 06.
	\item \textbf{Séptima edición:} 2023-01 cambios en formato y en serialización.
\end{itemize}

\subsection{Anexos}

\begin{itemize}
	\item \cref{AP1}
\end{itemize}