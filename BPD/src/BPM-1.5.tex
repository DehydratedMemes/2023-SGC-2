\renewcommand{\Codigo}{BPD-PROG}
\renewcommand{\FechaPub}{2023-01}
\renewcommand{\Edit}{03}
\renewcommand{\Titulo}{Reglamento de almacén}
\section{\Titulo}

\begin{itemize}
	\item Mantener las manos limpias.
	\item Prohibido portar joyería (anillos, cadenas, aretes, pulseras, piercings en lugares visibles etc.)
	\item No se permite fumar
	\item No se permite masticar chicle
	\item Prohibido comer o beber en las áreas de Producto
	\item Prohibido escupir
	\item Usar ropa y/o uniforme limpio
	\item Prohibido usar pantalones deshilachados o con pedrería
	\item Prohibido entrar en shorts y camisas sin mangas
	\item Usar cofias en áreas de dentro de Almacén
	\item Usar cubrebocas en áreas de Almacén
	\item No portar artículos en bolsillos superiores a la cintura
	\item Prohibido introducir artículos de vidrio
	\item No golpear ni dañar los productos
	\item Mantener limpias y en orden las áreas y las herramientas de trabajo.
	\item Prohibido el uso de zapatos abiertos.
	\item No pisar los productos ni las tarimas
	\item No colocar en el piso material de contacto con producto (ejem.\ emplaye, cajas plástico, etc.)
	\item No utilizar celulares personales
\end{itemize}

\subsection{Historial de modificaciones}

\begin{itemize}
	\item \textbf{Tercera edición:} cambio de fecha de 01 de noviembre 2017 a 01 de noviembre 2018, se realizó cambio de código de RA- BPD a IN-001. Y no se realizaron cambios en el documento de la revisión 02 a 03.
	\item \textbf{Cuarta edición:} enero 2020 se hizo cambio de formato y cambio de código del IN-001 al RE-OP-002. Y no se realizaron cambios de la revisión 03 a la 04.
	\item \textbf{Quinta edición:} febrero 2021 no se realizaron cambios de la revisión 04 a la 05.
	\item \textbf{Sexta edición:} febrero 2022 no se realizaron cambios de la revisión 05 a la 06.
	\item \textbf{Séptima edición:} 2023-01: cambios en formato y serialización.
\end{itemize}
