% ORDEN DE ENTRADA
% 	\item \textbf{caja de la unidad de transporte (CUT)}
% 	\begin{itemize}
% 		\item unidad, generalmente con temperatura controlada, unida a un vehículo de transporte.
% 		\item \textbf{Nota:} en la sección \emph{C - Servicios} se usa el término de \emph{caja} en referencia a cantidad unitaria de producto.
% 	\end{itemize}
% 	\item \textbf{carga}
% 	\begin{itemize}
% 		\item El total de \emph{tarimas con producto} solicitadas en una orden de salida por el \emph{cliente} o algún representante de él que puedan ser transportadas en una sola caja.
% 		\item \textbf{Nota:} si en la \emph{orden de salida} se requiere de más de una caja para surtir la demanda, no se considera como \emph{carga} al total solicitado en ésta, si no al total de producto transportado por cada caja.
% 	\end{itemize}
% 	\item \textbf{cliente}
% 	\begin{itemize}
% 		\item Empresa o persona a la que RDF le brinda el servicio de almacenamiento de producto alimenticio.
% 	\end{itemize}
% 	\item \textbf{condición de la caja}
% 	\begin{itemize}
% 		\item Estado visible en general de las paredes, techo, piso y unidad de enfriamiento de la caja;
% 		\item si las paredes y techos tienen rasgaduras, se debe de marcar el campo como \emph{Mala.}
% 		\item \textbf{NOTA:} la condición (visible) de la caja \emph{no asegura} el correcto funcionamiento de la unidad de enfriamiento de la misma. Por lo que si la caja no llega a su destino a la temperatura a la que salió de RDF por algún fallo del sistema de enfriamiento del la misma, RDF no se responsabiliza de la carga;
% 		\item \textbf{NOTA:} si es requerido por el \emph{cliente} o el \emph{sub-cliente,} se pueden incorporar a las cajas termoregistradores proporcionados o pagados por dichas partes.
% 	\end{itemize}
% 	\item \textbf{congelación}
% 	\begin{itemize}
% 		\item En el apartado \emph{B - Condiciones del producto} se considera como congelación una especificación de almacenamiento de producto a óptimamente \SI{-18}{\celsius};
% 		\item \textbf{NOTA:} en caso de que las especificaciones de almacenamiento del producto sean diferentes a \SI{-18}{\celsius}, pero sean menores a \SI{0}{\celsius}, se marcará este campo.
% 	\end{itemize}
% 	\item \textbf{Emplaye (servicio)}
% 	\begin{itemize}
% 		\item proceso en el que se envuelve la tarima cargada con producto con emplaye.
% 	\end{itemize}
% 	\item \textbf{limpieza de la caja}
% 	\begin{itemize}
% 		\item se considerará como \emph{buena} la limpieza de una caja cuando no se encuentren las paredes, techo y piso manchados con residuos como (pero no limitados a):
% 		\begin{itemize}
% 			\item residuos de madera o plástico de cargas anteriores;
% 			\item manchas o salpicaduras de aceite;
% 			\item manchas o salpicaduras de fluidos excretados por cargas anteriores;
% 			\item otras.
% 		\end{itemize}
% 	\end{itemize}
% 	\item \textbf{maniobras de salida}
% 	\begin{itemize}
% 		\item proceso de carga de tarimas a la caja, considerando la extracción de la tarima cargada de la cámara de enfriamiento y su disposición de forma temporal en el área designada para efectuar la carga.
% 	\end{itemize}
% 	\item \textbf{picking}
% 	\begin{itemize}
% 		\item proceso manual en el que \emph{cargas unitarias} de producto se disponen en una tarima o palé vacío según especificaciones acordadas con el \emph{cliente} o \emph{sub-cliente} para su posterior carga en la caja.
% 	\end{itemize}
% 	\item \textbf{presencia de plagas}
% 	\begin{itemize}
% 		\item se debe de marcar positivamente la presencia de plagas si es que se encuentran:
% 		\begin{itemize}
% 			\item excretas;
% 			\item orina
% 			\item restos de plagas;
% 			\item insectos vivos;
% 			\item indicios de roídas;
% 			\item plumas;
% 			\item entre otros.
% 		\end{itemize}
% 	\end{itemize}
% 	\item \textbf{presencia de tarimas NO cargadas en RDF}
% 	\begin{itemize}
% 		\item si la caja viene con una carga parcial de productos que no fueron cargados en RDF, se debe de marcar el campo de forma positiva.
% 	\end{itemize}
% 	\item \textbf{sub-cliente}
% 	\begin{itemize}
% 		\item \emph{o prestador de servicio logístico} es la empresa o persona encargada del transporte de la carga.
% 	\end{itemize}
% 	\item \textbf{refrigeración}
% 	\item En el apartado \emph{B - Condiciones del producto} se considera como refrigeración una especificación de almacenamiento de producto a óptimamente \SI{4}{\celsius};
% 	\begin{itemize}
% 		\item \textbf{NOTA:} en caso de que las especificaciones de almacenamiento del producto sean diferentes a \SI{4}{\celsius} pero sean mayores a \SI{0}{\celsius}, se marcará este campo.
% 	\end{itemize}
% 	\item \textbf{requerimiento de temperatura del producto}
% 	\begin{itemize}
% 		\item especificación cualitativa indicada para el almacenamiento del producto según el \emph{cliente.}
% 	\end{itemize}
% 	\item \textbf{tarima}
% 	\begin{itemize}
% 		\item también conocida como \emph{palé,} es una estructura plana de transporte que soporta productos de manera estable para ser cargados por un montacargas;
% 	\end{itemize}
% 	\item \textbf{temperatura de salida}
% 	\begin{itemize}
% 		\item temperatura expresada en grados Celsius \si{\celsius} que indica la temperatura a la que se encuentra la caja una vez que fue cargada.
% 	\end{itemize}
% 	\item \textbf{temperatura de la CUT adecuada para cargar}
% 	\begin{itemize}
% 		\item se refiere a la temperatura en la que se encuentra la CUT cuando llega al andén y aún no se carga
% 	\end{itemize}
% \end{itemize}


% Abreviaciones
\newabbreviation{CUT}{CUT}{caja de la unidad de transporte}
\newabbreviation{PPRO}{PPRO}{Programa de Prerequisito(s) Operacional(es)}
\newabbreviation{PPR}{PPR}{Programa de Prerequisito(s)}
\newabbreviation{HACCP}{HACCP}{Hazard Analysis and Critical Control Points}