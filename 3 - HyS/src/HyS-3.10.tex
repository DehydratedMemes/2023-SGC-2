\thispagestyle{formato-PI}
\renewcommand{\MenorVer}{0}
\renewcommand{\MayorVer}{2}
%\renewcommand{\Edit}{\MayorVer.\MenorVer}
\renewcommand{\Codigo}{HYS-10-IT}
\renewcommand{\FechaPub}{2023--01}
\renewcommand{\Titulo}{Limpieza de patios de carga y colchonetas}

\section{\Titulo}
% \section{Limpieza de patios de carga y colchonetas}

\subsection{Objetivo}

\begin{itemize}
	\item Establecer procedimiento de limpieza que se debe de realizar en los patios de carga y colchonetas con el fin de dar mantenimiento de limpieza y minimizar los riesgos de contaminación para el producto.
\end{itemize}

\subsection{Alcance}

\begin{itemize}
	\item Patios de carga y colchonetas.
\end{itemize}

\subsection{Terminología y definiciones}
\begin{description}
\defglo{limpieza}
\defglo{sanitizacion}
\end{description}

%\subsection{Documentos y/o normas relacionadas}

%N/A

\subsection{Procedimiento}

\subsubsection{Materiales}

\begin{itemize}
	\item Preparación de Liquid K a la concentración indicada por el proveedor \qtyrange{0.13}{0.26}{\percent} o bien \qtyrange{1.3}{2.6}{\milli\liter\per\liter}
	\item Cubeta de \qty{20}{\litre}
	\item Cepillo para piso color amarillo
	\item Recogedor de plástico color amarillo
	\item Hidrolavadora
\end{itemize}

\subsubsection{Instrucciones}

\paragraph{Preoperativo}

\subparagraph{Limpieza de patios de carga}

\begin{enumerate}
	\item Barrer toda el área con el cepillo para piso.
	\item Deshacer cualquier tipo de desechos en bolsas plásticas o contenedor.
\end{enumerate}

\paragraph{Operativo}

\subparagraph{Limpieza diaria de pisos de carga y exteriores}

\begin{enumerate}
	\item Barrer toda el área con el cepillo para piso.
	\item Desechar cualquier tipo de desechos en bolsas plásticas o contenedor.
	\item Con ayuda de la hidrolavadora eliminar suciedad incrustada en pisos.
	\item En caso de un derrame o suciedad incrustada en el área de patio aplicar desengrasante a la concentración indicada por el proveedor.
	\item Tallar con ayuda del cepillo para pisos.
	\item Retire excedente de detergente con ayuda de la hidrolavadora.
	\item Para finalizar barrer todo el exceso de agua con el cepillo
	\item Elimine excedente de agua
\end{enumerate}

\subparagraph{Limpieza de Colchonetas}

\begin{itemize}
	\item Enjuagar con la hidrolavadora asegurándose de eliminar excedentes de polvo,
	\item Posteriormente continuar cepillando el área con detergente líquido (a la concentración indicada por el proveedor).
	\item Con la ayuda de una hidrolavadora con agua a presión retirar el exceso de detergente
	\item Si es requerido seque el área con toalla desechable.
\end{itemize}

\subsection{Responsables de la actividad}

\begin{itemize}
	\item \textbf{Ejecutado} por personal de limpieza
	\item \textbf{Monitoreado} por personal de mantenimiento
	\item \textbf{Verificado} por personal de Calidad.
\end{itemize}

\subsection{Acciones preventivas}

\begin{itemize}
	\item Se llevará a cabo una inspección diaria. Registrando el indicador y medida correspondiente
	\item Si la desviación se repite frecuentemente se dará curso de capacitación al personal de limpieza para que realicen eficientemente su trabajo.
	\item Si después de haber capacitado al personal de limpieza se siguen presentando desviaciones por causas injustificadas, se tomaran acciones más enérgicas con el personal por incumplimiento con sus deberes.
\end{itemize}

\subsection{Acciones correctivas}

\begin{itemize}
	\item La tarea es liberada una vez que el área se encuentran libre de polvo, suciedad, residuos de materiales extraños, residuos de jabón, y seca. En caso contrario la tarea se deberá volver a realizar como se indica en el procedimiento.
	\item En caso de no conformidad reportar en \RAC.
\end{itemize}

\subsection{Frecuencia}

\begin{itemize}
	\item[Superficial] Diario de lunes a viernes de 10:30 a 11:30 y sábados de 11:30 a 12:30;
	\item[Profundo] Trimestralmente.
\end{itemize}

\begin{changelog}[simple, sectioncmd=\subsection*,label=changelog-\thesection-\MayorVer.\MenorVer0]
	\begin{version}[v=\MayorVer.\MenorVer, date=2023--01, author=Pablo E. Alanis]
		\item Cambio de formato;
		\item Cambios en la serialización de versiones;
		\item Correcciones ortográficas y de estilo.
	\end{version}

	\begin{version}[v=1.7, date=2022--05, author=Alonso M.]
		\item cambio de fecha;
	\end{version}

	\shortversion{v=1.6, date=2021--05, changes=No hubo cambios}
\end{changelog}