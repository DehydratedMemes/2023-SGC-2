\thispagestyle{formato-PI}
\renewcommand{\MayorVer}{2}
\renewcommand{\MenorVer}{0}
%\renewcommand{\Edit}{\MayorVer.\MenorVer}
\renewcommand{\Titulo}{Limpieza de montacargas y patines}
\renewcommand{\Codigo}{HYS-5-IT}
\renewcommand{\FechaPub}{2023--01}

\section{\Titulo}\index{Limpieza!montacargas, de}\index{Limpieza!patines, de}
% \section{Limpieza de montacargas y patines}

\subsection{Objetivo}
Establecer el procedimiento de limpieza que se debe de realizar a los montacargas y patines de carga para su mantenimiento y minimizar riesgos de contaminación al producto.

\subsection{Alcance}
Todos los patines y montacargas ubicados dentro del almacén.

\subsection{Terminología y definiciones}
\begin{description}
	\defglo{limpieza}
	\defglo{sanitizacion}
\end{description}

\subsection{Procedimiento}

\subsubsection{Materiales}

\begin{itemize}
	\item Preparación de \textit{Liquid K} a la concentración indicada por el proveedor \qtyrange{0.13}{0.26}{\percent} o bien \qtyrange{1.3}{2.6}{\milli\liter\per\liter}
	\item Cubeta de \qty{20}{\litre}
	\item Toalla desechable
	\item Cepillo de mano
\end{itemize}

\subsubsection{Instrucciones}

\paragraph{Limpieza de montacargas y patines diario}

\subparagraph{Preoperativo}

\begin{enumerate}
	\item Desconecte equipos para evitar que se encuentren energizados.
	\item Prepare el detergente a la concentración recomendada por el proveedor, sumerja la toalla desechable en el detergente, elimine el exceso del mismo exprimiendo la toalla desechable limpie la estructura del equipo de carga.
	\item Elimine el excedente de humedad del equipo de carga utilizando toalla desechable.
	\item Limpiar el área después de terminar el trabajo.
\end{enumerate}

\subsection{Responsables de la actividad}

\begin{itemize}
	\item \textbf{Ejecutado} por personal de operación montacargas
	\item \textbf{Monitoreado} por personal de mantenimiento
	\item \textbf{Verificado} por personal de Calidad.
\end{itemize}

\subsection{Acciones preventivas}

\begin{itemize}
	\item Se llevará a cabo una inspección diaria. Registrando el indicador y medida correspondiente
	\item Si la desviación se repite frecuentemente se dará curso de capacitación al personal de almacén para que realicen eficientemente su trabajo.
	\item Si después de haber capacitado al personal de almacén se siguen presentando desviaciones por causas injustificadas, se tomaran acciones más enérgicas con el personal por incumplimiento con sus deberes.
\end{itemize}

\subsection{Acciones correctivas}

\begin{enumerate}
	\item La tarea es liberada una vez que el área se encuentran libre de polvo, suciedad, residuos de materiales extraños, residuos de jabón, y seca. En caso contrario la tarea se deberá volver a realizar como se indica en el procedimiento.
	\item En caso de no conformidad reportar en \RAC.
\end{enumerate}

\subsection{Frecuencia}

\begin{itemize}
	\item Diario al finalizar operaciones lunes a viernes 5:00 pm y sábado a la 1:00 pm
\end{itemize}

\begin{changelog}[simple, sectioncmd=\subsection*,label=changelog-\thesection-\MayorVer.\MenorVer]
	\begin{version}[v=\MayorVer.\MenorVer, date=2023--01, author=Pablo E. Alanis]
		\item Cambio de formato;
		\item Cambios en la serialización de versiones;
		\item Correcciones ortográficas y de estilo.
	\end{version}

	\begin{version}[v=1.7, date=2022--05, author=Alonso M.]
		\item cambio de fecha;
	\end{version}

	\shortversion{v=1.6, date=2021--05, changes=No hubo cambios}
\end{changelog}