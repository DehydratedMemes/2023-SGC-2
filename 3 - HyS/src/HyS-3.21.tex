\renewcommand{\MenorVer}{0}
\renewcommand{\MayorVer}{2}
%\renewcommand{\Edit}{\MayorVer.\MenorVer}
\renewcommand{\Codigo}{HYS-21-IT}
\renewcommand{\FechaPub}{2023--01}
\renewcommand{\Titulo}{Determinación de cloración en agua}

\section{\Titulo}
% \section{Determinación de cloración en agua}

\subsection{Objetivo}

\begin{itemize}
	\item Establecer el procedimiento de como determinar la cantidad de cloro en el agua.
\end{itemize}

\subsection{Alcance}

\begin{itemize}
	\item Este procedimiento de determinación de cloro es aplicable a todos los puntos de tuberías de tomas de agua ubicadas en las instalaciones de \gls{RDF}.
\end{itemize}

\subsection{Terminología y definiciones}

\begin{description}
\defglo{limpieza}
\defglo{sanitizacion}
\end{description}

\subsection{Documentos y/o normas relacionadas}

\begin{itemize}
	\item N/A
\end{itemize}

\subsection{Procedimiento}

\subsubsection{Precauciones de seguridad}

\begin{itemize}
	\item El indicador empleado para la determinación de cloro, 2-metilanilina, es cancerígeno. Se deben de seguir las indicaciones del MSDS y evitar la exposición directa a este compuesto.
\end{itemize}

\subsubsection{Materiales}

\begin{enumerate}
	\item Indicador (2-metilanilina);
	\item Indicados de concentración de cloro;
	\item Vial de muestra para agua.
\end{enumerate}

\subsubsection{Instrucciones}

\begin{enumerate}
	\item Deje correr agua por un promedio de 1.5 a 2 min para poder tomar una muestra.
	\item Posterior a este tiempo prosiga a llenar el vial del indicador de concentración de cloro.
	\item Vierta 5 gotas del indicador ortotolidina en la muestra de agua.
	\item Cierre el vial con el tapón.
	\item Agite el vial para mezclar la muestra con el indicador.
	\item Observe la tonalidad que adquiere la muestra y compare con el indicador de cloro.
	\item Registre el valor de cloro obtenido con la prueba
	\item El rango de valor de concentración ideal se encuentra entre 1 y 1.5 ppm
\end{enumerate}

\subsection{Responsables de la actividad}

\begin{itemize}
	\item \textbf{Ejecutado} por personal de aseguramiento de calidad.
\end{itemize}

\subsection{Acciones preventivas}

\begin{itemize}
	\item Se llevara a cabo una inspección diaria. Registrando el indicador y medida correspondiente
	\item Si la desviación se repite frecuentemente se dará curso de capacitación al personal de almacén y limpieza para que realicen eficientemente su trabajo.
	\item Si después de haber capacitado al personal de almacén y limpieza se siguen presentando desviaciones por causas injustificadas, se tomaran acciones más enérgicas con el personal por incumplimiento con sus deberes.
\end{itemize}

\subsection{Acciones correctivas}

\begin{itemize}
	\item En caso de no detectarse cloro en el agua de red municipal, el almacén cuenta con un dosificador de pulsos el cual es activado para ingresar el cloro a la línea de suministro del almacén; cabe mencionar que el agua no es utilizada para contacto con producto, solo para servicios y lavado de manos.
\end{itemize}

\subsection{Frecuencia}

\begin{itemize}
	\item Todos los días de lunes a sábado iniciando operaciones.
\end{itemize}

\subsection{Historial de modificaciones}

\begin{itemize}
	\item Primera Edición: cambio de formato.
	\item Segunda Edición: se realizó cambio de código de HS-P-DCA a PR-039 y no se realizaron cambios de la edición 01 a la 02.
	\item Tercera Edición: febrero 2020 se hizo cambio de formato y cambio de código de PR-039 a IT-OP-022. Y no se realizaron cambios de la revisión 02 a la 03.
	\item Cuarta edición: febrero 2021 no se realizaron cambios de la revisión 03 a la 04.
	\item Quinta edición: febrero 2022 no se realizaron cambios de la revisión 04 a la 05.
\end{itemize}
