\thispagestyle{formato-PI}
\renewcommand{\MayorVer}{2}
\renewcommand{\MenorVer}{0}
%\renewcommand{\Edit}{\MayorVer.\MenorVer}
\renewcommand{\Codigo}{HYS-4-IT}
\renewcommand{\FechaPub}{2023--01}
\renewcommand{\Titulo}{Lavado de rampas mecánicas}

\section{\Titulo}\index{Limpieza de!rampas mecánicas, de}

% \section{Lavado de rampas mecánicas}

\subsection{Objetivo}

\begin{itemize}
	\item \textbf{Establecer} el procedimiento de limpieza que se debe de realizar en a las rampas mecánicas para minimizar la presencia de peligros físicos, químicos o biológicos que pudieran generar un riesgo al producto.
\end{itemize}

\subsection{Alcance}
Todas las rampas mecánicas ubicadas en el almacén.

\subsection{Terminología y definiciones}

\begin{description}
	\defglo{limpieza}
	\defglo{sanitizacion}
\end{description}

%\subsection{Documentos y/o normas relacionadas}

%N/A

\subsection{Procedimiento}
\subsubsection{Materiales}
\begin{itemize}
	\item Cepillo para piso color rojo
	\item Cepillos manuales color rojo
	\item Recogedor de plástico
\end{itemize}

\subsubsection{Instrucciones}
\paragraph{Preoperativo}

\begin{enumerate}
	\item Poner el cono de aviso, levantar rampa y poner el seguro de la rampa.
	\item Retire con ayuda de escoba todos los residuos que se encuentren en la superficie y en el interior de la rampa.
	\item Posteriormente se procede a la recolección de los residuos con recogedor.
	\item Depositar los desechos en el contenedor para basura.
	\item Retire con ayuda del cepillo todos los residuos que se encuentran en la superficie y en el interior de la rampa.
	\item Posteriormente se procede a la recolección de los residuos con recogedor.
	\item Depositar los desechos en el contenedor para basura.
\end{enumerate}

\paragraph{Operativo}
\begin{enumerate}
	\item Retirar con ayuda de cepillo para piso todos los residuos (plásticos, pedazos de madera, etc.).
	\item Retire residuos de los cepillos de la rampa
	\item Posteriormente se procede a la recolección de los residuos con recogedor.
	\item Depositar los desechos en el contenedor para basura.
\end{enumerate}

\subsection{Responsables de la actividad}
\begin{itemize}
	\item \textbf{Ejecutado} por personal de limpieza
	\item \textbf{Monitoreado} por personal de mantenimiento
	\item \textbf{Verificado} por personal de Calidad.
\end{itemize}

\subsection{Acciones preventivas}

\begin{itemize}
	\item Se llevara a cabo una inspección diaria. Registrando el indicador y medida correspondiente
	\item Si la desviación se repite frecuentemente se dará curso de capacitación al personal de limpieza para que realicen eficientemente su trabajo.
	\item Si después de haber capacitado al personal de limpieza se siguen presentando desviaciones por causas injustificadas, se tomaran acciones más enérgicas con el personal por incumplimiento con sus deberes.
\end{itemize}

\subsection{Acciones correctivas}

\begin{itemize}
	\item La tarea es liberada una vez que el área se encuentran libre de polvo, suciedad, residuos de materiales extraños, residuos de jabón, y seca. En caso contrario la tarea se deberá volver a realizar como se indica en el procedimiento.
	\item En caso de no conformidad reportar en \RAC.
\end{itemize}

\subsection{Frecuencia}

\begin{itemize}
	\item[\textbf{Superficial}] Diaria de lunes a viernes de 10:30 a 11:00 y sábados de 11:30 a 12:00;
	\item[\textbf{Profunda:}] Anualmente, según el programa y la disponibilidad de espacio.
\end{itemize}

\begin{changelog}[simple, sectioncmd=\subsection*,label=changelog-3.7]
	\begin{version}[v=2.0, date=2023--01, author=Pablo E. Alanis]
		\item Cambio de formato;
		\item Cambios en la serialización de versiones;
		\item Correcciones ortográficas y de estilo.
	\end{version}

	\begin{version}[v=1.7, date=2022--05, author=Alonso M.]
		\item cambio de fecha;
	\end{version}

	\shortversion{v=1.6, date=2021--05, changes=No hubo cambios}
\end{changelog}