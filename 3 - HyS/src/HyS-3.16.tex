\thispagestyle{formato-PI}
\renewcommand{\MenorVer}{0}
\renewcommand{\MayorVer}{2}
%\renewcommand{\Edit}{\MayorVer.\MenorVer}
\renewcommand{\Codigo}{HYS-16-IT}
\renewcommand{\FechaPub}{2023--01}
\renewcommand{\Titulo}{Limpieza de baños de empleados}

\section{\Titulo}\index{Limpieza!baño de empleados, de}
% \section{Limpieza de baños de empleados}

\subsection{Objetivo}
Establecer un procedimiento que aseguren un mantenimiento y limpieza apropiada de los baños para colaboradores de la empresa.

\subsection{Alcance}

Todos los baños dentro de las instalaciones.

\subsection{Terminología y definiciones}

\begin{description}
\defglo{limpieza}
\defglo{sanitizacion}
\end{description}

%\subsection{Documentos y/o normas relacionadas}

%N/A

\subsection{Procedimiento}

\subsubsection{Materiales}

\begin{itemize}
	\item Preparación de Liquid K a la concentración indicada por el proveedor, \qtyrange{.13}{.26}{\percent}, o bien \qtyrange{1.3}{2.6}{\milli\liter\per\liter};
	\item Cubeta de \qty{20}{\litre};
	\item Cepillo para piso color azul;
	\item Cepillos manuales color azul;
	\item Recogedor de plástico;
	\item Jalador color azul;
	\item Trapeador color azul.
\end{itemize}

\subsubsection{Instrucciones}
\paragraph{Operativo}
\subparagraph{Mingitorios, Sanitarios, lavamanos y pisos}
\begin{enumerate}
	\item Preparar detergente a la concentración indicada por el proveedor
	\item Aplicar detergente en lavamanos
	\item Tallar con el cepillo de mano.
	\item Aplicar detergente a Mingitorios y sanitarios
	\item Tallar con el cepillo identificado para esta área.
	\item Enjuagar con agua limpia.
	\item Eliminar el excedente de agua hacia una coladera.
	\item Aplique detergente en pisos y talle con el cepillo para piso.
	\item Elimine excedente de agua.
\end{enumerate}

\subsection{Responsables de la actividad}

\begin{itemize}
	\item \textbf{Ejecutado} por personal de limpieza;
	\item \textbf{Monitoreado} por personal de mantenimiento;
	\item \textbf{Verificado} por personal de Calidad.
\end{itemize}

\subsection{Acciones preventivas}

\begin{itemize}
	\item Se llevará a cabo una inspección diaria. Registrando el indicador y medida correspondiente
	\item Si la desviación se repite frecuentemente se dará curso de capacitación al personal de limpieza para que realicen eficientemente su trabajo
	\item Si después de haber capacitado al personal de limpieza se siguen presentando desviaciones por causas injustificadas, se tomaran acciones más enérgicas con el personal por incumplimiento con sus deberes.
\end{itemize}

\subsection{Acciones correctivas}

\begin{itemize}
	\item La tarea es liberada una vez que el área se encuentran libre de polvo, suciedad, residuos de materiales extraños, residuos de jabón, y seca. En caso contrario la tarea se deberá volver a realizar como se indica en el procedimiento.
	\item En caso de no conformidad reportar en \RAC.
\end{itemize}

\subsection{Frecuencia}

Diario de lunes a viernes de 10:30 a 11:00 y sábados de 11:30 a 12:00.

\begin{changelog}[title=Registro de cambios,simple, sectioncmd=\subsection*,label=changelog-\thesection-\MayorVer.\MenorVer6]
	\begin{version}[v=\MayorVer.\MenorVer, date=2023--01, author=Pablo E. Alanis]
		\item Cambio de formato;
		\item Cambios en la serialización de versiones;
		\item Correcciones ortográficas y de estilo.
	\end{version}

	\begin{version}[v=1.7, date=2022--05, author=Alonso M.]
		\item cambio de fecha;
	\end{version}	

	\shortversion{v=1.6, date=2021--05, changes=No hubo cambios}
\end{changelog}