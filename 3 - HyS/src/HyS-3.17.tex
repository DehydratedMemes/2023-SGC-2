\thispagestyle{formato-PI}
\renewcommand{\MenorVer}{0}
\renewcommand{\MayorVer}{2}
%\renewcommand{\Edit}{\MayorVer.\MenorVer}
\renewcommand{\Codigo}{HYS-17-IT}
\renewcommand{\FechaPub}{2023--01}
\renewcommand{\Titulo}{Lavado del comedor}

\section{\Titulo}\index{Limpieza!comedor, del}
% \section{Lavado del comedor}

\subsection{Objetivo}
Establecer un procedimiento que aseguren un mantenimiento y limpieza apropiada del comedor.

\subsection{Alcance}
Comedor de almacén.

\subsection{Terminología y definiciones}
\begin{description}
\defglo{limpieza}
\defglo{sanitizacion}
\end{description}

%\subsection{Documentos y/o normas relacionadas}

%N/A

\subsection{Procedimiento}

\subsubsection{Materiales}

\begin{itemize}
	\item Preparación de Liquid K a la concentración indicada por el proveedor \qtyrange{.13}{.26}{\percent}, o bien \qtyrange{1.3}{2.6}{\milli\liter\per\liter}
	\item Cubeta de \qty{20}{\litre}
	\item Cepillo para piso color verde
	\item Fibra
	\item Toalla desechable
	\item Recogedor de plástico color verde
	\item Jalador color verde
	\item Trapeador color verde
\end{itemize}

\subsubsection{Instrucciones}
\paragraph{Operativo}
\subparagraph{Limpieza para mesa y equipo de comedor}
\begin{enumerate}
	\item Eliminar toda materia orgánica e inorgánica que se encuentre en estas aéreas y depositarla en los contenedores de basura
	\item Desconectar todos los equipos eléctricos.
	\item Preparar detergente a la concentración indicada por el proveedor
	\item Aplicar detergente, tallar con fibra la mesa y el equipo de comedor
	\item Enjuagar con agua limpia
	\item Eliminar excedente de humedad con una toalla desechable
\end{enumerate}

\subparagraph{Limpieza de pisos}
\begin{enumerate}
	\item Eliminar toda materia orgánica e inorgánica que se encuentre en estas aéreas y depositarlas en los contenedores de basura
	\item Prepare detergente a la concentración indicada por el proveedor
	\item Con ayuda de trapeador realizar limpieza del piso
	\item Enjuague el trapeador con agua limpia.
	\item Con el trapeador elimine el excedente de humedad.
\end{enumerate}

\subsection{Responsables de la actividad}

\begin{itemize}
	\item \textbf{Ejecutado} por personal de limpieza;
	\item \textbf{Monitoreado} por personal de mantenimiento;
	\item \textbf{Verificado} por personal de Calidad.
\end{itemize}

\subsection{Acciones preventivas}

\begin{itemize}
	\item Se llevará a cabo una inspección diaria. Registrando el indicador y medida correspondiente
	\item Si la desviación se repite frecuentemente se dará curso de capacitación al personal de limpieza para que realicen eficientemente su trabajo.
	\item Si después de haber capacitado al personal de limpieza se siguen presentando desviaciones por causas injustificadas, se tomaran acciones más enérgicas con el personal por incumplimiento con sus deberes.
\end{itemize}

\subsection{Acciones correctivas}

\begin{itemize}
	\item La tarea es liberada una vez que el área se encuentran libre de polvo, suciedad, residuos de materiales extraños, residuos de jabón, y seca. En caso contrario la tarea se deberá volver a realizar como se indica en el procedimiento.
	\item En caso de no conformidad reportar en \RAC.
\end{itemize}

\subsection{Frecuencia}

Diaria de lunes a viernes de 14:30 a 15:00 y sábados de 11:30 a 12:00.

\begin{changelog}[simple, sectioncmd=\subsection*,label=changelog-\thesection-\MayorVer.\MenorVer7]
	\begin{version}[v=\MayorVer.\MenorVer, date=2023--01, author=Pablo E. Alanis]
		\item Cambio de formato;
		\item Cambios en la serialización de versiones;
		\item Correcciones ortográficas y de estilo.
	\end{version}

	\begin{version}[v=1.7, date=2022--05, author=Alonso M.]
		\item cambio de fecha;
	\end{version}	

	\shortversion{v=1.6, date=2021--05, changes=No hubo cambios}
\end{changelog}