\thispagestyle{formato-PI}
\renewcommand{\MayorVer}{2}
\renewcommand{\MenorVer}{0}
%\renewcommand{\Edit}{\MayorVer.\MenorVer}
\renewcommand{\Titulo}{Limpieza de andén y cámaras desocupadas}
\renewcommand{\Codigo}{HYS-3-IT}
\renewcommand{\FechaPub}{2023--01}

\section{\Titulo}\index{Limpieza de!andén}\index{Limpieza de!cámaras desocupadas}
% \section{Limpieza de andén y cámaras desocupadas}

\subsection{Objetivo}
\begin{itemize}
	\item \textbf{Establecer} el procedimiento de limpieza que se debe de realizar en andén y cámara fuera de operación, buscando con ello eliminar o minimizar la presencia de peligros físicos, químicos o biológicos que pudieran generar un riesgo de contaminación en el producto.
\end{itemize}

\subsection{Alcance}
Este procedimiento es aplicable al andén y cámaras fuera de operación de \gls{RDF}.

\subsection{Terminología y definiciones}

\begin{description}
\defglo{limpieza}
\defglo{sanitizacion}
\end{description}

% \subsection{Documentos y/o normas relacionadas} %TODO

\subsection{Procedimiento}
\subsubsection{Materiales}

\begin{itemize}
	\item Preparación de \textit{Liquid K} a la concentración indicada por el proveedor (\qtyrange{1.3}{2.6}{\milli\liter\per\liter}).
	\item Cubeta
	\item Cepillo para piso color rojo
	\item Cepillos manuales color rojo
	\item Recogedor de plástico
	\item Jalador color rojo
Trapo de microfibra antibacterial.
\end{itemize}

\subsubsection{Instrucciones}
\paragraph{Preoperativo}
\begin{enumerate}
	\item Retire el producto del área donde se realizara la limpieza para evitar una posible contaminación durante el proceso de limpieza.
	\item Retire el material o suciedad visible como pudiera ser emplaye, madera, cartón, papel, entre otros hasta que el área quede libre de estos, deposite los desechos en el contenedor de basura del área.
	\item Con ayuda de la hidrolavadora pre enjuague los techos, paredes, protecciones, puertas, \textit{racks}, cortinas y pisos.
	\item En una cubeta preparar la solución de detergente \textit{Liquid K} (\qtyrange{1.3}{2.6}{\milli\liter\per\liter}).
	\item La limpieza del área se debe realizar en el siguiente orden: techos, muros, protecciones, puertas \textit{racks}, cortinas y pisos.\label{3.3:pt5}
	\item Sumerja el cepillo en el detergente \textit{Liquid K} y talle el área, realice esta operación las veces que sean necesarias hasta cubrir toda el área seleccionada, inicie el tallado de arriba hacia abajo según indica el orden en \cref{3.3:pt5}.
	\item Para el caso del piso, talle con el cepillo destinado para esta área.
	\item Con el uso de la hidrolavadora, realice el enjuague del área iniciando de la parte más alta y terminado con los pisos.
	\item Con el jalador rastrille el excedente de agua de los pisos y retire el agua con la ayuda de la aspiradora.
	\item Registre en el Programa Maestro de Limpieza y Sanitización (\RAC. la actividad realizada.
	\item Finalmente, notifique al Supervisor de Calidad para la verificación y liberación de la tarea.
	\item La tarea es liberada una vez que el área se encuentran libre de polvo, suciedad, residuos de materiales extraños, residuos de jabón, y seca. En caso contrario la tarea se deberá volver a realizar
	\item Con la escoba y recogedor retire el material o suciedad visible como pudiera ser emplaye, madera, cartón, papel entre otros hasta que el piso del área quede libre de estos;
	\item Retire de manera manual el emplaye, madera, cartón, papel que pudieran estar acumulados entre los \textit{racks.}
	\item Deposite los desechos en el contenedor de basura del área.
\end{enumerate}

\subsection{Responsables de la actividad}

\begin{itemize}
	\item \textbf{Ejecutado} por personal de limpieza
	\item \textbf{Monitoreado} por personal de mantenimiento
	\item \textbf{Verificado} por personal de Calidad.
\end{itemize}

\subsection{Acciones preventivas} %TODO: Hay que especificar que es lo que se verifica a diario.

\begin{itemize}
	\item Se llevará a cabo una inspección diaria. Registrando el indicador y medida correspondiente
	\item Si la desviación se repite frecuentemente se dará curso de capacitación al personal de almacén y limpieza para que realicen eficientemente su trabajo.
	\item Si después de haber capacitado al personal de almacén y limpieza se siguen presentando desviaciones por causas injustificadas, se tomaran acciones más enérgicas con el personal por incumplimiento con sus deberes.
\end{itemize}

\subsection{Acciones correctivas}

\begin{itemize}
	\item La tarea es liberada una vez que el área se encuentran libre de polvo, suciedad, residuos de materiales extraños, residuos de jabón, y seca. En caso contrario la tarea se deberá volver a realizar como se indica en el procedimiento. 
	\item En caso de no conformidad reportar en \RAC.
\end{itemize}

\subsection{Frecuencia}

\begin{itemize}
	\item[\textbf{Superficial:}]	Diaria, de lunes a viernes de 9:00 a 10:00 y sábados de 10:00 a 11:00;
	\item[\textbf{Profunda:}]		Trimestral según el programa general de limpieza.\footnote{En caso de que no haya disponibilidad para la limpieza profunda del andén en el intervalo establecido por el programa de limpieza, esta se puede reprogramar en un máximo de 2 meses.}
\end{itemize}

\begin{changelog}[simple, sectioncmd=\subsection*,label=changelog-3.3]
	\begin{version}[v=2.0, date=2023--01, author=Pablo E. Alanis]
		\item Cambio de formato;
		\item Cambios en la serialización de versiones;
		\item Correcciones ortográficas y de estilo.
	\end{version}

	\begin{version}[v=1.5, date=2022--05, author=Alonso M.]
		\item cambio de fecha;
	\end{version}

	\shortversion{v=1.4, date=2021--05, changes=No hubo cambios}
\end{changelog}