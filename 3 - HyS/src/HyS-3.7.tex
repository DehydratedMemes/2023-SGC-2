\thispagestyle{formato-PI}
\renewcommand{\MayorVer}{2}
\renewcommand{\MenorVer}{0}
%\renewcommand{\Edit}{\MayorVer.\MenorVer}
\renewcommand{\Codigo}{HYS-7-IT}
\renewcommand{\FechaPub}{2023--01}
\renewcommand{\Titulo}{Limpieza de aduana sanitaria}

\section{\Titulo}\index{Limpieza!aduana sanitaria, de}

\subsection{Objetivos}
Establecer el procedimiento de limpieza que aseguren un mantenimiento y limpieza apropiada de la aduana sanitaria.

\subsection{Alcance}
Este procedimiento es aplicable a la aduana sanitaria.

\subsection{Terminología y definiciones}
\begin{description}
\defglo{limpieza}
\defglo{sanitizacion}
\end{description}

\subsection{Procedimiento}
\subsubsection{Materiales}

\begin{itemize}
	\item Preparación de Liquid K a la concentración indicada por el proveedor \qtyrange{1.3}{2.6}{\percent} o bien \qtyrange{1.3}{2.6}{\milli\liter\per\liter}.
	\item Cubeta de \qty{20}{\litre}
	\item Cepillo para piso color rojo
	\item Cepillos manuales color rojo
	\item Recogedor de plástico
	\item Jalador color rojo
	\item Toallas desechables
\end{itemize}

\subsubsection{Instrucciones}

\paragraph{Preoperativo}

\subparagraph{Limpieza de aduana sanitaria}

\begin{enumerate}
	\item Prepare detergente a la concentración indicada por el proveedor.
	\item Con ayuda de un cepillo tallar el techo, paredes, tarja y accesorios de la aduana.
	\item Con agua corriente eliminar el excedente de detergente.
	\item Talle los pisos y junta sanitaria con ayuda de cepillo.
	\item Elimine el exceso de detergente con ayuda de agua corriente.
	\item Con el jalador elimine el excedente de agua hasta dejar el área libre de humedad.
\end{enumerate}

\paragraph{Limpieza de tapete sanitario, tarja y pisos}

\subparagraph{Pisos}

\begin{enumerate}
	\item Retire los desechos con ayuda de cepillo para piso.
	\item Deposite los desechos en contenedor.
	\item Prepare detergente a la concentración indicada por el proveedor.
	\item Con ayuda de cepillo, realice limpieza de pisos.
	\item Elimine excedente de humedad.
\end{enumerate}

\subparagraph{Tapete}

\begin{enumerate}
	\item Elimine el exceso de desinfectante
	\item Con agua corriente enjuague el tapete sanitario hasta eliminar cualquier residuo
\end{enumerate}

\subparagraph{Tarja}

\begin{enumerate}
	\item Limpie la tarja, elimine cualquier desecho que se encuentre en la misma.
	\item Prepare detergente a la concentración indicada por el proveedor.
	\item Talle con cepillo de mano la tarja
	\item Enjuague con agua corriente hasta elimine el excedente de detergente.
\end{enumerate}

\subsection{Responsables de la actividad}

\begin{itemize}
	\item \textbf{Ejecutado} por personal de limpieza;
	\item \textbf{Monitoreado} por personal de mantenimiento;
	\item \textbf{Verificado} por personal de Calidad.
\end{itemize}

\subsection{Acciones preventivas}

\begin{itemize}
	\item Se llevará a cabo una inspección diaria. Registrando el indicador y medida correspondiente
	\item Si la desviación se repite frecuentemente se dará curso de capacitación al personal de limpieza para que realicen eficientemente su trabajo.
	\item Si después de haber capacitado al personal de limpieza se siguen presentando desviaciones por causas injustificadas, se tomaran acciones más enérgicas con el personal por incumplimiento con sus deberes.
\end{itemize}

\subsection{Acciones correctivas}

\begin{itemize}
	\item La tarea es liberada una vez que el área se encuentran libre de polvo, suciedad, residuos de materiales extraños, residuos de jabón, y seca. En caso contrario la tarea se deberá volver a realizar como se indica en el procedimiento.
	\item En caso de no conformidad reportar en \RAC.
\end{itemize}

\subsection{Frecuencia}

\begin{description}
	\item[Profundo] Trimestralmente;
	\item[Superficial] Lunes a sábado de 8:00 a 8:30.
\end{description}

\begin{changelog}[simple, sectioncmd=\subsection*,label=changelog-\thesection-\MayorVer.\MenorVer]
	\begin{version}[v=\MayorVer.\MenorVer, date=2023--01, author=Pablo E. Alanis]
		\item Cambio de formato;
		\item Cambios en la serialización de versiones;
		\item Correcciones ortográficas y de estilo.
	\end{version}

	\begin{version}[v=1.5, date=2022--05, author=Alonso M.]
		\item cambio de fecha;
	\end{version}

	\shortversion{v=1.4, date=2021--05, changes=No hubo cambios}
\end{changelog}