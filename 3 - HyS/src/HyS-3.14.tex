\renewcommand{\MenorVer}{0}
\renewcommand{\MayorVer}{2}
%\renewcommand{\Edit}{\MayorVer.\MenorVer}
\renewcommand{\Codigo}{HYS-14-IT}
\renewcommand{\FechaPub}{2023--01}
\renewcommand{\Titulo}{Limpieza de oficinas}

\section{\Titulo}
% \section{Limpieza de oficinas}

\subsection{Objetivo}

Establecer un procedimiento que aseguren un mantenimiento y limpieza apropiada de Oficinas.

\subsection{Alcance}

Todas las oficinas del almacén.

\subsection{Terminología y definiciones}

\begin{description}
\defglo{limpieza}
\defglo{sanitizacion}
\end{description}

%\subsection{Documentos y/o normas relacionadas}

%N/A

\subsection{Procedimiento}

\subsubsection{Materiales}

\begin{itemize}
	\item Preparación de Liquid K a la concentración indicada por el proveedor (0.13\% a 0.26\% v/v)(1.3ml a 2.6ml en 1L de agua)
	\item Cubeta de \qty{20}{\litre}
	\item Cepillo para piso color verde
	\item Recogedor de plástico color verde
	\item Toalla desechable
	\item Trapeador color verde
\end{itemize}

\subsubsection{Instrucciones}

\paragraph{Limpieza de pisos}

\begin{enumerate}
	\item Barrer pisos de áreas de oficinas.
	\item Recoger la suciedad recolectada con recogedor.
	\item Colocar residuos en bolsa para basura.
	\item Prepara cubeta con agua y detergente a la concentración indicada por el proveedor para iniciar el trapeado de áreas.
	\item Trapear las áreas de oficinas.
	\item Enjuagar trapeador con agua limpia.
	\item Eliminar excedente de humedad.
\end{enumerate}

\paragraph{Limpieza de escritorios}

\begin{enumerate}
	\item Preparar disolución a dosificación indicada
	\item Humedecer una toalla desechable
	\item Realizar limpieza de superficie y base de escritorio
	\item Eliminar excedente de humedad
\end{enumerate}

\paragraph{Limpieza de ventanas:}

\begin{enumerate}
	\item Preparar disolución a dosificación indicada por el proveedor
	\item Humedecer una toalla desechable
	\item Realizar limpieza de superficie
	\item Eliminar excedente de humedad
\end{enumerate}

\subsection{Responsables de la actividad}

\begin{itemize}
	\item \textbf{Ejecutado} por personal de limpieza;
	\item \textbf{Monitoreado} por personal de mantenimiento;
	\item \textbf{Verificado} por personal de Calidad.
\end{itemize}

\subsection{Acciones preventivas}

\begin{itemize}
	\item Se llevara a cabo una inspección diaria. Registrando el indicador y medida correspondiente
	\item Si la desviación se repite frecuentemente se dará curso de capacitación al personal de limpieza para que realicen eficientemente su trabajo.
	\item Si después de haber capacitado al personal de limpieza se siguen presentando desviaciones por causas injustificadas, se tomaran acciones más enérgicas con el personal por incumplimiento con sus deberes.
\end{itemize}

\subsection{Acciones correctivas}

\begin{itemize}
	\item La tarea es liberada una vez que el área se encuentran libre de polvo, suciedad, residuos de materiales extraños, residuos de jabón, y seca. En caso contrario la tarea se deberá volver a realizar como se indica en el procedimiento.
	\item En caso de no conformidad reportar en \RAC.
\end{itemize}

\subsection{Frecuencia}

Diario de lunes a sábado de 8:00 a 10:30.

\subsection{Historial de modificaciones}

\begin{itemize}
	\item Cuarta edición: cambio de fecha de 14 de noviembre 2017 a 28 de enero 2019, se realizó cambio de código de HS-P-LO a PR-032. Y no se realizaron cambios de la revisión 003 a la 004.
	\item Quinta edición: febrero 2020 se hizo cambio de formato y cambio de código de PR-032 a IT-OP-015. Y no se realizaron cambio de la revisión 04 a la 05.
	\item Sexta edición: febrero 2021 no se realizaron cambios de la revisión 05 a la 06.
	\item Séptima edición: febrero 2022 no se realizaron cambios de la revisión 06 a la 07.
\end{itemize}
