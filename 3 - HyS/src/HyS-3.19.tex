\renewcommand{\MenorVer}{0}
\renewcommand{\MayorVer}{2}
%\renewcommand{\Edit}{\MayorVer.\MenorVer}
\renewcommand{\Codigo}{HYS-19-IT}
\renewcommand{\FechaPub}{2023--01}
\renewcommand{\Titulo}{Verificación de eficacia de POES}

\section{\Titulo}
% \section{Verificación de eficacia de POES}

\subsection{Objetivo}

Establecer los parámetros con los cuales se mide la eficacia de los procedimientos de opresión estándar de sanitización con los cuales se toma la decisión si el procedimiento se realizó de manera adecuada.

\subsection{Alcance}

El alcance es aplicado para todos los procedimientos de operación estándar de sanitización.

\subsection{Terminología y definiciones}

\begin{description}
\defglo{limpieza}
\defglo{sanitizacion}
\end{description}

%\subsection{Documentos y/o normas relacionadas}

%N/A

\subsection{Procedimiento}

\subsubsection{Materiales}

\begin{itemize}
	\item Lámpara
\end{itemize}

\subsubsection{Instrucciones}

\paragraph{Revisión visual de los equipos y áreas posterior a la realización de los procedimientos de operación estándar de sanitización}

\begin{itemize}
	\item El área debe de estar libre de polvo.
	\item El área debe de estar libre de astillas de madera generadas por las tarimas.
	\item El área debe de estar libre de emplaye.
	\item El área debe estar libre de material orgánico o suciedad.
	\item El área debe de estar libre de cualquier material ajeno a la construcción del área o equipo (tornillos, clavos, vidrios, etc.)
\end{itemize}

\subsection{Responsables de la actividad}

\begin{itemize}
	\item \textbf{Verificado} por personal de Calidad.
\end{itemize}

\subsection{Acciones preventivas}

La acción preventiva se lleva a cabo al tiempo de realizar la verificación del cumplimiento de los procedimientos de operación estándar de sanitización.

\subsection{Acciones correctivas}

\begin{itemize}
	\item La tarea es liberada una vez que el área se encuentran libre de polvo, suciedad, residuos de materiales extraños, residuos de jabón, y seca. En caso contrario la tarea se deberá volver a realizar como se indica en el procedimiento.
	\item En caso de no conformidad reportar en \RAC.
\end{itemize}

\subsection{Frecuencia}

En cada verificación de tarea realizada

\subsection{Historial de modificaciones}

\begin{itemize}
	\item Primera Edición: cambio de formato.
	\item Segunda Edición: se realizó cambio de código de PV a PR-037 y no se realizaron cambios de la edición 01 a la 02.
	\item Tercera edición: febrero 2020 se hizo cambio de formato y cambio de código de PR-037 a IT-OP-020. Y no se realizaron cambio de la revisión 02 a la 03.
	\item Cuarta edición: febrero 2021 no se realizaron cambios de la revisión 03 a la 04.
	\item Quinta edición: febrero 2022 no se realizaron cambios de la revisión 04 a la 05.
\end{itemize}
