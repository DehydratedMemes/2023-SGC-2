\thispagestyle{formato-PI}
\renewcommand{\MenorVer}{0}
\renewcommand{\MayorVer}{2}
%\renewcommand{\Edit}{\MayorVer.\MenorVer}
\renewcommand{\Codigo}{HYS-24-IT}
\renewcommand{\FechaPub}{2023--01}
\renewcommand{\Titulo}{Cloración de agua potable}

\section{\Titulo}\index{Cloración!agua potable, de}
% \section{Cloración de agua potable}

\subsection{Objetivo}
Establecer las instrucciones de trabajo para la cloración del agua potable almacenada en el tinaco.

\subsection{Alcance}
\begin{itemize}
	\item Personal de mantenimiento;
	\item personal de aseguramiento de calidad.
\end{itemize}

\subsection{Terminología y definiciones}
\begin{description}
\defglo{cloración}
\end{description}

\subsection{Documentos y/o normas relacionadas}
\begin{itemize}
	\item Codex Alimentarius
	\item \BitClor
\end{itemize}

\subsection{Procedimiento}
\subsubsection{Precauciones de seguridad}
\begin{itemize}
	\item El indicador empleado para la determinación de cloro, 2-metilanilina, es cancerígeno. Se deben de seguir las indicaciones del MSDS y evitar la exposición directa a este compuesto.
\end{itemize}

\subsubsection{Materiales}
\begin{itemize}
	\item \BitClor.
	% \item Bitácora de verificación de cloración de agua potable. %TODO: ponerlo algun dia
\end{itemize}

\subsubsection{Instrucciones}

\begin{enumerate}
	\item Limpiar tapa del tinaco;
	\item Abrir tapa del tinaco y colocarla sobre una superficie limpia;
	\item Colocar las pastillas de cloro requeridas según las especificaciones del fabricante.
	\item Cerrar tinaco y limpiar nuevamente la tapa;
	\item Anotar en bitácora de cloración de agua.
\end{enumerate}

\subsection{Responsables de la actividad}

\begin{itemize}
	\item \textbf{Ejecutado} por personal de mantenimiento;
	\item \textbf{Monitoreado} por personal de calidad;
	\item \textbf{Verificado} por personal de calidad.
\end{itemize}

\subsection{Acciones preventivas}

\begin{itemize}
	\item Se llevará a cabo una determinación cualitativa del cloro libre presente en el suministro de agua empleando 2-metilanilina (2-amino-1-metilbenceno) dos veces al mes y se registrará la concentración de cloro en la \emph{bitácora de verificación de cloración de agua potable.}
\end{itemize}

\subsection{Acciones correctivas}

\begin{itemize}
	\item Si se detecta que la concentración de cloro es inferior al punto de vire de la 2-metilanilina, debe de reportarse en la \emph{bitácora de verificación de cloración de agua potable} y reportarse a la gerencia.
\end{itemize}

\subsection{Frecuencia}

\begin{itemize}
	\item Mensualmente.
	% \item[Verificación] mensualmente.
\end{itemize}

\subsection{Historial de modificaciones}

\begin{changelog}[title=Registro de cambios,simple, sectioncmd=\subsection*,label=changelog-\thesection-\MayorVer.\MenorVer4]
	\shortversion{v=1.0, author=Pablo E. Alanis, date=2022--12, changes=Primera edición}
\end{changelog}