\renewcommand{\MenorVer}{0}
\renewcommand{\MayorVer}{2}
%\renewcommand{\Edit}{\MayorVer.\MenorVer}
\renewcommand{\Codigo}{HYS-22-IT}
\renewcommand{\FechaPub}{2023--01}
\renewcommand{\Titulo}{Procedimiento en caso de derrame de químicos}

\section{\Titulo}
% \section{Procedimiento en caso de derrame de químicos}

\subsection{Objetivo}

Establecer un procedimiento de limpieza que asegure un mantenimiento y retiro apropiada de los productos químicos que han sufrido un derrame, y se realice una limpieza correcta para garantizar la eliminación de trazas de este, con el fin de evitar la contaminación cruzada.

\subsection{Alcance}

Áreas afectadas por derrame de producto químico.

\subsection{Terminología y definiciones}

\begin{itemize}
	\item \textbf{Limpieza:} Proceso de remover los residuos de alimento y suciedad que pueden ser una fuente de contaminación. Los métodos de limpieza adecuados y los materiales dependerán de la naturaleza del alimento.
	\item \textbf{Sanitización:} Proceso realizado después de efectuar la limpieza profunda en las áreas afectadas por derrames la cual nos permite el reducir el número de microorganismos
\end{itemize}

\subsection{Documentos y/o normas relacionadas}

\begin{itemize}
	\item Manual de Buenas Prácticas de Manufactura
	\item Manual de Higiene.
\end{itemize}

\subsection{Procedimiento}

\subsubsection{Instrucciones}

\begin{enumerate}
	\item Verificar la no afectación de ningún producto.
	\item Contener el derrame con el cilindro absorbente.
	\item Retirar el producto derramado con ayuda de las toallas absorbentes.
	\item Depositar en bolsa plástica
	\item Enjuagar el piso (Si fuera necesario)
	\item Eliminar la humedad con el jalador asignado
	\item Pedir al supervisor que realice una revisión visual
	\item Registrar evento en formato de derrame de químicos.
\end{enumerate}

\subsection{Responsables de la actividad}

\begin{itemize}
	\item \textbf{Ejecutado} por personal de limpieza.
\end{itemize}

\subsection{Acciones correctivas}

\begin{itemize}
	\item Cuando se presente una no conformidad en la realización del procedimiento, detectada por el supervisor o encargado del área deberá de repetir este mismo hasta que sea corregida la desviación.
	\item Se procederá según las indicaciones del departamento de calidad del cliente.
\end{itemize}

\subsection{Frecuencia}

\begin{itemize}
	\item Cada que ocurra un evenmto de derrame de químicos
\end{itemize}

\subsection{Historial de modificaciones}

\begin{itemize}
	\item Segunda edición: cambio de fecha de 24 de mayo 2017 a 28de enero 2019, se realizó cambio de código de CA-HS-DQ a PR-040. Y no se realizaron cambios de la revisión 001 a la 002.
	\item Tercera edición: febrero 2020 se hizo cambio de formato y cambio de código de PR-040 a PRO-OP-018. Y no se realizaron cambios de la revisión 02 a la 03.
	\item Cuarta edición: febrero 2021 no se realizaron cambios de la revisión 03 a la 04.
	\item Quinta edición: febrero 2022 no se realizaron cambios de la revisión 04 a la 05.
\end{itemize}
