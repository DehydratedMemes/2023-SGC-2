\thispagestyle{formato-PI}
\renewcommand{\MenorVer}{0}
\renewcommand{\MayorVer}{2}
%\renewcommand{\Edit}{\MayorVer.\MenorVer}
\renewcommand{\Codigo}{HYS-22-IT}
\renewcommand{\FechaPub}{2023--01}
\renewcommand{\Titulo}{Limpieza en caso de derrame de químicos}

\section{\Titulo}\index{Limpieza!derrame de químicos, en caso de}
% \section{Procedimiento en caso de derrame de químicos}

\subsection{Objetivo}
Establecer un procedimiento de limpieza que asegure un mantenimiento y retiro apropiada de los productos químicos que han sufrido un derrame, y se realice una limpieza correcta para garantizar la eliminación de trazas de este, con el fin de evitar la contaminación cruzada.

\subsection{Alcance}
Áreas afectadas por derrame de producto químico.

\subsection{Terminología y definiciones}
\begin{description}
\defglo{limpieza}
\defglo{sanitizacion}
\end{description}

\subsection{Documentos y/o normas relacionadas}
\begin{itemize}
	\item Manual de Buenas Prácticas de Manufactura
	\item Manual de Higiene.
\end{itemize}

\subsection{Procedimiento}
\subsubsection{Instrucciones}
\begin{enumerate}
	\item Verificar la no afectación de ningún producto.
	\item Contener el derrame con el cilindro absorbente.
	\item Retirar el producto derramado con ayuda de las toallas absorbentes.
	\item Depositar en bolsa plástica
	\item Enjuagar el piso (Si fuera necesario)
	\item Eliminar la humedad con el jalador asignado
	\item Pedir al supervisor que realice una revisión visual
	\item Registrar evento en formato de derrame de químicos.
\end{enumerate}

\subsection{Responsables de la actividad}
\begin{itemize}
	\item \textbf{Ejecutado} por personal de limpieza.
\end{itemize}

\subsection{Acciones correctivas}
\begin{itemize}
	\item Cuando se presente una no conformidad en la realización del procedimiento, detectada por el supervisor o encargado del área deberá de repetir este mismo hasta que sea corregida la desviación.
	\item Se procederá según las indicaciones del departamento de calidad del cliente.
\end{itemize}

\subsection{Frecuencia}
\begin{itemize}
	\item Cada que ocurra un evento de derrame de químicos
\end{itemize}

\begin{changelog}[simple, sectioncmd=\subsection*,label=changelog-3.20]
	\begin{version}[v=2.0, date=2023--01, author=Pablo E. Alanis]
		\item Cambio de formato;
		\item Cambios en la serialización de versiones;
		\item Correcciones ortográficas y de estilo.
	\end{version}

	\begin{version}[v=1.5, date=2022--05, author=Alonso M.]
		\item cambio de fecha;
	\end{version}	

	\shortversion{v=1.4, date=2021--05, changes=No hubo cambios}
\end{changelog}