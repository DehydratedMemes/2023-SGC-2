\thispagestyle{formato-PI}
\renewcommand{\MayorVer}{2}
\renewcommand{\MenorVer}{0}
%\renewcommand{\Edit}{\MayorVer.\MenorVer}
\renewcommand{\Titulo}{Limpieza de cámara de secos}
\renewcommand{\Codigo}{HYS-2-IT}
\renewcommand{\FechaPub}{2023--01}
% \renewcommand{\Titulo}{Politica de calidad}

\section{\Titulo}\index{Limpieza de!cámara!secos, de}

% \section{}

\subsection{Objetivo}

\begin{itemize}
	\item \textbf{Establecer} el procedimiento de limpieza que se debe de realizar en el almacén de secos, buscando con ello eliminar o minimizar la presencia de peligros físicos, químicos o biológicos que pudieran generar un riesgo de contaminación en el producto.
\end{itemize}

\subsection{Alcance}

Este procedimiento es aplicable a todos los almacenes de secos de \gls{RDF}.

\subsection{Terminología y definiciones}

\begin{description}
	\defglo{limpieza}
	\defglo{sanitizacion}
\end{description}

% \subsection{Documentos y/o normas relacionadas}

\subsection{Procedimiento}
\subsubsection{Materiales}
\begin{itemize}
	\item Preparación de Liquid-K a la concentración indicada por el proveedor (\qtyrange{2.6}{3.0}{\milli\liter\per\liter})
	\item Cubeta
	\item Cepillo para piso color rojo
	\item Cepillos manuales color rojo
	\item Recogedor de plástico
	\item Jalador color rojo
	\item Trapo de microfibra antibacteriana.
\end{itemize}

\subsubsection{Instrucciones}
\paragraph{Limpieza}
\begin{enumerate}
	\item Retire el producto del área donde se realizara la limpieza para evitar una posible contaminación durante el proceso de limpieza.
	\item Retire el material o suciedad visible como pudiera ser emplaye, madera, cartón, papel entre otros hasta que el área quede libre de estos, deposite los desechos en el contenedor de basura del área
	\item Utilizar detergente \textit{Liquid K} (\qtyrange{1.3}{2.6}{\milli\liter\per\liter}) para la limpieza de techos, muros, \textit{racks} y pisos.
	\item La limpieza del área se debe realizar en el siguiente orden: techos, muros, \textit{racks} y pisos
	\item Primero realice la limpieza de techos y paredes, aplique el detergente \textit{Liquid K} directamente de manera superficial con un trapo de microfibra, deje el producto aplicado durante \qtyrange{10}{20}{\minute} aproximadamente para que genere su acción.
	\item Con otro trapo de microfibra retire el excedente de detergente, realice esta maniobra hasta finalizar el área seleccionada para la limpieza.
	\item Elimine del trapo de microfibra el excedente de detergente y continúe eliminando el excedente hasta finalizar de retirar el detergente del área.
	\item Continúe con la limpieza de los \textit{racks}, aplique el detergente \textit{Liquid K} de manera superficial con un trapo de microfibra, deje el producto aplicado durante \qtyrange{10}{20}{\minute} aproximadamente para que genere su acción
	\item Con otro trapo de microfibra retire el excedente de detergente, realice esta maniobra hasta finalizar el área seleccionada para la limpieza.
	\item Elimine del trapo de microfibra el excedente de detergente y continúe eliminando el excedente hasta finalizar de retirar el detergente del área
	\item Finalice con la limpieza de los pisos y zoclos, aplique el detergente \textit{Liquid K} en el piso con ayuda de la escoba para piso, deje el producto aplicado durante \qtyrange{10}{20}{\minute} aproximadamente para que genere su acción.
	\item Talle con la escoba el piso y zoclos.
	\item Con el jalador para pisos retire el excedente de detergente hasta que el área quede libre del mismo.
	\item Una vez que se encuentren libres de detergente las áreas limpiadas
	\item Registre en el Programa Maestro de Limpieza y Sanitización (\RAC. la actividad realizada.
	\item Finalmente, notifique al Supervisor de Calidad para la verificación y liberación de la tarea.
	\item La tarea es liberada una vez que el área se encuentran libre de polvo, suciedad, residuos de materiales extraños, residuos de jabón, y seca. En caso contrario la tarea se deberá volver a realizar.
\end{enumerate}

\subsection{Responsables de la actividad}
\begin{itemize}
	\item \textbf{Ejecutado} por personal de limpieza
	\item \textbf{Monitoreado} por personal de mantenimiento
	\item \textbf{Verificado} por personal de Calidad.
\end{itemize}

\subsection{Acciones preventivas}
\begin{enumerate}
	\item La tarea es liberada una vez que el área se encuentran libre de polvo, suciedad, residuos de materiales extraños, residuos de jabón, y seca. En caso contrario la tarea se deberá volver a realizar como se indica en el procedimiento.
	\item En caso de no conformidad reportar en \RAC.
\end{enumerate}

\subsection{Frecuencia}
\begin{itemize}
	\item[\textbf{Superficial:}] diario de lunes a viernes de 8:00 a 9:00, sábados de 9:00 a 10:00.
	\item[\textbf{Profunda:}] Anualmente, de acuerdo con el programa y disponibilidad de las cámaras.
\end{itemize}

\begin{changelog}[simple, sectioncmd=\subsection*,label=changelog-3.2]
	\begin{version}[v=2.0, date=2023--01, author=Pablo E. Alanis]
		\item Cambio de formato;
		\item Cambios en la serialización de versiones;
		\item Correcciones ortográficas y de estilo.
	\end{version}

	\begin{version}[v=1.5, date=2022--05, author=Alonso M.]
		\item cambio de fecha;
	\end{version}

	\shortversion{v=1.4, date=2021--05, changes=No hubo cambios}
\end{changelog}