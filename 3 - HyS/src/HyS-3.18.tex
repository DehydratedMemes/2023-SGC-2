\thispagestyle{formato-PI}
\renewcommand{\MenorVer}{0}
\renewcommand{\MayorVer}{2}
%\renewcommand{\Edit}{\MayorVer.\MenorVer}
\renewcommand{\Codigo}{HYS-18-IT}
\renewcommand{\FechaPub}{2023--01}
\renewcommand{\Titulo}{Limpieza de uniformes}

\section{\Titulo}\index{Limpieza!uniformes, de}
% \section{Limpieza de uniformes}

\subsection{Objetivo}
Asegurar el lavado de los uniformes que se utilizan en el almacén.

\subsection{Alcance}
Uniformes de trabajo.

\subsection{Terminología y definiciones}
\begin{description}
\defglo{limpieza}
\defglo{sanitizacion}
\end{description}

\subsection{Procedimiento}
\subsubsection{Materiales}
\begin{itemize}
	\item Bitácora de registros.
\end{itemize}

\subsubsection{Instrucciones}
\paragraph{Limpieza de uniformes}
\begin{enumerate}
	\item El personal dejara los uniformes con el personal de recepción o vigilancia firmara de entrega de uniforme, cantidad y la fecha.
	\item El personal entregara los uniformes al personal de lavandería el cual pasara a las instalaciones.
	\item La Empresa Externa (Servicio de Lavado) es responsable de lo siguiente:
	\item Brindar el servicio de lavado de buena calidad de la empresa \gls{RDF}.
	\item Verificar condiciones y cantidades de uniformes que la empresa le hace entrega, así como empleado, departamento y firma de entregado y recibido.
	\item Entregar el uniforme lavado en las fechas pactadas.
	\item Entregar las notas de remisión correspondientes a cada entrega realizada.
	\item Se entregara uniformes al personal y este firmara de acuerdo de recibido.
\end{enumerate}

\subsection{Responsables de la actividad}

\begin{itemize}
	\item \textbf{Ejecutado} por personal de limpieza;
	\item \textbf{Monitoreado} por personal de mantenimiento;
	\item \textbf{Verificado} por personal de Calidad.
\end{itemize}

\subsection{Acciones preventivas}

\begin{itemize}
	\item Se llevará a cabo una inspección diaria. Registrando el indicador y medida correspondiente
	\item Si la desviación se repite frecuentemente se dará curso de capacitación al personal de limpieza para que realicen eficientemente su trabajo.
	\item Si después de haber capacitado al personal de limpieza se siguen presentando desviaciones por causas injustificadas, se tomaran acciones más enérgicas con el personal por incumplimiento con sus deberes.
\end{itemize}

\subsection{Acciones correctivas}

\begin{itemize}
	\item Cuando se presente una desviación en la realización del procedimiento detectada por el personal se reportara a Gerencia.
\end{itemize}

\subsection{Frecuencia}

\begin{itemize}
	\item Quincenalmente.
\end{itemize}

\begin{changelog}[simple, sectioncmd=\subsection*,label=changelog-\thesection-\MayorVer.\MenorVer8]
	\begin{version}[v=\MayorVer.\MenorVer, date=2023--01, author=Pablo E. Alanis]
		\item Cambio de formato;
		\item Cambios en la serialización de versiones;
		\item Correcciones ortográficas y de estilo.
	\end{version}

	\begin{version}[v=1.7, date=2022--05, author=Alonso M.]
		\item cambio de fecha;
	\end{version}	

	\shortversion{v=1.6, date=2021--05, changes=No hubo cambios}
\end{changelog}