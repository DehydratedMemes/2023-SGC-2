\renewcommand{\MenorVer}{0}
%\renewcommand{\Edit}{\MayorVer.\MenorVer}
\renewcommand{\Codigo}{HYS-12-IT}
\renewcommand{\FechaPub}{2023--01}
\renewcommand{\Titulo}{Lavado de accesorios de limpieza}

\section{\Titulo}
% \section{Lavado de accesorios de limpieza}

\subsection{Objetivo}

Establecer procedimiento que aseguren un mantenimiento y limpieza apropiada de utensilios de limpieza.

\subsection{Alcance}

Utensilios de limpieza del área de almacén.

\subsection{Terminología y definiciones}

\begin{description}
\defglo{limpieza}
\defglo{sanitizacion}
\end{description}

%\subsection{Documentos y/o normas relacionadas}

%N/A

\subsection{Procedimiento}

\subsubsection{Materiales}

\begin{itemize}
	\item Preparación de Liquid K a la concentración indicada por el proveedor (0.13\% a 0.26\% v/v)(1.3ml a 2.6ml en 1L de agua)
	\item Cubeta de \qty{20}{\litre}
\end{itemize}

\subsubsection{Instrucciones}

\paragraph{Operativo}

\subparagraph{Limpieza de accesorios de limpieza}

\begin{enumerate}
	\item Enjuagar los utensilios como, cepillo de mano, recogedores, cubetas, jaladores que se utilizaron en el transcurso del día.
	\item Sumergir en una cubeta con agua y detergente
	\item Enjuagarlos perfectamente
	\item Elimine el excedente de humedad
	\item Colocarlas en su lugar correspondiente para terminar de secar.
	\item Lavar cepillos, jaladores y recogedores de la parte de arriba hacia abajo y enjuagar perfectamente.
	\item Después enjuagar y colocarlos en el lugar correspondiente.
	\item Enjuague perfectamente cubetas hasta eliminar residuos de suciedad.
	\item De igual manera elimine el excedente de humedad.
\textbf{NOTA:} NO permita la acumulación de agua en cubetas o utensilios de limpieza
\textbf{NOTA:} La limpieza de los utensilios se realiza de acuerdo a su mismo código de color.
\end{enumerate}

\subsection{Responsables de la actividad}

\begin{itemize}
	\item \textbf{Ejecutado} por personal de limpieza;
	\item \textbf{Monitoreado} por personal de mantenimiento;
	\item \textbf{Verificado} por personal de Calidad.
\end{itemize}

\subsection{Acciones preventivas}

\begin{itemize}
	\item Se llevara a cabo una inspección diaria. Registrando el indicador y medida correspondiente
	\item Si la desviación se repite frecuentemente se dará curso de capacitación al personal de limpieza para que realicen eficientemente su trabajo.
	\item Si después de haber capacitado al personal de limpieza se siguen presentando desviaciones por causas injustificadas, se tomaran acciones más enérgicas con el personal por incumplimiento con sus deberes.
\end{itemize}

\subsection{Acciones correctivas}

\begin{itemize}
	\item La tarea es liberada una vez que el área se encuentran libre de polvo, suciedad, residuos de materiales extraños, residuos de jabón, y seca. En caso contrario la tarea se deberá volver a realizar como se indica en el procedimiento.
	\item En caso de no conformidad reportar en \RAC.
\end{itemize}

\subsection{Frecuencia}

\begin{itemize}
	\item Diario de lunes a viernes de 16:00 a 16:30 y sábados de 11:30 a 12:00.
\end{itemize}

\subsection{Historial de modificaciones}

\begin{itemize}
	\item Cuarta edición: cambio de fecha de 24 de mayo 2017 a 28 de enero 2019, se realizó cambio de código de HS-P-LAL a PR-030. Y no se realizaron cambios de la revisión 003 a la 004.
	\item Quinta edición: febrero 2020 se hizo cambio de formato y cambio de código de PR-030 a IT-OP-013. Y no se realizaron cambios de revisión 04 a la 05.
	\item Sexta edición: febrero 2021 no se realizaron cambios de revisión 05 a la 06.
	\item Séptima edición: febrero 2022 no se realizaron cambios de la revisión 06 a la 07.
\end{itemize}
