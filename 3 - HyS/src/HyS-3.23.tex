\renewcommand{\MenorVer}{0}
\renewcommand{\MayorVer}{2}
%\renewcommand{\Edit}{\MayorVer.\MenorVer}
\renewcommand{\Codigo}{HYS-23-IT}
\renewcommand{\FechaPub}{2023--01}
\renewcommand{\Titulo}{Limpieza y desinfección de cisterna}

\section{\Titulo}
% \section{Limpieza y desinfección de cisterna}

\subsection{Objetivo}

Establecer un método para la correcta limpieza y desinfección de cisterna.

\subsection{Alcance}

Aplica a cisternas de agua.

\subsection{Terminología y definiciones}

\begin{itemize}
	\item \textbf{cisterna} es un depósito subterráneo que se utiliza para recoger y guardar agua. También se denomina así a los receptáculos usados para contener líquidos, generalmente agua, y a los vehículos que los transportan. En algunos lugares se denomina también tinaco. Su capacidad va desde unos litros a miles de metros cúbicos.
	\begin{itemize}
		\item Como almacenan agua para uso o consumo humano, es rigurosamente indispensable realizar la limpieza de las cisternas con regularidad.
	\end{itemize}
\end{itemize}

\subsection{Procedimiento}

N/A

\begin{itemize}
	\item \textbf{AVISO: Red de Fríos S.A. de C.V. NO CUENTA CON CISTERNA DE AGUA dentro de sus operaciones.}
	\item \textbf{AVISO: Red de Fríos S.A. de C.V. NO realiza actividades de LAVADO Y DESINFECCION DE CISTERNA DE AGUA como práctica común dentro de sus procedimientos durante la operación.}
	\item \textbf{AVISO: Red de Fríos S.A. de C.V. NO cuenta con NINGUN tipo de almacenamiento de Agua dentro de las instalaciones del almacén.}
	\item \textbf{\gls{RDF} utiliza agua suministrada por la red municipal de Monterrey, esta agua no tiene contacto directo en ningún momento con el producto.}
\end{itemize}
