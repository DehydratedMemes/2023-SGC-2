\renewcommand{\MenorVer}{0}
\renewcommand{\MayorVer}{2}
%\renewcommand{\Edit}{\MayorVer.\MenorVer}
\renewcommand{\Codigo}{HYS-11-IT}
\renewcommand{\FechaPub}{2023--01}
\renewcommand{\Titulo}{Lavado de patio de montacargas}

\section{\Titulo}\index{Lavado!patio de montacargas, de}
% \section{Lavado de área de montacargas}

\subsection{Objetivo}
Establecer procedimiento de limpieza del área de montacargas.

\subsection{Alcance}
Área de carga de montacargas

\subsection{Terminología y definiciones}
\begin{itemize}
	\item \textbf{Limpieza} Proceso de remover los residuos Orgánicos y suciedad que pueden ser una fuente de contaminación. Los métodos de limpieza adecuados y los materiales dependerán de la naturaleza del alimento.
	\item \textbf{Sanitización} Proceso realizado después de efectuar la limpieza profunda de equipos, maquinaria e instalaciones la cual nos permite el reducir el número de microorganismos.
\end{itemize}

%\subsection{Documentos y/o normas relacionadas}

%N/A

\subsection{Procedimiento}

\subsubsection{Materiales}

\begin{itemize}
	\item Preparación de Liquid K a la concentración indicada por el proveedor (0.13\% a 0.26\% v/v)(1.3ml a 2.6ml en 1L de agua)
	\item Cubeta de \qty{20}{\litre}
	\item Cepillo para piso color rojo
	\item Cepillos manuales color rojo
	\item Recogedor de plástico
	\item Jalador color rojo
	\item Toalla desechable
\end{itemize}

\subsubsection{Instrucciones}

\paragraph{Preoperativo}

\subparagraph{Limpieza de paredes y techos}

\begin{enumerate}
	\item Preparar el área que se va a lavar (retirando la herramienta que se encuentre en el área)
	\item Prepare detergente a la concentración indicada por el proveedor
	\item Aplicar el detergente
	\item Tallar con el cepillo empezando por techos, paredes, dejar reposar 2 a 5 min sin permitir que la espuma se seque
	\item Retire el excedente de la solución iniciando de la parte más alta finalizando en la parte más baja.
	\item Aplicar detergente en el piso
	\item Tallar con cepillo
	\item Retire exceso de humedad con el jalador para piso
\end{enumerate}

\subsection{Responsables de la actividad}

\begin{enumerate}
	\item Ejecutado por personal de limpieza
	\item Monitoreado por personal de mantenimiento
	\item Verificado por personal de Calidad.
\end{enumerate}

\subsection{Acciones preventivas}

\begin{itemize}
	\item Se llevara a cabo una inspección diaria. Registrando el indicador y medida correspondiente
	\item Si la desviación se repite frecuentemente se dará curso de capacitación al personal de limpieza para que realicen eficientemente su trabajo.
	\item Si después de haber capacitado al personal de limpieza se siguen presentando desviaciones por causas injustificadas, se tomaran acciones más enérgicas con el personal por incumplimiento con sus deberes.
\end{itemize}

\subsection{Acciones correctivas}

\begin{itemize}
	\item La tarea es liberada una vez que el área se encuentran libre de polvo, suciedad, residuos de materiales extraños, residuos de jabón, y seca. En caso contrario la tarea se deberá volver a realizar como se indica en el procedimiento.
	\item En caso de no conformidad reportar en \RAC.
\end{itemize}

\subsection{Frecuencia}

\begin{itemize}
	\item Diario de lunes a viernes de 13:30pm a 14:30pm y sábados de 10:30am a 11:30am / Trimestral revisión del programa general de limpieza.
\end{itemize}

\subsection{Historial de modificaciones}

\begin{itemize}
	\item Cuarta edición: cambio de fecha de 24 de mayo 2017 a 28 de enero 2019, se realizó cambio de código de HS-P-LAM a PR-029 Y no se realizaron cambios de la revisión 003 a la 004.
	\item Quinta edición: febrero 2020 se hizo cambio de formato y cambio de código de PR-029 a IT-OP-012. Y no se realizaron cambios de la revisión 04 a la 05.
	\item Sexta edición: febrero 2021 no se realizaron cambios de la revisión 05 a la 06.
	\item Séptima edición: febrero 2022 no se realizaron cambios de la revisión 06 a la 07.
\end{itemize}
