\thispagestyle{formato-PI}
\renewcommand{\MenorVer}{0}
\renewcommand{\MayorVer}{2}
%\renewcommand{\Edit}{\MayorVer.\MenorVer}
\renewcommand{\Codigo}{HYS-20-IT}
\renewcommand{\FechaPub}{2023--01}
\renewcommand{\Titulo}{Limpieza de manos}

\section{\Titulo}\index{Limpieza!manos, de}
% \section{Limpieza de manos}

\subsection{Objetivo}
\begin{itemize}
	\item Establecer el procedimiento del correcto lavado de manos.
\end{itemize}

\subsection{Alcance}
\begin{itemize}
	\item Todo personal y visitante de \gls{RDF}.
\end{itemize}

\subsection{Terminología y definiciones}
\begin{description}
\defglo{limpieza}
\defglo{sanitizacion}
\end{description}

\subsection{Procedimiento}
\subsubsection{Materiales}
\begin{enumerate}
	\item Jabón para manos.
	\item Toallas desechables
	\item desinfectante de manos
\end{enumerate}

\subsubsection{Instrucciones}

\begin{enumerate}
	\item Moje sus manos con agua. En algunos casos deberá accionar el flujo de agua con la con su rodilla o pie.
	\item Coloque suficiente jabón para cubrir toda la superficie de manos y dedos.
	\item Con movimientos circulares frote las palmas de su mano contra la otra.
	\item Con movimientos ascendentes frote el dorso de ambas manos y entre los dedos.
	\item Entrelace sus dedos y con movimientos ascendentes y descendentes talle entre sus dedos.
	\item Entrelace sus dedos y frote para lavar la parte frontal de sus dedos.
	\item Con la ayuda de sus palmas y movimientos giratorios, talle sus dedos pulgares.
	\item Sobre la palma de sus manos, frote la punta de los dedos de la mano contraria.
	\item Enjuague con agua hasta eliminar por completo el jabón de sus manos.
	\item Con ayuda de toallas desechables o secador de aire caliente elimine el agua de sus manos.
	\item Coloque suficiente gel desinfectante para cubrir manos y dedos, frote hasta que este se evapore por completo.
	\item Sus manos se encuentran limpias para poder ingresar o regresar a sus áreas de trabajo.
\end{enumerate}

\subsection{Responsables de la actividad}

\begin{itemize}
	\item \textbf{Ejecutado} por todo el personal y visitantes.
\end{itemize}

\subsection{Acciones preventivas}
\begin{itemize}
	\item La acción preventiva se lleva a cabo al tiempo de realizar la verificación del cumplimiento de los procedimientos de operación estándar de sanitización.
\end{itemize}

\subsection{Acciones correctivas}

\begin{itemize}
	\item La tarea es liberada una vez que el área se encuentran libre de polvo, suciedad, residuos de materiales extraños, residuos de jabón, y seca. En caso contrario la tarea se deberá volver a realizar como se indica en el procedimiento;
	\item En caso de no conformidad reportar en \RAC.
\end{itemize}

\subsection{Frecuencia}

\begin{itemize}
	\item Diaria, antes de ingresar al almacén, en el cambio de actividades, después de ir al baño, antes de comidas.
\end{itemize}

\begin{changelog}[title=Registro de cambios,simple, sectioncmd=\subsection*,label=changelog-\thesection-\MayorVer.\MenorVer0]
	\begin{version}[v=\MayorVer.\MenorVer, date=2023--01, author=Pablo E. Alanis]
		\item Cambio de formato;
		\item Cambios en la serialización de versiones;
		\item Correcciones ortográficas y de estilo.
	\end{version}

	\begin{version}[v=1.5, date=2022--05, author=Alonso M.]
		\item cambio de fecha;
	\end{version}	

	\shortversion{v=1.4, date=2021--05, changes=No hubo cambios}
\end{changelog}