\thispagestyle{formato-PI}
\renewcommand{\MayorVer}{2}
\renewcommand{\MenorVer}{0}
%\renewcommand{\Edit}{\MayorVer.\MenorVer}
\renewcommand{\Titulo}{Limpieza de artículos de andén}
\renewcommand{\Codigo}{HYS-6-IT}
\renewcommand{\FechaPub}{2023--01}

\section{\Titulo}\index{Limpieza!artículos del andén, de}
% \section{Limpieza de artículos de andén}

\subsection{Objetivo}

Establecer un procedimiento que aseguren un mantenimiento y limpieza apropiada de los artículos de andén.

\subsection{Alcance}

Mesa de inspección y lavamanos.

\subsection{Terminología y definiciones}

\begin{description}
\defglo{limpieza}
\defglo{sanitizacion}
\end{description}

\subsection{Procedimiento}

\subsubsection{Materiales}

\begin{itemize}
	\item Preparación de Liquid K a la concentración indicada por el proveedor \qtyrange{0.13}{0.26}{\percent} o bien \qtyrange{1.3}{2.6}{\milli\liter\per\liter}
	\item Cubeta de \qty{20}{\litre}
	\item Cepillos manuales color rojo
	\item Toallas desechables
\end{itemize}

\subsubsection{Instrucciones}
\paragraph{Mesa de trabajo}
\subparagraph{Preoperativo}
Cuando la mesa y lavamanos no sea utilizada Se llevará a cabo la \emph{limpieza en seco,} que consiste en rociar desinfectante con el aspersor manual en la superficie de la mesa y lavamanos y con una toalla desechable frotar hasta secar el mismo.

\subsection{Mesa después de inspección}

\begin{enumerate}
	\item En caso de que se llegara a presentar derrame de residuos orgánicos (líquido y/o pedazos de algún producto cárnico), se recoge los residuos y se depositan en el contenedor.
	\item Pre enjuagar la mesa y lavamanos, se procede a aplicar el desengrasante en el área específica a la concentración indicada por el proveedor.
	\item Tallando con el cepillo iniciando de la parte superior hasta la inferior.
	\item Enjuague con agua hasta eliminar el detergente.
	\item Secar con toalla desechable.
	\item Desinfectar con la ayuda de un Atomizador rociando la Mesa
	\item Frotar la Mesa con una toalla desechable hasta que quede seca
\end{enumerate}

\subsection{Responsables de la actividad}

\begin{itemize}
	\item \textbf{Ejecutado} por personal de limpieza
	\item \textbf{Monitoreado} por personal de mantenimiento
	\item \textbf{Verificado} por personal de Calidad.
\end{itemize}

\subsection{Acciones preventivas}

\begin{itemize}
	\item Se llevará a cabo una inspección diaria. Registrando el indicador y medida correspondiente
	\item Si la desviación se repite frecuentemente se dará curso de capacitación al personal de limpieza para que realicen eficientemente su trabajo.
	\item Si después de haber capacitado al personal de limpieza se siguen presentando desviaciones por causas injustificadas, se tomaran acciones más enérgicas con el personal por incumplimiento con sus deberes.
\end{itemize}

\subsection{Acciones correctivas}

\begin{enumerate}
	\item La tarea es liberada una vez que el área se encuentran libre de polvo, suciedad, residuos de materiales extraños, residuos de jabón, y seca. En caso contrario la tarea se deberá volver a realizar como se indica en el procedimiento.
	\item En caso de no conformidad reportar en \RAC.
\end{enumerate}

\subsection{Frecuencia}

\begin{itemize}
	\item Diaria de lunes a viernes de 16:00 a 16:30 y sábados de 12:00 a 12:30.
\end{itemize}

\begin{changelog}[title=Registro de cambios,simple, sectioncmd=\subsection*,label=changelog-\thesection-\MayorVer.\MenorVer]
	\begin{version}[v=\MayorVer.\MenorVer, date=2023--01, author=Pablo E. Alanis]
		\item Cambio de formato;
		\item Cambios en la serialización de versiones;
		\item Correcciones ortográficas y de estilo.
	\end{version}

	\begin{version}[v=1.7, date=2022--05, author=Alonso M.]
		\item cambio de fecha;
	\end{version}

	\shortversion{v=1.6, date=2021--05, changes=No hubo cambios}
\end{changelog}