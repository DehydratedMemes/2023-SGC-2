\thispagestyle{formato-PI}
\renewcommand{\MayorVer}{2}
\renewcommand{\MenorVer}{0}
%\renewcommand{\Edit}{\MayorVer.\MenorVer}
\renewcommand{\Codigo}{HYS-8-IT}
\renewcommand{\FechaPub}{2023--01}
\renewcommand{\Titulo}{Preparación y aplicación de solución desinfectante para unidades de transporte}

\section{\Titulo}
% \section{Preparación y aplicación de solución desinfectante para unidades de transporte}

\subsection{Objetivo}

\begin{itemize}
	\item Establecer un procedimiento para realizar la preparación de la solución para el desinfectado de las unidades de transporte.
\end{itemize}

\subsection{Alcance}

\begin{itemize}
	\item Solución desinfectante.
\end{itemize}

\subsection{Terminología y definiciones}

\begin{description}
	\defglo{limpieza}
	\defglo{sanitizacion}
\end{description}


\subsection{Procedimiento}

\subsubsection{Materiales}

\begin{itemize}
	\item Preparación de \emph{WHISPER V} a la concentración indicada por el proveedor \qty{0.2}{\percent} o bien (\qty{2}{\milli\liter\per\liter}).
	\item Bote aspersor.
	\item Probeta dosificadora.
\end{itemize}

\subsubsection{Instrucciones}
\paragraph{Desinfección}
\begin{enumerate}
	\item Se procede al preparado de la solución desinfectante.
	\item Agregar en \qty{7}{\liter} de agua dentro del aspersor y posteriormente se adiciona Whisper V \qty{14}{\milli\liter} para alcanzar una concentración de \qty{150}{\milli\gram\per\liter}. Con tiempo de contacto de \qty{60}{\second}.
	\item Se verifica que la unidad venga limpia y libre de cualquier sólido.
	\item Se aplica el producto desinfectante directamente al interior de la unidad (caja) antes de embarcar producto, aplicándolo de la parte más alta a la parte más baja de la caja y de adentro hacia afuera de la unidad.
	\item Se llevará a cabo una inspección diaria. Registrando el indicador y medida correspondiente
	\item Si la desviación se repite frecuentemente se dará curso de capacitación al personal de almacén para que realice eficientemente su trabajo.
	\item Si después de haber capacitado al personal de almacén se siguen presentando desviaciones por causas injustificadas, se tomaran acciones más enérgicas con el personal por incumplimiento con sus deberes.
\end{enumerate}

\subsection{Responsables de la actividad}
\begin{itemize}
	\item \textbf{Ejecutado} por personal de limpieza;
	\item \textbf{Monitoreado} por personal de mantenimiento;
	\item \textbf{Verificado} por personal de Calidad.
\end{itemize}

\subsection{Acciones correctivas}
\begin{itemize}
	\item La tarea es liberada una vez que el área se encuentran libre de polvo, suciedad, residuos de materiales extraños, residuos de jabón, y seca. En caso contrario la tarea se deberá volver a realizar como se indica en el procedimiento.
	\item En caso de no conformidad reportar en \RAC.
\end{itemize}

\subsection{Frecuencia}
\begin{itemize}
	\item Diario de lunes a sábado.
\end{itemize}

\begin{changelog}[simple, sectioncmd=\subsection*,label=changelog-\thesection-\MayorVer.\MenorVer]
	\begin{version}[v=\MayorVer.\MenorVer, date=2023--01, author=Pablo E. Alanis]
		\item Cambio de formato;
		\item Cambios en la serialización de versiones;
		\item Correcciones ortográficas y de estilo.
	\end{version}

	\begin{version}[v=1.7, date=2022--05, author=Alonso M.]
		\item cambio de fecha;
	\end{version}

	\shortversion{v=1.6, date=2021--05, changes=No hubo cambios}
\end{changelog}