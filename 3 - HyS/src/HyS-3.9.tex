\thispagestyle{formato-PI}
\renewcommand{\MenorVer}{0}
\renewcommand{\MayorVer}{2}
%\renewcommand{\Edit}{\MayorVer.\MenorVer}
\renewcommand{\Codigo}{HYS-9-IT}
\renewcommand{\FechaPub}{2023--01}
\renewcommand{\Titulo}{Preparación de solución desinfectante}

\section{\Titulo}\index{Preparación de soluciones!desinfectante}

\subsection{Objetivo}
\begin{itemize}
	\item Establecer un procedimiento para realizar la preparación de la solución desinfectante del tapete sanitario.
\end{itemize}

\subsection{Alcance}
\begin{itemize}
	\item Tapete sanitario de aduana de ingreso.
\end{itemize}

\subsection{Terminología y definiciones}

\begin{description}
\defglo{limpieza}
\defglo{sanitizacion}
\end{description}

\subsection{Procedimiento}

\subsubsection{Materiales}
\begin{itemize}
	\item Preparación de \emph{WHISPER V} a la concentración indicada por el proveedor \qty{1.00}{\percent} o bien \qty{10}{\milli\liter\per\liter}.
	\item Cubeta de \qty{20}{\litre}.
	\item Tapete sanitario.
\end{itemize}

\subsubsection{Instrucciones}

\paragraph{Limpieza y desinfección}

\begin{enumerate}
	\item Después del lavado del tapete sanitario se procede al preparado de la solución desinfectante para tapete sanitario (De aduana Sanitaria).
	\item Agregar en \qty{20}{\litre} de agua dentro del tapete sanitario y posteriormente se adiciona Whisper V (\qty{200}{\milli\liter}) para alcanzar una concentración de \qtyrange{400}{1200}{\milli\gram\per\liter}.
	\item Realizar verificaciones de concentración 1 vez por día para garantizar que la solución preparada se encuentra dentro de los valores establecidos.
\end{enumerate}

\subsection{Acciones preventivas}

\begin{itemize}
	\item La tarea es liberada una vez que el área se encuentre libre de polvo, suciedad, residuos de materiales extraños, residuos de jabón, y seca. En caso contrario la tarea se deberá volver a realizar.
	\item Se llevará a cabo una inspección diaria. Registrando el indicador y medida correspondiente
	\item Si la desviación se repite frecuentemente se dará curso de capacitación al personal de limpieza para que realicen eficientemente su trabajo.
	\item Si después de haber capacitado al personal de limpieza se siguen presentando desviaciones por causas injustificadas, se tomaran acciones más enérgicas con el personal por incumplimiento con sus deberes.
\end{itemize}

\subsection{Acciones correctivas}
\begin{itemize}
	\item Cuando se presente una no conformidad en la realización del procedimiento, detectada por el supervisor o encargado del área se deberá de repetir este mismo hasta que sea corregida la desviación
	\item Registrar en formato de acciones correctivas correspondientes.
\end{itemize}

\subsection{Frecuencia}
\begin{itemize}
	\item Diario de lunes a sábado de 08:00 a 08:15.
\end{itemize}

\begin{changelog}[title=Registro de cambios,simple, sectioncmd=\subsection*,label=changelog-\thesection-\MayorVer.\MenorVer]
	\begin{version}[v=\MayorVer.\MenorVer, date=2023--01, author=Pablo E. Alanis]
		\item Cambio de formato;
		\item Cambios en la serialización de versiones;
		\item Correcciones ortográficas y de estilo.
	\end{version}

	\begin{version}[v=1.7, date=2022--05, author=Alonso M.]
		\item cambio de fecha;
	\end{version}

	\shortversion{v=1.6, date=2021--05, changes=No hubo cambios}
\end{changelog}