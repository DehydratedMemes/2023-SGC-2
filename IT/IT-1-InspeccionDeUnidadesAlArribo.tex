%\paragraph{Inspección física del alimento}\index{Instrucción!inspección fisica de los alimentos}
\begin{enumerate}
    \item A la llegada del alimento, se verifica la documentación del mismo.\ \emph{Si se descubre la documentación no es la adecuada,} no se puede descargar el alimento y el \emph{Supervisor de Almacén} debe ser contactado inmediatamente para obtener instrucciones adicionales.
          \begin{enumerate}
              \item Documentos TIF;\footnote{En caso de ser alimentos de procedencia TIF.}
              \item Listado de alimentos con cantidades;
              \item Especificaciones de almacenamiento del alimento.
          \end{enumerate}
    \item Se verifica la \emph{temperatura programada} de la unidad;
    \item Se revisa que el número de sello coincida con la guía.\
        \begin{itemize}
            \item \emph{Si se descubre que el sello está roto a la llegada,} no se puede descargar el alimento y el \emph{Supervisor de Almacén} debe ser contactado inmediatamente para obtener instrucciones adicionales;
        \end{itemize}
    \item Se \textit{enrampa} la unidad para descargar el alimento;
    \item La condición física del alimento (caja, cubetas, etc.) debe ser inspeccionada, cualquier daño debe ser documentado en el registro de \textit{\Oent;}
    \item Se inspecciona la unidad: Se debe de revisar:
            \begin{itemize}
              \item Limpieza (presencia de suciedad, escombros, basura, etc.);
              \item Evidencia de cualquier actividad de insectos (excrementos, etc.);
              \item Daños físicos (bolsas rotas, cajas dañadas, fugas, etc.);
              \item Daño físico a la unidad que comprometa la inocuidad del alimento;
              \item Evidencia de violación de los materiales de empaque (caja o alimento abierto);
              \item Presencia visible de agentes químicos.\footnote{También es importante reportar olores químicos, como olor a solvente o a plaguicida excesivos.}
          \end{itemize}
    \item Verificar la temperatura del alimento.\footnote{Todo alimento refrigerado o congelado recibido en la instalación debe ser revisado para asegurar que fue enviado de origen a la temperatura adecuada, verificando la documentación correspondiente.}
    \item Se revisarán todos los documentos de envío \emph{(números de lote, fechas, número de orden, etc.)} y la información requerida deberá ser registrada en el formulario \Oent;
    \item Se comprobará que si cumple con los criterios de acción para la aceptación (ver \cref{criterios:aceptacion})
    \item si es conforme la temperatura, se procederá a almacenar el alimento en la camara designada según sus especificaciones.
          \begin{itemize}
              \item si el \gls{alimento} no requiere de condiciones de almacenamiento especiales\footnote{diferentes a las especificaciones genéricas de almacenamiento (ver \cref{esp:generica}).} se almacenará en la cámara adecuada.
              \item si el \gls{alimento} requiere de condiciones de almacenamiento especiales y el cliente alquiló una camara exclusiva para su alimento, entonces se almacenará ahí;
              \item de no ser así, se almacenará el \gls{alimento} en aquella camara compartida que coincida con el rango de las especificaciones del alimento.
          \end{itemize}
\end{enumerate}

\begin{scheme}
    \centering
    IT/IT-1-InspeccionDeUnidades.pdf
\end{scheme}