\paragraph{Inspección física de la unidad de transporte}\label{sec:InspeccionTransporte}\index{Instrucción!inspección fisica de los alimentos}
\begin{enumerate}
\item Verificar la temperatura del alimento.\footnote{Todo alimento refrigerado o congelado recibido en la instalación debe ser revisado para asegurar que fue enviado de origen a la temperatura adecuada, verificando la documentación correspondiente.}
    \item Se revisarán todos los documentos de envío \emph{(números de lote, fechas, número de orden, etc.)} y la información requerida deberá ser registrada en el formulario \Oent;
    \item Se comprobará que si cumple con los criterios de acción para la aceptación (ver \cref{criterios:aceptacion})
    \item si es conforme la temperatura, se procederá a almacenar el alimento en la camara designada según sus especificaciones.
          \begin{itemize}
              \item si el \gls{alimento} no requiere de condiciones de almacenamiento especiales\footnote{diferentes a las especificaciones genéricas de almacenamiento (ver \cref{esp:generica}).} se almacenará en la cámara adecuada.
              \item si el \gls{alimento} requiere de condiciones de almacenamiento especiales y el cliente alquiló una camara exclusiva para su alimento, entonces se almacenará ahí;
              \item de no ser así, se almacenará el \gls{alimento} en aquella camara compartida que coincida con el rango de las especificaciones del alimento.
          \end{itemize}
\end{enumerate}