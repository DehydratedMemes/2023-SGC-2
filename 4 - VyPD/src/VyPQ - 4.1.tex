\thispagestyle{formato-PI}
\renewcommand{\MayorVer}{2}
\renewcommand{\MenorVer}{0}
\renewcommand{\Titulo}{Programa de control de vidrio y plástico quebradizo}
\renewcommand{\TipoID}{PRO}
\renewcommand{\FechaPub}{2023--01}
% \renewcommand{\Titulo}{Politica de calidad}

\section{\Titulo}\index{Programa!control de!vidrio y plástico quebradizo}
\renewcommand{\Codigo}{\Prog--\thesection--\TipoID}

\subsection{Objetivo}
Establecer procedimiento de verificación y mantenimiento de los artículos o instalaciones que contengan vidrio o plástico quebradizo para evitar o minimizar un posible riesgo de contaminación del producto por estos artículos.

\subsection{Alcance}
Todo material de vidrio o plástico quebradizo ubicados dentro del almacén.

\subsection{Terminología y definiciones}
\begin{description}
	\defglo{limpieza-y-sant}
\end{description}

\subsection{Documentos y/o normas relacionadas}
\begin{itemize}
	\item Programa de Buenas Prácticas de Distribución;
	\item Programa de Seguridad Alimentaria.
\end{itemize}

\subsection{Procedimiento}
El desarrollo contempla que el vidrio requiere estar protegido dentro de la planta, cuáles son las actividades que se tienen que realizar si se presenta una ruptura de vidrio o plástico quebradizo dentro de las áreas de Operación y como se direcciona el almacenamiento y uso de empaque de vidrio o plástico quebradizo. Además de cómo prevenir la contaminación por ruptura o astillamiento.
El programa contempla:

\begin{itemize}
	\item Frecuencia de verificación de control de vidrio y plástico quebradizo o quebradizo.
	\item Limpieza de áreas y equipo por contaminación de vidrio y plástico quebradizo.
	\item Procedimiento de que hacer en caso de contaminación de equipo, área, producto y material de empaque.
	\item Listado de vidrio y plástico quebradizo por área y mapas de su ubicación
	\item Lista de verificación.
	\item Todos los vidrios mencionados en la lista, están protegidos para evitar astillamiento o ruptura y así evitar contaminación del producto.
	\item Algunos ya son de fábrica inastillable, como todos los vidrios de las ventanas y lámparas del área de Almacenamiento.
	\item Las que no se encuentran en esta categoría deberán estar recubiertas con plástico adherente, capuchones plásticos o protectores (cubre polvos) para evitar los astillamientos en caso de ruptura.
	\item La verificación del control de vidrio y plástico quebradizo en las áreas de operación se realiza de manera MENSUAL por parte de parte del personal de mantenimiento.
	\item En caso de cualquier evento este se registrará en el momento ocurrido.
	\item El supervisor de almacén deberá informar inmediatamente de vidrio roto o plástico que sea detectado en el almacén.
	\item El supervisor de almacén debe inspeccionar la rotura para determinar si algún producto o empaque de alimentos ha sido contaminado. Con base en las observaciones del Supervisor de Almacén, se determinará si el producto puede ser salvado o, si está contaminado, debe ser enviado a cuarentena y se la dará aviso al cliente del evento sucedido
	\item En caso de ruptura se procederá al procedimiento de limpieza de áreas y equipo por contaminación de Vidrio y plástico quebradizo.
	\item\textbf{NOTA:} Se anexa el formato de inventario por áreas con el contenido de materiales de vidrio y plástico quebradizo, así como el mapa de la planta, con la ubicación en cada una de las áreas.
\end{itemize}

\subsection{Acciones correctivas}
\begin{itemize}
	\item Cuando se presente una no conformidad en la realización del procedimiento detectada por el supervisor o encargado del área se deberá repetir este mismo.
	\item Registrar acciones correctivas correspondientes.
\end{itemize}

\subsection{Frecuencia}
\begin{itemize}
	\item Mensualmente se llena la lista de verificación de materiales de vidrio y plástico quebradizo;
	\item cada evento por contaminación de vidrio o plástico quebradizo.
\end{itemize}


\begin{changelog}[simple, sectioncmd=\subsection*,label=changelog-\thesection-\MayorVer.\MenorVer]
	\begin{version}[v=\MayorVer.\MenorVer, date=2023--01, author=Pablo E. Alanis]
		\item Cambio de formato;
		\item Cambios en la serialización de versiones;
		\item Correcciones ortográficas y de estilo.
	\end{version}

	\begin{version}[v=1.7, date=2022--05, author=Alonso M.]
		\item cambio de fecha;
	\end{version}

	\shortversion{v=1.6, date=2021--05, changes=No hubo cambios}
\end{changelog}

