\thispagestyle{formato-PI}
\renewcommand{\MayorVer}{2}
\renewcommand{\MenorVer}{0}
\renewcommand{\Titulo}{Lavado de áreas y equipos por contaminación de vidrio y plástico quebradizo}
\renewcommand{\TipoID}{IT}
\renewcommand{\FechaPub}{2023--01}

\section{\Titulo}\index{Programa!control de!vidrio y plástico quebradizo!limpieza en caso de contaminación por VyPQ}
\renewcommand{\Codigo}{\Prog--\thesection--\TipoID}

\subsection{Objetivo}
Establecer procedimiento que asegure un mantenimiento y limpieza apropiados, manejo de los desechos y vigilancia de la efectividad de los procedimientos de las áreas, equipos y producto que estén involucrados en contaminación con vidrio y/o plástico quebradizo.

\subsection{Alcance}
Todas las áreas afectadas por partículas de vidrio y plástico quebradizo.

\subsection{Terminología y definiciones}
\begin{description}
	\defglo{limpieza-y-sant}
\end{description}

\subsection{Procedimiento}

\subsubsection{Materiales}

\begin{itemize}
	\item Garrafa identificada con agua y detergente a la concentración especificada por el proveedor.
	\item Cepillos para equipo de color rojo con franjas naranja.
	\item Cepillo para piso (escoba) color rojo con franjas naranja.
	\item Recogedor rojo con franjas naranja.
	\item Aspiradora.
	\item Bolsa de basura.
	\item Papel absorbente desechable.
	\item Contenedor para desperdicio.
\end{itemize}

\subsubsection{Instrucciones}

\paragraph{Contaminación de equipo y área}

\begin{itemize}
	\item Limpie el equipo.
	\item Lave el equipo (SEGÚN EL PROCEDIMIENTO DE LAVADO DE CADA EQUIPO)
	\item Lave o limpie los equipos que se encuentran en un radio de 15 metros (CADA EQUIPO SEGÚN EL PROCEDIMIENTO DE LIMPIEZA DE CADA UNO DE ELLOS)
	\item Lavar estanteros y todo inmobiliario que se encuentre en un radio de 5 metros.
	\item Barra con escoba en el piso en un radio de 5 metros.
	\item Junte basura con recogedor color rojo con franja naranja
	\item Coloque en bolsa para basura. (eliminar al momento y cambie la bolsa de basura)
	\item Aplicar agua con jabón en todas las partes a lavar del piso
	\item Cepille muy bien el piso con cepillo de piso color rojo con franja naranja.
	\item Enjuague con agua corriente.
\end{itemize}

\paragraph{De vidrio y/o plástico quebradizo}

\begin{itemize}
	\item En caso de evaluar un proceso de contaminación MUY GRAVE, separa el producto al área de rechazo e identifique el producto con el problema.
	\item Se tendrá comunicación con el cliente.
	\item Se procederá según las indicaciones del departamento de calidad del cliente.
	\item (Elimine el producto dañado)
	\item A continuación se procede a la limpieza de equipos y áreas
	\item Elimine material de empaque que se localice en un radio de 5mts (este se aparta y se manda al área de materiales no conformes)
	\item Lave el área (SEGUN EL PROCEDIMIENTO DE LIMPIEZA DE CADA EQUIPO)
	\item Lave o limpie los equipos que se encuentran en un radio de 5 metros (CADA EQUIPO SEGÚN EL PROCEDIMIENTO DE LIMPIEZA DE CADA UNO DE ELLOS)
	\item Lavar estanteros y todo inmobiliarios que se encuentre en un radio de 5 metros.
	\item Barra con escoba el piso en un radio de 5 metros.
	\item Junte basura con recogedor rojo con franja naranja.
	\item Coloque en bolsa para basura. (Eliminar al momento y cambie la bolsa de basura)
	\item Aplicar agua con jabón en todas las partes a lavar del piso
	\item Cepille muy bien el piso con cepillo de piso color rojo con franja naranja.
	\item Enjuague con agua corriente.
	\item Elimine el exceso de agua con jalador para piso color rojo con franja naranja.
	\item Recoger todos los artículos que fueron utilizados para ser acomodados en el área que corresponde.
	\item Avise al jefe inmediato que el trabajo ya está para su supervisión
	\item Registre el evento en el formato “registro de eventos de contaminación con vidrio y plástico quebradizo” numero de orden de mantenimiento si requiere cambio o reparación y el procedimiento de limpieza realizado de acuerdo al programa.
\end{itemize}

\subsection{Responsables de la actividad}

NOTA: cada uno se analizara y se tomara las acciones correctivas según su gravedad. Las personas autorizadas para valorar esta gravedad (Si la presencia lo permite la decisión se toma en conjunto) son:

\begin{itemize}
	\item Gerente de Operaciones.
	\item Cliente
	\item Jefe de Almacén
	\item Gerente General
\end{itemize}

\subsection{Acciones correctivas}

\begin{itemize}
	\item Cuando se presente una no conformidad en la realización del procedimiento, detectada por el supervisor o encargado del área se deberá de repetir este mismo hasta que sea corregida la desviación.
	\item Registrar en formato de acciones correctivas correspondientes.
\end{itemize}

\subsection{Frecuencia}

\begin{itemize}
	\item Cuando se presente evento de rotura de vidrio y/o plástico quebradizo.
\end{itemize}

\begin{changelog}[title=Registro de cambios,simple, sectioncmd=\subsection*,label=changelog-\thesection-\MayorVer.\MenorVer]
	\begin{version}[v=\MayorVer.\MenorVer, date=2023--01, author=Pablo E. Alanis]
		\item Cambio de formato;
		\item Cambios en la serialización de versiones;
		\item Correcciones ortográficas y de estilo.
	\end{version}

	\begin{version}[v=1.7, date=2022--05, author=Alonso M.]
		\item cambio de fecha;
	\end{version}

	\shortversion{v=1.6, date=2021--05, changes=No hubo cambios}
\end{changelog}

